\documentclass[rep_main]{subfiles}

\begin{document}

% \setcounter{}{}
\chapter{ベクトル空間の話}

\cite{Humphreys1972introduction}が前提としていそうな線型代数の知識をまとめておく.
$\mathbb{K}$ を\underline{任意の}体とする.また,選択公理を認める.

\section{Hom ベクトル空間}

\begin{mydef}[label=def:hom-vec]{Hom ベクトル空間}
    任意の $\mathbb{K}$-ベクトル空間 $V,\, W \in \Obj{\VEC{\mathbb{K}}}$ を与える.このとき
    \begin{itemize}
        \item 台集合を
        \begin{align}
            \bm{\Hom{\mathbb{K}} (V,\, W)} \coloneqq \bigl\{\, f \colon V \lto W \bigm| \mathbb{K}\text{-線型写像} \,\bigr\} 
        \end{align}
        とする.
        \item 加法とスカラー乗法を,$\forall f,\, f_1,\, f_2 \in \Hom{\mathbb{K}}(V,\, W),\; \forall \lambda \in \mathbb{K}$ について
        \begin{align}
            f_1 + f_2 \colon V &\lto W,\; v \lmto f_1(v) + f_2(v) \\
            \lambda f \colon V &\lto W,\; v \lmto \lambda \bigl(f(v)\bigr)
        \end{align}
        と定める.
    \end{itemize}
    ことで得られる $\mathbb{K}$-ベクトル空間のことを $\bm{\Hom{\mathbb{K}}(V,\, W)}$ と書く.
    \tcblower
    特に $\mathbb{K}$ 自身を $\mathbb{K}$-ベクトル空間と見做したとき,
    \begin{align}
        \bm{V^*} \coloneqq \Hom{\mathbb{K}}(V,\mathbb{K})
    \end{align}
    と書いて\textbf{双対ベクトル空間} (dual vector space) と呼ぶ.
\end{mydef}

特に\underline{有限次元} $\mathbb{K}$-ベクトル空間の場合は,双対ベクトル空間の基底が次のように構成される:

\begin{myprop}[label=prop:dual-basis]{双対ベクトル空間の基底}
    \underline{有限次元} $\mathbb{K}$-ベクトル空間 $V$ とその基底 $\{e_\mu\}$ を与える.
    このとき $\dim V$ 個の線型写像 $\varepsilon^\mu \in V^*$ を
    \begin{align}
        \varepsilon^\mu (e_\nu) \coloneqq \delta^\mu_\nu
    \end{align}
    により定義すると,$\{\varepsilon^\mu\}$ は $V^*$ の基底になる.
\end{myprop}


\section{ベクトル空間のテンソル積}


\subsection{普遍性による定義}

$\mathbb{K}$-ベクトル空間 $V,\, W,\, Z$ を与える.このとき写像 $f \colon V \times W \lto Z$ が\textbf{双線型} (bilinear) であるとは,2つの引数について\underline{それぞれ}\footnote{つまり,\hyperref[def:univ-vec-sum]{直積ベクトル空間} $V \times W$ から $Z$ への線型写像\underline{ではない}.} $\mathbb{K}$-線型であることを言う.i.e. $\forall v,\, v_1,\, v_2 \in V,\; \forall w,\, w_1,\, w_2 \in W,\; \forall \lambda \in \mathbb{K}$ に対して
\begin{align}
    f(v_1 + v_2,\, w) &= f(v_1,\, w) + f(v_2,\, w) \\
    f(v,\, w_1 + w_2) &= f(v,\, w_1) + f(v,\, w_2) \\
    f(\lambda v,\, w) &= \lambda f(v,\, w) \\
    f(v,\, \lambda w) &= \lambda f(v,\, w)
\end{align}
が成り立つこと.
同様に $\mathbb{K}$-ベクトル空間 $V_1,\, \dots,\, V_n,\, W$ に関して,写像 $f \colon V_1 \times \cdots \times V_n \lto W$ が\textbf{多重線型} (multillinear) であるとは,全ての引数について\underline{それぞれ} $\mathbb{K}$-線型であることを言う.

\begin{mydef}[label=def:univ-vec-tensor]{ベクトル空間のテンソル積}
    $\mathbb{K}$ を任意の体とし,$\mathbb{K}$-ベクトル空間 $V,\, W$ を与える.
    \begin{itemize}
        \item $\mathbb{K}$-ベクトル空間 $\bm{V \otimes W}$ 
        \item 双線型写像 $\bm{\Phi} \colon V \times W \lto \bm{V \otimes W}$ 
    \end{itemize}
    の2つ組 $(V \otimes W,\, \Phi)$ が $V,\, W$ の\textbf{テンソル積} (tensor product) であるとは,以下の性質を充たすことをいう:
    \begin{description}
        \item[\textbf{(テンソル積の普遍性)}] 
        
        任意の $\mathbb{K}$-ベクトル空間 $\textcolor{blue}{Z}$ および任意の双線型写像 $\textcolor{blue}{f} \colon V \times W \lto \textcolor{blue}{Z}$ に対して,以下の図式を可換にする線型写像 $\textcolor{red}{\overline{f}} \colon \bm{V \otimes W} \lto \textcolor{blue}{Z}$ が一意的に存在する:
        \begin{center}
            \begin{tikzcd}[row sep=large, column sep=large]
                V \times W \ar[d, "\bm{\Phi}"']\ar[r, blue, "f"] &\forall \textcolor{blue}{Z} \\
                \bm{V \otimes W} \arrow[ur, red, dashed, "\exists!\bar{f}"']&
            \end{tikzcd}
        \end{center}
    \end{description}
\end{mydef}

テンソル積をとる操作は結合的かつ対称である.i.e. $V_1 \otimes (V_2 \otimes V_3) \cong (V_1 \otimes V_2) \otimes V_3$ および $V_1 \otimes V_2 \cong V_2 \otimes V_1$ が成り立つ.
従って以降では3つ以上のベクトル空間のテンソル積を括弧を省略して書く.

\begin{myprop}[label=prop:unique-tensor-vec]{テンソル積の一意性}
	\hyperref[def:univ-vec-tensor]{テンソル積}は,存在すればベクトル空間の同型を除いて一意である.
\end{myprop}

\begin{proof}
	$\mathbb{K}$ 上のベクトル空間 $V,\, W \in \Obj{\VEC{\mathbb{K}}}$ を与える.
	体 $\mathbb{K}$ 上のベクトル空間と双線型写像の組 $(\bm{T},\, \bm{\Phi} \colon V \times W \lto \bm{T})$ および $(\bm{T},\, \bm{\Phi'} \colon V \times W \lto \bm{T'})$ がどちらも $V,\, W$ の\hyperref[def:univ-vec-tensor]{テンソル積}であるとする.
	このとき\hyperref[def:univ-vec-tensor]{テンソル積の普遍性}から $\VEC{\mathbb{K}}$ の\hyperref[def:commutative]{可換図式}
	\begin{figure}[H]
		\centering
		\begin{subfigure}{0.4\columnwidth}
			\centering
			\begin{tikzcd}[row sep=large, column sep=large]
				&V \times W \ar[d, "\bm{\Phi}"] \ar[r, blue, "\Phi'"] &\textcolor{blue}{T'} \\
				&\bm{T} \ar[ur, red, dashed, "\exists! u"] &
			\end{tikzcd}
		\end{subfigure}
		\hspace{5mm}
		\begin{subfigure}{0.4\columnwidth}
			\centering
			\begin{tikzcd}[row sep=large, column sep=large]
				&V \times W \ar[d, "\bm{\Phi'}"] \ar[r, blue, "\Phi"] &\textcolor{blue}{T} \\
				&\bm{T'} \ar[ur, red, dashed, "\exists! u'"] &
			\end{tikzcd}
		\end{subfigure}
	\end{figure}%
	が成り立つので,これらの図式を併せた $\VEC{\mathbb{K}}$ の可換図式
	\begin{center}
		\begin{tikzcd}[row sep=large, column sep=large]
			&\bm{T} & \\
			&V \times W \ar[u, "\bm{\Phi}"]\ar[d, "\bm{\Phi}"] \ar[r, blue, "\Phi'"] &\textcolor{blue}{T'}\ar[ul, red, dashed, "\exists! u'"] \\
			&\bm{T} \ar[ur, red, dashed, "\exists! u"] &
		\end{tikzcd}
	\end{center}
	が存在する.然るに $\VEC{\mathbb{K}}$ の可換図式
	\begin{center}
		\begin{tikzcd}[row sep=large, column sep=large]
			&V \times W \ar[d, "\bm{\Phi}"] \ar[r, "\bm{\Phi}"] &\bm{T} \\
				&\bm{T} \ar[ur, red, dashed, "\mathrm{id}_T"] &
		\end{tikzcd}
	\end{center}
	も成り立ち,\hyperref[def:univ-vec-tensor]{テンソル積の普遍性}より赤点線で書いた線型写像は一意でなくてはならないので,
	\begin{align}
		\textcolor{red}{u'} \circ \textcolor{red}{u} = \mathrm{id}_T
	\end{align}
	がわかる.同様の議論から
	\begin{align}
		\textcolor{red}{u} \circ \textcolor{red}{u'} = \mathrm{id}_{T'}
	\end{align}
	も従うので,線型写像 $\textcolor{red}{u} \colon \bm{T} \lto \bm{T'},\; \textcolor{red}{u'} \colon \bm{T'} \lto \bm{T}$ は互いに逆写像,i.e. 同型写像である.
\end{proof}


命題\ref{prop:unique-tensor-vec}からテンソル積の一意性が言えたが,そもそもテンソル積が存在しなければ意味がない.
そこで,\hyperref[ax.ring]{体} $\mathbb{K}$ 上の任意のベクトル空間 $V,\, W \in \Obj{\VEC{\mathbb{K}}}$ を素材にして\hyperref[def:univ-vec-tensor]{テンソル積} $(V \otimes W,\, \Phi \colon V \times W \lto V \otimes W)$ を具体的に構成してみよう.

$\mathbb{K} \in \Obj{\VEC{\mathbb{K}}}$ なので,任意の集合 $S$ に対して\hyperref[def:univ-vec-sum]{ベクトル空間の直和}
\begin{align}
	\mathbb{K}^{\oplus S} \in \Obj{\VEC{\mathbb{K}}}
\end{align}
を考えることができる.$\mathbb{K}^{\oplus S}$ の元 $f$ とは,命題\ref{prop:subvec-directsum}からわかるように有限個の元 $x_1,\, \dots,\; x_n \in S$ を除いた全ての $x \in S$ に対して値 $0 \in \mathbb{K}$ を返すような $\mathbb{K}$ 値関数
$f \colon S \lto \mathbb{K}$ のことである
% \footnote{これは集合論の記法である:ある集合 $\Lambda$ から集合 $A$ への写像 $a \colon \Lambda \lto A$ のことを $\Lambda$ によって\textbf{添字づけられた} $A$ の元の\textbf{族}と呼び,$\forall \lambda \in \Lambda$ に対して $a(\lambda) \in A$ のことを $\bm{a_\lambda}$ と書き,$\bm{a \colon \Lambda \lto A}$ \textbf{自身のこと}を $\bm{(a_\lambda)_{\lambda \in \Lambda}}$ と書くのである.
% なお,$\Familyset[\big]{a_\lambda}{\lambda \in \Lambda}$ と書いたときは $A$ の部分集合 $\bigl\{\, a(\lambda) \bigm| \lambda \in \Lambda \,\bigr\} \subset A$ のことを意味する.}.

ところで,$\forall x \in S$ に対して次のような関数 $\delta_{x} \in \mathbb{K}^{\oplus S}$ が存在する:
\begin{align}
	\delta_{x} \bigl(y\bigr) =
	\begin{cases}
		1, &y = x \\
		0, &y \neq x
	\end{cases}
\end{align}
この $\delta_{x}$ を $x$ そのものと同一視してしまうことで,先述の $f \in \mathbb{K}^{\oplus (V \times W)}$ は
\begin{align}
	f = \sum_{i=1}^n \lambda_i x_i \quad \WHERE \lambda_i \coloneqq f (x_i) \in \mathbb{K}
\end{align}
の形に一意的に書ける.
\footnote{
	というのも,このように書けば $\forall y \in S$ に対して
	\begin{align}
		f (y)  &= \sum_{i=1}^n \lambda_i \delta_{x_i}(y) =
		\begin{cases}
			f (x_i), &y = x_i \\
			0, &\text{otherwise}
		\end{cases}
	\end{align}
	が言えるので.特に,この式の中辺は $\mathbb{K}$ の元の有限和なので意味を持つ.
}
この意味で,$\mathbb{K}^{\oplus (V \times W)}$ は $V \times W$ の元の\textbf{形式的な} $\bm{\mathbb{K}}$ \textbf{係数線形結合}全体がなす $\mathbb{K}$ ベクトル空間と言うことができ,集合 $V \times W$ 上の\textbf{自由ベクトル空間}と呼ばれる.
\hyperref[def:free-mod]{自由加群}の特別な場合と言っても良い.自由ベクトル空間は次の普遍性によって特徴づけられる:

\begin{mylem}[label=lem:univ-free-vec, breakable]{自由ベクトル空間の普遍性}
	任意の\underline{集合} $S$ および任意の $\mathbb{K}$ ベクトル空間 $\textcolor{blue}{Z} \in \Obj{\VEC{\mathbb{K}}}$ を与える.
	包含写像
	\begin{align}
		\iota \colon S \lto \mathbb{K}^{\oplus S},\; x \lmto \delta_x
	\end{align}
	を考える.
	このとき,任意の写像 $\textcolor{blue}{f} \colon S \lto \textcolor{blue}{Z}$ に対して線型写像
	$\textcolor{red}{u} \colon \mathbb{K}^{\oplus S} \lto \textcolor{blue}{Z}$ が一意的に存在して,図式\ref{cmtd:free-vec}を可換にする:
	\begin{figure}[H]
		\centering
		\begin{tikzcd}[row sep=large, column sep=large]
			&S \ar[r, blue, "f"]\ar[d, "\iota"] &\forall \textcolor{blue}{Z} \\
			&\mathbb{K}^{\oplus S} \ar[ur, red, dashed, "\exists! u"] &
		\end{tikzcd}
		\caption{自由ベクトル空間の普遍性}
		\label{cmtd:free-vec}
	\end{figure}%
\end{mylem}

\begin{proof}
	写像
	\begin{align}
		\textcolor{red}{u} \colon \mathbb{K}^{\oplus S} \lto \textcolor{blue}{Z},\; \sum_{i=1}^n \lambda_i \delta_{x_i} \lmto \sum_{i=1}^n \lambda_i \textcolor{blue}{f}(x_i)
	\end{align}
	は右辺が有限和なのでwell-definedであり,$\forall x \in S$ に対して $\textcolor{red}{u} \bigl( \iota(x) \bigr) = \textcolor{blue}{f}(x)$ を充たす.
	
	別の線型写像 $g \colon \mathbb{K}^{\oplus S} \lto \textcolor{blue}{Z}$ が $g \circ \iota = \textcolor{blue}{f}$ を充たすとする.このとき $\forall x \in S$ に対して $g(\delta_x) = \textcolor{blue}{f}(x)$ であるから,
	$\forall v = \sum_{i=1}^n \lambda_i \delta_{x_i} \in \mathbb{K}^{\oplus S}$ に対して
	\begin{align}
		g(v) = g \left( \sum_{i=1}^n \lambda_i \delta_{x_i} \right) = \sum_{i=1}^n \lambda_i g(\delta_{x_i}) = \sum_{i=1}^n \lambda_i \textcolor{blue}{f}(x_i) = \textcolor{red}{u} (v)
	\end{align}
	が言える.よって $g = \textcolor{red}{u}$ である.
\end{proof}

さて,\hyperref[cmtd:free-vec]{自由加群の普遍性の図式}と\hyperref[def:univ-vec-tensor]{テンソル積の普遍性の図式}はとても似ているので,$\bm{V \otimes W} \in \Obj{\VEC{\mathbb{K}}}$ の候補として $\mathbb{K}^{\oplus (V \times W)}$ を考えてみる.
しかしそのままでは $\iota \colon V \times W \lto \mathbb{K}^{\oplus (V \times W)}$ が双線型写像になってくれる保証はない.
そこで,
\begin{align}
	\iota(\lambda v,\, w) &\sim \lambda \iota(v,\, w), \\
	\iota(v,\, \lambda w) &\sim \lambda \iota(v,\, w), \\
	\iota(v_1 + v_2,\, w) &\sim \iota(v_1,\, w) + \iota(v_2,\, w), \\
	\iota(v,\, w_1 + w_2) &\sim \iota(v,\, w_1) + \iota(v,\, w_2)
\end{align}
を充たすような上手い\hyperref[ax.eq]{同値関係}による\hyperref[prop:quotient-vec]{商ベクトル空間}を構成する.
% そこで,$\iota (V \times W) \in \mathbb{K}^{\oplus (V \times W)}$ の上に適切な関係式を課してできる $\mathbb{K}^{\oplus (V \times W)}$ の部分集合が\hyperref[prop:gen-submodule]{生成する部分ベクトル空間}で $\mathbb{K}^{\oplus (V \times W)}$ の\hyperref[prop:quotient-vec]{商ベクトル空間}を作ることを試みる.
% \hyperref[prop:universality-dp]{直和の普遍性}から $\forall \textcolor{blue}{V} \in \Obj{\VEC{\mathbb{K}}}$ に対して可換図式
% \begin{center}
% 	\begin{tikzcd}[row sep=large, column sep=large]
% 		&V \times W \ar[d, "\bm{\Phi}"] \ar[r, blue, "f"] &\forall \textcolor{blue}{Z} \\
% 		&\bm{V \otimes W} \ar[ur, red, dashed, "\exists! u"] &
% 	\end{tikzcd}
% \end{center}


\begin{myprop}[label=prop:tensor-vec, breakable]{テンソル積の構成}
	$\mathbb{K}^{\oplus (V \times W)}$ の部分集合
	\begin{align}
		S_1 &\coloneqq \bigl\{\iota(\lambda v,\, w) - \lambda \iota(v,\, w) \bigm| \forall v \in V,\, \forall w \in W,\, \forall \lambda \in \mathbb{K} \bigr\} , \\
		S_2 &\coloneqq \bigl\{\iota(v,\, \lambda w) - \lambda \iota(v,\, w) \bigm| \forall v \in V,\, \forall w \in W,\, \forall \lambda \in \mathbb{K} \bigr\}, \\
		S_3 &\coloneqq \bigl\{\iota(v_1 + v_2,\, w) - \iota(v_1,\, w) - \iota(v_2,\, w) \bigm| \forall v_1,\, \forall v_2 \in V,\, \forall w \in W,\, \forall \lambda \in \mathbb{K} \bigr\}, \\
		S_4 &\coloneqq \bigl\{\iota(v,\, w_1 + w_2) - \iota(v,\, w_1) - \iota(v,\, w_2) \bigm| \forall v \in V,\, \forall w_1,\, w_2 \in W,\, \forall \lambda \in \mathbb{K} \bigr\}
	\end{align}
	の和集合 $S_1 \cup S_2 \cup S_3 \cup S_4$ が\hyperref[prop:gen-submodule]{生成する} $\mathbb{K}$ ベクトル空間\footnote{これらの元の形式的な$\mathbb{K}$ 係数線型結合全体が成すベクトル空間のこと.}を $\mathcal{R}$ と書き,
	\hyperref[prop:quotient-vec]{商ベクトル空間}
	$\mathbb{K}^{\oplus (V \times W)} / \mathcal{R}$ 
	の商写像を
	\begin{align}
		\pi \colon \mathbb{K}^{\oplus (V \times W)} \lto \mathbb{K}^{\oplus (V \times W)} / \mathcal{R},\; \sum_{i=1}^n \lambda_i \iota(v_i,\, w_i) \lmto \left( \sum_{i=1}^n \lambda_i \iota(v_i,\, w_i) \right) + \mathcal{R}
	\end{align}
	と書き,$\bm{v \otimes w} \coloneqq \pi \bigl( \iota (v,\, w) \bigr)$ とおく.
	このとき,
	\begin{itemize}
		\item $\mathbb{K}$ ベクトル空間 $\bm{\mathbb{K}^{\oplus (V \times W)} / \mathcal{R}}$
		\item 写像 $\bm{\Phi} = \pi \circ \iota \colon V \times W \lto \bm{\mathbb{K}^{\oplus (V \times W)} / \mathcal{R}},\; (v,\, w) \lmto v \otimes w$
	\end{itemize}
	の組は $V,\, W$ の\hyperref[def:univ-vec-tensor]{テンソル積}である.
\end{myprop}

\begin{proof}
	まず,$\bm{\Phi}$ が双線型写像であることを示す.\hyperref[prop:quotient-vec]{商ベクトル空間}の和とスカラー乗法の定義から
	\begin{align}
		\Phi (\lambda v,\, w) &= \iota(v,\, w) + \mathcal{R} = \bigl( \lambda \iota(v,\, w) + \iota (\lambda v,\, w) - \lambda \iota (v,\, w) \bigr) + \mathcal{R} \\
		&= \lambda \iota (v,\, w) + \mathcal{R} = \lambda \bigl( \iota(v,\, w) + \mathcal{R} \bigr) = \lambda \Phi(v,\, w) \\
		\Phi (v_1 + v_2,\, w) &= \iota(v_1 + v_2,\, w) + \mathcal{R} = \bigl( \iota(v_1,\, w) + \iota (v_2,\, w) + \iota(v_1+v_2,\, w) - \iota (v_1,\, w) - \iota(v_2,\, w) \bigr) + \mathcal{R} \\
		&= \bigl(\iota (v_1,\, w) + \iota (v_2,\, w) \bigr) + \mathcal{R} = \bigl( \iota(v_1,\, w) + \mathcal{R} \bigr) + \bigl( \iota(v_2,\, w) + \mathcal{R} \bigr) \\
		&= \Phi(v_1,\, w) + \Phi(v_2,\, w)
	\end{align}
	が言える.第2引数に関しても同様であり,$\bm{\Phi}$ は双線型写像である.

	次に,上述の構成が\hyperref[def:univ-vec-tensor]{テンソル積の普遍性}を充たすことを示す.$\forall  \textcolor{blue}{Z} \in \Obj{\VEC{\mathbb{K}}}$ と任意の双線型写像 $\textcolor{blue}{f} \colon V \times W \lto \textcolor{blue}{Z}$ を与える.
	\hyperref[lem:univ-free-vec]{自由ベクトル空間の普遍性}から $\VEC{\mathbb{K}}$ の可換図式
	\begin{center}
		\begin{tikzcd}[row sep=large, column sep=large]
			&V \times W \ar[r, blue, "f"]\ar[d, "\iota"'] &\forall \textcolor{blue}{Z} \\
			&\mathbb{K}^{\oplus (V \times W)} \ar[ur, red, dashed, "\exists! \overline{f}"] &
		\end{tikzcd}
	\end{center}
	が存在する.$\textcolor{blue}{f}$ が双線型なので,
	\begin{align}
		\textcolor{red}{\overline{f}} \bigl( \iota (\lambda v,\, w) \bigr) &= f (\lambda v,\, w) = \lambda f(v,\, w) \\ 
		&= \lambda\textcolor{red}{\overline{f}} \bigl( \iota (v,\, w) \bigr) = \textcolor{red}{\overline{f}}\bigl( \lambda \iota(v,\, w) \bigr) , \\
		\textcolor{red}{\overline{f}} \bigl( \iota (v_1 + v_2,\, w) \bigr) &= f (v_1 + v_2,\, w) = f(v_1,\, w) + f(v_2,\, w)\\
		&= \textcolor{red}{\overline{f}} \bigl( \iota(v_1,\, w) \bigr) + \textcolor{red}{\overline{f}} \bigl(\iota(v_2,\, w) \bigr) = \textcolor{red}{\overline{f}} \bigl( \iota(v_1,\, w) + \iota(v_2,\, w)\bigr) 
	\end{align}
	が成り立つ.第2引数についても同様なので,$\mathcal{R} \subset \Ker \textcolor{red}{\overline{f}}$ である.よって\hyperref[thm.homo1]{準同型定理}から
	$\VEC{\mathbb{K}}$ の可換図式
	\begin{center}
		\begin{tikzcd}[row sep=large, column sep=large]
			&V \times W \ar[r, blue, "f"]\ar[d, "\iota"'] &\forall \textcolor{blue}{Z} \\
			&\mathbb{K}^{\oplus (V \times W)} \ar[d, "\pi"']\ar[ur, red, dashed, "\exists! \overline{f}"] & \\
			&\mathbb{K}^{\oplus (V \times W)} /\mathcal{R}\ar[uur, red, dashed, "\exists! u"] &
		\end{tikzcd}
	\end{center}
	が存在する.この図式の外周部は\hyperref[def:univ-vec-tensor]{テンソル積の普遍性の図式}である.
\end{proof}

\subsection{多重線型写像とテンソル積}

任意の $\mathbb{K}$-ベクトル空間\footnote{有限次元でなくても良い.} $V_1,\, \dots,V_n,\, W \in \Obj{\VEC{\mathbb{K}}}$ に対して,集合
\begin{align}
	\bm{L(V_1},\,\dots,\, \bm{V_n;\, W)} \coloneqq \bigl\{\, F \colon V_1 \times \cdots \times V_n \lto W \bigm| F\;\text{は多重線型写像} \,\bigr\} 
\end{align}
を考える.$L(V_1,\, \dots,\, V_n;\, W)$ の上の加法とスカラー乗法を $\forall v_i \in V_i,\, \forall \lambda \in \mathbb{K}$ に対して
\begin{align}
	(F + G)(v_1,\, \dots,\, v_n) &\coloneqq F(v_1,\, \dots,\, v_n) + G(v_1,\, \dots,\, v_n), \\
	(\lambda F)(v_1,\, \dots,\, v_n) &\coloneqq \lambda \bigl(F(v_1,\, \dots,\, v_n)\bigr)
\end{align}
と定義すると $L(V_1,\, \dots,\, V_n;\, W)$ は $\mathbb{K}$ ベクトル空間になる.
特に,\hyperref[def:hom-vec]{Hom の定義}から $\mathbb{K}$-ベクトル空間の等式として
\begin{align}
    L(V;\, W) = \Hom{\mathbb{K}} (V,\, W)
\end{align}
が成り立つ.\hyperref[def:univ-vec-tensor]{テンソル積の普遍性}はこの等式を多重線型写像について一般化するものである.

\begin{myprop}[label=prop:tensor-multillinear]{多重線型写像とテンソル積}
    任意の $\mathbb{K}$-ベクトル空間 $V_1,\, \dots,\, V_n,\, W \in \Obj{\VEC{\mathbb{K}}}$ に対して,$\mathbb{K}$-ベクトル空間として
    \begin{align}
        L(V_1,\, \dots,\, V_n;\, W) \cong \Hom{\mathbb{K}}(V_1 \otimes \cdots \otimes V_n,\, W)
    \end{align}
    が成り立つ.
\end{myprop}

\begin{proof}
    \hyperref[def:univ-vec-tensor]{テンソル積の普遍性}から,$\mathbb{K}$-線型写像
    \begin{align}
        \alpha \colon \Hom{\mathbb{K}}(V_1 \otimes \cdots \otimes V_n,\, W) \lto L(V_1,\, \dots,\, V_n;\, W),\; f \lmto f \circ \Phi
    \end{align}
    は全単射,i.e. $\mathbb{K}$-ベクトル空間の同型写像である.
\end{proof}

\subsection{有限次元ベクトル空間のテンソル積}

\begin{myprop}[label=prop:basis-tensor]{有限次元テンソル積の基底}
	\underline{有限次元} $\mathbb{K}$-ベクトル空間 $V,\, W$($\dim V \eqqcolon n,\; \dim W \eqqcolon m$)を与える.
    $V,\, W$ の基底をそれぞれ $\{e_1,\, \dots,\, e_n\},\; \{f_1,\, \dots,\, f_m\}$ と書く.
	このとき,集合
	\begin{align}
		\mathcal{E} \coloneqq \bigl\{\, e_\mu \otimes f_\nu \bigm| 1 \le \mu \le n,\, 1 \le \nu \le m \,\bigr\} 
	\end{align}
	は $V \otimes W$ の基底である.従って $\dim V \otimes W = nm$ である.
\end{myprop}

\begin{proof}
	\hyperref[prop:tensor-vec]{テンソル積の構成}から,$\forall t \in V \otimes W$ は有限個の $(v_i,\, w_i) \in V \times W\; (i=1,\, \dots l)$ を使って
	\begin{align}
		t = \left(\sum_{i=1}^l t_{i} \iota(v_i,\, w_i)\right) = \sum_{i=1}^l t_i v_i \otimes w_i
	\end{align}
	と書ける.$v_i = v_i{}^\mu e_\mu,\; w_i = w_i{}^{\mu} f_\mu$ のように展開することで,
	\begin{align}
		t &= \sum_{i=1}^l t_i (v_i{}^\mu e_\mu) \otimes (w_i{}^\nu f_\nu) \\
		&= \sum_{i=1}^l t_i v_i{}^\mu w_i{}^\nu e_{\mu} \otimes f_\nu
	\end{align}
	と書ける.ただし添字 $\mu,\, \nu$ に関してはEinsteinの規約を適用した.従って $\mathcal{E}$ は $V \otimes W$ を生成する.

	$\mathcal{E}$ の元が線型独立であることを示す.
	\begin{align}
		t^{\mu\nu} e_\mu \otimes e_{\nu} = 0
	\end{align}
	を仮定する.$\{e_\mu\},\, \{f_\mu\}$ の\hyperref[def.basisforDVS]{双対基底}をそれぞれ $\{\varepsilon^\mu\},\, \{\eta^\nu\}$ と書き,
	全ての添字の組み合わせ $(\mu,\, \nu) \in \{	1,\, \dots,\, n\} \times \{1,\, \dots,\, m\}$ に対して双線型写像
	\begin{align}
		\tau^{\mu\nu} \colon V \times W \lto \mathbb{R},\; (v,\, w) \lmto \varepsilon^\mu (v) \eta^\nu (w)
	\end{align}
	を定める.$\tau^{\mu\nu}$ は双線型なので\hyperref[cmtd:tensor-vec]{テンソル積の普遍性}から $\VEC{\mathbb{K}}$ の可換図式
	\begin{center}
		\begin{tikzcd}[row sep=large, column sep=large]
			&V \times W \ar[d, "\pi \circ \iota"'] \ar[r,"\tau^{\mu\nu}"] &\mathbb{R} \\
			&\bm{V \otimes W} \ar[ur, red, dashed, "\exists! \overline{\tau}^{\mu\nu}"'] &
		\end{tikzcd}
	\end{center}
	が存在する.このことは,
	\begin{align}
		0 &= \textcolor{red}{\overline{\tau}^{\mu\nu}} (t^{\rho\sigma} e_{\rho} \otimes f_{\sigma}) \\
		&= t^{\rho\sigma} (\textcolor{red}{\overline{\tau}^{\mu\nu}} \circ \pi \circ \iota) (e_\rho ,\,  f_\sigma) \\
		&= t^{\rho\sigma} \tau^{\mu\nu}(e_\rho ,\,  f_\sigma) = t^{\mu\nu}
	\end{align}
	を意味する.従って$\mathcal{E}$ の元は線型独立である.
\end{proof}

これでもまだ直接の計算には向かない.より具体的な構成を探そう.
$\forall \omega_i \in V_i^*$ に対して,$\bm{\omega_1 \otimes \cdots \otimes \omega_n}$ と書かれる $L(V_1,\, \dots,\, V_n;\, \mathbb{K})$ の元を
\begin{align}
	\bm{\omega_1 \otimes \cdots \otimes \omega_n} \colon V_1 \times \cdots \times V_n \lto \mathbb{K},\; (v_1,\, \dots,\, v_n) \lmto \prod_{i=1}^n \omega_i(v_i)
\end{align}
によって定義する.ただし右辺の総積記号は $\mathbb{K}$ の積についてとる.

\begin{myprop}[label=prop:basis-L]{}
	有限次元 $\mathbb{K}$-ベクトル空間 $V,\, W$($\dim V \eqqcolon n,\, \dim W \eqqcolon m$)の基底をそれぞれ $\{e_\mu\},\, \{f_\nu\}$ と書き,その\hyperref[def.basisforDVS]{双対基底}をそれぞれ $\{\varepsilon^\mu\},\, \{\eta^\mu\}$ と書く.このとき,集合
	\begin{align}
		\mathcal{B} \coloneqq \bigl\{\, \varepsilon^\mu \otimes \eta^\nu \bigm| 1 \le \mu \le n,\, 1 \le \nu \le m \,\bigr\} 
	\end{align}
	は $L(V,\, W;\, \mathbb{K})$ の基底である.従って $\dim L(V,\, W;\, \mathbb{K}) = nm$ である.
\end{myprop}

\begin{proof}
	$\forall F  \in L(V,\, W;\, \mathbb{K})$ を1つとり,全ての添字の組み合わせ $(\mu,\, \nu)$ に対して
	\begin{align}
		F_{\mu\nu} \coloneqq F(e_\mu,\, f_\nu)
	\end{align}
	とおく.$\forall (v,\, w) \in V \times W$ を $v = v^\mu e_\mu,\; w = w^\nu f_\nu$ と展開すると,
	\begin{align}
		F_{\mu\nu} \varepsilon^\mu \otimes \eta^\nu (v,\, w) &= F_{\mu\nu} \varepsilon^{\mu}(v) \eta^\nu(w) \\
		&= F_{\mu\nu} v^\mu w^\nu
	\end{align}
	が成り立つ.一方,双線型性から
	\begin{align}
		F(v,\, w) = v^\mu w^\nu F(e_\mu,\, f_\nu) = F_{\mu\nu} v^\mu w^\nu
	\end{align}
	も成り立つので $F = F_{\mu\nu} \varepsilon^\mu \otimes \eta^\nu$ が言えた.i.e. 集合 $\mathcal{B}$ は $L(V,\, W;\, \mathbb{K})$ を生成する.

	次に,$\mathcal{B}$ の元が線型独立であることを示す.
	\begin{align}
		F_{\mu\nu} \varepsilon^\mu \otimes \eta^\nu = 0
	\end{align}
	を仮定する.全ての添字の組み合わせについて,$(e_\mu,\, f_\nu)$ に左辺を作用させることで,$F_{\mu\nu} = 0$ が従う.i.e. $\mathcal{B}$ の元は互いに線型独立である.
\end{proof}


\begin{myprop}[label=prop:fin-tensor-dual]{テンソル積の構成その2}
	任意の\underline{有限次元} $\mathbb{K}$-ベクトル空間 $V,\, W$ に対して
    \begin{align}
		L(V,\, W;\, \mathbb{K}) \cong V^* \otimes W^*
	\end{align}
\end{myprop}

\begin{marker}
    命題\ref{prop:tensor-multillinear}より,これは 
    \begin{align}
        L(V,\, W;\, \mathbb{K}) \cong \Hom{\mathbb{K}}(V \otimes W,\, \mathbb{K}) = (V \otimes W)^* \cong V^* \otimes W^*
    \end{align}
    と同値である.
\end{marker}

\begin{proof}
	写像
	\begin{align}
		\Phi \colon V^* \times W^* &\lto L(V,\, W;\, \mathbb{K}), \\
		(\omega,\, \eta) &\lmto \bigl( (v,\, w) \lmto \omega(v) \eta(w) \bigr) 
	\end{align}
	は双線型写像なので
	\hyperref[cmtd:tensor-vec]{テンソル積の普遍性}から $\VEC{\mathbb{K}}$ の可換図式
	\begin{center}
		\begin{tikzcd}[row sep=large, column sep=large]
			&V^* \times W^* \ar[d, "\pi \circ \iota"'] \ar[r,"\Phi"] &L(V,\, W;,\, \mathbb{K}) \\
			&\bm{V^* \otimes W^*} \ar[ur, red, dashed, "\exists! \overline{\Phi}"'] &
		\end{tikzcd}
	\end{center}
	が存在する.
	$V,\, W$($\dim V = n,\, \dim W = m$)の基底をそれぞれ $\{e_\mu\},\, \{f_\nu\}$ と書き,その\hyperref[def.basisforDVS]{双対基底}をそれぞれ $\{\varepsilon^\mu\},\, \{\eta^\mu\}$ と書く.
	命題\ref{prop:basis-tensor}より $V^* \otimes W^*$ の基底として
	\begin{align}
		\mathcal{E} \coloneqq \bigl\{\, \varepsilon^\mu \otimes \eta^\nu \bigm| 1 \le \mu \le n,\, 1 \le \nu \le m \,\bigr\} 
	\end{align}
	がとれ,命題\ref{prop:basis-L}より $L(V,\, W;\, \mathbb{K})$ の基底として
	\begin{align}
		\mathcal{B} \coloneqq \bigl\{\, \varepsilon^\mu \otimes \eta^\nu \bigm| 1 \le \mu \le n,\, 1 \le \nu \le m \,\bigr\} 
	\end{align}
	がとれる(記号が同じだが,違う定義である).
	このとき,$\forall (v,\, w) \in V \times W$ に対して
	\begin{align}
		\textcolor{red}{\overline{\Phi}}(\varepsilon^\mu \otimes \eta^\nu )(v,\, w) = \textcolor{red}{\overline{\Phi}}\circ \pi \circ \iota(\varepsilon^\mu,\,\eta^\nu )(v,\, w) = \Phi(\varepsilon^\mu,\, \eta^\nu)(v,\, w) = \varepsilon^\mu (v) \eta^\nu (w) = \varepsilon^\mu \otimes \eta^\nu(v,\, w)
	\end{align}
	が成り立つ(ただし,左辺の $\otimes$ は命題\ref{prop:tensor-vec},右辺は命題\ref{prop:basis-L}で定義したものである)ので,
	$\textcolor{red}{\overline{\Phi}}$ は $\mathcal{E}$ の元と $\mathcal{B}$ の元の1対1対応を与える.i.e. 同型写像である.
\end{proof}

\begin{myprop}[label=prop:tensor-hom]{テンソル積と Hom の同型}
    任意の\underline{有限次元} $\mathbb{K}$-ベクトル空間 $V,\, W$ に対して
    \begin{align}
        \Hom{\mathbb{K}} (V,\, W) \cong V^* \otimes W
    \end{align}
\end{myprop}

\begin{proof}
    % 定義から $L(V;\, W) = \Hom{\mathbb{K}}(V,\, W)$ である.
    写像
    \begin{align}
        \Phi \colon V^* \times W &\lto \Hom{\mathbb{K}}(V,\, W), \\
        (\omega,\, w) &\lmto \bigl( v \lmto \omega(v) w \bigr) 
    \end{align}
    は双線型なので,\hyperref[def:univ-vec-tensor]{テンソル積の普遍性}から $\VEC{\mathbb{K}}$ の可換図式
    \begin{center}
		\begin{tikzcd}[row sep=large, column sep=large]
			&V^* \times W \ar[d, "\pi \circ \iota"'] \ar[r,"\Phi"] &\Hom{\mathbb{K}}(V,\, W) \\
			&\bm{V^* \otimes W} \ar[ur, red, dashed, "\exists! \overline{\Phi}"'] &
		\end{tikzcd}
	\end{center}
	が存在する.
	$V,\, W$($\dim V = n,\, \dim W = m$)の基底をそれぞれ $\{e_\mu\},\, \{f_\nu\}$ と書き,その\hyperref[def.basisforDVS]{双対基底}をそれぞれ $\{\varepsilon^\mu\},\, \{\eta^\mu\}$ と書く.
	命題\ref{prop:basis-tensor}より $V^* \otimes W$ の基底として
	\begin{align}
		\mathcal{E} \coloneqq \bigl\{\, \varepsilon^\mu \otimes f_\nu \bigm| 1 \le \mu \le n,\, 1 \le \nu \le m \,\bigr\} 
	\end{align}
	がとれる.一方,$\forall \omega \in V^*,\; \forall w \in W$ に対して
    \begin{align}
        \label{eq:tensorbasis}
        \omega \otimes w \coloneqq \Phi(\omega,\, w) \colon V \lto W,\; v \lmto \omega(v) w
    \end{align}
    とおくと $\Hom{\mathbb{K}}(V,\, W)$ の基底として
	\begin{align}
		\mathcal{B} \coloneqq \bigl\{\, \varepsilon^\mu \otimes f_\nu \bigm| 1 \le \mu \le n,\, 1 \le \nu \le m \,\bigr\} 
	\end{align}
	がとれる
    \footnote{
        $\forall F \in \Hom{\mathbb{K}}(V,\, W)$ をとる.$F_{\mu}{}^\nu \coloneqq \eta^\nu \bigl( F(e_\mu) \bigr)$ とおく.このとき $\forall v = v^\mu e_\mu \in V$ に対して
        \begin{align}
            F_{\mu}{}^{\nu} \varepsilon^\mu \otimes f_\nu (v) = F_\mu{}^\nu \varepsilon^\mu(v) f_\nu = F_\mu{}^\nu v^\mu f_\nu
        \end{align}
        一方で,線形性および双対基底の定義から
        \begin{align}
            F(v) = v^\mu F(e_\mu) = v^\mu \eta^\nu \bigl( F(e_\mu) \bigr) f_\nu = v^\mu F_\mu{}^\nu f_\nu
        \end{align}
        が成り立つので $F = F_\mu{}^\nu \varepsilon^\mu \otimes f_\nu$ が言えた.i.e. $\mathcal{B}$ は $\Hom{\mathbb{K}}(V,\, W)$ を生成する.

        次に,$\mathcal{B}$ の元が線型独立であることを示す.
        \begin{align}
            F_{\mu}{}^\nu \varepsilon^\mu \otimes f_\nu = 0
        \end{align}
        を仮定する.$1 \le \forall \mu \le \dim V$ について右辺を $e_\mu$ に作用させることで $F_\mu{}^\nu f_\nu = 0$ が従うが,$f_\nu$ の線型独立性から $F_{\mu}{}^\nu = 0$ である.
    }
    (記号が同じだが,$\mathcal{E}$ とは違う定義である).
	このとき,$\forall v \in V$ に対して
	\begin{align}
		\textcolor{red}{\overline{\Phi}}(\varepsilon^\mu \otimes f_\nu )(v) = \textcolor{red}{\overline{\Phi}} \circ \pi \circ \iota(\varepsilon^\mu,\, f_\nu )(v) = \Phi(\varepsilon^\mu,\, f_\nu)(v) = \varepsilon^\mu (v) f_\nu = \varepsilon^\mu \otimes f_\nu(v)
	\end{align}
	が成り立つ(ただし,左辺の $\otimes$ は命題\ref{prop:tensor-vec},右辺は\eqref{eq:tensorbasis}で定義したものである)ので,
	$\textcolor{red}{\overline{\Phi}}$ は $\mathcal{E}$ の元と $\mathcal{B}$ の元の1対1対応を与える.i.e. 同型写像である.
\end{proof}


\begin{mycol}[label=col:tensor-multillinear]{}
    任意の\underline{有限次元} $\mathbb{K}$-ベクトル空間 $V_1,\, \dots,\, V_n,\, W$ に対して
    \begin{align}
        L(V_1,\, \dots,\, V_n;\, W) \cong V_1^* \otimes \cdots \otimes V_n^* \otimes W
    \end{align}
\end{mycol}

\begin{proof}
    命題\ref{prop:tensor-multillinear}および命題\ref{prop:fin-tensor-dual}, \ref{prop:tensor-hom}から
    % , \ref{prop:tensor-hom}の証明と同様の議論をすることで
    \begin{align}
        L(V_1,\, \dots,\, V_n;\, W) 
        &\cong \Hom{\mathbb{K}}(V_1 \otimes \cdots \otimes V_n,\, W) \\
        &\cong (V_1 \otimes \cdots \otimes V_n)^* \otimes W \\
        &\cong V_1^* \otimes \cdots \otimes V_n^* \otimes W
    \end{align}
    を得る.
\end{proof}

\begin{mycol}[label=col:tensor-hom-adj]{Tensor-Hom adjunction}
    任意の\underline{有限次元} $\mathbb{K}$-ベクトル空間 $V,\, W,\, Z$ に対して,
    \begin{align}
        \Hom{\mathbb{K}} (V \otimes W,\, Z) \cong \Hom{\mathbb{K}} \bigl( V,\, \Hom{\mathbb{K}} (W,\, Z) \bigr) 
    \end{align}
\end{mycol}

\begin{proof}
    命題\ref{prop:fin-tensor-dual}, \ref{prop:tensor-hom}およびテンソル積の結合則より
    \begin{align}
        \Hom{\mathbb{K}} (V \otimes W,\, Z) 
        &\cong (V \otimes W)^* \otimes Z \\
        &\cong V^* \otimes W^* \otimes Z \\
        &\cong V^* \otimes (W^* \otimes Z) \\
        &\cong V^* \otimes \Hom{\mathbb{K}} (W,\, Z) \\
        &\cong \Hom{\mathbb{K}} \bigl( V,\, \Hom{\mathbb{K}} (W,\, Z) \bigr) 
    \end{align}
\end{proof}

\begin{marker}
    系\ref{col:tensor-hom-adj}は,有限次元ベクトル空間の圏が閉圏 (closed category) であることを意味する.
\end{marker}


\section{ベクトル空間の直積・直和}

\subsection{普遍性による定義}

$A$ を集合とする.集合 $I$ を\textbf{添字集合} (index set) とする $A$ の元の\textbf{族} (family) とは,写像 $a \colon I \lto A$ のことを言う.$\forall i \in I$ に対して $\bm{a_i} \coloneqq a(i)$ と略記し,写像 $a \colon I \lto A$ 自身のことを $\bm{(a_i)_{i \in I}}$ や $\bm{\{a_i\}_{i \in I}}$ と略記する.
集合族 $\{A_i\}_{i \in I}$ の\textbf{直積} (Cartesian product) とは,写像の集合
\begin{align}
    \bm{\prod_{i \in I} A_i} \coloneqq \Bigl\{\, a \colon I \lto \bigcup_{i \in I} A_i \Bigm| \forall i \in I,\; a(i) \in A_i \,\Bigr\} 
\end{align}
のこと.集合族の直積について定まる自然な全射
\begin{align}
    \pi_i \colon \prod_{j \in I} A_j \lto A_i,\; a \lmto a(i)
\end{align}
のことを\textbf{標準的射影} (canonical projection) と呼ぶ.
族としての略記を使うと,直積は
\begin{align}
    \prod_{i \in I} A_i = \Bigl\{\, (a_i)_{i \in I} \Bigm| \forall i \in I,\; a_i \in A_i \,\Bigr\} 
\end{align}
と書くことができる.
選択公理を認めたので空でない集合族の直積は空でない.

\begin{mydef}[label=def:univ-vec-sum,breakable]{直積・直和-普遍性による定義}
	添字集合 $I$,および $\mathbb{K}$-ベクトル空間の族 $\Familyset[\big]{V_i}{i\in I}$ を与える.
    % 添字 $\mu \in I$ に対する標準的射影,標準的包含をそれぞれ $\pi_\mu,\, \iota_\mu$ と書く.
	\begin{enumerate}
		\item \begin{itemize}
            \item $\mathbb{K}$-ベクトル空間 $\bm{P}$
            \item $\mathbb{K}$-線型写像の族 $\Familyset[\big]{ \bm{\pi_i} \colon \bm{P} \lto V_i}{i \in I}$
        \end{itemize}
        の2つ組が $\mathbb{K}$-ベクトル空間の族 $\Familyset[\big]{V_i}{i \in I}$ の\textbf{直積} (product) であるとは,以下の性質を充たすことを言う:
        \begin{description}
            \item[\textbf{(ベクトル空間の直積の普遍性)}] 
            
            $\forall \textcolor{blue}{W} \in \Obj{\VEC{\mathbb{K}}}$ および任意の $\mathbb{K}$-線型写像の族 $\Familyset[\big]{\textcolor{blue}{f_i} \colon \textcolor{blue}{W} \lto V_i}{i \in I}$ に対して,以下の図式を可換にする $\mathbb{K}$-線型写像 $\textcolor{red}{f} \colon \textcolor{blue}{W} \lto \bm{P}$ が一意的に存在する:
            \begin{center}
                \begin{tikzcd}[column sep=huge,row sep=large]
                    &\forall \textcolor{blue}{W} \ar[d, red, dashed, "\exists ! f"'] \ar[dr, "\textcolor{blue}{f_i}"] & \\
                    &\bm{P} \ar[r, "\bm{\pi_i}"'] &V_i
                \end{tikzcd}
            \end{center}
        \end{description}
        
		\item \begin{itemize}
            \item $\mathbb{K}$-ベクトル空間 $\bm{S}$
            \item $\mathbb{K}$-線型写像の族 $\Familyset[\big]{\bm{\iota_i} \colon V_i \lto \bm{S}}{i \in I}$
        \end{itemize}
        の2つ組が $\mathbb{K}$-ベクトル空間の族 $\Familyset[\big]{V_i}{i \in I}$ の\textbf{直和} (direct sum) であるとは,以下の性質を充たすことを言う:
        \begin{description}
            \item[\textbf{(ベクトル空間の直和の普遍性)}] 
            
            $\forall \textcolor{blue}{W} \in \Obj{\VEC{\mathbb{K}}}$ および任意の $\mathbb{K}$-線型写像の族 $\Familyset[\big]{\textcolor{blue}{f_i} \colon V_i \lto \textcolor{blue}{W}}{i \in I}$ に対して,以下の図式を可換にする $\mathbb{K}$-線型写像 $\textcolor{red}{f} \colon \bm{S} \lto \textcolor{blue}{W}$ が一意的に存在する:
            \begin{center}
                \begin{tikzcd}[column sep=huge,row sep=large]
                    &\forall \textcolor{blue}{W} \ar[from=d, red, dashed, "\exists ! f"] \ar[from=dr, "\textcolor{blue}{f_i}"'] & \\
                    &\bm{S} \ar[from=r, "\bm{\iota_i}"] &V_i
                \end{tikzcd}
            \end{center}
        \end{description}
	\end{enumerate}
\end{mydef}

\begin{myprop}[label=prop:unique-vecsum]{}
    \hyperref[def:univ-vec-sum]{ベクトル空間の直和・直積}は,存在すれば同型を除いて一意である.
\end{myprop}

\begin{proof}
    \hyperref[prop:unique-tensor-vec]{テンソル積の場合}と全く同様.
\end{proof}

\begin{myprop}[label=prop:sum-vec,breakable]{ベクトル空間の直和・直積の構成}
    添字集合 $I$,および $\mathbb{K}$-ベクトル空間の族 $\Familyset[\big]{V_i}{i\in I}$ を与える.
    \begin{enumerate}
        \item \begin{itemize}
            \item 集合族の直積 $\prod_{i \in I} V_i$ の上に加法とスカラー乗法を
            \begin{align}
                (v_i)_{i \in I} + (w_i)_{i \in I} &\coloneqq (v_i + w_i)_{i \in I} \\
                \lambda (v_i)_{i \in I}  &\coloneqq (\lambda v_i)_{i \in I}
            \end{align}
            と定義することで得られる $\mathbb{K}$-ベクトル空間 $\prod_{i \in I} V_i$
            \item $\forall i \in I$ に対して定まる $\mathbb{K}$-線型写像\footnote{\underline{集合} $\prod_{i \in I} V_i$ に入れたベクトル空間の構造の定義から,自動的に $\pi_i$ は $\mathbb{K}$-線型写像になる.}
            \begin{align}
                \pi_i \colon \prod_{j \in I} V_j \lto V_i,\; (v_j)_{j \in I} \lmto v_i
            \end{align}
        \end{itemize}
        の2つ組は $\Familyset[\big]{V_i}{i\in I}$ の\hyperref[def:univ-vec-sum]{直積}である.
        \item \begin{itemize}
            \item 集合族の直積 $\prod_{i \in I} V_i$ の部分集合
            \begin{align}
                \bigoplus_{i \in I} V_i \coloneqq \Bigl\{\, (v_i)_{i \in I} \in \prod_{i \in I} V_i \Bigm| \substack{\text{有限個の添字を除いた全ての}\, j \in I \\ \text{について}\, v_j = 0 \in V_j} \,\Bigr\} 
            \end{align}
            の上に加法とスカラー乗法を
            \begin{align}
                (v_i)_{i \in I} + (w_i)_{i \in I} &\coloneqq (v_i + w_i)_{i \in I} \\
                \lambda (v_i)_{i \in I}  &\coloneqq (\lambda v_i)_{i \in I}
            \end{align}
            と定義することで得られる $\mathbb{K}$-ベクトル空間 $\bigoplus_{i \in I} V_i$
            \item $\forall i \in I$ に対して定まる $\mathbb{K}$-線型写像\footnote{\underline{集合} $\bigoplus_{i \in I} V_i$ に入れたベクトル空間の構造の定義から,自動的に $\iota_i$ は $\mathbb{K}$-線型写像になる.}
            \begin{align}
                &\iota_i \colon V_i \lto \bigoplus_{j \in I} V_j,\; v \lmto (y_j)_{j \in I} \\
                &\WHERE y_j = \begin{cases}
                    v, &j=i \\
                    0, &j\neq i
                \end{cases}
            \end{align}
        \end{itemize}
        の2つ組は $\Familyset[\big]{V_i}{i\in I}$ の\hyperref[def:univ-vec-sum]{直和}である.
    \end{enumerate}
    
\end{myprop}

\begin{marker}
    構成から明らかに,添字集合 $I$ が有限集合ならベクトル空間の直積と直和は同型である.
\end{marker}

\begin{proof}
    \begin{enumerate}
		\item 
		\begin{description}
			\item[\textbf{$\textcolor{red}{f}$ の存在}] 
            
            任意の $\mathbb{K}$-ベクトル空間 $W$ および任意の $\mathbb{K}$-線型写像の族 $\Familyset[\big]{\, f_\lambda \colon N \to M_\lambda\,}{\lambda\in\Lambda}$ を与える.
            このとき写像 $\textcolor{red}{f}$ を
			\begin{align}
				\textcolor{red}{f} \colon W \lto \prod_{i\in I}V_i,\; w \lmto \Dpmember[\big]{f_i(w)}{i\in I}
			\end{align}
			と定義すると,$\textcolor{red}{f}$ は $\prod_{i\in I}V_i$ の $\mathbb{K}$-ベクトル空間としての構造の定義から自動的に $\mathbb{K}$-線型写像になり,
            $\forall i \in I$ および $\forall w \in W$ に対して
			\begin{align}
				(\pi_i \circ \textcolor{red}{f})(w) = f_i(w)
			\end{align}
			が成り立つ.i.e. \hyperref[def:univ-vec-sum]{直積の普遍性の可換図式}が成り立つ.
			\item[\textbf{$\textcolor{red}{f}$ の一意性}] 
            
            \hyperref[def:univ-vec-sum]{直積の普遍性の可換図式}を充たす別の $\mathbb{K}$-線型写像 $g \colon W \to \displaystyle \prod_{i\in I}V_i$ が存在したとする.このとき $\forall i \in I$ および $\forall w \in W$ に対して
			\begin{align}
				\pi_i \bigl( g(w) \bigr) = f_i(w) = \pi_i \bigl( \textcolor{red}{f}(w) \bigr) 
			\end{align}
			が成り立つので $\textcolor{red}{f} = g$ が言える.i.e. $\textcolor{red}{f}$ は一意である.
		\end{description}
		\item 
		\begin{description}
			\item[\textbf{$\textcolor{red}{f}$ の存在}] 
            
            任意の $\mathbb{K}$-ベクトル空間 $W$ および任意の $\mathbb{K}$-線型写像の族 $\Familyset[\big]{\, f_i \colon M_i \to N\,}{i \in I}$ が与えられたとき,写像 $\textcolor{red}{f}$ を
			\begin{align}
				\textcolor{red}{f} \colon \bigoplus_{i \in I} V_i \to W,\; \Dpmember{v_i}{i \in I} \mapsto \sum_{i \in I} f_i(v_i)
			\end{align}
			と定義する.右辺は有限和なので意味を持ち,$\textcolor{red}{f}$ は $\bigoplus_{i\in I}V_i$ の $\mathbb{K}$-ベクトル空間としての構造の定義から自動的に $\mathbb{K}$-線型写像になる.
            このとき $\forall i\in I$ および $\forall v \in V_i$ に対して
			\begin{align}
				\textcolor{red}{f} \bigl( \iota_i(v) \bigr) = f_i(v) + \sum_{j \in I \setminus \{i\}} f_j(0) = f_i(v)
			\end{align}
			が成り立つ.i.e. \hyperref[def:univ-vec-sum]{直和の普遍性の可換図式}が成り立つ.

			\item[\textbf{$\textcolor{red}{f}$ の一意性}] 
            
            \hyperref[def:univ-vec-sum]{直和の普遍性の可換図式}を充たす別の $\mathbb{K}$-線型写像 $g \colon \displaystyle \bigoplus_{i \in I}V_i \to W$ が存在したとする.このとき $\forall \Dpmember{v_i}{i \in I} \in \bigoplus_{i \in I} V_i$ に対して
			\begin{align}
				g \bigl( \Dpmember{v_i}{i \in I} \bigr) = g \left(\sum_{i \in I} \iota_i (v_i)\right) = \sum_{i \in I} g \bigl( \iota_i(v_i) \bigr) = \sum_{i \in I} f_i(v_i) = \textcolor{red}{f} \bigl( \Dpmember{v_i}{i \in I} \bigr) 
			\end{align}
			が成り立つので $\textcolor{red}{f} = g$ が言える.i.e. $\textcolor{red}{f}$ は一意である.
		\end{description}
	\end{enumerate}
\end{proof}

\begin{myprop}[label=prop:hom-sum]{Hom と直積・直和の交換}
    任意の $\mathbb{K}$-ベクトル空間 $W$,添字集合 $I$,および $\mathbb{K}$-ベクトル空間の族 $\Familyset[\big]{V_i}{i\in I}$ を与える.
    このとき以下の2つの $\mathbb{K}$-ベクトル空間の同型が成り立つ:
    \begin{enumerate}
        \item 
        \begin{align}
            \Hom{\mathbb{K}} \Bigl( W,\, \prod_{i \in I} V_i \Bigr) \cong \prod_{i \in I} \Hom{\mathbb{K}}(W,\, V_i)
        \end{align}
        \item \begin{align}
            \Hom{\mathbb{K}} \Bigl(\bigoplus_{i \in I} V_i,\, W\Bigr) \cong \prod_{i \in I} \Hom{\mathbb{K}}(V_i,\, W)
        \end{align}
    \end{enumerate}
    
\end{myprop}

\begin{proof}
    \begin{enumerate}
        \item \hyperref[def:univ-vec-sum]{直積の普遍性}より,$\mathbb{K}$-線型写像
        \begin{align}
            \alpha\colon \Hom{\mathbb{K}} \Bigl( W,\, \prod_{i \in I} V_i \Bigr) \lto \prod_{i \in I} \Hom{\mathbb{K}}(W,\, V_i),\; f \lmto (\pi_i \circ f)_{i \in I}
        \end{align}
        は全単射である.
        \item \hyperref[def:univ-vec-sum]{直和の普遍性}より,$\mathbb{K}$-線型写像
        \begin{align}
            \beta \colon \Hom{\mathbb{K}} \Bigl(\bigoplus_{i \in I} V_i,\, W\Bigr) \lto \prod_{i \in I} \Hom{\mathbb{K}}(V_i,\, W),\; f \lmto (f \circ \iota_i)_{i \in I}
        \end{align}
        は全単射である.
    \end{enumerate}
\end{proof}

\begin{marker}
    命題\ref{prop:hom-sum}の同型は,$\mathbb{K}$-ベクトル空間の圏 $\VEC{\mathbb{K}}$ における(圏論的)極限,余極限に関する同型
    \begin{align}
        \Hom{\mathbb{K}} (W,\, \lim_{i \in I} V_i ) &\cong \lim_{i \in I} \Hom{\mathbb{K}}(W,\, V_i) \\
        \Hom{\mathbb{K}} (\underset{i \in I}{\colim} V_i,\, W) &\cong \lim_{i \in I} \Hom{\mathbb{K}}(V_i,\, W)
    \end{align}
    の一例である.
\end{marker}


\subsection{部分ベクトル空間の直和}

$\mathbb{K}$-ベクトル空間 $V$ の部分ベクトル空間の族 $\Familyset[\big]{W_i}{i \in I}$ の和空間とは,$V$ の部分 $\mathbb{K}$-ベクトル空間
\begin{align}
    \bm{\sum_{i \in I} W_i} \coloneqq \Bigl\{\, \sum_{i \in I} w_i \in V \Bigm| \substack{\forall i \in I,\; w_i \in W_i\;\text{かつ,有限個の添字を除いた} \\ \text{全ての}\, j \in I\, \text{について}\, w_j = 0 \in W_j} \,\Bigr\} 
\end{align}
のこと.集合 $\bigcup_{i \in I} W_i$ が生成する(張る)部分 $\mathbb{K}$-ベクトル空間と言っても良い.

\begin{myprop}[label=prop:subvec-directsum,breakable]{部分ベクトル空間の直和}
    $\mathbb{K}$-ベクトル空間 $V$ の部分ベクトル空間の族 $\Familyset[\big]{W_i}{i \in I}$ が
    \begin{align}
        \label{eq:subvec-directsum}
        \forall i \in I,\; W_i \cap \left(\sum_{j \in I \setminus \{i\}} W_j\right) = 0
    \end{align}
    を充たすとする.このとき
    \begin{itemize}
        \item 和空間 $\sum_{i \in I} W_i$
        \item $\forall i \in I$ に対して定まる $\mathbb{K}$-線型写像
        \begin{align}
            \iota_i \colon W_i \lto \sum_{j \in I} W_j,\; w \lmto w
        \end{align}
    \end{itemize}
    の2つ組は $\Familyset[\big]{W_i}{i \in I}$ の\hyperref[def:univ-vec-sum]{直和}である.
\end{myprop}

% \begin{marker}
%     条件\eqref{eq:subvec-directsum}が満たされているとき,この命題によって $\sum_{i \in I}W_i$ と命題\ref{prop:sum-vec}の $\bigoplus_{i \in I} W_i$ を同一視できる.
% \end{marker}

\begin{proof}
    $\forall w \in \sum_{i \in I} W_i$ を1つとる.まず,条件\eqref{eq:subvec-directsum}が充たされているならば $w = w_{i_1} + \cdots + w_{i_n}\; (1 \le \forall k \le n,\; v_{i_k} \in W_{i_k})$ と書く方法が一意的に定まることを示す.
    $w = w_{i_1} + \cdots + w_{i_n} = w'_{j_1} + \cdots + w'_{j_m}$ が成り立つと仮定する.$A \coloneqq \{i_1,\, \dots,\, i_n\} \cup \{j_1,\, \dots,\, j_m\} \subset I$ とおき,$\forall r \in A \setminus \{i_1,\, \dots,\, i_n\}$ に対しては $w_{r} = 0$,$\forall s \in A \setminus \{j_1,\, \dots,\, j_m\}$ に対しては $w'_s = 0$ とおくことで
    \begin{align}
        w = \sum_{r \in A} w_r = \sum_{r \in A} w'_r \in \sum_{i \in I} W_i
    \end{align}
    と書ける\footnote{$\sharp A \le n + m < \infty$ なので左辺は有限和である.}.
    このとき $\forall r \in A$ について
    \begin{align}
        w_r - w'_r = \sum_{s \in A \setminus \{r\}} (w'_s - w_s) \in W_r \cap \left(\sum_{s \in A \setminus \{r\}} W_s\right)
    \end{align}
    が言えるが,条件\eqref{eq:subvec-directsum}より左辺は $0$ である.i.e. $\forall r \in A$ に対して $w_r = w'_r$ が言えた.

    次に\hyperref[def:univ-vec-sum]{直和の普遍性}を示す.任意の $\mathbb{K}$-ベクトル空間 $\textcolor{blue}{W}$ および $\mathbb{K}$-線型写像の族 $\Familyset[\big]{\textcolor{blue}{f_i} \colon V_i \lto \textcolor{blue}{W}}{i \in I}$ を与える.このとき
    $\mathbb{K}$-線型写像
    \begin{align}
        \textcolor{red}{f} \colon \sum_{i \in I} W_i \lto \textcolor{blue}{W},\; \sum_{i \in I} w_i \lmto \sum_{i \in I} \textcolor{blue}{f_i} (w_i)
    \end{align}
    は先程の議論からwell-definedであり\footnote{右辺は有限和なので意味を持つ.},かつ $\forall i \in I,\; \forall w \in W_i$ に対して 
    \begin{align}
        \textcolor{red}{f}(w) = \textcolor{blue}{f_i} (w) = \textcolor{blue}{f_i} \circ \iota_i (w)
    \end{align}
    を充たす.i.e. $\forall i \in I$ に対して図式
    \begin{center}
        \begin{tikzcd}[column sep=huge,row sep=large]
            &\forall \textcolor{blue}{W} \ar[from=d, red, dashed, "\exists ! f"] \ar[from=dr, "\textcolor{blue}{f_i}"'] & \\
            &\sum_{i \in I} W_i \ar[from=r, "\iota_i"] &W_i
        \end{tikzcd}
    \end{center}
    は可換である.
    別の $\mathbb{K}$-線型写像 $g \colon \sum_{i \in I} W_i \lto \textcolor{blue}{W}$ がこの図式を可換にするならば,線型性から $\forall w = \sum_{i \in I} w_i \in \sum_{i \in I} W_i$ に対して
    \begin{align}
        g(w) &= g \left( \sum_{i \in I} w_i \right) = \sum_{i \in I} g(w_i) = \sum_{i \in I} g \circ \iota_i (w_i) = \sum_{i \in I} \textcolor{blue}{f_i}(w_i) = \textcolor{red}{f}(w)
    \end{align}
    が言える.i.e. $g = \textcolor{red}{f}$ であり,$\textcolor{red}{f}$ は一意である.
\end{proof}

\begin{mycol}[label=col:subvec-directsum]{内部直和}
    $\mathbb{K}$-ベクトル空間 $V$ の部分ベクトル空間の族 $\Familyset[\big]{W_i}{i \in I}$ が
    \begin{align}
        \forall i \in I,\; W_i \cap \left(\sum_{j \in I \setminus \{i\}} W_j\right) = 0
    \end{align}
    を充たすとする.このとき
    \begin{align}
        \sum_{i \in I} W_i \cong \bigoplus_{i \in I} W_i
    \end{align}
    が成り立つ.ただし右辺は命題\ref{prop:sum-vec}で構成した\hyperref[def:univ-vec-sum]{ベクトル空間の直和}である.
\end{mycol}

\begin{marker}
    結局同型ではあるのだが,系\ref{col:subvec-directsum}において $\sum_{i \in I} W_I$ と $\bigoplus_{i \in I} W_i$ を区別するときは,前者を\textbf{内部直和} (internal direct sum),後者を\textbf{外部直和} (external direct sum) と呼ぶ.
\end{marker}

\begin{proof}
    命題\ref{prop:unique-vecsum}より従う.
\end{proof}

\section{階数・退化次数の定理}

\subsection{有限次元の場合}

$V,\, W$ を\textbf{有限次元} $\mathbb{K}$ ベクトル空間とし,線型写像 $T \colon V \lto W$ を与える.
$V,\, W$ の\hyperref[def:free-mod]{基底} $\{\vb{e}_1,\, \dots,\, \vb{e}_{\dim V}\},\; \{\vb{f}_1,\, \dots,\, \vb{f}_{\dim W}\}$ をとり,
\begin{align}
	T(\vb{e}_\mu) = T^\nu{}_{\mu} \vb{f}_\nu
\end{align}
のように左辺を展開したときに得られる行列
\begin{align}
	\mqty[T^1{}_1 & \cdots &T^1{}_{\dim V} \\ \vdots &\ddots &\vdots \\ T^{\dim W}{}_1 &\cdots &T^{\dim W}{}_{\dim V}]
\end{align}
は基底 $\{\vb{e}_1,\, \dots,\, \vb{e}_{\dim V}\},\; \{\vb{f}_1,\, \dots,\, \vb{f}_{\dim W}\}$ に関する $T$ の\textbf{表現表列}と呼ばれる.
$\forall \bm{v} = v^\nu \vb{e}_\nu \in V$ に対して
\begin{align}
	T(\bm{v}) = T(v^\nu \vb{e}_\nu) = v^\nu T(\vb{e}_\nu) = v^\nu T^\mu{}_{\nu} \vb{f}_\mu
\end{align}
と書けるので,成分表示だけを見ると $T$ はその表現行列を左から掛けることに相当する:
\begin{align}
	\mqty[T^1{}_1 & \cdots &T^1{}_{\dim V} \\ \vdots &\ddots &\vdots \\ T^{\dim W}{}_1 &\cdots &T^{\dim W}{}_{\dim V}] \mqty[v^1 \\ \vdots \\ v^{\dim V}]
\end{align}

\begin{mydef}[label=def:rank]{線型写像の階数}
	$\Im T$ の次元のことを $T$ の\textbf{階数} (rank) と呼び,$\bm{\rank T}$ と書く.
\end{mydef}

\begin{myprop}[label=prop:canonical-matrix]{表現行列の標準形}
	$V,\, W$ を\textbf{有限次元}ベクトル空間とし,任意の線型写像 $T \colon V \lto W$ を与える.
	このとき $V,\, W$ の基底であって,$T$ の表現行列を
	\begin{align}
		\mqty[I_{\rank T} & 0 \\ 0 & 0]
	\end{align}
	の形にするものが存在する.
\end{myprop}

\begin{proof}
	$\Im T$ の基底 $\{\vb{f}_1,\, \dots,\, \vb{f}_{\rank T}\}$ および $\Ker T$ の基底 $\{\vb{k}_1,\, \dots \vb{k}_{\dim (\Ker T)}\}$ を勝手にとる.
	\hyperref[def:ker-module]{像の定義}から,$1 \le \forall \mu \le \rank T$ に対して $\vb{e}_\mu \in V$ が存在して $\vb{f}_\mu = T(\vb{e}_\mu)$ を充たす.
	
	まず $\vb{e}_1,\, \dots ,\, \vb{e}_{\rank T},\, \vb{k}_1,\, \dots,\, \vb{k}_{\dim (\Ker T)}$ が $V$ の基底を成すことを示す.
	\begin{description}
		\item[\textbf{線型独立性}] 
		\begin{align}
			\sum_{\mu = 1}^{\rank T} a^\mu \vb{e}_\mu + \sum_{\nu = 1}^{\dim (\Ker T)} b^\nu \vb{k}_\nu = 0
		\end{align}
		を仮定する.左辺に $T$ を作用させることで
		\begin{align}
			\sum_{\mu = 1}^{\rank T} a^\mu \vb{f}_\mu = 0
		\end{align}
		がわかるが,$\vb{f}_1,\, \dots,\, \vb{f}_{\rank T}$ は $\Im T$ の基底なので線型独立であり,$1 \le \forall \mu \le \rank T$ に対して $a_\mu = 0$ が言える.
		故に仮定から
		\begin{align}
			\sum_{\nu = 1}^{\dim (\Ker T)} b^\nu \vb{k}_\nu = 0
		\end{align}
		であるが,$\vb{k}_1,\, \dots,\, \vb{k}_{\dim (\Ker T)}$ は $\Ker T$ の基底なので線型独立であり,$1 \le \forall \nu \le \dim (\Ker T)$ に対して $b_\nu = 0$ が言える.
		i.e. $\vb{e}_1,\, \dots ,\, \vb{e}_{\rank T},\, \vb{k}_1,\, \dots,\, \vb{k}_{\dim (\Ker T)}$ は線型独立である.
		\item[\textbf{$\bm{V}$ を生成すること}] 
		$\forall \bm{v} \in V$ を1つとる.このとき $T(\bm{v}) \in \Im T$ なので
		\begin{align}
			T(\bm{v}) = \sum_{\mu = 1}^{\rank T} v^\mu \vb{f}_\mu
		\end{align}
		と展開できる.ここで
		$\bm{w} \coloneqq \sum_{\mu=1}^{\rank T} v^\mu \vb{e}_\mu \in V$ 
		とおくと,$T(\bm{v}) = T(\bm{w})$ が成り立つが,$T$ が線型写像であることから $T(\bm{v} - \bm{w}) = 0 \iff \bm{v} - \bm{w} \in \Ker T$ が言えて
		\begin{align}
			\bm{v} - \bm{w} = \sum_{\nu = 1}^{\dim (\Ker T)} w^\nu \vb{k}_\nu
		\end{align}
		と展開できる.従って
		\begin{align}
			\bm{v} = \bm{w} + (\bm{v} - \bm{w})
			= \sum_{\mu=1}^{\rank T} v^\mu \vb{e}_\mu + \sum_{\nu = 1}^{\dim (\Ker T)} w^\nu \vb{k}_\nu
		\end{align}
		であり,$\vb{e}_1,\, \dots ,\, \vb{e}_{\rank T},\, \vb{k}_1,\, \dots,\, \vb{k}_{\dim (\Ker T)}$ は $V$ を生成する.
	\end{description}
	$\vb{f}_1,\, \dots,\, \vb{f}_{\rank T}$ と線型独立な $\dim W - \rank T$ 個のベクトル $\tilde{\vb{f}}_{\rank T + 1},\, \dots,\, \tilde{\vb{f}}_{\dim W}$ をとると,
	\begin{itemize}
		\item $V$ の基底 $\{\vb{e}_1,\, \dots ,\, \vb{e}_{\rank T},\, \vb{k}_1,\, \dots,\, \vb{k}_{\dim (\Ker T)}\}$ 
		\item $W$ の基底 $\{\vb{f}_1,\, \dots,\, \vb{f}_{\rank T},\, \tilde{\vb{f}}_{\rank T + 1},\, \dots,\, \tilde{\vb{f}}_{\dim W}\}$
	\end{itemize}
	に関する $T$ の表現行列は
	\begin{align}
		\mqty[I_{\rank T} & 0 \\ 0 & 0]
	\end{align}
	になる.
\end{proof}

\begin{mycol}[label=col:rank-nullity]{階数・退化次数の定理(有限次元)}
	$V,\, W$ を\textbf{有限次元}ベクトル空間とし,任意の線型写像 $T \colon V \lto W$ を与える.
	このとき
	\begin{align}
		\dim V = \dim (\Im T) + \dim (\Ker T)
	\end{align}
	が成り立つ.
\end{mycol}

\begin{proof}
	命題\ref{prop:canonical-matrix}の証明より従う.
\end{proof}

系\ref{col:rank-nullity}から便利な補題がいくつか従う:

\begin{mylem}[label=lem:finvec-basic]{有限次元ベクトル空間に関する小定理集}
	$V,\, W$ を\textbf{有限次元}ベクトル空間とし,任意の線型写像 $T \colon V \lto W$ を与える.
	このとき以下が成り立つ:
	\begin{enumerate}
		\item $\rank T \le \dim V.$ 特に $\rank T = \dim V \iff T$ は単射
		\item $\rank T \le \dim W.$ 特に $\rank T = \dim W \iff T$ は全射
		\item $\dim V = \dim W$ かつ $T$ が単射 $\IMP$ $T$ は同型写像
		\item $\dim V = \dim W$ かつ $T$ が全射 $\IMP$ $T$ は同型写像
	\end{enumerate}
\end{mylem}

\begin{proof}
	\begin{enumerate}
		\item 系\ref{col:rank-nullity}より
		\begin{align}
			\dim V = \rank T + \dim (\Ker T) \ge \rank T
		\end{align}
		が成り立つ.特に命題\ref{prop.kernel1}から $T$ が単射 $\iff \Ker T = 0 \iff \dim (\Ker T) = 0 \iff \rank T = \dim V$ が従う.
		\item \hyperref[def:rank]{rankの定義}より $\rank T \le \dim W$ は明らか.特に次元の等しい有限次元ベクトル空間は同型なので,$T$ が全射 $\iff \Im T \cong W \iff \dim (\Im T) = \rank T = \dim W$ が言える.
		\item $\dim V = \dim W$ かつ $T$ が単射とする.$T$ が単射なので (1) より $\rank T = \dim V = \dim W$ が従い,(2) より $T$ は全射でもある.
		\item $\dim V = \dim W$ かつ $T$ が全射とする.$T$ が全射なので (2) より $\rank T = \dim W = \dim V$ が従い,(1) より $T$ は単射でもある.
	\end{enumerate}
\end{proof}

\subsection{分裂補題と射影的加群}

実は,系\ref{col:rank-nullity}は有限次元でなくとも成り立つ.
それどころか,左 $R$ 加群の場合の\hyperref[lem:splitting]{分裂補題}に一般化される.

\begin{mylem}[label=lem:splitting]{分裂補題}
    左 $R$ 加群の短完全列
    \begin{align}
        \label{SES:3-1}
        0 \lto M_1 \xrightarrow{i_1} M \xrightarrow{p_2} M_2 \lto 0
    \end{align}
    が与えられたとする.このとき,以下の二つは同値である:
    \begin{enumerate}
        \item 左 $R$ 加群の準同型 $i_2 \colon M_2 \lto M$ であって $p_2 \circ i_2 = 1_{M_2}$ を充たすものが存在する
        \item 左 $R$ 加群の準同型 $p_1 \colon M \lto M_1$ であって $p_1 \circ i_1 = 1_{M_1}$ を充たすものが存在する
    \end{enumerate}
\end{mylem}

\begin{proof}
    \begin{description}
        \item[\textbf{(1) $\Longrightarrow$ (2)}] 写像
        \begin{align}
            p_1' \colon M \lto M,\; x \lmto x - i_2 \bigl( p_2(x) \bigr) 
        \end{align}
        を定義すると,
        \begin{align}
            p_2 \bigl( p_1'(x) \bigr) = p_2(x) - ((p_2 \circ i_2) \circ p_2)(x) = p_2(x) - p_2(x) = 0
        \end{align}
        が成り立つ.従って,\eqref{SES:3-1}が完全列であることを使うと $p_1'(x) \in \Ker p_2 = \Im i_1$ である.さらに $i_1$ が単射であることから
        \begin{align}
            \exists ! y \in M_1,\; p_1'(x) = i_1(y)
        \end{align}
        が成り立つ.ここで写像
        \begin{align}
            p_1 \colon M \lto M_1,\; x \lmto y
        \end{align}
        を定義するとこれは準同型写像であり,$\forall x \in M_1$ に対して
        \begin{align}
            p_1' \bigl( i_1(x) \bigr) = i_1(x) - (i_2 \circ (p_2 \circ i_1))(x) = i_1(x)
        \end{align}
        が成り立つ\footnote{\eqref{SES:3-1}が完全列であるため,$p_2 \circ i_1 = 0$}ことから
        \begin{align}
            (p_1 \circ i_1)(x) = x
        \end{align}
        とわかる.i.e. $p_1 \circ i_1 = 1_{M_1}$
        \item[\textbf{(1) $\Longleftarrow$ (2)}] \eqref{SES:3-1}は完全列であるから $M_2 = \Ker 0 =  \Im p_2$ である.
        従って $\forall x \in M_2 = \Im p_2$ に対して,$x = p_2(y)$ を充たす $y \in M$ が存在する.ここで写像
        \begin{align}
            i_2 \colon M_2 \lto M,\; x \lmto y - i_1 \bigl( p_1(y) \bigr) 
        \end{align}
        はwell-definedである.$x = p_2(y')$ を充たす勝手な元 $y' \in M$ をとってきたとき,$p_2(y-y') = 0$ より $y-y' \in \Ker p_2 = \Im i_1$ だから,$i_1$ の単射性から
        \begin{align}
            \exists ! z \in M_1,\quad y-y' = i_1(z)
        \end{align}
        が成り立ち,このとき
        \begin{align}
           	\Bigl( y - i_1 \bigl( p_1(y) \bigr) \Bigr) - \Bigl( y' - i_1 \bigl( p_1(y') \bigr)  \Bigr) = i_1(z) - (i_1 \circ (p_1 \circ i_1))(z) = i_1(z) - i_1(z) = 0
        \end{align}
        とわかるからである.$i_2$ は準同型写像であり,$\forall x \in M_2$ に対して
        \begin{align}
            (p_2 \circ i_2)(x) = p_2(y) - ((p_2 \circ i_1) \circ p_1)(y) = p_2(y) = x
        \end{align}
        なので $p_2 \circ i_2 = 1_{M_2}$.
    \end{description}
\end{proof}

\begin{mycol}[label=col:split]{}
    左 $R$ 加群の短完全列
    \begin{align}
        0 \lto M_1 \xrightarrow{i_1} M \xrightarrow{p_2} M_2 \lto 0
    \end{align}
    が補題\ref{lem:splitting}の条件を充たすならば
    \begin{align}
        M \cong M_1 \oplus M_2 
    \end{align}
\end{mycol}

\begin{proof}
    補題\ref{lem:splitting}の条件 (1) が満たされているとする.このとき補題\ref{lem:splitting}証明から $\forall x \in M$ に対して
    \begin{align}
        i_1  \bigl( p_1(x) \bigr) = p_1'(x) = x - i_2 \bigl( p_2(x) \bigr) \IFF i_1 \bigl(p_1(x)  \bigr) + i_2 \bigl( p_2(x) \bigr) = x
    \end{align}
    また,完全列の定義から $p_2 \bigl( i_1(x) \bigr) = 0$ であるから $\forall x \in M_2$ に対して
    \begin{align}
        p_1' \bigl( i_2(x) \bigr) = i_2(x) - ((i_2 \circ p_2) \circ i_2)(x) = 0 = i_1(0)
    \end{align}
    であり,結局 $p_1 \bigl( i_2(x) \bigr) = 0$ とわかる.

    ここで準同型写像
    \begin{align}
        &f\colon M_1 \oplus M_2 \lto M,\; (x,\, y) \lmto i_1(x) + i_2(y), \\
        &g\colon M \lto M_1 \oplus M_2,\; x \lmto \bigl( p_1(x),\, p_2(x) \bigr) 
    \end{align}
    を定めると
    \begin{align}
        (g \circ f)(x,\, y) &= \bigl( p_1(i_1(x)) + p_1(i_2(y)) ,\, p_2(i_1(x)) + p_2(i_2(x)) \bigr) = (x,\, y) , \\
        (f\circ g)(x) &= i_1(p_1(x)) + i_2(p_2(x)) = x
    \end{align}
    なので $f,\, g$ は同型写像.
\end{proof}

\begin{mydef}[label=def:split]{分裂}
    左 $R$ 加群の短完全列
    \begin{align}
        0 \lto M_1 \xrightarrow{i_1} M \xrightarrow{p_2} M_2 \lto 0
    \end{align}
    が\textbf{分裂} (split) するとは,補題\ref{lem:splitting}の条件を充たすことをいう.
\end{mydef}


% \subsection{射影的加群}

\begin{mydef}[label=def:proj-mod]{射影的加群}
    左 $R$ 加群 $P$ が\textbf{射影的加群} (projective module) であるとは,
    任意の左 $R$ 加群の\textbf{全射準同型} $f\colon M \lto N$ および任意の準同型写像 $g \colon P \lto N$ に対し,左 $R$ 加群の準同型写像 $ h \colon P \lto M$ であって $f \circ h = g$ を充たすものが存在することを言う(図式\ref{fig:proj-mod}).
\end{mydef}

\begin{figure}[H]
    \centering
    \begin{tikzcd}[row sep=large, column sep=large]
        &P \ar[d, "g"]\ar[dl, red, dashed, "\exists h"] \\
        M \ar[r, two heads, "f"] &N
    \end{tikzcd}
    \caption{射影的加群}
    \label{fig:proj-mod}
\end{figure}%

\begin{myprop}[label=prop:proj-mod-split]{}
    左 $R$ 加群の完全列
    \begin{align}
        0 \lto L \xrightarrow{f} M \xrightarrow{g} \textcolor{red}{N} \lto 0
    \end{align}
    は,$\textcolor{red}{N}$ が\hyperref[def:proj-mod]{射影的加群}ならば\hyperref[def:split]{分裂}する.
\end{myprop}

\begin{proof}
    射影的加群の定義において $P = N$ とすることで,左 $R$ 加群の準同型写像 $s \colon N \lto M$ であって $g \circ s = 1_N$ を充たすものが存在する.
\end{proof}

\begin{myprop}[label=prop:proj-mod-dp]{射影的加群の直和}
    左 $R$ 加群の族 $\Familyset[\big]{P_\lambda}{\lambda \in \Lambda}$ に対して以下の2つは同値:
    \begin{enumerate}
        \item $\forall \lambda \in \Lambda$ に対して $P_\lambda$ が射影的加群
        \item $\displaystyle\bigoplus_{\lambda \in \Lambda} P_\lambda$ が射影的加群
    \end{enumerate}
\end{myprop}

\begin{proof}
    \hyperref[def:inj-proj]{標準的包含}を $\iota_\lambda \colon P_\lambda \hookrightarrow \bigoplus_{\lambda \in  \Lambda}P_\lambda$と書く.
    \begin{description}
        \item[\textbf{(1)$\bm{\Longrightarrow}$(2)}] 仮定より,$\forall \lambda \in \lambda$ に対して,任意の全射準同型写像
        $f \colon M \lto N$ および任意の準同型写像 $g \colon \bigoplus_{\lambda \in \Lambda}P_\lambda \lto N$ に対して,準同型写像 $h_\lambda \colon P_\lambda \lto M$ であって $f \circ h_\lambda = g \circ \iota_\lambda$ を充たすものが存在する.
        従って\hyperref[prop:univ-dp]{直和の普遍性}より
        準同型写像
        \begin{align}
            h \colon \bigoplus_{\lambda \in \Lambda} P_\lambda \lto M
        \end{align}
        であって $f \circ h_\lambda = h \circ \iota_\lambda$ を充たすものが一意的に存在する.
        このとき 
        \begin{align}
            (f \circ h) \circ \iota_\lambda = f \circ h_\lambda = g \circ \iota_\lambda
        \end{align}
        であるから,$h$ の一意性から $f \circ h = g$.
        \item[\textbf{(1)$\bm{\Longleftarrow}$(2)}] $\lambda \in \Lambda$ を一つ固定し,任意の全射準同型写像
        $f \colon M \lto N$ および任意の準同型写像 $g \colon P_\lambda \lto M$ を与える. 
        \hyperref[prop:univ-dp]{直和の普遍性}より
        準同型写像
        \begin{align}
            h \colon \bigoplus_{\lambda \in \Lambda} P_\lambda \lto N
        \end{align}
        であって $h \circ \iota_\lambda = g$($\forall \mu \in \Lambda \setminus \{\lambda\},\; h \circ \iota_\lambda = 0$)を充たすものが一意的に存在する.
        さらに仮定より,準同型写像
        \begin{align}
            \alpha \colon \bigoplus_{\lambda \in \Lambda} \lto M
        \end{align}
        であって $f \circ \alpha = h$ を充たすものが存在する.このとき
        \begin{align}
            f \circ (\alpha \circ \iota_\lambda) = h \circ \iota_\lambda = g
        \end{align}
        なので $\beta \coloneqq h \circ \iota_\lambda$ とおけば良い.
    \end{description}
\end{proof}

\begin{mycol}[label=col:free-proj]{自由加群は射影的加群}
    環 $R$ 上の自由加群は射影的加群である
\end{mycol}

\begin{proof}
    $R$ が射影的加群であることを示せば命題\ref{prop:proj-mod-dp}より従う.

    左 $R$ 加群の全射準同型写像と準同型写像 $f \colon M \lto N,\; g\colon R \lto N$ を任意に与える.このとき
    ある $x \in M$ が存在して $f(x) = g(1)$ となる.この $x$ に対して準同型写像 $h \colon R \lto M,\; a \lmto ax$ を定めると,$\forall a \in R$ に対して
    \begin{align}
        f \bigl( h(a) \bigr)  = f(ax) = af(x) = ag(1) = g(a)
    \end{align}
    が成り立つので $f \circ h = g$ となる.
\end{proof}

$V,\, W$ を任意の(有限次元とは限らない) $\mathbb{K}$ ベクトル空間,$T \colon V \lto W$ を任意の線型写像とする.
\begin{align}
	i_1 &\colon \Ker T  \lto V,\; \bm{v} \lmto \bm{v}, \\
	p_2 &\colon V \lto \Im T,\; \bm{v} \lmto T(\bm{v}),
\end{align}
と定めると,$i_1$ は単射,$p_2$ は全射で,かつ $p_2 \circ i_1 = 0$ が成り立つ.よって $\VEC{\mathbb{K}}$ の図式
\begin{align}
	\label{eq:split-vec}
	0 \lto \Ker T \xrightarrow{i_1} V \xrightarrow{p_2} \Im T \lto 0
\end{align}
は短完全列だが,$\Im T$ はベクトル空間なので\hyperref[def:free-mod]{自由加群}であり,系\ref{col:free-proj}より\hyperref[def:proj-mod]{射影的加群}でもある.
従って命題\ref{prop:proj-mod-split}より短完全列\eqref{eq:split-vec}は\hyperref[def:split]{分裂}し,系\ref{col:split}から
\begin{align}
	V \cong \Im T \oplus \Ker T
\end{align}
が言える.

\begin{mytheo}[label=thm:rank-nullity]{階数・退化次数の定理}
	$V,\, W$ をベクトル空間とし,任意の線型写像 $T \colon V \lto W$ を与える.
	このとき
	\begin{align}
		\dim V = \dim (\Im T) + \dim (\Ker T)
	\end{align}
	が成り立つ.
\end{mytheo}

\section{Jordan標準形}

この節では体 $\mathbb{K}$ は\underline{代数閉体}であるとし,$V$ を\underline{有限次元} $\mathbb{K}$-ベクトル空間とする.

\begin{mydef}[label=def:eigen,breakable]{広義固有空間}
    % $V$ を\underline{有限次元} $\mathbb{K}$-ベクトル空間とし,
    $x \in \End V$ を与える.
    $\forall \lambda \in \mathbb{K}$ に対して
    \begin{align}
        \bm{V(\lambda)} &\coloneqq \Ker (x - \lambda\, \mathrm{id}_V) = \bigl\{\, v \in V \bigm| (x - \lambda\, \mathrm{id}_V) v = 0 \,\bigr\} \\
        \bm{W(\lambda)} &\coloneqq \bigl\{\, v \in V \bigm| \exists k > 0,\; (x - \lambda\, \mathrm{id}_V)^k v = 0 \,\bigr\}
    \end{align}
    とおく.
    \begin{itemize}
        \item $\lambda$ が $x$ の\textbf{固有値} (eigenvalue) であるとは,$V(\lambda) \neq 0$ であることを言う.
        \item $V(\lambda)$ が固有値 $\lambda$ に属する $x$ の\textbf{固有空間} (eigenspace) であるとは,$V(\lambda) \neq 0$ であることを言う.
        \item $W(\lambda)$ が固有値 $\lambda$ に属する $x$ の\textbf{広義固有空間} (generalized eigenspace) であるとは,$W(\lambda) \neq 0$ であることを言う.
    \end{itemize}
\end{mydef}

$V(\lambda) \subset W(\lambda) \subset V$ が $x$-不変な部分ベクトル空間であることは明らかである.

\subsection{上三角化}

\begin{mylem}[label=lem:flag,breakable]{旗と基底}
    $V$ の\hyperref[def:flag]{旗}
    \begin{align}
        0 = V_0 \subset V_1 \subset \cdots \subset V_{\dim V} = V
    \end{align}
    を与え,
    \begin{align}
        V_1 &= \mathbb{K}e_1 \\
        V_2 / V_1 &= \mathbb{K} (e_2 + V_1) \\
        V_3 / V_2 &= \mathbb{K} (e_3 + V_2) \\
        &\vdots \\
        V_{\dim V} / V_{\dim V - 1} &= \mathbb{K} (e_{\dim V} + V_{\dim V - 1})
    \end{align}
    を充たす $e_i \in V_i\quad (1 \le i \le \dim V)$ をとる.
    このとき $1 \le \forall i \le \dim V$ に対して
    \begin{align}
        V_i &= \Span \bigl\{\, e_1,\, \dots,\, e_i  \,\bigr\}
    \end{align}
    が成り立つ.
\end{mylem}

\begin{proof}
    $1 \le i \le \dim V$ に関する数学的帰納法により示す.$i = 1$ のときは明らか.

    $i > 1$ とする.
    \begin{align}
        \lambda_{i} e_i + \sum_{j=1}^{i-1} \lambda_j e_j = 0
    \end{align}
    を仮定する.両辺に標準的射影 $p_i \colon V_i \lto V_i / V_{i-1}$ を作用させることで
    \begin{align}
        \lambda_i  p(e_i) = \lambda_i (e_i + V_{i-1}) = 0
    \end{align}
    がわかるが,$e_i \in V_i$ の選び方から $e_i + V_{i-1} \neq 0$ なので $\lambda_i = 0$ が言える.よって
    \begin{align}
        \sum_{j=1}^{i-1} \lambda_j e_j = 0 \in V_{i-1}
    \end{align}
    となるが,帰納法の仮定より $e_1,\, \dots,\, e_{i-1}$ は線型独立なので $1 \le \forall j \le i,\; \lambda_j = 0$ が言えた.i.e. $e_1,\, \dots,\, e_i$ は線型独立である.
    旗の定義から $\dim V_i = i$ なので $e_1,\, \dots,\, e_i$ は $V_i$ の基底であり,帰納法が完成した.
\end{proof}



$x \in \End V$ が\textbf{上三角化可能}であるとは,$x$ によって\hyperref[def:flag]{安定化される} $V$ の\hyperref[def:flag]{旗}が存在することを言う.
というのも,このとき補題\ref{lem:flag}の基底をとると
\begin{align}
    x(e_1) &= x^1{}_1 e_1 \in V_1, \\
    x(e_2) &= x^1{}_2 e_1 + x^2{}_2 e_2 \in V_2, \\
        &\vdots \\
    x(e_{\dim V}) &= x^1{}_{\dim V} e_1 + \cdots + x^{\dim V}{}_{\dim V} e_{\dim V} \in V_{\dim V}
\end{align}
と書けるので,この基底に関する $x$ の表現行列が
\begin{align}
    \mqty[x^1{}_1 & x^1{}_2 & \cdots &x^1{}_{\dim V} \\ 0 & x^2{}_2 & \cdots &x^2{}_{\dim V} \\ \vdots &\ddots &\ddots &\vdots \\ 0 & 0 &\cdots &x^{\dim V}{}_{\dim V}]
\end{align}
という上三角行列になるのである.

\begin{mytheo}[label=thm:upper-triangle]{上三角化}
    $\forall x \in \End V$ に対して,$x$ によって\hyperref[def:flag]{安定化される} $V$ の\hyperref[def:flag]{旗}が存在する.i.e. $\forall x \in \End V$ は上三角化可能である.
\end{mytheo}

\begin{proof}
    $\dim V$ に関する数学的帰納法により示す.$\dim V = 0$ のときは明らかなので
    $\dim V > 0$ とする.
    $\forall x \in \End V$ を1つ固定する.仮定より $\mathbb{K}$ は代数閉体なので,$x$ は重複も込めてちょうど $\dim V$ 個の\hyperref[def:eigen]{固有値}をもつ.それを $\lambda_1,\, \dots,\, \lambda_{\dim V}$ とおこう.
    $v_1 \in V(\lambda_1) \setminus \{0\}$ をとり,$V_1 \coloneqq \mathbb{K}v_1$ とおく.
    標準的射影を $p \colon V \lto V/V_1$ と書くと,$\forall v \in V_1$ に対して $p \circ x(v) = p(\lambda_1 v) = 0$ が成り立つので $V_1 \subset \Ker p \circ x$ である.故に\hyperref[prop:homo]{商ベクトル空間の普遍性}を使うことができ,$\mathbb{K}$-ベクトル空間の図式
    \begin{center}
        \begin{tikzcd}[row sep=large, column sep=large]
            V \ar[d, "p"']\ar[r, "x"] &V \ar[r, "p"] &V/V_1 \\
            V/V_1 \arrow[urr, red, dashed, "\exists!\bar{x}"']& &
        \end{tikzcd}
    \end{center}
    を可換にするような $\mathbb{K}$-線型写像 $\overline{x} \in \End (V/V_1)$ が一意的に存在する.$\dim (V/V_1) = \dim V - 1$ なので帰納法の仮定が使えて,
    $\overline{x} \in \End (V/V_1)$ によって安定化される $V/V_1$ の\hyperref[def:flag]{旗}
    \begin{align}
        0 = W_0 \subset W_1 \subset \cdots \subset W_{\dim V - 1} = V/V_1
    \end{align}
    が存在する.このとき $0 \le \forall i \le \dim V - 1$ を1つ固定すると
    $\forall w + V_1 \in W_i$ に対して $\overline{x} (w+V_1) \in W_i$ が成り立つ.\hyperref[prop:homo]{商ベクトル空間の普遍性}から $\overline{x} (w + V_1) = \overline{x} \circ p(w) = p \circ x(w)$ なので,$\forall v \in p^{-1}(W_i)$ に対して $p \circ x(v) = \overline{x} \circ p(v) \in W_i \IFF x(v) \in p^{-1}(W_i)$ が分かった.i.e. $V$ の部分空間の増大列
    \begin{align}
        0 \subset p^{-1}(W_0) \subset p^{-1}(W_1) \subset \cdots \subset p^{-1}(W_{\dim V - 1}) = V
    \end{align}
    は $x$ によって安定化される $V$ の旗であり,帰納法が完成する.
\end{proof}

\begin{marker}
    定理\ref{thm:upper-triangle}の証明から,もし $\mathbb{K}$ が代数閉体でなくても,$x \in \End V$ の固有値が全て $\mathbb{K}$ に含まれるならば $x$ は上三角化可能である.
\end{marker}

\subsection{広義固有空間分解}

\begin{mytheo}[label=thm:eigen-decomp-1]{広義固有空間分解-1}
    $x \in \Mat{n}{\mathbb{K}}$ を与え,
    \begin{itemize}
        \item $x$ の相異なる全ての\hyperref[def:eigen]{固有値}を $\lambda_1,\, \dots,\, \lambda_r$
        \item $\lambda_i$ の重複度を $p_i$
    \end{itemize} 
    とおく.このとき $\mathbb{K}^n$ の\hyperref[prop:subvec-directsum]{内部直和}分解
    \begin{align}
        \mathbb{K}^n = \bigoplus_{i=1}^r W(\lambda_i)
    \end{align}
    が一意的に存在し,かつ $\dim W(\lambda_i) = p_i$ である.
\end{mytheo}

\begin{proof}
    \begin{description}
        \item[$\bm{\dim W(\lambda_i) \ge p_i}$] 
        
        仮定より $\mathbb{K}$ は代数閉体なので,定理\ref{thm:upper-triangle}より $u \in \LGL (n,\, \mathbb{K})$ が存在して $u^{-1}xu$ が上三角行列になる.
        $1 \le i \le r$ を1つ固定し,$u^{-1} x u$ の対角成分の最初の $p_i$ 個が $\lambda_i$ であると仮定しても一般性を失わない.
        このとき $u^{-1}(x - \lambda_i \, \unity_n) u$ の対角成分の最初の $p_i$ 個は $0$ になるので $u^{-1} (x - \lambda_i \, \unity_n)^{p_i} u$ の第 $1 \le j \le p_i$ 列は全て $0$ となる.よって $u = [e_1\, \dots\, e_n]$ とおいたとき $\Span \{e_1,\, \dots,\, e_{p_i}\} \subset W(\lambda_i)$ である.$u$ は正則行列なので $e_1,\, \dots,\, e_{p_i}$ は線型独立であり,$\dim W(\lambda_i) \ge p_i$ だと分かった.

        \item[\textbf{和空間} $\bm{\sum_{i=1}^r W(\lambda_i)}$ \textbf{が内部直和}] 
        
        % 次に,相異なる $r$ 個の固有値 $\lambda_1,\, \dots,\, \lambda_r$ に関して和空間 $\sum_{i=1}^r W(\lambda_i)$ が\hyperref[prop:subvec-directsum]{内部直和}であることを,
        $r$ に関する数学的帰納法により示す.
        $r = 1$ のときは明らかなので $r > 1$ とする.
        $w_i \in W(\lambda_i)$ に対して
        \begin{align}
            \label{eq:eigen-1}
            \sum_{i=1}^r w_i = 0
        \end{align}
        が成り立つと仮定する.このとき $w_1 = \cdots = w_r = 0$ であることを示せば良い.
        \hyperref[def:eigen]{広義固有空間の定義}からある $k_r > 0$ が存在して $(x - \lambda_r \unity_n)^{k_r} w_r = 0$ が成り立つので,
        \begin{align}
            (x - \lambda_r \unity_n)^{k_r}\left(\sum_{i=1}^rw_i \right) = \sum_{i=1}^{r-1} (x - \lambda_r \unity_n)^{k_r}w_i = 0
        \end{align}
        が言える.$W(\lambda_i)$ は $x$-不変なので,帰納法の仮定から $1 \le \forall i \le r-1$ に対して $(x - \lambda_r \unity_n)^{k_r}w_i = 0$ である.
        ここである $1 \le j \le r-1$ が存在して $w_j \neq 0$ であると仮定する.このとき $k_j > 0$ を $(x - \lambda_j \unity_n)^{k_j} w_j = 0$ を充たす最小の自然数とする.
        このとき $x (x - \lambda_j \unity_n)^{k_j-1} w_j = \lambda_j (x - \lambda_j \unity_n)^{k_j-1} w_j$ なので
        \begin{align}
            0 
            &= (x - \lambda_j \unity_n)^{k_j-1} (x - \lambda_r \unity_n)^{k_r}w_j \\
            &= (x - \lambda_r \unity_n)^{k_r}(x - \lambda_j \unity_n)^{k_j-1} w_j \\
            &= (\lambda_j - \lambda_r)^{k_r}(x - \lambda_j \unity_n)^{k_j-1} w_j
        \end{align}
        が成り立つが,$\lambda_j \neq \lambda_r$ なので $(x - \lambda_j \unity_n)^{k_j-1} w_j = 0$ と言うことになって $k_j$ の最小性に矛盾する.よって背理法から $1 \le \forall j \le r-1$ に対して $w_j = 0$ が言えた.
        仮定\eqref{eq:eigen-1}より $w_r = 0$ も言えて帰納法が完成する.
    \end{description}

    代数学の基本定理より $\sum_{i=1}^r p_i = n$ が言える.$\bigoplus_{i=1}^r W(\lambda_i) \subset \mathbb{K}^n$ であることも踏まえると,以上の議論から $\dim W(\lambda_i) = p_i$ が分かった.従って $\mathbb{K}^n = \bigoplus_{i=1}^r W(\lambda_i)$ である.\hyperref[def:univ-vec-sum]{直和の普遍性}よりこの内部直和分解は一意である.
\end{proof}

\subsection{多項式環・最小多項式}

定理\ref{thm:eigen-decomp-1}の証明に多項式環を利用することもできる.
まず一般論から入る.
$R$ を環とし,$R$ の\textbf{単元} (unit)\footnote{\textbf{可逆元} (invertible element) と言う場合もある.}全体がなす集合を $\bm{R^\times}$ と書く.$R^\times$ は環の積に関して群となる.
特に,\underline{$R$ が体ならば $R^\times = R \setminus \{0\}$ となる}.

$n$ 個の変数 $t_1,\, \dots,\, t_n$ を持つ $R$ 係数多項式全体\footnote{\hyperref[prop:sum-vec]{直和の構成}と同様に,非ゼロの係数は有限個であるとする.より厳密には,写像 $\mathbb{Z}_{\ge 0}^n \lto R$ であって有限個の $(i_1,\, \dots,\, i_n) \in \mathbb{Z}_{\ge 0}^n$ を除いて $0$ を返すようなものと変数 $(t_1,\, \dots,\, t_n)$ の組のことである.}の集合を $R[t_1,\, \dots,\, t_n]$ と書く.
$R[t_1,\, \dots,\, t_n]$ の上の加法と乗法をそれぞれ
% $f(t_1,\, \dots,\, t_n) = \sum_{i_1,\, \dots,\, i_n} a_{i_1\dots i_n} t_1^{i_1} \cdots t_n^{i_n},\; g(t_1,\, \dots,\, t_n) = \sum_{i_1,\, \dots,\, i_n} b_{i_1\dots i_n} t_1^{i_1} \cdots t_n^{i_n} \in R[t_1,\, \dots,\, t_n]$ の間の加法と乗法を
\begin{align}
    f(t_1,\, \dots,\, t_n) + g(t_1,\, \dots,\, t_n)  &\coloneqq \sum_{i_1,\, \dots,\, i_n}(a_{i_1\dots i_n} + b_{i_1\dots i_n}) t_1^{i_1} \cdots t_n^{i_n}, \\
    f(t_1,\, \dots,\, t_n) g(t_1,\, \dots,\, t_n) &\coloneqq \sum_{i_1,\, \dots,\, i_n}\left(\sum_{j_1 + k_1 = i_1} \cdots \sum_{j_n + k_n = i_n}a_{j_1\dots j_n}b_{k_1\dots k_n}\right) t_1^{i_1} \cdots t_n^{i_n}
\end{align}
で定義する\footnote{乗法の記号は慣例に従って省略した.}と $R[t_1,\, \dots,\, t_n]$ 自身が環になる.ただし $f(t_1,\, \dots,\, t_n) = \sum_{i_1,\, \dots,\, i_n} a_{i_1\dots i_n} t_1^{i_1} \cdots t_n^{i_n},\; g(t_1,\, \dots,\, t_n) = \sum_{i_1,\, \dots,\, i_n} b_{i_1\dots i_n} t_1^{i_1} \cdots t_n^{i_n} \in R[t_1,\, \dots,\, t_n]$ とおいた.
ある1つの変数に注目して項を整理することで,帰納的に自然な(環の)同型 $R[t_1,\, \dots,\, t_n] \cong (R[t_1,\, \dots,\, t_{n-1}])[t_n] \cong \dots \cong (\cdots((R[t_1])[t_2])\cdots [t_{n-1}])[t_n]$ が得られる.

\begin{mydef}[label=def:poly-basic]{多項式の基本概念}
    $f(t_1,\, \dots,\, t_n) = \sum_{i_1,\, \dots,\, i_n} a_{i_1 \dots i_n} t_1^{i_1} \cdots t_n^{i_n}  \in R[t_1,\, \dots,\, t_n]$ を与える.
    \begin{itemize}
        \item $f(t_1,\, \dots,\, t_n) \neq 0$ ならば,$f$ の\textbf{次数} (degree) を
        \begin{align}
            \bm{\deg f} \coloneqq \max \bigl\{ i_1 + \cdots + i_n \bigm| a_{i_1 \dots i_n} \neq 0 \bigr\} 
        \end{align}
        と定義する.$f(t_1,\, \dots,\, t_n) \neq 0$ ならば $\deg f \coloneqq -\infty$ と定義する.
        \item $(x_1,\, \dots,\, x_n) \in R^n$ の $f$ への\textbf{代入}とは,値
        \begin{align}
            \sum_{i_1,\, \dots,\, i_n} a_{i_1\dots i_n} x_1^{i_1} \cdots x_n^{i_n} \in R
        \end{align}
        のこと\footnote{非ゼロな係数は有限個なので,この和は意味を持つ.}.記号としては $f(x_1,\, \dots,\, x_n)$ と書く.
        \item $f(x_1,\, \dots,\, x_n) = 0$ を充たす $(x_1,\, \dots,\, x_n) \in R^n$ のことを $f$ の\textbf{根} (root) と呼ぶ.
        \item $n=1$ のとき,$f$ の最高次係数が $1 \in R$ ならば $f$ は\textbf{モニック} (monic) であると呼ばれる.
    \end{itemize}
\end{mydef}

最高次係数が単元であるような多項式によって,多項式を割り算することができる:

\begin{myprop}[label=prop:poly-divide,breakable]{多項式の割り算}
    1変数多項式 $f(t),\, g(t) \in R[t]$ および単元 $u \in R^\times$ を与える.

    このとき $f(t)$ が\hyperref[def:poly-basic]{モニック}ならば,以下を充たす $q(t),\, r(t) \in R[t]$ が一意的に存在する:
    \begin{align}
        g(t) = q(t) \bigl( uf(t) \bigr) + r(t),\quad \deg r < \deg f
    \end{align}
    
\end{myprop}

\begin{proof}
    $f(t)$ はモニックであるとする.
    \begin{description}
        \item[\textbf{$\bm{q(t)},\, \bm{r(t)}$ の存在}] 
        
        $g(t) = 0$ のときは $q(t) = r(t) = 0$ とすれば良い.
        $g(t) \neq 0$ のとき,$\deg g$ に関する数学的帰納法により示す.
        $\deg g < \deg f$ のときは $q(t) = 0,\; r(t) = g(t)$ とおけば良い.
        
         $\deg g \ge \deg f$ とする.
        \begin{align}
            f(t) \eqqcolon \sum_{i=0}^{\deg f} a_i t^i,\quad g(t) \eqqcolon \sum_{i=0}^{\deg g} b_i t^i
        \end{align}
        とおくと仮定から $a_{\deg f} = 1$ である.このとき
        \begin{align}
            q_1(t) \coloneqq b_{\deg g} \bm{u^{-1}} t^{\deg g - \deg f},\quad g_1(t) \coloneqq g(t) - q_1(x) \bigl( uf(t) \bigr)
        \end{align}
        と定義すると $\deg g_1 < \deg g$ が成り立つ.$g_1(t) = 0$ ならば $q(t) = q_1(t),\; r(t) = 0$ とすれば良い.
        $g_1(t) \neq 0$ ならば,帰納法の仮定より $q_2(t),\, r_2(t) \in R[t]$ が存在して
        \begin{align}
            g_1(t) = q_2(t) \bigl( u f(t) \bigr) + r_2(t)
        \end{align}
        と書け,かつ $\deg r_2 < \deg f$ が成り立つ.すると
        \begin{align}
            g(t) = \bigl( q_1(t) + q_2(t) \bigr) \bigl( u f(t) \bigr) + r_2(t)
        \end{align}
        であるから,$q(t) = q_1(t) + q_2(t),\; r(t) = r_2(t)$ とすれば良い.

        \item[\textbf{$\bm{q(t)},\, \bm{r(t)}$ の一意性}] 
        
        $g(t) = q_1(t) \bigl( u f(t) \bigr) + r_1(t) =  q_2(t) \bigl( u f(t) \bigr) + r_2(t) \AND \deg r_i < \deg f$ が成り立つとする.このとき
        \begin{align}
            \bigl( q_1(t) - q_2(t) \bigr) \bigl( uf(t) \bigr) = r_2(t) - r_1(t)
        \end{align}
        が成り立つが,$q_1(t) - q_2(t) \neq 0$ だとすると $\deg f \le \max \{\deg r_1,\, \deg r_2\}$ と言うことになり矛盾.よって背理法から $q_1(t) = q_2(t)$ が言えて,$r_1(t) = r_2(t)$ も従う.
    \end{description}
\end{proof}

特に $R$ が体ならば $R^\times = R\setminus \{0\}$ なので,任意のゼロでない多項式同士の割り算をすることができる.
多変数の場合も,ある1つの変数に関する最高次係数が単元ならば割り算ができる.

\begin{mydef}[label=def:ideal-ring]{環のイデアル}
    $R$ を環とする.
    \begin{itemize}
        \item 部分集合 $I \subset R$ が左 (resp. 右)\textbf{イデアル} (ideal) であるとは,$\forall r \in R,\; \forall x \in I$ に対して $rx \in I$(resp. $xr \in I$)が成り立つことを言う.
        左イデアルかつ右イデアルのとき\textbf{両側イデアル}と呼ぶ.$R$ が可換環のときは左,右イデアルの区別はなく,単に\textbf{イデアル} (ideal) と呼ぶ.
        \item \underline{可換環} $R$ のイデアル $I\subsetneq R$ が\textbf{素イデアル} (prime ideal) であるとは,
        \begin{align}
            r,\, s \not\in I \IMP rs \not\in I
        \end{align}
        が成り立つことを言う.
        \item \underline{可換環} $R$ のイデアル $I \subsetneq R$ が\textbf{極大イデアル} (maximal ideal) であるとは,$I \subset J \subsetneq R$ なる任意のイデアル $J$ に対して $I = J$ が成り立つことを言う.
    \end{itemize}
\end{mydef}

\begin{mydef}[label=def:domain-basic,breakable]{整域}
    $R$ を\underline{可換環}とする.
    \begin{itemize}
        \item $R$ が\textbf{整域} (integral domain) であるとは,$\forall a,\, b \in R \setminus \{0\}$ に対して $ab \neq 0$ が成り立つことを言う.
        \item \textbf{整域} $R$ および $a \in R,\, b \in R \setminus \{0\}$ を与える.$b$ が $a$ の\textbf{約元}であるとは,ある $c \in R$ が存在して $a = bc$ となることを言い,$\bm{b \mid a}$ と書く.
        \item \textbf{整域} $R$ および $(a_1,\, \dots,\, a_n) \in R^n \setminus \{(0,\, \dots,\, 0)\},\, b \in R$ を与える.
        $b$ が $a_1,\, \dots,\, a_n$ の\textbf{公約元}であるとは,
        \begin{align}
            1 \le \forall i \le n,\; b \mid a_i
        \end{align}
        が成り立つことを言う.他の任意の公約元 $c \in R$ に対して $c \mid b$ が成り立つとき,$b$ は\textbf{最大公約元} (greatest common deviser; GCD) と呼ばれ,$\bm{\mathrm{gcd}}\bm{(a_1,\, ...,\, a_n)}$ と書く.
        \item \textbf{整域} $R$ および $b_1,\, \dots,\, b_n \in R \setminus \{0\},\, a \in R$ を与える.
        $a$ が $b_1,\, \dots,\, b_n$ の\textbf{公倍元}であるとは,
        \begin{align}
            1 \le \forall i \le n,\; b_i \mid a
        \end{align}
        が成り立つことを言う.他の任意の公倍元 $c \in R$ に対して $b \mid c$ が成り立つとき,$b$ は\textbf{最小公倍元} (least common multiple; LCM) と呼ばれ,$\bm{\mathrm{lcm}}\bm{(a_1,\, ...,\, a_n)}$ と書く.
        \item $R$ が\textbf{単項イデアル整域} (principal ideal domain; PID) であるとは,$R$ の任意の\hyperref[def:ideal-ring]{イデアル} $I \subset R$ に対してある $r \in R$ が存在して $I = Rr$ を充たすこと.
    \end{itemize}
    \tcblower
    $R$ を整域とし,$r \in R \setminus \{0\}$ を与える.
    \begin{itemize}
        \item $r$ が\textbf{素元}であるとは,$r$ が生成するイデアル $Rr \subset R$ が\hyperref[def:ideal-ring]{素イデアル}であることを言う.
        \item $r$ が\textbf{既約元}であるとは,$r \not\in R^\times$ で,かつ $\forall a,\, b \in R$ に対して
        \begin{align}
            r = ab \IMP a \in R^\times \OR b \in R^\times
        \end{align}
        が成り立つことを言う.$r$ が既約でなければ\textbf{可約}と言う.
        \item 既約元 $r,\, s$ が\textbf{同伴}であるとは,ある $u \in R^\times$ が存在して $r =su$ と書けることを言う.
    \end{itemize}
\end{mydef}


\begin{myprop}[label=prop:ideal-ring-basic]{素イデアルと極大イデアルの特徴付け}
    \underline{可換環} $R$ の\hyperref[def:ideal-ring]{イデアル} $I$ に対し,以下が成り立つ:
    \begin{enumerate}
        \item $I$ が\hyperref[def:ideal-ring]{素イデアル} $\IFF$ $R/I$ は\hyperref[def:domain-basic]{整域}
        \item $I$ が\hyperref[def:ideal-ring]{極大イデアル} $\IFF$ $R/I$ は体
    \end{enumerate}
\end{myprop}

\begin{proof}
    \begin{enumerate}
        \item 
    \end{enumerate}
\end{proof}

GCD, LCM は存在すれば単元の積を除いて一意である.


\begin{mydef}[label=def:domains,breakable]{種々の整域}
    $R$ を\hyperref[def:domain-basic]{整域}とする.
    \begin{itemize}
        \item $R$ が\textbf{Euclid整域} (Euclid domain) であるとは,
        以下の条件を充たす写像 $\phi \colon R \setminus \{0\} \lto \mathbb{Z}_{\ge 0}$ が存在することを言う:
        \begin{description}
            \item[\textbf{(ED)}] $\forall a \in R,\; \forall b \in R \setminus \{0\}$ に対して $q,\, r \in R$ が存在して
            \begin{align}
                a = qb + r
            \end{align}
            かつ
            \begin{align}
                r = 0 \OR \phi(r) < \phi(b)
            \end{align}
            が成り立つ.
        \end{description}
        
        \item $R$ が\textbf{単項イデアル整域} (principal ideal domain; PID) であるとは,$R$ の任意の\hyperref[def:ideal-ring]{イデアル} $I \subset R$ に対してある $r \in R$ が存在して $I = Rr$ を充たすこと.
        \item $R$ が\textbf{一意分解整域} (unique factorization domain: UFD) であるとは,以下の2条件を充たすこと:
        \begin{description}
            \item[\textbf{(UFD-1)}] $\forall r \in R \setminus (R^\times \cup \{0\})$ に対して,\hyperref[def:domain-basic]{素元} $p_1,\, \dots,\, p_n \in R$ が存在して $r = p_1 \cdots p_n$ と書ける.このとき $p_1,\, \dots,\, p_n$ を $r$ の\textbf{素因子},素因子の積 $p_1\cdots p_n$ を $a$ の\textbf{素元分解}と呼ぶ.
            \item[\textbf{(UFD-2)}] $p_1,\, \dots,\, p_n,\; q_1,\, \dots,\, q_m \in R$ が素元ならば $n = m$ であり,かつある $\sigma \in \mathfrak{S}_n$ が存在して,$1 \le \forall i \le n$ に対して $p_i$ と $q_{\sigma(i)}$ が\hyperref[def:domain-basic]{同伴}となる.
        \end{description}
    \end{itemize}
    
\end{mydef}

次の定理は整域の理論において極めて重要である:

\begin{mytheo}[label=thm:ED-PID-UFD]{}
    \hyperref[def:domains]{Euclid整域} $\IMP$ \hyperref[def:domains]{単項イデアル整域} $\IMP$ \hyperref[def:domains]{一意分解整域}
\end{mytheo}

\begin{proof}
    \begin{description}
        \item[\textbf{Euclid整域 $\IMP$ PID}] 
        
        $R$ が\hyperref[def:domains]{Euclid整域}であるとする.
        $R = \{0\}$ ならば $R = R0$ となるのでPIDである.
        
         $R \neq \{0\}$ とする.$R$ の任意の非自明な\hyperref[def:ideal-ring]{イデアル} $I$ をとる.
        $\mathbb{Z}_{\ge 0}$ は整列集合なので $\phi(x) \coloneqq \min \bigl\{\, \phi(y) \in \mathbb{Z}_{\ge 0} \bigm| y \in I \setminus \{0\} \,\bigr\}$ が存在する.
        Euclid整域の定義から $\forall z \in I$ に対してある $q,\, r \in R$ が存在して
        \begin{align}
            z = qx + r
        \end{align}
        かつ $r = 0$ または $\phi(r) < \phi(x)$ が成り立つが,もし $r \neq 0$ ならば $r \in I \setminus \{0\}$ となり $x$ の最小性に矛盾する.よって $r = 0$ であり,$I = Rx$ が言えた.

        \item[\textbf{PID $\IMP$ UFD}] 
        

    \end{description}
    
\end{proof}

\begin{mycol}[label=col:poly-PID]{}
    任意の体\footnote{定義から可換である.} $\mathbb{K}$ に対して $\mathbb{K}[t]$ は\hyperref[def:domains]{Euclid整域}である.
    従ってPIDでもありUFDでもある.
\end{mycol}

\begin{proof}
    命題\ref{prop:poly-divide}より従う.
\end{proof}


\begin{myprop}[label=prop:PID-prime]{}
    \underline{可換環} $R$ について以下が成り立つ:
    \begin{enumerate}
        \item $R$ が\hyperref[def:domain-basic]{整域}ならば,$R$ の任意の\hyperref[def:domain-basic]{素元}は\hyperref[def:domain-basic]{既約元}である.
        \item $R$ が\hyperref[def:domains]{UFD}ならば,$R$ の任意の\hyperref[def:domain-basic]{既約元}は\hyperref[def:domain-basic]{素元}である.
        \item $R$ が\hyperref[def:domains]{PID}ならば,$R$ の任意の $\{0\}$ でない\hyperref[def:ideal-ring]{素イデアル}は\hyperref[def:ideal-ring]{極大イデアル}である.
    \end{enumerate}
    
\end{myprop}

\begin{proof}
    \begin{enumerate}
        \item $R$ が整域だとする.このとき任意の素元 $p \in R \setminus \{0\}$ をとると,$Rp \subsetneq R$ なので $p \notin R^\times$ である.$a,\, b \in R$ に対して $p = ab$ が成り立つとする.
        $Rp$ が\hyperref[def:ideal-ring]{素イデアル}なので $a \in Rp$ または $b \in Rp$ が成り立つ.$R$ は可換なので $a \in Rp$ としても一般性を失わない.このときある $q \in R$ が存在して $a = qp$ と書ける.
        $ab = p$ なので $qpb = pqb = p \IFF p(qb-1) = 0$ が成り立つ.$R$ は整域で $p \neq 0$ なので $qb-1 = 0 \IFF qb = 1$,i.e. $b \in R^\times$ が分かった.よって $p$ は既約である.
        \item $R$ がUFDだとする.このとき任意の既約元 $p \in R \setminus (R^\times \cup \{0\})$ をとると,その素元分解 $p = p_1 \cdots p_n\quad (n \ge 1)$ が存在する.$n \ge 2$ だとすると,$p_1 \notin R^\times$ なので $p$ の既約性から $q \coloneqq p_2 \cdots p_n \in R^\times$ となる.然るにこのとき $Rq \subset Rp_2 \subsetneq R$ となり矛盾.よって背理法から $n=1$ が言えた.
        \item $R$ のゼロでない任意の素イデアル $I$ をとる.$R$ はPIDなので $p \in R$ が存在して $I = Rp$ と書けるが,素元の定義からこのとき $p \in R$ は素元であり,および命題\ref{thm:ED-PID-UFD}-(1) から既約元である.
        
        ここで,$I \subset J \subsetneq R$ なる任意のイデアル $J$ をとる.$R$ はPIDなので $q \in R$ が存在して $J = Rq$ と書けるが,$Rq \subsetneq R$ なので $q \notin R^\times$ である.
        このときある $u \in R$ が存在して $p = qu$ と書けるが,$p$ は既約元なので $u \in R^\times$ でなくてはならない.よって $Rp = Rq$ となり,$Rp$ は極大イデアルである.命題\ref{prop:ideal-ring-basic}-(2) より $R/Rp$ は体である.
    \end{enumerate}
\end{proof}


\begin{myprop}[label=prop:PID-Bezout]{単項イデアル整域におけるB\'{e}zoutの等式}
    $R$ を\hyperref[def:domains]{PID}とする.
    このとき,$\forall (a_1,\, a_2) \in R^2  \setminus \{(0,\, 0)\},\; \forall b \in R$ に対して以下は同値である:
    \begin{enumerate}
        \item $Ra_1 + Ra_2 = Rb$
        \item $\mathrm{\gcd}(a_1,\, a_2) = b$
    \end{enumerate}
\end{myprop}

\begin{proof}
    \item[\textbf{(1) $\bm{\Longrightarrow}$ (2)}] 
    
    $Ra_1 + Ra_2 = Rb$ とする.このとき $a_i \in Rb\quad (i = 1,\, 2)$ なのである $q_i \in R$ が存在して $a_i = q_i b$ と書ける.i.e. $b \mid a_i$ である.
    
    $c \in R$ を $a_1,\, a_2$ の任意の\hyperref[def:domain-basic]{公約元}とする.このときある $r_i \in R$ が存在して $a_i = r_i c$ と書ける.
    仮定より $b \in Ra_1 + Ra_2$ なので,
    $s_i \in R$ が存在して
    \begin{align}
        b = s_1a_1 + s_2a_2 = (s_1 r_1 + s_2r_2) c
    \end{align}
    と書ける.i.e. $c \mid b$ が言えた.
    \item[\textbf{(1) $\bm{\Longleftarrow}$ (2)}] 
    
    $\mathrm{\gcd}(a_1,\, a_2) = b$ とする.
    $R$ はPIDなので,ある $c \in R$ が存在して $Ra_1 + Ra_2 = Rc$ と書ける.必要性の証明から $c = \mathrm{gcd}(a_1,\, a_2)$ が言えるので,$u \in R^\times$ が存在して $b = uc$ と書ける.よって $Ra_1 + Ra_2 = Rc = Rb$ である.
\end{proof}

\begin{mycol}[label=col:PID-Bezout]{単項イデアル整域におけるB\'{e}zoutの等式}
    $R$ を\hyperref[def:domains]{PID}とする.このとき,$\forall (a_1,\, \dots,\, a_n) \in R^n \setminus \{(0,\, \dots,\, 0)\},\; \forall b \in R$ に対して以下は同値である:
    \begin{enumerate}
        \item $Ra_1 + \cdots + Ra_n = Rb$
        \item $\mathrm{\gcd}(a_1,\, \dots,\, a_n) = b$
    \end{enumerate}
\end{mycol}

\begin{mydef}[label=def:field-extention]{体の拡大}
    \begin{itemize}
        \item 体 $\mathbb{L}$ が体 $\mathbb{K}$ の\textbf{拡大体} (extention field) であるとは,$\mathbb{K}$ が $\mathbb{L}$ の部分体になっていることを言う.
        このことを記号で $\bm{\mathbb{L}/\mathbb{K}}$ と書き,\textbf{体の拡大} (field extention) であると言う.
        \item $\mathbb{L} / \mathbb{K}$ を体の拡大とするとき,部分体 $\mathbb{M} \subset \mathbb{L}$ であって $\mathbb{K} \subset \mathbb{M}$ を充たすものを $\mathbb{L}/\mathbb{K}$ の\textbf{中間体}と呼ぶ.
        \item $\mathbb{L} / \mathbb{K}$ を体の拡大とするとき,$\mathbb{L}$ を $\mathbb{K}$-ベクトル空間と見做したときの次元を $\mathbb{L}$ の $\mathbb{K}$ 上の\textbf{拡大次数}と呼び,$\bm{[\mathbb{L}:\mathbb{K}]}$ と書く.
        \item $\mathbb{L} / \mathbb{K}$ を体の拡大とする.$x \in \mathbb{L}$ が $\mathbb{K}$ 上\textbf{代数的}であるとは,$x$ がある $\mathbb{K}$-係数多項式 $f(t) \in \mathbb{K}[t] \setminus \{0\}$ の\hyperref[def:poly-basic]{根}となっていること\footnote{$x \in \mathbb{K}$ とは限らないので,先ほど採用した定義によると厳密には根とは呼べない.}を言う.
        $\forall x \in \mathbb{L}$ が $\mathbb{K}$ 上代数的ならば,体の拡大 $\mathbb{L}/\mathbb{K}$ は\textbf{代数拡大}であるという.
    \end{itemize}
\end{mydef}

\begin{myprop}[label=prop:minimal-poly]{最小多項式の存在}
    $\mathbb{L}/\mathbb{K}$ を\hyperref[def:field-extention]{体の代数拡大}とし,$\forall \alpha \in \mathbb{L}$ を1つ与える.
    このとき $I_\alpha \coloneqq \bigl\{\, f(t) \in \mathbb{K}[t] \bigm| f(\alpha) = 0 \,\bigr\}$ とおくと,以下の条件を充たす $f(t) \in I_\alpha$ が定数倍を除いて一意的に存在する:
    \begin{enumerate}
        \item $f(t) \neq 0$
        \item $f(t)$ は $I_\alpha$ の元のうち $\deg f$ が最小のものの1つである.
        \item $f(t)$ は $I_\alpha$ の全ての元の\hyperref[def:domains]{約元}である.
        \item $f(t)$ は\hyperref[def:domain-basic]{既約}
    \end{enumerate}
\end{myprop}

\begin{proof}
    $\mathbb{K}$-\hyperref[ax:Alg]{結合代数}の準同型
    \begin{align}
        \phi \colon \mathbb{K}[t] \lto \mathbb{L},\; g(t) \lmto g(\alpha)
    \end{align}
    に関して $\Ker \phi = I_\alpha$ であり,従って $I_\alpha$ は $\mathbb{K}[t]$ の\hyperref[def:ideal-ring]{イデアル}である.
    $\alpha$ は $\mathbb{K}$ 上代数的なので $I_\alpha \neq \{0\}$ であり,
    系\ref{col:poly-PID}より $\mathbb{K}[t]$ は\hyperref[def:domains]{PID}なので,ある $f(t) \in \mathbb{K}[t]$ が存在して $I_\alpha = \mathbb{K}[t] f(t)$ と書ける.このとき定理\ref{thm:ED-PID-UFD}の証明から $f(t)$ は $I_\alpha\setminus \{0\}$ の元のうち次数が最小であり,
    $\mathbb{K}[t]^\times = \mathbb{K} \setminus \{0\}$ なので定数倍を除いて一意に定まる.

    あとは $f(x)$ の既約性を示す.実際,準同型定理から $\mathbb{K}[t]/I_\alpha \cong \Im \phi = \mathbb{K}[\alpha]$ が言えるが,$\mathbb{K}[\alpha]$ は体なので命題\ref{prop:ideal-ring-basic}-(2) より $I_\alpha = \mathbb{K}[t] f(t)$ は\hyperref[def:ideal-ring]{素イデアル}であり,\hyperref[def:domain-basic]{素元の定義}から $f(t)$ は素元.よって命題\ref{prop:PID-prime}-(1) から $f(x)$ は既約元である.
\end{proof}

\begin{mydef}[label=def:minimal-poly]{最小多項式}
    命題\ref{prop:minimal-poly}の $f(t) \in I_\alpha$ が\hyperref[def:poly-basic]{モニック}ならば,$f(t)$ のことを $\alpha$ の $\mathbb{K}$ 上の\textbf{最小多項式} (minimal polynomial) と呼ぶ.
\end{mydef}

命題\ref{prop:minimal-poly}では $\mathbb{L}/\mathbb{K}$ を体の代数拡大としたが,$\mathbb{L}$ を $\Mat{n}{\mathbb{K}}$ に置き換えても (1)-(3) はほぼ同じ証明により成り立つ\footnote{$I_\alpha \neq \{0\}$ の証明のみ変更を要する.}.このようなときも $f(t) \in I_\alpha$ のことを $\alpha \in \Mat{n}{\mathbb{K}}$ の\textbf{最小多項式}と呼ぶ.
では,定理\ref{thm:eigen-decomp-1}の別証明を与えよう:

\begin{mytheo}[label=thm:eigen-decomp-2]{広義固有空間分解-2}
    $x \in \Mat{n}{\mathbb{K}}$ を与え,
    \begin{itemize}
        \item $x$ の相異なる全ての\hyperref[def:eigen]{固有値}を $\lambda_1,\, \dots,\, \lambda_r$
        \item $\lambda_i$ の重複度を $p_i$
    \end{itemize} 
    とおく.このとき $\mathbb{K}^n$ の\hyperref[prop:subvec-directsum]{内部直和}分解
    \begin{align}
        \mathbb{K}^n = \bigoplus_{i=1}^r W(\lambda_i)
    \end{align}
    が一意的に存在し,かつ $\dim W(\lambda_i) = p_i$ である.
\end{mytheo}

\begin{proof}
    $1 \le \forall i \le k$ に対して $\phi_i (t) \coloneqq \prod_{j \neq i} (t-\lambda_i)^{p_i} \in \mathbb{K}[t]$ とおく.
\end{proof}


% \section{旗と上三角行列}

% $V$ を有限次元 $\mathbb{K}$ ベクトル空間とし,$V$ の\hyperref[def:flag]{旗}
% \begin{align}
%     V_0 \subset V_1 \subset \cdots \subset V_{\dim V}
% \end{align}
% を与える.$x \in \End V$ がこの旗を\hyperref[def:flag]{安定化する}としよう.
% このとき,まず $v_1 \in V_1$ をとり $W_1 \coloneqq V_1 / \mathbb{K} v_1$ とおく.
\end{document}