\documentclass[rep_main]{subfiles}

\begin{document}

\setcounter{chapter}{5}

\chapter{表現定理}
この章では,$\mathfrak{g}$は標数$0$の代数閉体$\mathbb{K}$上の\hyperref[def:semisimple-LieAlg]{半単純Lie代数}とし,$\mathfrak{h}$を$\mathfrak{g}$の極大\hyperref[def:toral-subLieAlg]{トーラス},$\Phi$を\hyperref[ax:root-system-2]{ルート系},$\Delta = \{\alpha_1, \ldots, \alpha_l\}$を$\Phi$の\hyperref[def:base-root]{底},$\mathscr{W}$を\hyperref[def:Weylgroup]{Weyl群}とする.

\section{表現のウエイトと極大ベクトル}
$\lsl{2}{\mathbb{K}}$で考えたウエイトを一般化する.$\lsl{2}{\mathbb{K}}$の場合,極大\hyperref[def:toral-subLieAlg]{トーラス}$\mathfrak{h}$は$1$次元線型空間だったので,ウエイトは固有値として定義された.$\mathfrak{h}$が一般の次元の場合は,ルートのように$\mathfrak{h}^*$の元として定義する.ルートと異なる点は,表現が随伴表現とは限らないことだけである.
\subsection{ウエイト空間}
\begin{mydef}[label=def:weight-rep]{表現のウエイト,ウエイト空間,ウエイトの集合}
	$V$を\hyperref[ax:g-module]{$\mathfrak{g}$-加群}とし,極大\hyperref[def:toral-subLieAlg]{トーラス}$\mathfrak{h}$を一つ固定する.$\lambda \in h^*$に対し,
	\begin{align}
		V_\lambda \coloneqq \bigl\{\, v \in V \bigm| h \btr v = \lambda(h) v \,\bigr (\forall h \in \mathfrak{h})\bigr\}
	\end{align}
	が定義される.この内$V_\lambda \neq 0$のものを,\textbf{ウエイト空間} (weight space) と呼び,$\lambda$を$V$の\textbf{ウエイト} (weight) (より正確には$\mathfrak{h}$の$V$上のウエイト)と呼ぶ.\\
	また$V$の\textbf{ウエイトの集合}を
	\begin{equation}
		\Pi(V) \coloneqq \bigl\{\, \mu \in \mathfrak{h}^* \bigm| V_\mu \neq 0\bigr\}
	\end{equation}
	と定義する.
\end{mydef}
\begin{mylem}[label=lem:weight-rep]{}
	\begin{enumerate}
		\item $V' \coloneqq \sum_{\lambda \in \mathfrak{h}^*} V_\lambda$は\hyperref[def:univ-vec-sum]{ベクトル空間の直和}で,$V$の\hyperref[def:sub-g-module]{部分$\mathfrak{g}$-加群}.
		\item $V$が有限次元の場合,$V = V'$.
		\item $\mathfrak{g}_\alpha \btr V_\lambda = V_{\lambda + \alpha},\ (\forall \lambda \in \mathfrak{h}^*,\ \forall \alpha \in \Phi)$.
	\end{enumerate}
\end{mylem}
\begin{proof}
	\begin{enumerate}
		\item \hyperref[def:weight-rep]{ウエイト}$\lambda, \lambda' \in \mathfrak{h}$に対し,$v \in V_\lambda \cap V_\lambda' \setminus \{0\}$とする.任意の$h$に対し,
		\begin{equation}
			h \btr v = \lambda(h)v = \lambda'(h)v  \IFF  (\lambda(h) - \lambda'(h))v = 0 
		\end{equation}
		であるから,$v \neq 0$より$\lambda = \lambda'$となる.よって,\hyperref[def:univ-vec-sum]{直和}となる.
		\item $\mathfrak{g}$は\hyperref[def:semisimple-LieAlg]{半単純}なので,系\ref{col:JC}より$\mathfrak{h}^*$の元を同時対角化できる.
		\item $x \in \mathfrak{g}_\alpha,\ h \in \mathfrak{h},\ v \in V_\lambda$とすると,
		\begin{equation}
			h \btr (x \btr v) = x \btr (h \btr v) + [h, x] \btr v = (\lambda(h) + \alpha(h))(x \btr v)
		\end{equation}
		より,$\mathfrak{g}_\alpha \btr V_\lambda = V_{\lambda + \alpha}$となる.
	\end{enumerate}
\end{proof}
\subsection{最高ウエイト加群}
極大ベクトルも同様に一般化する.
\begin{mydef}[label=def:maximal-vector-rep]{ウエイトの極大ベクトル}
	\hyperref[ax:g-module]{$\mathfrak{g}$-加群}$V$に対し,ルートの\hyperref[def:base-root]{底}$\Delta$を固定する.$V$のその\hyperref[def:weight-rep]{ウエイト}$\lambda$の固有空間の元$v^+ \neq 0$が,
	\begin{align}
		\mathfrak{g}_\alpha \btr v^+ = 0\quad  (\forall \alpha \succ 0)  \IFF  \mathfrak{g}_\alpha \btr v^+ = 0\quad  (\forall \alpha \in \Delta)
	\end{align}
	を満たすとき,$v^+$を\textbf{極大ベクトル}(maximal vector)と呼ぶ.
\end{mydef}
$\Longleftarrow$は,命題\ref{prop:root-decomp-int}の$[g_\alpha, g_\beta] = g_{\alpha + \beta}$より言える.\\
無限次元では存在は保証されない.一方有限次元ではBorel部分代数(極大\hyperref[def:solvable-LieAlg]{可解}部分代数,第4章)$\mathfrak{b}(\Delta) = \mathfrak{h} \oplus \mathfrak{n}^+\ (\mathfrak{n}^+ = \oplus_{\alpha \succ 0} \mathfrak{g}_\alpha)$を考えると,\hyperref[thm:Lie]{Lieの定理}と\hyperref[root-decomp-basic1]{$g_\alpha$の冪零性}から,$\mathfrak{g}_\alpha$の作用で$0$になるような$\mathfrak{b}(\Delta)$の共通の固有ベクトルが存在する.

\hyperref[def:maximal-vector-rep]{極大ベクトル}で生成される\hyperref[ax:g-module]{$\mathfrak{g}$-加群}を考える.
\begin{mydef}[label=def:highest-weight-module]{最高ウエイト加群}
	\hyperref[def:weight-rep]{ウエイト}$\lambda$の\hyperref[def:maximal-vector-rep]{極大ベクトル}$v^+$と,$\mathfrak{g}$の\hyperref[def:univ-env-alg]{普遍包絡代数}$\mathfrak{U}(\mathfrak{g})$に対し,
	\begin{equation}
		V = \mathfrak{U}(\mathfrak{g}) \btr v^+
	\end{equation}を満たすとき,$\lambda$を$V$の\textbf{最高ウエイト}(highest weight)と呼び,$V$は(\hyperref[def:weight-rep]{ウエイト}$\lambda$の)\textbf{最高ウエイト加群}(または標準巡回(stnadard cyclic)加群)と呼ぶ.
\end{mydef}
\hyperref[def:highest-weight-module]{最高ウエイト加群}の構造は単純.命題\ref{prop:root-decomp-ortho}より,\hyperref[def:base-root]{正ルート}$\alpha$に対し,$x_\alpha \in \mathfrak{g}_\alpha\setminus\{0\}$を選ぶと,$y_\alpha \in \mathfrak{g}_{-\alpha}$が存在し,$[x_\alpha, y_\alpha] = h_\alpha \in \mathfrak{h}$となる.\\
\hyperref[ax:root-system-2]{ルート系}$(\mathbb{E}, \Phi)$で定義された\hyperref[def:base-root,breakable]{半順序}同様に,Lie代数の\hyperref[def:weight-rep]{ウエイト}の\textbf{半順序}$\prec \subset \mathfrak{h}^* \times \mathfrak{h}^*$を
\begin{equation}
	\mu \prec \lambda  \DEF  \exists! \Familyset[\big]{k_\alpha}{\alpha \in \Delta} \in \prod_{\alpha \in \Delta} \mathbb{R}_{\ge 0},\; \lambda - \mu = \sum_{\alpha \in \Delta} k_\alpha \alpha
\end{equation}
と定義する.

\begin{mylem}[label=lem:max-sub-quot-mod]{}
	加群$M$の部分加群$N$に対し,
	\begin{equation}
		\text{$N$が極大} \IFF \text{商加群$M/N$は既約}
	\end{equation}
\end{mylem}
\begin{proof}
	$\Leftrightarrow$のどちらも対偶を示す.\\
	($\Rightarrow$)商加群の非自明部分加群$K$が存在したとすると,$N \subsetneq K + N \subsetneq M$より,$N$は極大でない.\\
	($\Leftarrow$)$N \subsetneq K \subsetneq M$を満たす部分加群$K$が存在したとすると,$M / K$は$M / N$の非自明部分加群である.
\end{proof}
\begin{mytheo}[label=thm:hwmodule]{}
	$V$を\hyperref[def:highest-weight-module]{最高ウエイト$\mathfrak{g}$-加群},$\lambda$を\hyperref[def:highest-weight-module]{最高ウエイト},$v^+ \in V_\lambda$を\hyperref[def:maximal-vector-rep]{極大ベクトル},$\Phi^+ = \{\beta_1, \ldots, \beta_m\}$を\hyperref[def:base-root]{正ルート}の集合とする.
	\begin{enumerate}
		\item\label{thm:hwmodule-a} $V = \Span\{y_{\beta_1}^{i_1}\cdots y_{\beta_m}^{i_m} \btr v^+\ (i_j \in \mathbb{Z}_{\geq 0})\}$.特に,\hyperref[def:weight-rep]{ウエイト空間}の\hyperref[def:univ-vec-sum]{直和}である.
		\item\label{thm:hwmodule-b} $V$の任意の\hyperref[def:weight-rep]{ウエイト}について,$\mu \prec \lambda$(最高ウエイトと呼ばれる所以).
		\item\label{thm:hwmodule-c} $\forall \mu \in \mathfrak{h}^*$に対し,$\dim V_\mu < \infty$.また,$\dim V_\lambda = 1$.
		\item\label{thm:hwmodule-d} 任意の$V$の\hyperref[def:sub-g-module]{部分加群}は\hyperref[def:weight-rep]{ウエイト空間}の\hyperref[def:univ-vec-sum]{直和}.
		\item\label{thm:hwmodule-e} $V$は\hyperref[def:irr]{直既約$\mathfrak{g}$-加群}であり,唯一の極大\hyperref[def:sub-g-module]{部分加群}と対応する\hyperref[def:irr]{既約}な商$\mathfrak{g}$-加群をもつ.
		\item 全ての非零な準同型な$V$の像も\hyperref[def:weight-rep]{ウエイト}$\lambda$の\hyperref[def:maximal-vector-rep]{最高ウエイト加群}.
	\end{enumerate}
\end{mytheo}
\begin{proof}
	\begin{description}
		\item[\textbf{(1)}] $\mathfrak{n}^- = \bigoplus_{\alpha \prec 0}\mathfrak{g}_\alpha,\quad  \mathfrak{b} = \mathfrak{b}(\Delta)$をBorel部分代数とする.\hyperref[def:univ-env-alg-module]{普遍包絡代数上の加群の定義}と\hyperref[thm:PBW]{PBW定理}(の系\ref{col:PBW-D})より,
		\begin{equation}
			\mathfrak{U}(\mathfrak{g}) \btr v^+ = \mathfrak{U}(\mathfrak{n}^-)\mathfrak{U}(\mathfrak{b}) \btr v^+ = \mathfrak{U}(\mathfrak{n}^-) \btr (\mathbb{K}v^+)\quad  (\because \text{$\mathfrak{b}$の共通の固有ベクトル})
		\end{equation}
		$\mathfrak{U}(\mathfrak{n}^-)$の基底は$y_{\beta_1}^{i_1}\cdots y_{\beta_m}^{i_m}$だったから,$V$はこのベクトルの集合で張られる.
		\item[\textbf{(2)}] 補題\ref{lem:weight-rep}より,
		\begin{equation}
			\label{eq:hwmodule-v}
			v \coloneqq y_{\beta_1}^{i_1}\cdots y_{\beta_m}^{i_m}v^+
		\end{equation}
		の\hyperref[def:weight-rep]{ウエイト}は$\mu = \lambda - \sum_j i_j\beta_j$であり,$\beta_j \in \Phi^+$は\hyperref[def:base-root]{単純ルート}の和だったから,$\mu \prec \lambda$.
		\item[\textbf{(3)}] $\sum_j i_j\beta_j = \sum_{i=1}^l k_i\alpha_i$を満たす$(i_j,\ \beta_j)$の組み合わせは高々有限個だから,$\dim V_\mu < \infty$.特に,\hyperref[def:weight-rep]{ウエイト}$\lambda$を持つ\eqref{eq:hwmodule-v}の形のベクトルは$v^+$しかないので,$\dim V_\mu = 1$.
		\item[\textbf{(4)}] $W$を$V$の\hyperref[def:sub-g-module]{部分加群}とする,$\forall w \in W \subset V$は異なる\hyperref[def:weight-rep]{ウエイト空間}$V_{\mu_i}$の元$v_i$の和で書ける.$\forall v_i \in W$を背理法で示す.\\
		$W$は線型空間なので,$w = v_1 + \cdots + v_n \in W,\ (\forall v_i \notin W)$を(背理法の仮定として)仮定して良い.$n=1$はあり得ないので,$n > 1$とする.ここで,
		\begin{equation}
			\exists h \in \mathfrak{h},\quad  h \btr w = \sum_{i=1}^n \mu_i(h) v_i \in W\ \mu_1(h) \neq \mu_2(h)
		\end{equation}
		とでき,
		\begin{equation}
			(h - \mu_1(h)1) \btr w = \sum_{i=2}^n (\mu_i(h) - \mu_1(h)) v_i \in W \setminus \{0\}
		\end{equation}
		となる.これを繰り返すと,$v_n \in W$となり矛盾する.
		\item[\textbf{(5)}] $V$の\hyperref[def:sub-g-module]{部分加群}が$V_\lambda$を含むと,(1)より$V$自身となる.(4)と合わせて,$V$の非自明な\hyperref[def:sub-g-module]{部分$\mathfrak{g}$-加群}は$V_\lambda$以外の\hyperref[def:weight-rep]{ウエイト空間}の\hyperref[def:univ-vec-sum]{直和}でかけるから,これら全ての\hyperref[def:sub-g-module]{部分加群}の和$W$も\hyperref[def:sub-g-module]{部分$\mathfrak{g}$-加群}である.よって$V$は,唯一の極大\hyperref[def:sub-g-module]{部分加群}$W$と唯一の\hyperref[def:irr]{既約}な商$\mathfrak{g}$-加群(補題\ref{lem:max-sub-quot-mod})をもつ.全ての非自明\hyperref[def:sub-g-module]{部分加群}は$W$に含まれ,$V_\lambda$を含まないから,$V$は\hyperref[def:sub-g-module]{部分加群}の和で表せない.i.e. \hyperref[def:irr]{直既約}$\mathfrak{g}$-加群である.\\
		\item[\textbf{(6)}] \textbf{(1)}の生成系を実際に準同型で送ったものなので,\hyperref[def:highest-weight-module]{最高ウエイト加群}.
	\end{description}
\end{proof}

\begin{mycol}[label=col:highest-weight-uni]{$\mathfrak{g}$-加群の極大ベクトルの唯一性}
	\hyperref[def:irr]{既約}な\hyperref[def:highest-weight-module]{最高ウエイト$\mathfrak{g}$-加群}$V$の,(\hyperref[def:highest-weight-module]{最高ウエイト}$\lambda$の)$v^+$は高々スカラー倍の違いを除いて唯一.
\end{mycol}
\begin{proof}
	$w^+$を\hyperref[def:weight-rep]{ウエイト}$\lambda'$の\hyperref[def:maximal-vector-rep]{極大ベクトル}とすると,$\mathfrak{U}(\mathfrak{g}) \btr w^+$は$V$の\hyperref[def:sub-g-module]{部分加群}である.$V$は\hyperref[def:irr]{既約}だったので,$\mathfrak{U}(\mathfrak{g}) \btr w^+ = V$.定理\ref{thm:hwmodule}\ref{thm:hwmodule-b}より,$\lambda = \lambda'$.よって定理\ref{thm:hwmodule}\ref{thm:hwmodule-c}より,$w^+$は$v^+$のスカラー倍.
\end{proof}

\subsection{存在と唯一性}
各\hyperref[def:weight-rep]{ウエイト}$\lambda \in \mathfrak{h}^*$に対し,(同型を除いて)唯一の\hyperref[def:irr]{既約}な\hyperref[def:highest-weight-module]{最高ウエイト$\mathfrak{g}$-加群}が存在する事を示す(無限次元でも成立).唯一性の方の証明は,第4章14.2の定理の証明と似ている上に単純になっているらしい.
\begin{mytheo}[label=thm:hwmodule-unique]{唯一性}
	$V, W$を\hyperref[def:highest-weight-module]{最高ウエイト}$\lambda$の\hyperref[def:highest-weight-module]{最高ウエイト加群}とする.$V, W$が共に\hyperref[def:irr]{既約}ならば同型.
\end{mytheo}
\begin{proof}
	\hyperref[ax:g-module]{$\mathfrak{g}$-加群}$X  = V \oplus W$を考える.$v^+, w^+$をそれぞれ$V, W$の\hyperref[def:weight-rep]{ウエイト}$\lambda$の\hyperref[def:maximal-vector-rep]{極大ベクトル}とすると,
	\begin{equation}
		X = V \oplus W = (\mathfrak{U}(\mathfrak{g}) \btr v^+) \oplus (\mathfrak{U}(\mathfrak{g}) \btr w^+) = \mathfrak{U}(\mathfrak{g}) \btr (v^+, w^+)
	\end{equation}
	より$x^+ = (v^+, w^+)$は$X$の\hyperref[def:weight-rep]{ウエイト}$\lambda$の\hyperref[def:maximal-vector-rep]{極大ベクトル}となる.$x^+$で生成される$X$の(\hyperref[def:highest-weight-module]{最高ウエイト})\hyperref[def:sub-g-module]{部分加群}を$Y$とし,$p:\ Y \to V,\ p':\ Y \to W$をそれぞれ$X$の第一,第二引数を取ってくる射影に誘導される写像とすると,
	\begin{align}
		&p(x^+) = v^+,\quad  p'(x^+) = w^+ \\
		&\Im p = V,\quad  \Im p' = W
	\end{align}
	より,$p,\ p'$は共に\hyperref[ax:g-module]{$\mathfrak{g}$-加群}の全射\hyperref[g-module-hom]{準同型}.
	\begin{center}
		\begin{tikzcd}[row sep=large, column sep=large]
			Y \ar[d, "p"']\ar[r, "p''"] & W \\
			V \ar[ur, "\sim"']&
		\end{tikzcd}
	\end{center}
	よって加群の同型定理と定理\ref{thm:hwmodule}\ref{thm:hwmodule-e}より,$V, W$は\hyperref[def:highest-weight-module]{最高ウエイト加群}$Y$の\hyperref[def:irr]{既約}な商加群として同型.
\end{proof}
存在について,まず,\textbf{Verma加群}と呼ばれる\hyperref[def:highest-weight-module]{最高ウエイト加群}$M(\lambda)$を2通りの方法で構成する.

まずは誘導加群の方法.Borel部分代数を$\mathfrak{b} = \mathfrak{b}(\Delta)$とすると,$\mathfrak{b}$-加群としての\hyperref[def:highest-weight-module]{最高ウエイト加群}は\hyperref[def:maximal-vector-rep]{極大ベクトル}$v^+$で生成される一次元\hyperref[def:sub-g-module]{部分加群}を含む.よってまずは$\lambda \in \mathfrak{h}^*$を固定し,$v^+$を基底に持つ一次元ベクトル空間$\mathbb{K}_\lambda$に対し,$\mathbb{K}_\lambda$上の$\mathfrak{b}$の作用を
\begin{align}
	h \btr v^+ &= \lambda(h) v^+  &&(h \in \mathfrak{h}) \\
	x_\alpha \btr v^+ &= 0  &&(x_\alpha \in \mathfrak{g}_\alpha,\ \alpha \in \Phi^+)
\end{align}
と定義する.$\mathbb{K}_\lambda$は$\mathfrak{b}$-加群となる.それと同時に$\mathbb{K}_\lambda$は$\mathfrak{U}(\mathfrak{b})$-加群でもあるので,テンソル積
\begin{equation}
	M(\lambda) = \mathfrak{U}(\mathfrak{g}) \otimes_{\mathfrak{U}(\mathfrak{b})} \mathbb{K}_\lambda
\end{equation}
を定義すると,自然な左作用
\begin{equation}
	x \btr (y \otimes v) = (xy) \otimes v,\quad  (x, y \in \mathfrak{U}(\mathfrak{g}),\ v \in \mathbb{K}_\lambda)
\end{equation}
により$\mathfrak{U}(\mathfrak{g})$-加群となる.\\
次に$M(\lambda)$が\hyperref[def:weight-rep]{ウエイト}$\lambda$の\hyperref[def:highest-weight-module]{最高ウエイト加群}であることを示す.$1 \otimes v^+$は$M(\lambda)$を生成する.また,\hyperref[thm:PBW]{PBW定理}の系\ref{col:PBW-D}より,$\mathfrak{U}(\mathfrak{g}) \simeq \mathfrak{U}(\mathfrak{n}^-)\otimes\mathfrak{U}(\mathfrak{b})$は自由$\mathfrak{U}(\mathfrak{b})$-加群であったから,$1 \otimes v^+$は非零である.よって,$1 \otimes v^+$は\hyperref[def:weight-rep]{ウエイト}$\lambda$の\hyperref[def:maximal-vector-rep]{極大ベクトル}である.よって,$M(\lambda)$を$\mathfrak{U}(\mathfrak{n}^-)$-加群と見ることができ,
\begin{equation}
	M(\lambda) \simeq \mathfrak{U}(\mathfrak{n}^-) \otimes \mathbb{K} \simeq  \mathfrak{U}(\mathfrak{n}^-)
\end{equation}
となる.特に右側の同型に対応して,$1 \otimes v^+ \simeq a \in \mathbb{K}\setminus\{0\} \subset \mathfrak{U}(\mathfrak{n}^-)$.

次に生成系と関係式の方法で構成し,同型であることを示す.\\
\hyperref[def:base-root]{正ルート}の集合$\Phi^+$および,$h_\alpha - \lambda(h_\alpha)1\ (\alpha \in \Phi)$で生成される左イデアルを$I(\lambda)$とする.
\begin{equation}
	I(\lambda) \btr v^+ = 0
\end{equation}
より,$\mathfrak{U}(\mathfrak{g})$-加群の\hyperref[g-module-hom]{準同型}定理から,$\mathfrak{U}(\mathfrak{g}) / I(\lambda) \to M(\lambda)$は$1 + I(\lambda) \mapsto 1\otimes v^+$となる.再び,\hyperref[thm:PBW]{PBW定理}の系\ref{col:PBW-D}より,$\mathfrak{U}(\mathfrak{b}) + I(\lambda) \mapsto \mathbb{K}(1\otimes v^+)$となる.よって,この標準的射影は一対一対応であり,
\begin{equation}
	U(\mathfrak{g}) / I(\lambda) \simeq \mathfrak{U}(\mathfrak{n}^-) \otimes \mathbb{K} \simeq M(\lambda)
\end{equation}

\begin{mytheo}[label=thm:hwmodule-exist]{既約最高ウエイト加群の存在}
	$\forall \lambda \in \mathfrak{h}^*$に対し,\hyperref[def:weight-rep]{ウエイト}$\lambda$の\hyperref[def:irr]{既約}な\hyperref[def:highest-weight-module]{最高ウエイト加群}$L(\lambda)$が存在する.
\end{mytheo}
\begin{proof}
	上で構成された$M(\lambda)$は,\hyperref[def:weight-rep]{ウエイト}$\lambda$の\hyperref[def:highest-weight-module]{最高ウエイト加群}で,唯一の極大\hyperref[def:sub-g-module]{部分加群}$Y(\lambda)$をもつ(定理\ref{thm:hwmodule}\ref{thm:hwmodule-d}).よって,
	\begin{equation}
		L(\lambda) = M(\lambda) / Y(\lambda)
	\end{equation}
	は\hyperref[def:weight-rep]{ウエイト}$\lambda$の\hyperref[def:irr]{既約}\hyperref[def:highest-weight-module]{最高ウエイト加群}(定理\ref{thm:hwmodule}\ref{thm:hwmodule-e}).
\end{proof}
\hyperref[def:weight-rep]{ウエイト}$\lambda \in \mathfrak{h}^*$の\hyperref[thm:hwmodule-exist]{既約最高ウエイト加群}$L(\lambda)$は一意的に定まるから,$L(\lambda)$の\hyperref[def:weight-rep]{ウエイトの集合}を$\Pi(\lambda) \coloneqq \Pi(V)$と書く.
\begin{mytheo}[label=thm:finite-irr-mod]{有限次元既約加群の構造}
	$V$を有限次元\hyperref[def:irr]{既約$\mathfrak{g}$-加群}とすると,
	\begin{equation}
		\exists \lambda \in \mathfrak{h}^*,\quad  V \simeq L(\lambda)
	\end{equation}
\end{mytheo}
\begin{proof}
	有限次元には,\hyperref[def:weight-rep]{ウエイト}$\lambda$の\hyperref[def:maximal-vector-rep]{極大ベクトル}$v^+$の存在が保証されていた.$L(\lambda) = \mathfrak{U}(\mathfrak{g}) \btr v^+$は\hyperref[def:irr]{既約$\mathfrak{g}$-加群}$V$の\hyperref[def:sub-g-module]{部分加群}で,非零なので$V \simeq L(\lambda)$.
\end{proof}
\begin{mydef}[label=def:fundamental-rep]{基本表現}
	\hyperref[def:semisimple-LieAlg]{半単純Lie代数}$\mathfrak{g}$に対し,\hyperref[def:fundamental-weight]{基本ウエイト}$\lambda_i$に対する表現$\rho \colon \mathfrak{g} \lto L(\lambda_i)$を$\mathfrak{g}$の\textbf{基本表現}(fundamental representation)と呼ぶ.
\end{mydef}

\section{有限次元加群}
\hyperref[thm:hwmodule-exist]{既約(最高ウエイト)$\mathfrak{g}$-加群}$L(\lambda)$が有限次元であるための\hyperref[def:weight-rep]{ウエイト}$\lambda$の条件を調べる.
\subsection{有限次元である必要条件}
各\hyperref[def:base-root]{単純ルート}$\alpha_i$に対し,
	\begin{equation}
		\mathfrak{s}_i \coloneqq \mathfrak{g}_{\alpha_i} \oplus \mathfrak{g}_{-\alpha_i} \oplus [\mathfrak{g}_{\alpha_i}, \mathfrak{g}_{-\alpha_i}] \simeq \Lsl(2,\, \mathbb{K})
	\end{equation}
	とする($\simeq$は定理\ref{prop:root-decomp-ortho}).$L(\lambda)$は有限次元$\mathfrak{s}_i$-加群でもあり,$\mathfrak{g}$の\hyperref[def:maximal-vector-rep]{極大ベクトル}$v^+$は$\mathfrak{s}_i$の\hyperref[def:maximal-vector-rep]{極大ベクトル}でもある.特に,$\mathfrak{s}_i$の極大\hyperref[def:toral-subLieAlg]{トーラス}$\mathfrak{h}_i = [\mathfrak{g}_{\alpha_i}, \mathfrak{g}_{-\alpha_i}]$に対し,$h_i \in \mathfrak{h}_i$の作用は,固有値$\lambda(h_i)$で完全に決まる.その固有値が非負整数になることは既に知っている.
\begin{mytheo}[label=thm:necessary-for-finite]{ウエイトの整性の必要性}
	\hyperref[def:highest-weight-module]{最高ウエイト}$\lambda$の\hyperref[thm:hwmodule-exist]{有限次元既約$\mathfrak{g}$-加群}を$L(\lambda)$,\hyperref[def:base-root]{単純ルート}を$\alpha_i$,$h_i \in \mathfrak{h}_{\alpha_i} = [\mathfrak{g}_{\alpha_i}, -\mathfrak{g}_{\alpha_i}]$とすると,$\lambda(\mathfrak{h}_i)$は非負整数.
\end{mytheo}
\begin{proof}
	定理\ref{thm:irr-sl2}.
\end{proof}
この定理の系\ref{col:sl2}より,\hyperref[def:highest-weight-module]{最高ウエイト}でない任意の$V$の\hyperref[def:weight-rep]{ウエイト}$\mu$でも成り立つ:
\begin{equation}
	\label{eq:int-weight}
	\mu(h_i) = \sspair{\mu}{\alpha_i} \in \mathbb{Z},\quad  (1 \leq i \leq l)
\end{equation}
これは,\hyperref[def:root-lattice]{ルート系の整ウエイト}に対応している.これをLie代数の\textbf{整ウエイト}と呼ぶのは自然だろう.\hyperref[def:domweight]{優,強い優ウエイト}\footnote{ルート系で定義した優ウエイトについては,優かつ整と呼ぶ方が親切かもしれないが,ここでは単に優と呼ぶことにする.},\hyperref[def:fundamental-weight]{基本優ウエイト}も同様に定義され,当然,ルート系の整ウエイトに対する定理は全て成り立つ.\\

\subsection{有限次元である十分条件}
\begin{mytheo}[label=thm:suff-for-finite]{ウエイトの優性の十分性}
	$\lambda \in \mathfrak{h}^*$を優ウエイトとする.このとき,\\
	\hyperref[thm:hwmodule-exist]{既約$\mathfrak{g}$-加群}$V = L(\lambda)$は有限次元.また,$V$の\hyperref[def:weight-rep]{ウエイトの集合}$\Pi(\lambda)$は,\hyperref[def:Weylgroup]{Weyl群}$\mathscr{W}$の作用によって置換され,$\dim V_\mu = \dim V_{\sigma\mu},\ \forall \sigma \in \mathscr{W}$を満たす.
\end{mytheo}
\begin{mycol}[label=col:suff-for-finite]{}
	$\lambda \mapsto L(\lambda)$は,優ウエイト$\Lambda^*$と有限次元既約$\mathfrak{g}$-加群(の同型の類)の一対一対応を誘導する.
\end{mycol}
\begin{proof}
	優ウエイトは整ウエイトなので,有限次元であるための\hyperref[thm:necessary-for-finite]{必要条件}を満たす.定理\ref{thm:finite-irr-mod}より一対一対応.
\end{proof}
十分条件を証明しよう.
\begin{mylem}[label=lem:suff-for-finite]{}
	\hyperref[def:semisimple-LieAlg]{半単純Lie代数}$\mathfrak{g}$の\hyperref[def:univ-env-alg]{普遍包絡代数}を$\mathfrak{U}(\mathfrak{g})$,\hyperref[def:base-root]{単純ルート}を$\alpha_1, \ldots, \alpha_l$,$x_i \in \mathfrak{g}_{\alpha_i} \setminus \{0\},\ y_\alpha \in \mathfrak{g}_{-\alpha_i} \setminus \{0\}, h_i = [x_i, y_i] \in \mathfrak{h}$とすると,任意の$k \in \mathbb{Z}_{\geq 0},\ 1 \leq i, j \leq l$に対し以下が成り立つ.
	\begin{enumerate}
		\item $[h_j,\ y_i^{k+1}] = -(k+1)\alpha_i(h_j)y_i^{k+1}$
		\item $[x_j,\ y_i^{k+1}] = -(k+1)y_i^k(k - h_i)\delta_{ij}$
	\end{enumerate}
	ただし,$\delta_{ij}$はKroneckerのデルタ.
\end{mylem}
\begin{proof}
	\begin{description}
		\item[\textbf{(a)}] $k$についての数学的帰納法.$k=0$は\hyperref[prop:semisimple-Lie-alg-relation]{半単純Lie代数の関係式}より明らか.ある$k$で成り立つとき,
		\begin{equation}
			[h_j, y_i^{k+1}] = [h_j, y_i]y_i^k + y_i[h_j, y_i^k] = -\alpha_i(h_j)y_i^{k+2} - (k+1)\alpha_i(h_j)y_i^{k+2} = - (k+2)\alpha_i(h_j)y_i^{k+2}
		\end{equation}
		より任意の$k$で成り立つ.
		\item[\textbf{(b)}] $i \neq j$の場合は,補題\ref{lem:base}より$\alpha_j - \alpha_i$がルートでないことから従う.\\
		$i = j$のとき,$k = 0$は$y_i, h_i$の選び方より明らか.ある$k$で成り立つとき,
		\begin{equation}
			[x_i, y_i^{k+1}] = [x_i, y_i]y_i^k + y_i^k[x_i, y_i] = h_iy_i^k - (k+1)y_i^{k+1}(k - h_i) = -(k+2)y_i^{k+1}(k - h_i)
		\end{equation}
		より任意の$k$で成り立つ.
	\end{description}
\end{proof}
\hyperref[thm:suff-for-finite]{十分条件の証明}のポイントは,$V$の\hyperref[def:weight-rep]{ウエイト}が$\mathscr{W}$で置換されることから$\Pi(\lambda)$が有限個であることを示すことである(\hyperref[thm:Serre]{Serreの定理}の証明同様).
\begin{proof}
	$g$-加群$V$を表現と見たい場合は$\phi$とする.$V$の\hyperref[def:weight-rep]{ウエイト}$\lambda$の\hyperref[def:maximal-vector-rep]{極大ベクトル}を$v^+$とし,$m_i = \lambda(h_i)$とする(仮定より$\lambda$は優ウエイトなので非負整数).
	\begin{description}
		\item[\textbf{Step 1: $w_i \coloneqq (y_i^{m_i + 1} \btr v^+) = 0$}] 
		
		$x_i \btr v^+ = 0$と補題\ref{lem:suff-for-finite}(1)に注意すると,
		\begin{equation}
			x_i \btr w_i = y_i^{m_i + 1} \btr (x_j \btr v^+) - (m_i + 1)y_i^{m_i} \btr (m_i - m_i)v^+ = 0
		\end{equation}
		となる.また,補題\ref{lem:base}より$\alpha_j - \alpha_i$はルートでないから$x_j \btr w_i = 0\ (1 \leq j \leq l)$.もし$w \neq 0$とすると,\hyperref[def:highest-weight-module]{最高ウエイト}$\lambda - (m_i + 1)\alpha_i \neq \lambda$が存在することになり.\hyperref[thm:hwmodule-exist]{既約最高ウエイト加群}の\hyperref[col:highest-weight-uni]{最高ウエイトの唯一性}に反する.
		\item[\textbf{Step 2: $V$に非零な有限次元$\mathfrak{s}_i$-加群が含まれる}]  
		
		部分空間$V_i = \{v^+,\ y_i \btr v^+,\ \ldots,\ y_i^{m_i}\btr v^+ \}$を考える.\textbf{(Step 1)}と合わせると$y_i$の作用について不変.補題\ref{lem:suff-for-finite}(1)より$x_i, h_i$の作用についても不変.よって,$\mathfrak{s}_i$-加群として,$V_i$は$V$の\hyperref[def:sub-g-module]{部分加群}.
		\item[\textbf{Step 3: $V$は有限次元部分$\mathfrak{s}_i$-加群の和}]  
		
		$V' = \sum_{i = 1}^l V_i$とする.ここで,$W$を$V$の任意の有限次元$\mathfrak{s}_i$-\hyperref[def:sub-g-module]{部分加群}とすると,$x_i \btr W,\ y_i \btr W$も有限次元$\mathfrak{s}_i$-\hyperref[def:sub-g-module]{部分加群}.命題\ref{root-decomp-int}より,任意のルート$\alpha \in \Phi$,$x_\alpha = \mathfrak{g}_\alpha \setminus \{0\}$に対し,$x_\alpha \btr W$は有限次元$\mathfrak{s}_i$-\hyperref[def:sub-g-module]{部分加群}となり,$W$の部分空間.よって,$V'$は\hyperref[ax:g-module]{$\mathfrak{g}$-加群}.\\
		$V$は\hyperref[def:irr]{既約}で,\textbf{(Step 2)}より$V' \subset V$は非零だったから,$V = V'$.
		\item[\textbf{Step 4: $1 \leq i \leq l$に対し,$\phi(x_i),\ \phi(y_i)$は$V$の局所冪零自己準同型}] 
		
		補題\ref{lem:suff-for-finite},\textbf{(Step 1)}より,$x_i, y_i$の作用は\hyperref[def:locally-nilpotent]{局所冪零}.\hyperref[col:JC]{Jordan分解の保存}より,$\phi(x_i),\ \phi(y_i)$も局所冪零.
		\item[\textbf{Step 5: (Step 4)より$s_i = \exp\phi(x_i)\exp\phi(-y_i)\exp\phi(x_i)$はwell-defined}] \hyperref[def:locally-nilpotent]{局所冪零}の定義の後の文章参照のこと
		\item[\textbf{Step 6: $\mu$を$V$のウエイトとすると,$s_i(V_\mu) = V_{\sigma_i\mu}\ (\sigma_i: \alpha_i\text{に関する鏡映})$}] 
		
		\textbf{(Step 3)}より$V_\mu$は有限次元$\mathfrak{s}_i$-\hyperref[def:sub-g-module]{部分加群}$V' = V$に含まれる.$s_i|_{V'}$は存在しない$\Lsl(2, \mathbb{K})$の\hyperref[def:irr]{既約表現}の分類の自己同型$\tau$で,$s_i(V_\mu) = V_{\sigma_i\mu}$が従う.
		\item[\textbf{Step 7: ウエイトの集合$\Pi(\lambda)$は$\mathscr{W}$の作用で不変.また,$\dim V_\mu = \dim V_{\sigma\mu}\ (\forall \mu \in \Pi(\lambda),\ \forall \sigma \in \mathscr{W})$}] 
		
		\hyperref[def:Weylgroup]{Weyl群}$\mathscr{W}$は\hyperref[def:base-root]{単純ルート}に関する鏡映で生成されたから,\textbf{(Step 6)}より従う.
		\item[\textbf{Step 8: $\Pi(\lambda)$は有限集合.}] 
			
		補題\ref{lem:dom-weight-B}より,$\mu \prec \lambda$を満たす優ウエイト$\mu$の集合は有限なので,$\mathscr{W}$で写した集合も有限.定理\ref{thm:hwmodule}\ref{thm:hwmodule-b}より,$\Pi(\lambda)$はこの部分集合だから,有限集合.
		\item[\textbf{Step 9: $V$は有限次元.}] 
		
		定理\ref{thm:hwmodule}\ref{thm:hwmodule-c}より$V_\mu\ (\mu \in \Pi(\lambda))$は有限次元.$\mu \in \Pi(\lambda)$上の和は有限和なので,定理\ref{thm:hwmodule}\ref{thm:hwmodule-a}より$V$は有限次元.
	\end{description}
\end{proof}

\subsection{ウエイトstringとウエイト図}
ここでも優ウエイト$\lambda \in \Lambda^+$に対する\hyperref[thm:hwmodule-exist]{有限次元既約$\mathfrak{g}$-加群}$V = L(\lambda)$を考える.\\
補題\ref{lem:weight-rep}より,\hyperref[def:sub-g-module]{部分加群}$W = \oplus_{i \in \mathbb{Z}} V_{\mu + i\alpha} \subset V$は$S_\alpha$-不変.\\
$(\mu + \alpha\mathbb{Z}) \cap \Pi(\lambda)$なる整ウエイトの集合を\textbf{$\bm{\alpha}$-string through $\bm{\mu}$}という.$\Pi(\lambda)$は有限集合であったから,
\begin{align}
	r &\coloneqq \max \bigl\{\, i \in \mathbb{Z}_{\ge 0} \bigm| \mu - i \alpha \in \Pi(\lambda) \,\bigr\}, \\
	q &\coloneqq \max \bigl\{\, i \in \mathbb{Z}_{\ge 0} \bigm| \mu + i \alpha \in \Pi(\lambda) \,\bigr\} 
\end{align}
が定義されるので,命題\ref{prop:a-string-basic}と同様のものが成り立つ.
\begin{myprop}[label=prop:weight-diagram]{$\alpha$-string through $\mu$の性質}
	$\alpha \neq \pm \mu$を充たす任意の$\alpha \in \Phi,\ \mu \in \Pi(\lambda)$に対して
	\begin{align}
		r &\coloneqq \max \bigl\{\, i \in \mathbb{Z}_{\ge 0} \bigm| \mu - i \alpha \in \Pi(\lambda) \,\bigr\}, \\
		q &\coloneqq \max \bigl\{\, i \in \mathbb{Z}_{\ge 0} \bigm| \mu + i \alpha \in \Pi(\lambda) \,\bigr\} 
	\end{align}
	とおく.このとき以下が成り立つ:
	\begin{enumerate}
		\item 
		$\alpha$-string through $\mu$は $\mathbb{E}$ の部分集合
		\begin{align}
			\label{eq:a-string}
			\bigl\{\, \mu + i\alpha \in \mathbb{E} \bigm| -r \le \lambda \le q \,\bigr\} 
		\end{align}
		に等しい.i.e. 
		\begin{align}
			i \in \mathbb{Z} \AND \mu + i\alpha \in \Pi(\lambda) \IMP -r \le i \le q
		\end{align}
		である.
		\item $\sigma_{\alpha}(\mu + i\alpha) = \mu - i\alpha$.i.e. $\alpha$-string through $\mu$は鏡映 $\sigma_\alpha$ の作用の下で不変である.
		\item $r-q = \sspair{\mu}{\alpha}$.特に$\alpha$-string through $\mu$の長さは $4$ 以下である.
		\item 優ウエイト$\lambda$に対し,$\Pi(\lambda)$は\hyperref[def:weight-saturated]{飽和集合}.
		\item 任意の\hyperref[def:weight-rep]{ウエイト}$\mu$について
		\begin{equation}
			\mu \in \Pi(\lambda)  \IFF  \sigma(\mu) \prec \lambda\quad  (\forall \sigma \in \mathscr{W})
		\end{equation}
	\end{enumerate}
\end{myprop}
\begin{proof}
	(1)-(3): 命題\ref{prop:a-string-basic}の証明の$\Phi$を$\Pi(\lambda)$に,$\beta$を$\mu$に直したもの.\\
	(4): 以上から$\Pi(\lambda)$は\hyperref[def:weight-saturated]{飽和集合}の定義そのものを満たすと言える.\\
	(5): 補題\ref{lem:dom-weight-A}と補題\ref{lem:weight-saturated-B}を組み合わせると成り立つ.
\end{proof}
ルートと$\Pi(\lambda)$の元を書いた図を\textbf{ウエイト図}(weight diagram)と呼ぶ.あとで定義する\hyperref[def:mutiplicity]{重複度}も書き込みたいので,具体例は次節に回す.

\subsection{$L(\lambda)$の生成系と関係式}
$\mathfrak{U}(\mathfrak{g}) \to M(\lambda) \to L(\lambda)$という\hyperref[g-module-hom]{準同型}を$\lambda$が優ウエイトの場合にさらに考察する.\\
$M(\lambda) = \mathfrak{U}(\mathfrak{g}) / I(\lambda)$であり,$I(\lambda)$は$\mathfrak{U}(\mathfrak{g})$の左イデアルで,$M(\lambda)$の\hyperref[def:maximal-vector-rep]{極大ベクトル}$v^+$に対し,$I(\lambda) \btr v^+ = 0$となるものであった.\\
同様に$\mathfrak{U}(\mathfrak{g})$の左イデアルで,$L(\lambda)$の\hyperref[def:maximal-vector-rep]{極大ベクトル}に作用すると$0$となるようなものを$J(\lambda)$とする.$I(\lambda) \subset J(\lambda)$より,\hyperref[g-module-hom]{準同型}
\begin{equation}
	M(\lambda) = \mathfrak{U}(\mathfrak{g}) / I(\lambda) \lto L(\lambda) \simeq \mathfrak{U}(\mathfrak{g}) / J(\lambda)
\end{equation}
を誘導する.また,定理\ref{thm:suff-for-finite}の証明\textbf{(Step 1)}より,$y_i^{m_i + 1} \in J(\lambda)$である.
\begin{mylem}[label=lem:Lie-bracket-power]{}
	$A$を標数$0$の体$\mathbb{K}$上の結合代数,$\ad y(z) = yz - zy\ (y, z \in A)$とすると,\\
	$\forall y, z \in A,\ \forall k \in \mathbb{N}$に対し,以下が成り立つ.
	\begin{equation}
		(\ad y^k)(z) = \sum_{i=1}^k \binom{k}{i}(\ad y)^i(z)y^{k-i}
	\end{equation}
\end{mylem}
\begin{proof}
	$(\ad y^1)(z) = (\ad y)^1(z)$.ある$k$で成り立つとき,以下より$k+1$でも成り立つ.
	\begin{align}
		(\ad y^{k+1})(z) &= \bigl(\ad y(z)y^k + \ad y^k(\ad y(z))\bigr) + \ad y^k(z)y = \sum_{i=1}^{k+1} \Biggl(\binom{k}{i-1} + \binom{k}{i}\Biggr)(\ad y)^i(z)y^{k+1-i} \\
		&= \sum_{i=1}^{k+1} \binom{k+1}{i}(\ad y)^i(z)y^{k+1-i}
	\end{align}
\end{proof}
\begin{mytheo}[label=thm:generator-of-J]{$J(\lambda)$の生成系}
	$\lambda \in \Lambda^+,\ m_i = \sspair{\lambda}{\alpha_i}\ (1 \leq i \leq l)$とすると,$J(\lambda)$は$I(\lambda)$と$y_i^{m_i + 1}\  (1 \leq i \leq l)$で生成される.
\end{mytheo}
\begin{proof}
	$I(\lambda)$と$y_i^{m_i + 1}\  (1 \leq i \leq l)$で生成されるイデアルを$J'(\lambda)$とする.まず,$L'(\lambda) = \mathfrak{U}(\mathfrak{g}) / J'(\lambda)$が有限次元であることを示す.\\
	$I(\lambda) \subset J'(\lambda)$より,$L'(\lambda)$は$M(\lambda)$の\hyperref[def:maximal-vector-rep]{極大ベクトル}に関する\hyperref[def:highest-weight-module]{最高ウエイト加群}.定理\ref{thm:suff-for-finite}の証明\textbf{(Step 3)}「有限次元$\mathfrak{s}_i$-\hyperref[def:sub-g-module]{部分加群}の和」を示せば,\textbf{(Step 4)}以降はそのまま使えて$L'(\lambda)$は有限次元となる.その\textbf{(Step 3)}の証明の中でも,$x_i,\ y_i$が$L'(\lambda)$上\hyperref[def:locally-nilpotent]{局所冪零}であることを示せば,有限次元となる.\\
	$x_i^k$は\hyperref[def:weight-rep]{ウエイト}を$k\alpha_i$増やすので,ある$k$で\hyperref[def:highest-weight-module]{最高ウエイト}を越えるため\hyperref[def:locally-nilpotent]{局所冪零}.$y_i^k$については,まず定理\ref{thm:hwmodule}\ref{thm:hwmodule-a}より,
	\begin{equation}
		V'(\lambda) = \Span\bigl\{y_{i_1}\cdots y_{i_t} \bigm| 0 \leq i_j \leq l,\ t \in \mathbb{Z}_{\geq 0} \bigr\}
	\end{equation}
	である.ここで,補題\ref{lem:Lie-bracket-power}を$A = \mathfrak{U}(\mathfrak{g})$に対して考えると$\alpha$-stringの長さは高々4だから,和は$i=3$で止まる.つまり,$[y^{k+3}, z] = (\text{多項式})\times y_i^k$.よって,
	\begin{equation}
		y_i^ky_{i_1}\cdots y_{i_t} \in J'(\lambda) \IMP  y_i^{k+3}y_{i_0}y_{i_1}\cdots y_{i_t} \in J'(\lambda)
	\end{equation}
	となる.$J'(\lambda)$の定義より$y_i^{m_i + 1} \in J'(\lambda)$だから,単項式の長さ$t$に関する帰納法より,$y_i$は局所冪零.
	
	よって,$L'(\lambda)$は有限次元\hyperref[def:highest-weight-module]{最高ウエイト加群}(または$0$)なので,定理\ref{thm:hwmodule}\ref{thm:hwmodule-e}より\hyperref[def:irr]{直既約}で,\hyperref[thm:Weyl]{完全可約性に関するWeylの定理}より\hyperref[def:irr]{既約}(または$0$).一方定義より$J'(\lambda) \subset J(\lambda)$なので,$L(\lambda) \subset L'(\lambda)$(より$0$でない).同型でないと,$L'(\lambda)$の\hyperref[def:irr]{既約}性に矛盾するから,
	\begin{equation}
		L(\lambda) \simeq L'(\lambda)  \IFF  J(\lambda) \simeq J'(\lambda)
	\end{equation}
\end{proof}

\section{重複度公式}
この節では有限次元加群を考える.
\begin{mydef}[label=def:mutiplicity]{ウエイトの重複度}
	$\mu \in \mathfrak{h}^*$を整ウエイト,$\lambda \in \Lambda^+$を優ウエイトとする.\hyperref[thm:hwmodule-exist]{有限次元既約$\mathfrak{g}$-加群}$L(\lambda)$に対し,\hyperref[def:weight-rep]{ウエイト}$\mu$の\hyperref[def:weight-rep]{ウエイト空間}の次元
	\begin{equation}
		m_\lambda(\mu) \coloneqq \dim L(\lambda)_\mu
	\end{equation}
	のことを$L(\lambda)$における$\mu$の\textbf{重複度}(multiplicity)と呼ぶ($L(\lambda)$のウエイトでなければ$0$とする).
\end{mydef}
目標は,\hyperref[def:weight-rep]{ウエイト}の\hyperref[def:mutiplicity]{重複度}$m_\lambda(\mu)$に対する再帰的公式である,Freudenthalの公式を示すこと.

\subsection{普遍Casimir演算子}
\hyperref[thm:Weyl]{完全可約性に関するWeylの定理}の証明で出てきた,\hyperref[def:semisimple-LieAlg]{半単純Lie代数}$\mathfrak{g}$の表現$\phi$に対して定義された\hyperref[def:Casimir]{Casimir演算子}$c_\phi = c_\phi(\beta)$を思い出そう.これを\hyperref[def:univ-env-alg]{普遍包絡代数}$\mathfrak{U}(\mathfrak{g})$の表現$\phi$に対する演算子と見ると,
\begin{align}
	c_\phi &\coloneqq \sum_{\mu=1}^{\dim\mathfrak{g}} \phi(e_\mu)\phi(e^\mu) = \sum_{\mu=1}^{\dim\mathfrak{g}} \phi(e_\mu e^\mu) \\
	\label{eq:univ-Casimir-beta}
	&= \phi(c_\mathfrak{g}(\beta))\quad  \biggl(c_\mathfrak{g}(\beta) \coloneqq \sum_{\mu=1}^{\dim\mathfrak{g}} e_\mu e^\mu \biggr)
\end{align}
と書ける.$c_\mathfrak{g}(\beta)$は陽に$\phi$に依存していない\footnote{$\beta(x, y) = \Tr(\phi(x)\phi(y))$としていたので陰に依存している}.そこで$\phi$として$\mathfrak{g}$の随伴表現を考える.その跡形式を\hyperref[def:Killing-form]{Killing形式}$\kappa$と呼んでいた.\hyperref[thm:semisimple-LieAlg-iff]{半単純Lie代数のKilling形式は非退化}であったから,$\kappa$に関する双対基底(ここでも補題\ref{lem:Casimir}のように,$\mathfrak{g}^*$の基底ではなく$\mathfrak{g}$の基底)を構成できる.\\
命題\ref{prop:root-decomp-basic1}より,$\alpha,\ \beta \in \mathfrak{h}^*$に対し,$\alpha \neq -\beta$であれば$g_\alpha$と$g_\beta$は直交する.よって,$\mathfrak{h} = \mathfrak{g}_0$及び$\mathfrak{g}_\alpha \oplus \mathfrak{g}_{-\alpha}$それぞれで\hyperref[def:Killing-form]{Killing形式}を考えれば良い.\\
まず,$\{x_\alpha \in \mathfrak{g}\ (\forall \alpha \in \Phi)\}$と\hyperref[def:base-root]{単純ルート}$\alpha_i$に対する$\{h_i = [x_{\alpha_i}, x_{-\alpha_i}]\ (1 \leq i \leq l)\}$で$\mathfrak{g}$の基底を張る.\\
$\kappa$に関する\hyperref[lem:Casimir]{双対}をそれぞれ$x^\alpha,\ h^i$とすると,命題\ref{prop:root-decomp-ortho}より$x^\alpha \in \mathfrak{g}_{-\alpha}$で,
\begin{equation}
	[x_\alpha, x^\alpha] = t_\alpha = \frac{(\alpha, \alpha)}{2}h_\alpha
\end{equation}
となる.このときの\eqref{eq:univ-Casimir-beta}で定義した$c_\mathfrak{g}(\beta) = c_\mathfrak{g}(\kappa)$に注目する.
\begin{mydef}[label=def:univ-Casimir-op]{普遍Casimir演算子}
	\hyperref[def:semisimple-LieAlg]{半単純Lie代数}$\mathfrak{g}$の\hyperref[def:univ-env-alg]{普遍包絡代数}を$\mathfrak{U}(\mathfrak{g})$,極大\hyperref[def:toral-subLieAlg]{トーラス}$\mathfrak{h}$の基底を$\{h_i\}_{i=1}^l$,$x_\alpha \in \mathfrak{g}_\alpha \setminus \{0\}$,$h_i, x_\alpha$を\hyperref[def:Killing-form]{Killing形式}$\kappa$に関して\hyperref[lem:Casimir]{双対}をとったものをそれぞれ$h^i, x^\alpha$とする.
	\begin{equation}
		c_\mathfrak{g} \coloneqq \sum_{i=1}^l h_ih^i + \sum_{\alpha \in \Phi} x_\alpha x^\alpha \in \mathfrak{U}(\mathfrak{g})
	\end{equation}
	を(\hyperref[def:Killing-form]{Killing形式}に関する)\textbf{普遍Casimir演算子}(universal Casimir operator)と呼ぶ.
\end{mydef}
縮約をとっているので,$c_\mathfrak{g}$は基底によらない.命題\ref{prop:Casimir-basic}より$c_\phi(\kappa) = \phi(c_\mathfrak{g})$と$\phi(\mathfrak{g})$は可換なので,$\phi$が\hyperref[def:irr]{既約}なら$\phi(c_\mathfrak{g})$はスカラーとして作用する.\\
$c_\phi$と$\phi(c_\mathfrak{g})$の関係を調べる.そのために$\phi$の跡形式$\beta(x, y) = \Tr(\phi(x)\phi(y))$と$\kappa$の関係を調べる.
\begin{mylem}[label=lem:trace-form-simple]{単純Lie代数の非退化対称結合双線型形式の同型性}
	$\mathfrak{g}$を\hyperref[def:simple-LieAlg]{単純Lie代数}とし,$f(x, y),\ g(x, y)$を$\mathfrak{g}$上の非退化で対称な双線型形式とし,$\beta = f, g$について
	\begin{equation}
		\label{eq:associative-bilinear}
		\beta(x, [y, z]) = \beta([x, y], z)
	\end{equation}
	を満たすとすると,
	\begin{equation}
		\exists a \in \mathbb{K},\quad  f = ag
	\end{equation}
\end{mylem}
\begin{proof}
	各非退化双線型形式$f, g$で定まる同型$\pi_f, \pi_g\colon \mathfrak{g} \lto \mathfrak{g}^*,\ x \mapsto s_f, s_g$を
	\begin{equation}
		s_f(y) = f(x, y),\quad  s_g(y) = g(x, y)
	\end{equation}
	で定義する.$\mathfrak{g}^*$は$\Lgl(\mathbb{K})$を返す\hyperref[ax:g-module]{$\mathfrak{g}$-加群}と見れる.$\mathbb{K}$は1次元線型空間なので,\eqref{eq:associative-bilinear}より,$\{s_f\}, \{s_g\}$は\hyperref[ax:g-module]{$\mathfrak{g}$-加群}として同型.\\
	よって,\hyperref[ax:g-module]{$\mathfrak{g}$-加群}$\pi_g^{-1}\circ\pi_f\colon \mathfrak{g} \to \mathfrak{g}$は$\mathfrak{g}$上の同型写像.$\mathfrak{g}$は\hyperref[def:simple-LieAlg]{単純}なので,\hyperref[def:irr]{既約$\mathfrak{g}$-加群}となり,\hyperref[col:Schur-closed]{代数閉体上のSchurの補題}より$\pi_f^{-1}\pi_g$はスカラー倍.i.e.
	\begin{equation}
		\forall x \in \mathfrak{g},\ \exists a \in \mathbb{K},\quad  \pi_g^{-1}\circ\pi_f(x) = ax  \IFF  \forall y \mathfrak{g},\ f(x, y) = g(ax, y)
	\end{equation}
	となり,$g$の双線型性より$f = ag$.
\end{proof}
\hyperref[ax:LieAlg]{Lie代数}$\mathfrak{g}$の非零の\hyperref[def:faithful]{忠実}な表現を$\phi$とする.$\Ker\phi \subsetneqq \mathfrak{g}$は$\mathfrak{g}$のイデアルとなる.\\
$\mathfrak{g}$が\hyperref[def:simple-LieAlg]{単純}のとき,$\Ker\phi = \{0\}$だから,$\beta(x, y) \coloneqq \Tr(\phi(x)\phi(y))$は非退化で\eqref{eq:associative-bilinear}を満たす.$\phi = \ad$の場合の\hyperref[def:Killing-form]{Killing形式}$\kappa$も含まれるから,補題\ref{lem:trace-form-simple}より,$\kappa = a\beta\ (a \in \mathbb{K} \setminus \{0\})$と書ける.$\beta$に関する\hyperref[lem:Casimir]{双対}ベクトルは,$\kappa$に関する\hyperref[lem:Casimir]{双対}ベクトルの$1/a$倍となるから,\eqref{eq:univ-Casimir-beta}より,
\begin{equation}
	c_\phi(\kappa) = \phi(c_\mathfrak{g}) = \phi(c_\mathfrak{g}(a\beta)) = \frac{1}{a}\phi(c_\mathfrak{g}(\beta)) = \frac{1}{a}c_\phi
\end{equation}

次に,$\mathfrak{g}$が\hyperref[def:semisimple-LieAlg]{半単純Lie代数}のとき.$\mathfrak{g}$は\hyperref[def:simple-LieAlg]{単純Lie代数}$g_i\ (1 \leq i \leq t)$の\hyperref[thm:semisimple-decomp]{直和}
\begin{equation}
	\mathfrak{g} = \bigoplus_{i=1}^t\mathfrak{g}_i
\end{equation}
で書ける.基底は各$\mathfrak{g}_i$の基底の和集合にとれるので,$\mathfrak{g}$の\hyperref[def:univ-Casimir-op]{普遍Casimir演算子}は,$\mathfrak{g}_i$の\hyperref[def:univ-Casimir-op]{普遍Casimir演算子}$c_{\mathfrak{g}_i}$の和で
\begin{equation}
	c_\mathfrak{g} = \sum_{i=1}^t c_{\mathfrak{g}_i}
\end{equation}
と書ける.定理\ref{thm:semisimple-decomp}より,各$\mathfrak{g}_i$の\hyperref[def:Killing-form]{Killing形式}は$\kappa|_{\mathfrak{g}_i \times \mathfrak{g}_i}$に等しい.よって,
\begin{equation}
	\phi(c_{\mathfrak{g}_i}) = \sum_{i=1}^t \frac{1}{a_i}c_{\phi|_{\mathfrak{g}_i}}\quad  (a_i \in \mathbb{K})
\end{equation}
と書ける.次に,この$a_i$の値の求め方を示す.
\subsection{Freudenthalの公式}
$V$の\hyperref[def:weight-rep]{ウエイト}$\mu \in \Pi(\lambda)$に対する\hyperref[def:weight-rep]{ウエイト空間}$V_\mu$に対し,$\phi(x_\alpha)\phi(x^\alpha)$は$V_\mu \lto V_{\mu - \alpha} \lto V_\mu$より$V_\mu$上の線型変換なので,$\phi(c_\mathbb{g})|_{V_\mu}$は$V_\mu$上の線型変換と見ることができる.ここで,$V_\mu$上の跡を調べる.
\begin{mylem}[label=lem:Freudenthal]{ウエイト空間上の既約表現の跡形式}
	優ウエイト$\lambda \in \Lambda^+$の\hyperref[thm:hwmodule-exist]{既約な$\mathfrak{g}$-加群}を$V = L(\lambda)$(表現と見たものを$\phi$と書く),$V$の\hyperref[def:weight-rep]{ウエイト}を$\mu \in \Lambda$,その\hyperref[def:mutiplicity]{重複度}を$m(\mu)$とすると,
	\begin{align}
		\Tr_{V_\mu} \phi(c_\mathfrak{g}) &= (\mu, \mu)m(\mu) + \sum_{\alpha \in \Phi} \sum_{i=1}^\infty m(\mu + i\alpha)(\mu + i\alpha, \alpha) \\
		&= (\mu, \mu + 2\delta)m(\mu) + 2\sum_{\alpha \succ 0} \sum_{i=1}^\infty m(\mu + i\alpha)(\mu + i\alpha, \alpha)
	\end{align}
	を満たす.
\end{mylem}
\begin{proof}
	$\mu \in \Lambda \setminus \Pi(\lambda)$の場合は$V_\mu = 0$より成立.以下,$\mu \in \Pi(\lambda)$を考える.\\
	上述のように,$\mathfrak{h}$の基底$\{h_i\}_{i=1}^l$,$\mathfrak{g}_\alpha$の基底$\{x_\alpha\}$を固定し,\hyperref[def:Killing-form]{Killing形式}$\kappa$に関する\hyperref[lem:Casimir]{双対基底}を取る.\\
	$\mu \in \Pi(\lambda)$として,$V_\mu$上の$x_\alpha x^\alpha,\ h_ih^i$の作用を直接見て,その和を調べる.
	\begin{description}
		\item[($x_\alpha x^\alpha$)] $\mu + \alpha \notin \Pi(\lambda)$とすると,$\alpha$-string through $\mu$は$\{\mu,\ \mu - \alpha,\ \ldots,\ \mu - r\alpha\}\ (r = \sspair{\mu}{\alpha})$となる.\hyperref[thm:Weyl]{完全可約性に関するWeylの定理}より,ウエイト空間の直和
		\begin{equation}
			W \coloneqq V_\mu \oplus \cdots \oplus V_{\mu - r\alpha}
		\end{equation}
		はある\hyperref[def:irr]{既約}$\mathfrak{s}_\alpha$-加群の\hyperref[def:gmod-directsum]{直和}で表せる.$w_0 \in V_\mu$を$\mathfrak{s}_\alpha$-加群としての\hyperref[def:maximal-vector-rep]{極大ベクトル}とすると,
		\begin{equation}
			w_i \coloneqq \left\{\begin{aligned}
				 &0 &&(i = -1) \\
				 &\frac{(\alpha, \alpha)^i}{2^i}y^i \btr w_0  &&(i \geq 0)
			\end{aligned} \right.
		\end{equation}
		に対し,$x^\alpha = \frac{(\alpha, \alpha)}{2}y_\alpha,\ t_\alpha = \frac{(\alpha, \alpha)}{2}h_\alpha$と補題\ref{lem:sl2-2}より,
		\begin{enumerate}
			\item $t_\alpha \btr w_i = (r - 2i)\frac{(\alpha, \alpha)}{2}w_i$
			\item $x^\alpha \btr w_i = w_{i+1}$
			\item $x_\alpha \btr w_i = i(r - i+1)\frac{(\alpha, \alpha)}{2}w_{i-1}  \WHERE i \ge 0$
		\end{enumerate}
		を満たす.特に,$w_m \in V_{\mu - m\alpha}$.よって,
		\begin{equation}
			\label{eq:Freudenthal-proof-1}
			x_\alpha x^\alpha \btr w_i = (r - i)(i + 1)\frac{(\alpha, \alpha)}{2}w_i
		\end{equation}
		鏡映$\sigma_\alpha \in \mathscr{W}$に対し,$\sigma_\alpha(\mu - i\alpha) = \mu + (r - i)\alpha$と定理\ref{thm:suff-for-finite}より,
		\begin{equation}
			m(\mu - i\alpha) = m(\mu - (r - i)\alpha)\quad  (0 \leq i \leq r)
		\end{equation}
		となる.\hyperref[def:highest-weight-module]{最高ウエイト}$r - 2i = (\mu - i\alpha)(h_\alpha)$をもつベクトルの数を$n_i\ (0 \leq i \leq r/2)$とすると,
		\begin{equation}
			m(\mu - i\alpha) = \sum_{j = 0}^i n_i  \IMP  n_i = m(\mu - i\alpha) - m(\mu - (i - 1)\alpha)
		\end{equation}
		となる.$0 \leq j \leq i \leq r$とする.\hyperref[def:highest-weight-module]{最高ウエイト}$r - 2j$の\hyperref[def:weight-rep]{ウエイト空間}は,\eqref{eq:Freudenthal-proof-1}の$r \mapsto r - 2j,\ i \mapsto j - i$より,
		\begin{equation}
			\phi(x_\alpha)\phi(x^\alpha)w_{i - j} = (r - j - i)(i - j + 1)\frac{(\alpha, \alpha)}{2}w_{i - j}
		\end{equation}
		よって$0 \leq i \leq r/2$に対しては,
		\begin{align}
			\Tr_{V_{\mu - i\alpha}} \phi(x_\alpha)\phi(x^\alpha) &= \sum_{j = 0}^i n_j(r - j - i)(i - j + 1)\frac{(\alpha, \alpha)}{2} \\
			&= \sum_{j = 0}^i (m(\mu - j\alpha) - m(\mu - (j - 1)\alpha))(r - j - i)(i - j + 1)\frac{(\alpha, \alpha)}{2} \\
			&= \sum_{j = 0}^i m(\mu - j\alpha)((r - j - i)(i - j + 1) - (r - j - i - 1)(i - j))\frac{(\alpha, \alpha)}{2} \\
			&= \sum_{j = 0}^i m(\mu - j\alpha)(r - 2j)\frac{(\alpha, \alpha)}{2} \\
			&= \sum_{j = 0}^i m(\mu - j\alpha)((\mu, \alpha) - j(\alpha, \alpha))\quad  \Biggl(r = \sspair{\mu}{\alpha} = 2\frac{(\mu, \alpha)}{(\alpha, \alpha)}\Biggr) \\
			\label{eq:Freudenthal-proof-2}
			&= \sum_{j=0}^i m(\mu - j\alpha)(\mu - j\alpha, \alpha)\quad  \Bigl(0 \leq i \leq \frac{r}{2}\Bigr)
		\end{align}
		となる.$r/2 \leq i \leq r$の場合は和が$r - i$までになるだけ.ただし,$\phi(x^\alpha)w_{i - (r - i)} = 0$だから,
		\begin{align}
			\label{eq:Freudenthal-proof-3}
			\Tr_{V_{\mu - i\alpha}} \phi(x_\alpha)\phi(x^\alpha) = \sum_{j=0}^{r - i - 1} m(\mu - j\alpha)(\mu - j\alpha, \alpha)\quad  \Bigl(\frac{r}{2} < i \leq r\Bigr)
		\end{align}
		となる.ここで,$m(\mu - j\alpha) = m(\mu - (r - j)\alpha)$より,
		\begin{align}
			m(\mu - j\alpha)(\mu - j\alpha, \alpha) + m(\mu - (r - j)\alpha)(\mu - (r - j)\alpha, \alpha) &= m(\mu - j\alpha)(2\mu - r\alpha, \alpha) \\
			\label{eq:Freudenthal-proof-4}
			&= 0
		\end{align}
		なので,\eqref{eq:Freudenthal-proof-2}を$i \geq r/2$で考えた際の$r - i,\ldots, i$の項の部分和は$0$\footnote{$\character\mathbb{K} \neq 2$より,$j = r/2$の項も$0$.}となり,\eqref{eq:Freudenthal-proof-3}に等しい.i.e. \eqref{eq:Freudenthal-proof-2}は任意の$i = 0, \ldots, r$で成り立つ.
		
		任意の$\nu \in \Pi(\lambda)$は,$\nu + j\alpha = \mu$の形にできる.$m(\nu + (j + i)\alpha) = 0\ (i > 0)$より,
		\begin{equation}
			\Tr_{V_\mu} \phi(x_\alpha)\phi(x^\alpha) = \sum_{i=0}^\infty m(\mu + i\alpha)(\mu + i\alpha)\quad  (\mu \in \Pi(\lambda))
		\end{equation}

	\item[\textbf{($h_ih^i$)}] \hyperref[col:torus-centralizer]{$\kappa|_{\mathfrak{h}\times\mathfrak{h}}$は非退化}だったから,
		\begin{equation}
			\exists t_\mu \in \mathfrak{h},\ \forall h \in \mathfrak{h},\quad  \mu(h) = \kappa(t_\mu, h)
		\end{equation}
		を満たす.特に,$t_\mu = \kappa(t_\mu, h^i)h_i = \mu(h^i)h_i$である.よって,
		\begin{align}
			\sum_{i=1}^l\Tr_{V_\mu} \phi(h_i)\phi(h^i) &= m(\mu)\sum_{i=1}^l\mu(h_i)\mu(h^i) = m(\mu)\sum_{i,j=1}^l\mu(h^j)\kappa(h_j, h_i)\mu(h^i) \\
			&= m(\mu)\kappa(t_\mu, t_\mu) = m(\mu)(\mu, \mu)
		\end{align}
	\end{description}
	以上より,
	\begin{equation}
		\Tr_{V_\mu} \phi(c_\mathfrak{g}) = (\mu, \mu)m(\mu) + \sum_{\alpha \in \Phi} \sum_{i=0}^\infty m(\mu + i\alpha)(\mu + i\alpha, \alpha)
	\end{equation}
	ここで,\eqref{eq:Freudenthal-proof-4}の$j \in \mathbb{Z}_{\geq r/2}$の和をとると,
	\begin{equation}
		\sum_{i=-\infty}^\infty m(\mu + i\alpha)(\mu + i\alpha, \alpha) = 0
	\end{equation}
	となるから,$\alpha \prec 0$に対する$i \in \mathbb{Z}_{\geq 0}$の和を$i < 0$で書き直すと,
	\begin{align}
		\Tr_{V_\mu} \phi(c_\mathfrak{g}) &= (\mu, \mu)m(\mu) + \sum_{\alpha \succ 0} m(\mu)(\mu, \alpha) + 2\sum_{\alpha \succ 0} \sum_{i=1}^\infty m(\mu + i\alpha)(\mu + i\alpha, \alpha) \\
		&= (\mu, \mu + 2\delta)m(\mu) + 2\sum_{\alpha \succ 0} \sum_{i=1}^\infty m(\mu + i\alpha)(\mu + i\alpha, \alpha)
	\end{align}
	となる($\delta = \frac{1}{2}\sum_{\alpha \succ 0} \alpha$).
\end{proof}

\begin{mytheo}[label=thm:Freudenthal]{Freudenthalの公式}
	\hyperref[def:highest-weight-module]{最高ウエイト}$\lambda \in \Lambda^+$の非零で\hyperref[def:faithful]{忠実}な既約$\mathfrak{g}$-加群を$L(\lambda)$,整ウエイト$\mu \in \Lambda$の\hyperref[def:mutiplicity]{重複度}を$m(\lambda)$とすると,以下を満たす.
	\begin{equation}
		\big((\lambda + \delta, \lambda + \delta) - (\mu + \delta, \mu + \delta)\big)m(\mu) = 2\sum_{\alpha \succ 0}\sum_{i=1}^\infty m(\mu + i\alpha)(\mu + i\alpha, \alpha)
	\end{equation}
\end{mytheo}
\begin{proof}
	補題\ref{lem:Freudenthal}の証明の続き.まず,命題\ref{prop:Casimir-basic}より$\phi(c_\mathfrak{g})$はスカラー倍だったから,
	\begin{equation}
		\Tr_{V_\mu} \phi(c_\mathfrak{g}) = \phi(c_\mathfrak{g})m(\mu)
	\end{equation}
	となる.ここで,
	\begin{equation}
		(\mu, \mu + 2\delta) = (\mu + \delta, \mu + \delta) - (\delta, \delta)
	\end{equation}
	であり,$\mu = \lambda$のときのこれは$\phi(c_\mathfrak{g})$(の固有値)に等しいから,Freudenthalの公式が成り立つ.
\end{proof}
$\lambda$が最高ウエイトで,\hyperref[thm:hwmodule-c]{$m(\lambda)=1$}だったから,この公式から$m(\mu)$を一意的に求めることが可能.
\subsection{具体例}

\subsection{代数的指標}

\begin{mydef}[label=def:group-ring]{群環}
	$G$を群,$R$を環とする.$G$で生成される自由$R$-加群$R[G]$に対し,その積を$R$-係数の畳み込み
	\begin{equation}
		\biggl(\sum_{g \in G} a(g)g \biggr)\biggl(\sum_{h \in G} b(h)h \biggr) \coloneqq \sum_{g, h \in G} a(g)b(h)(gh) = \sum_{g \in G} a *b(g)g\quad  \biggl(a * b(g) \coloneqq \sum_{h \in G} a(h)b(h^{-1}g)\biggr)
	\end{equation}
	で定めた($G$の和は実際は有限和なのでwell-defined)$R$上の代数を\textbf{群環}(group ring)と呼ぶ.
\end{mydef}
\hyperref[def:root-lattice]{ウエイト格子}$\Lambda \subset \mathfrak{h}^*$は\hyperref[def:fundamental-weight]{基本ウエイト}で生成される$\mathbb{Z}$-加群だったので,スカラー倍の構造を忘れば加法群となる.よって,$\{e^\lambda | \lambda \in \Lambda\}$で生成される\hyperref[def:group-ring]{群環}$Z[\Lambda]$が定義される(生成系を別の文字にしたのは,$Z[\Lambda]$の和$+$と$\Lambda$の積$+$を区別するため.$e^\lambda e^\mu = e^{\lambda + \mu}$).\\
また$Z[\Lambda]$上の\hyperref[def:Weylgroup]{Weyl群}$\mathscr{W}$の作用を
\begin{equation}
	\sigma e^\lambda = e^{\sigma\lambda}\ (\sigma \in \mathscr{W})
\end{equation}
で定義する.一般の有限次元\hyperref[ax:g-module]{$\mathfrak{g}$-加群}$V$は,補題\ref{lem:weight-rep}より\hyperref[def:weight-rep]{ウエイト空間}の\hyperref[def:univ-vec-sum]{直和}で書けるから,$V$の\hyperref[def:weight-rep]{ウエイトの集合}$\Pi(V)$は有限集合.これに注意して指標を定義する.
\begin{mydef}[label=def:alg-character]{代数的指標}
	\hyperref[def:semisimple-LieAlg]{半単純Lie代数}$\mathfrak{g}$の\hyperref[def:root-lattice]{ウエイト格子}を$\Lambda \subset \mathfrak{h}^*$,有限次元\hyperref[ax:g-module]{$\mathfrak{g}$-加群}を$V$,\hyperref[def:weight-rep]{ウエイト}$\mu \in \Pi(V)$の\hyperref[def:mutiplicity]{重複度}を$m_\lambda(\mu)$とする.\hyperref[def:group-ring]{群環}$\mathbb{Z}[\Lambda]$の元
	\begin{equation}
		\ch_V \coloneqq \sum_{\mu \in \Pi(V)} m_\lambda(\mu)e^\mu = \sum_{\mu \in \Lambda} m_\lambda(\mu)e^\mu
	\end{equation}
	を\textbf{代数的指標}(algebraic character)や\textbf{形式的指標}(formal character),または単に\textbf{指標}と呼ぶ.\\
	特に\hyperref[thm:hwmodule-exist]{既約加群}$L(\lambda)$に対しては,$\ch_{L(\lambda)} = \ch_\lambda$とも書く.
\end{mydef}
\begin{mylem}[label=lem:Weyl-stable-irr-ch]{既約最高ウエイト加群の指標の$\mathscr{W}$-不変性}
	$\ch_\lambda$は\hyperref[def:Weylgroup]{Weyl群}$\mathscr{W}$の作用の下で不変.
\end{mylem}
\begin{proof}
	定理\ref{thm:thm:suff-for-finite}より,$m_\lambda(\mu) = m_\lambda(\mathscr{W}\mu)$なので$\mathscr{W}$-不変.
\end{proof}
有限次元加群$V$について,\hyperref[thm:Weyl]{完全可約性に関するWeylの定理}より,\hyperref[def:gmod-directsum]{直和}分解$V = \bigoplus_{i=1}^t L(\lambda_i)$に対し,
\begin{equation}
	\ch_V = \sum_{i=1}^t \ch_{\lambda_i}
\end{equation}
と書ける.よって$\{\ch_{\lambda}\}$が線型独立であれば,加群の直和と代数的指標の和が一対一に対応する.\\
また補題\ref{lem:Weyl-stable-irr-ch}より,$\ch_V$も$\mathscr{W}$の作用の下で不変.\\
よって線型独立性と,$\mathscr{W}$-不変性が有限次元加群の\hyperref[def:alg-character]{代数的指標}となる必要十分条件であることを示そう.
\begin{myprop}[label=prop:alg-character-finite]{有限次元加群の代数的指標,代数的指標の和}
	\hyperref[def:invariant-polynomial]{不変式}$\mathbb{Z}[\Lambda]^{\mathscr{W}}$は$\{\ch_\lambda | \lambda \in \Lambda^+ \}$で生成される.
	
	i.e. $f \in \mathbb{Z}[\Lambda]$が\hyperref[def:Weylgroup]{Weyl群}$\mathscr{W}$の作用の下で不変とすると,$f$は$\ch_\lambda\ (\lambda \in \Lambda^+)$の$\mathbb{Z}$-係数線型結合で一意的に書ける.特に,
	\begin{equation}
		\ch_\lambda = \sum_{\mu \in \mathscr{W}\lambda} e^\mu
	\end{equation}
\end{myprop}
\begin{proof}
	\begin{description}
		\item[(存在)] $f = \sum_{\lambda \in \Lambda} c(\lambda)e^\lambda\ (c(\lambda) \in \mathbb{Z})$と書くと,$\mathscr{W}$の作用で不変だから,
		\begin{equation}
			f = \sum_{\lambda \in \Pi} c(\lambda) \sum_{\mu \in \mathscr{W}\lambda} e^\mu\quad  \biggl(\Pi = \bigl\{\lambda \in \Lambda^+ \bigm| c(\lambda) \neq 0\bigr\}\biggr)
		\end{equation}
		と書ける.$M_f = \bigcup_{\lambda \in \Pi} \{\mu \in \Lambda^+ | \mu \prec \lambda\}$とすると,補題\ref{lem:dom-weight-B}より有限集合.\\
		ここで,$\lambda \in M_f$が極大のとき,補題\ref{lem:Weyl-stable-irr-ch}より,$f' = f - c(\lambda)\ch_\lambda$も$\mathscr{W}$の作用の下で不変.$L(\lambda)$の\hyperref[def:weight-rep]{ウエイトの集合}$\Pi(\lambda)$は\hyperref[def:weight-saturated]{飽和集合}だから,$\mu \prec \lambda$なる優ウエイトを全て含み,かつ$\lambda \notin M_{f'}$より,$M_{f'} \subsetneq M_f$.\\
		よって,$|M_f|$についての帰納法で,$|M_f'| = 0$になるまでこれを繰り返せば,$\mathscr{W}$-不変性より$f' = 0$になる.特に$|M_f| = 1$の場合,
		\begin{equation}
			f = c(\lambda) \sum_{\mu \in \mathscr{W}\lambda} e^\mu
		\end{equation}
		であり,$f = \ch_\lambda$のとき,\hyperref[thm:hwmodule-c]{最高ウエイト加群のウエイト空間の\hyperref[def:mutiplicity]{重複度}は$1$}だから,$c(\lambda) = 1$.
		\item[(一意性)] $f = \sum_{\lambda \in \Pi} c(\lambda)\ch_\lambda = \sum_{\lambda' \in \Pi'} c'(\lambda')\ch_{\lambda'}\ \bigl(0 \notin c(\Pi), c'(\Pi')\bigr)$と書けたとする.\\
		$\lambda \in \Pi$のうち,極大なものを$\lambda_0$とすると,$\lambda_0 \prec \lambda'_0$を満たす$\lambda'_0 \in \Pi'$が存在する.すると,$\lambda'_0 \prec \lambda$となる$\lambda \in \Pi$が存在するが,それは$\lambda_0$しかないから,
		\begin{equation}
			\lambda_0 \prec \lambda'_0 \prec \lambda_0  \IFF  \lambda_0 = \lambda'_0
		\end{equation}
		次に,$f - c(\lambda_0)\ch_{\lambda_0}$を考えると,同様の議論から,$c(\lambda_0) - c'(\lambda'_0) = 0$となる.これを繰り返せば,$\Pi = \Pi', c = c'$が言えるので,$f$の展開は一意.
	\end{description}
\end{proof}
次に\hyperref[def:alg-character]{代数的指標}の積の性質を確認しよう.
\begin{myprop}[label=prop:alg-character-multiply]{代数的指標の積}
	$V, W$を有限次元\hyperref[ax:g-module]{$\mathfrak{g}$-加群}とすると,
	\begin{equation}
		\ch_{V \otimes W} = \ch_V\ch_W
	\end{equation}
\end{myprop}
\begin{proof}
	\hyperref[def:gmod-tensor]{$\mathfrak{g}$-加群のテンソル積の定義}より,\hyperref[def:weight-rep]{ウエイト空間}どうしのテンソル積$V_\mu \otimes W_\nu$は\hyperref[def:weight-rep]{ウエイト}$\mu + \nu$の\hyperref[def:weight-rep]{ウエイト空間}.つまり,
	\begin{equation}
		m_{V \otimes W}(\mu + \nu) = \sum_{\pi + \pi' = \mu + \nu} m_V(\pi)m_W(\pi')
	\end{equation}
	となる.右辺は$\ch_V\ch_W$の$e^{\mu + \nu}$の係数に等しい.
\end{proof}
特に,$V, W$の\hyperref[def:highest-weight-module]{最高ウエイト}をそれぞれ$\lambda_V, \lambda_W$とすると,\hyperref[ax:g-module]{$\mathfrak{g}$-加群}$V \otimes W$の\hyperref[def:irr]{既約}分解には,$L(\lambda_V + \lambda_W)$が含まれる.

\section{指標}
\subsection{不変式論}

\begin{mydef}[label=def:polynomial-function-ring]{多項式函数環}
	体$\mathbb{K}$上の線型空間$V$に対し,$V^*$に対する\hyperref[def:sym-alg]{対称代数}
	\begin{equation}
		\mathbb{K}[V] \coloneqq S(V^*)
	\end{equation}
	を$V$上の\textbf{多項式函数環}(ring of polynomial functions)と呼ぶ.
\end{mydef}
$V$を有限次元とし,$V^*$の基底を$(f^1,\ldots , f^n)$と取ると,$\mathbb{K}[V]$は$n$変数多項式代数$\mathbb{K}[f^1, \ldots, f^n]$と同型である.\\
\hyperref[def:root-lattice]{ウエイト格子}$\Lambda$は$\mathfrak{h}^*$を張るから,$\lambda \in \Lambda$の多項式は$\mathbb{K}[\mathfrak{h}]$を張る.
\begin{mydef}[label=def:poly-func-ring-modular]{誘導される多項式函数環の加群}
	群を$G$,体$\mathbb{K}$上の線型空間を$V$とする.$V$が$G$-加群のとき,$V^*$上の$G$-加群が
	\begin{equation}
		(g \btr f)(v) \coloneqq f(g^{-1} \btr v)\quad  (\forall v \in V,\ g \in G,\ f \in V^*)
	\end{equation}
	により定義される.\\
	また,$V^*$が$G$-加群の(または誘導された)とき,\hyperref[def:polynomial-function-ring]{多項式函数環}$\mathbb{K}[V]$上の$G$-加群が
	\begin{equation}
		g \btr (f_1 \cdots f_n) \coloneqq (g \btr f_1) \cdots (g \btr f_n)
	\end{equation}
	を線型に拡張としたものとして定義される.
\end{mydef}
\begin{mydef}[label=def:invariant-polynomial]{不変式}
	$G$を群,体$\mathbb{K}$上の線型空間を$V$とする.$\mathbb{K}[V]$が(定義\ref{def:poly-func-ring-modular}により誘導された)$G$-加群のとき,この作用で不変な集合
	\begin{equation}
		\mathbb{K}[V]^G \coloneqq \bigl\{f \in \mathbb{K}[V] \bigm| G \btr f = f \bigr\}
	\end{equation}
	は$\mathbb{K}$-可換結合代数をなす.これを$V$上の$G$-\textbf{不変多項式}($G$-invariant polynomials on $V$)あるいは単に\textbf{不変式}(invariants)と呼ぶ.
\end{mydef}
$\mathbb{K}[V]^G$が$G$の作用で閉じていることを具体的に示すと以下($g \in G, \lambda \in \mathbb{K},\ f_1, f_2 \in \mathbb{K}[V]^G$):
\begin{align}
	g \btr (f_1 + f_2) &= g \btr f_1 + g \btr f_2 = f_1 + f_2 \\
	g \btr (\lambda f_1) &= \lambda (g \btr f_1) = \lambda f_1 \\
	g \btr (f_1f_2) &= (g \btr f_1)(g \btr f_2) = f_1f_2
\end{align}
\hyperref[def:Weylgroup]{Weyl群}$\mathscr{W}$は$\mathfrak{h}^*$に作用するから,$\mathscr{W}$-不変式$\mathbb{K}[\mathfrak{h}]^\mathscr{W}$が定義される.また,$\mathfrak{g}$の\hyperref[def:inner-LieAlg]{内部自己同型}$\Inn\mathfrak{g}$は$\mathfrak{g}$に作用するから,$\Inn\mathfrak{g}$-不変式$\mathbb{K}[\mathfrak{g}]^{\Inn\mathfrak{g}}$が定義される.
\begin{mydef}[label=def:coinvariant-polynomial]{余不変式}
	$G$を群,体$\mathbb{K}$上の線型空間を$V$とする.多項式代数$\mathbb{K}[V]$が(定義\ref{def:poly-func-ring-modular}により誘導された)$G$-加群のとき,軌道空間
	\begin{equation}
		\mathbb{K}[V]_G \coloneqq \mathbb{K}[V] / G
	\end{equation}
	は$\mathbb{K}$-可換結合代数をなす.これを$V$上の$G$-\textbf{余不変多項式}($G$-coinvariant polynomials on $V$)あるいは単に\textbf{余不変式}(coinvariants)と呼ぶ.
\end{mydef}
\begin{mytheo}[label=def:coinv-inv-isomorphism]{有限群上の不変式と余不変式の同型性}
	有限群上の不変式と余不変式は同型.
\end{mytheo}
\begin{proof}
	\begin{equation}
		p_G \colon \mathbb{K}[V] \lto \mathbb{K}[V],\quad  x \mapsto \frac{1}{|G|}\sum_{g \in G} (g \btr x)
	\end{equation}
	は不変式$\mathbb{K}[V]^G$への(代数の)射影.よって,$\mathbb{K}[V]^G$はある商代数と同型だが,
	\begin{equation}
		p_G(x) = p_G(y)  \IFF  \exists g \in G,\ x = g \btr y\quad  (x, y \in \mathbb{K}[V])
	\end{equation}
	より$\mathbb{K}[V]^G$と$\mathbb{K}[V]_G$は同型.
\end{proof}
$\phi$を$\mathfrak{g}$の表現とする.$\mathbb{K}[\mathfrak{g}]^{\Inn\mathfrak{g}}$の元として,\textbf{跡多項式}(trace polynomial)
\begin{equation}
	t\colon x \mapsto \Tr(\phi(x)^k)\quad  (x \in \mathfrak{g})
\end{equation}
があることを見よう.まず$\tr \circ \phi$は線型なので,$\tr \circ \phi \in \mathfrak{g}^*$.

\hyperref[]{テンソル積とHomの同型}より$\End(V) \simeq V^* \otimes V$なので,$\Tr \colon \End(V) \lto \mathbb{K}$は
\begin{equation}
	V^* \otimes V \lto \End(V),\quad  h \otimes v \mapsto (w \mapsto h(w)v)
\end{equation}
$\Tr(h \otimes v) = h(v)$.

より,$t \in \mathbb{K}[\mathfrak{g}]$となる.


\hyperref[def:inner-LieAlg]{内部自己同型}$\tau_\alpha\ (\alpha \in \Phi)$に対し,$\tau_\alpha|_\mathfrak{h} = \sigma_\alpha \in \mathscr{W}$であり,$\sigma_\alpha$は$\mathscr{W}$を生成した.\\
よって射影$\mathfrak{g}^* \lto \mathfrak{h}^*$を多項式函数へ拡張したものを$p\colon \mathbb{K}[\mathfrak{g}] \lto \mathbb{K}[\mathfrak{h}]$とすると,$p$は不変式の代数準同型
\begin{equation}
	\theta \colon \mathbb{K}[\mathfrak{g}]^{\Inn\mathfrak{g}} \lto \mathbb{K}[\mathfrak{h}]^\mathscr{W}
\end{equation}
を誘導する.
\begin{mytheo}[label=thm:Chevalley-restriction]{Chevalleyの制限定理}
	$\mathfrak{g}$を\hyperref[def:semisimple-LieAlg]{半単純Lie代数},$\mathfrak{h}$をその極大\hyperref[def:toral-subLieAlg]{トーラス},$\mathscr{W}$を\hyperref[def:Weylgroup]{Weyl群}とする.不変式の代数準同型
	\begin{equation}
		\theta\colon \mathbb{K}[\mathfrak{g}]^{\Inn\mathfrak{g}} \lto \mathbb{K}[\mathfrak{h}]^\mathscr{W}
	\end{equation}
	は同型.
\end{mytheo}
\begin{proof}(Steinberg)
	\begin{description}
		\item[\textbf{全射性}] $p_{\Inn \mathfrak{g}}(\lambda^k)\ (\lambda \in \Lambda^+,\ k \in \mathbb{Z}_{\geq 0})$を考えれば良い.
		\hyperref[def:fundamental-weight]{基本ウエイト}
		\item[\textbf{単射性}] 省略
	\end{description}
\end{proof}
跡多項式の集合を$T \subset \mathbb{K}[\mathfrak{g}]^{\Inn\mathfrak{g}}$とする.\hyperref[thm:Chevalley-restriction]{Chevalleyの制限定理}の証明より,$\theta|_T$は全単射となる.つまり,次の定理が成り立つ.
\begin{mytheo}[label=thm:Inn-G-invariant-structure]{$\mathbb{K}[\mathfrak{g}]^{\Inn\mathfrak{g}}$の構造}
	$\mathbb{K}[\mathfrak{g}]^{\Inn\mathfrak{g}}$は跡多項式で生成される.
\end{mytheo}

\subsection{最高ウエイト加群と指標}
\hyperref[def:univ-env-alg]{普遍包絡代数}$\mathfrak{U}(\mathfrak{g})$の中心を$\mathfrak{Z}(\mathfrak{g}) = Z(\mathfrak{U}(\mathfrak{g}))$とする.$\mathfrak{g}$の\hyperref[def:auto-LieAlg]{自己同型}$\tau \in \Aut\mathfrak{g}$は,$\mathfrak{U}(\mathfrak{g})$の自己同型に一意的に拡張される.
\begin{mytheo}[label=thm:Inng-invariant-Ug]{}
	$\mathfrak{g}$を\hyperref[def:semisimple-LieAlg]{半単純Lie代数},$\mathfrak{U}(\mathfrak{g})$をその\hyperref[def:univ-env-alg]{普遍包絡代数},$\mathfrak{Z}(\mathfrak{g})$をその中心とすると,
	\begin{equation}
		\mathfrak{U}(\mathfrak{g})^{\Inn \mathfrak{g}} = \mathfrak{Z}(\mathfrak{g})
	\end{equation}
\end{mytheo}
\begin{proof}
	$\mathfrak{Z}(\mathfrak{g})$は中心だから,
	\begin{equation}
		0 = [x, \mathfrak{Z}(\mathfrak{g})] = \ad x(\mathfrak{Z}(\mathfrak{g}))  \IMP  \exp\ad x(z) = z\quad  (\forall z \in \mathfrak{Z}(\mathfrak{g}))
	\end{equation}
	つまり,任意の$\sigma \in \Inn\mathfrak{g}$に対し$z \in \mathfrak{Z}(\mathfrak{g})$は$\sigma$-不変だから,$\mathfrak{U}(\mathfrak{g})^{\Inn \mathfrak{g}} \supset \mathfrak{Z}(\mathfrak{g})$.\\
	逆の包含を示す.$x_\alpha \in \mathfrak{g} \setminus \{0\}$を固定する.$(\ad x_\alpha)^t \neq 0,\ (\ad x_\alpha)^{t+1} = 0$とする.\\
	今,体$\mathbb{K}$は無限集合だったから,異なるスカラー$a_1, \ldots, a_{t+1}$を選べる.このとき,仮定より$\exp \ad a_ix_\alpha = \exp (a_i\ad x_\alpha)$も$x$を固定する.ここで,Vandermonde行列式について,
	\begin{equation}
		\left|\begin{array}{ccccc}
			1 & a_1 & \frac{a_1^2}{2!} & \cdots & \frac{a_1^t}{t!} \\
			\vdots &&&& \vdots \\
			1 & a_{t+1} & \frac{a_{t+1}^2}{2!} & \cdots & \frac{a_{t+1}^t}{t!}
		\end{array}\right| = \prod_{k=0}^{t+1}\frac{1}{k!} \prod_{1 \leq i < j \leq t+1} (a_j - a_i) \neq 0
	\end{equation}
	だから,$\Span\{\exp (a_i\ad x_\alpha)\} = \Span\{1, \ad x_\alpha, \ldots (\ad x_\alpha)^t\}$.よって,
	\begin{equation}
		\exists b_i \in \mathbb{K},\quad  \ad x_\alpha = \sum_{i=1}^{t+1} b_i \exp (a_i\ad x_\alpha) = \sum_{i=1}^{t+1} b_i \exp(\ad (a_i x_\alpha)) = \sum_{i=0}^{t+1} b_i
	\end{equation}
	と書ける.$\ad x_\alpha$が冪零だから$\sum_{i=1}^{t+1} b_i = 0$となり,$\ad x_\alpha(\mathfrak{U}(\mathfrak{g})^{\Inn \mathfrak{g}}) = 0$.ところで,$x_\alpha$は$\mathfrak{g}$を生成するから,$\mathfrak{U}(\mathfrak{g})^{\Inn \mathfrak{g}} \subset \mathfrak{Z}(\mathfrak{g})$.
\end{proof}
ここで,任意の\hyperref[def:weight-rep]{ウエイト}$\lambda \in \mathfrak{h}^*$に対するVerma加群$M(\lambda)$を考える.$h \in \mathfrak{h}$とすると,\hyperref[def:weight-rep]{ウエイト}$\lambda$の\hyperref[def:maximal-vector-rep]{極大ベクトル}$v^+$に対し,
\begin{align}
	h \btr z \btr v^+ &= z \btr h \btr v^+ = \lambda(h)z \btr v^+\quad  (z \in \mathfrak{Z}(\mathfrak{g})) \\
	\mathfrak{g}_\alpha \btr z \btr v^+ &= z \btr \mathfrak{g}_\alpha \btr v^+ = 0\quad  (\alpha \in \Phi^+)
\end{align}
となるから,$z \btr v^+$は\hyperref[def:highest-weight-module]{最高ウエイト}$\lambda$の\hyperref[def:maximal-vector-rep]{極大ベクトル}.定理\ref{thm:hwmodule}より\hyperref[def:highest-weight-module]{最高ウエイト}の\hyperref[def:mutiplicity]{重複度}$m(\lambda)$は$1$だから,$z \btr v^+ = \chi_\lambda(z) v^+$と書ける.
\begin{mydef}[label=def:character]{指標}
	\hyperref[def:highest-weight-module]{最高ウエイト}$\lambda \in \mathfrak{h}^*$のVerma加群に対し,\hyperref[def:maximal-vector-rep]{極大ベクトル}に作用したときの固有値を返す$\mathbb{K}$-代数の準同型
	\begin{equation}
		\chi_\lambda \colon \mathfrak{Z}(\mathfrak{g}) \lto \mathbb{K},\quad  z \mapsto \chi_\lambda(z)
	\end{equation}
	を$\lambda$の\textbf{指標}(character)と呼ぶ.
\end{mydef}
$\forall x \in \mathfrak{U}(\mathfrak{g})$に対し,$x \btr v^+$は$\mathfrak{Z}(\mathfrak{g})$の固有ベクトル.$M(\lambda)$は\hyperref[def:highest-weight-module]{最高ウエイト加群}だったから,$M(\lambda)$の全体が$\mathfrak{Z}(\mathfrak{g})$の作用で不変.よって,任意の$M(\lambda)$の\hyperref[def:sub-g-module]{部分$\mathfrak{g}$-加群}は$\mathfrak{Z}(\mathfrak{g})$の作用で不変で,同じ\hyperref[def:character]{指標}$\chi_\lambda$をもつ.
\begin{myprop}[label=prop:hmmodule-character]{}
	整ウエイト$\lambda \in \Lambda$,\hyperref[def:base-root]{単純ルート}$\alpha \in \Delta$,整数$m = \sspair{\lambda}{\alpha}$に対し,$m \geq 0$のとき,\\
	$y_\alpha^{m+1} + I(\lambda)\ (M(\lambda) = \mathfrak{U}(\mathfrak{g}) / I(\lambda))$は\hyperref[def:weight-rep]{ウエイト}$\lambda - (m+1)\alpha$の\hyperref[def:maximal-vector-rep]{極大ベクトル}.
\end{myprop}
\begin{proof}
	$h_\alpha - \sspair{\lambda}{\alpha} \in I(\lambda)$なので,補題\ref{lem:suff-for-finite}より従う.
\end{proof}
\hyperref[def:weight-rep]{ウエイト}$\lambda - (\sspair{\lambda}{\mu}+1)\alpha$の\hyperref[def:maximal-vector-rep]{極大ベクトル}で生成される\hyperref[def:highest-weight-module]{最高ウエイト加群}は$M(\lambda)$の\hyperref[def:sub-g-module]{部分$\mathfrak{g}$-加群}.\hyperref[def:character]{指標の定義}の直後の議論よりこの命題\ref{prop:hmmodule-character}の系として以下が成り立つ.
\begin{mycol}[label=col:hmmodule-character]{}
	$\sspair{\lambda}{\alpha} \geq 0$のとき,$\mu = \lambda - (\sspair{\lambda}{\alpha}+1)\alpha$とすると,$\chi_\mu = \chi_\lambda$.
\end{mycol}
極大\hyperref[def:toral-subLieAlg]{トーラス}$\mathfrak{h}$とWeyl群$\mathscr{W}$に対し,軌道空間$\mathfrak{h}^* / \mathscr{W}$は同値類であった.よって,$(\mathfrak{h}^* + \delta) / \mathscr{W}$も同値類となる.
\begin{mycol}[label=col:hmmodule-character-2]{}
	任意の整ウエイト$\lambda, \mu \in \Lambda$に対し,
	\begin{equation}
		\mathscr{W}(\mu + \delta) = \mathscr{W}(\lambda + \delta)  \IMP  \chi_\lambda = \chi_\mu
	\end{equation}
\end{mycol}
\begin{proof}
	$\mathscr{W} = \ev{\sigma_\alpha | \alpha \in \Delta}$だから,$\mu = \sigma_\alpha(\lambda + \delta) - \delta$の場合を示せば良い.系\ref{col:simpleroot-B}と$\sigma_\alpha$の定義より,
	\begin{equation}
		\mu = \sigma_\alpha\lambda - \alpha = \lambda - (\sspair{\lambda}{\alpha} + 1)\alpha
	\end{equation}
	となる.$\lambda$は整ウエイトだから,$\sspair{\lambda}{\alpha} \in \mathbb{Z}$.\\
	$\sspair{\lambda}{\alpha} = -1$のとき,$\lambda = \mu$なので$\chi_\lambda = \chi_\mu$は明らか.\\
	そうでないとき,$\sspair{\mu}{\alpha} + 1 = -(\sspair{\lambda}{\alpha} + 1)$だから,$\sspair{\lambda}{\alpha}, \sspair{\mu}{\alpha}$のいずれかは非負.よっていずれかが系\ref{col:hmmodule-character}の仮定を満たすから,$\chi_\lambda = \chi_\mu$は等しい.
\end{proof}
$\mathfrak{g}$の基底を$\{x_\alpha, y_\alpha, | \alpha \prec 0\}$($x_\alpha \in \mathfrak{g}_\alpha,\ y_\alpha \in \mathfrak{g}_{-\alpha}$)と$\{h_i | 1 \leq i \leq l\}$(\hyperref[def:base-root]{ルート系の底}$\Delta = \{\alpha_i, \ldots, \alpha_l\}$に対し$h_i = h_{\alpha_i}$)ととる.$\mathfrak{U}(\mathfrak{g}),\ \mathfrak{U}(\mathfrak{h})$の\hyperref[col:PBW-C]{PBW基底}は$\{y_\alpha\}, \{h_i\},\ \{x_\alpha\}$の順に全順序となるようにとる.また,$p\colon \mathfrak{U}(\mathfrak{g}) \lto \mathfrak{U}(\mathfrak{h})$を射映とする.\\
極大\hyperref[def:toral-subLieAlg]{トーラス}$\mathfrak{h}$は可換だったから,$\mathfrak{U}(\mathfrak{h}) = S(\mathfrak{h})$なので,以下では特に$S(\mathfrak{h})$と書く.

\hyperref[thm:hwmodule-exist]{既約最高ウエイト加群}$L(\lambda)$の\hyperref[def:maximal-vector-rep]{極大ベクトル}$v^+$に対する$z \in \mathfrak{Z}(\mathfrak{g})$の作用を考える.\\
単項式$\prod_{\alpha \succ 0}y_\alpha^{i_\alpha}\prod_{i = 1}^l h_\alpha^{k_\alpha}\prod_{\alpha \succ 0}x_\alpha^{j_\alpha}$はある$\alpha$に対して$j_\alpha > 0$ならば,$v^+$に作用すると$0$.逆に全ての$\alpha$に対して$j_\alpha = 0$の場合はウェイト$\lambda - i_\alpha\alpha$のベクトルに送る.つまり,\hyperref[def:character]{指標}$\chi_\lambda$に関与するのは,$S(\mathfrak{h})$の線型結合部分のみであり,
\begin{equation}
	\label{eq:Harish-Chandra-1}
	\chi_\lambda = \lambda \circ p|_{\mathfrak{Z}(\mathfrak{g}) \cap S(\mathfrak{h})}
\end{equation}
となる($\lambda\colon \mathfrak{U}(\mathfrak{h}) \lto \mathbb{K}$は$\lambda\colon \mathfrak{h} \lto \mathbb{K}$の標準的拡大).$\chi_\lambda, \lambda$は代数準同型だから,$p|_{\mathfrak{Z}(\mathfrak{g}) \cap S(\mathfrak{h})}$は代数準同型となる.

$\mathfrak{h}$の同型$h_i \mapsto h_i - 1\ (\forall i = 1, \ldots ,k)$を$\mathfrak{U}(\mathfrak{h}) \simeq S(\mathfrak{h})$に線型に拡張したものを$\eta$とすると,可換\hyperref[g-module-hom]{Lie代数の同型}.$\delta$は\hyperref[def:fundamental-weight]{基本ウエイト}の和だったから,$\delta(h_i) = 1$.$\delta$も同様に$S(\mathfrak{h})$に拡大すると,
\begin{equation}
	(\lambda + \delta) \circ \eta(h_i) = (\lambda + \delta)(h_i - 1) = (\lambda(h_i) + 1) - (\lambda + \delta)1 = \lambda(h_i)
\end{equation}
\begin{equation}
	\therefore (\lambda + \delta) \circ \eta = \lambda
\end{equation}
となる.$\psi = \eta \circ p|_{\mathfrak{Z}(\mathfrak{g}) \cap S(\mathfrak{h})}$とすると,\eqref{eq:garish-chandra-1}と合わせて,
\begin{equation}
	\chi_\lambda = (\lambda + \delta) \circ \psi
\end{equation}
\begin{mylem}[label=lem:Harish-Chandra-1]{}
	任意の\hyperref[def:weight-rep]{ウエイト}$\lambda, \mu \in \mathfrak{h}^*$に対し,
	\begin{equation}
		\chi_\lambda = \chi_\mu  \IMP  \mathscr{W} \lambda = \mathscr{W} \mu
	\end{equation}
\end{mylem}
\begin{proof}
	先程定義した$\psi$は$\lambda$に(汎函数として)依存しないので,系\ref{col:hmmodule-character-2}より,整ウエイト$\lambda \in \Lambda$に対し$\chi_\lambda = (\mathscr{W}\lambda) \circ \psi$となる.つまり,$\psi$は$\mathscr{W}$不変で$\Lambda$は$S(\mathfrak{h})$を生成するから,$\Im\psi \subset S(\mathfrak{h})^\mathscr{W} \subset S(\mathfrak{h})$.\\
	逆にこのとき,$\forall \lambda \in \mathfrak{h}^*$で$\chi_\lambda = \chi_{\mathscr{W}\lambda}$.つまり,系\ref{col:hmmodule-character-2}が$\lambda, \mu \in \mathfrak{h}^*$でも成り立つことになる.
\end{proof}
\begin{mylem}[label=lem:Harish-Chandra-2]{}
	$\lambda_1, \lambda_2 \in \mathfrak{h}^*$について,$\mathscr{W}\lambda_1 \neq \mathscr{W}\lambda_2$とすると,
	\begin{equation}
		\exists x \in S(\mathfrak{h})^\mathscr{W} (= \mathbb{K}[\mathfrak{h}^*]),\quad  \lambda_1(x) \neq \lambda_2(x)
	\end{equation}
\end{mylem}
\begin{proof}
	$\lambda_1(y) \neq 0$を満たす$y \in \mathfrak{h}$を選ぶ.$\mathscr{W}$は有限群なので,
	\begin{equation}
		y\prod_{\mu \in (\mathscr{W}\lambda \setminus \lambda_1) \cup \mathscr{W}\lambda_2} (y - \mu(y)) \in S(\mathfrak{h})
	\end{equation}
	が定義でき,さらに$\mathscr{W}$の作用の和をとったもの$x$が定義できる.すると$x \in S(\mathfrak{h})^\mathscr{W}$かつ,
	\begin{equation}
		\lambda_1(x) \neq 0 = \lambda_2(x) 
	\end{equation}
\end{proof}

\begin{mytheo}[label=thm:Harish-Chandra]{Harish-Chandra同型}
	$\lambda, \mu \in \mathfrak{h}^*$に対し,
	\begin{equation}
		\chi_\lambda = \chi_\mu  \IFF  \mathscr{W} \lambda = \mathscr{W} \mu
	\end{equation}
\end{mytheo}
\begin{proof}
	補題\ref{lem:Harish-Chandra-1}を示したから,その逆(つまり$\Longleftarrow$)を示せば良い.
	
	補題\ref{lem:Harish-Chandra-2}を示したから,$\chi_\lambda = (\lambda + \delta) \circ \psi$より$\Im\psi \subset S(\mathfrak{h})^\mathscr{W}$を示せば良い.
	
	命題\ref{prop:graded-alg-by-filtration}の意味での結合代数の次数付けについて$S \simeq \gr S$となること,\hyperref[thm:PBW]{PBW定理},さらに群作用の不変性が$\Ker$が$0$の準同型で保存されること,定理\ref{thm:Inng-invariant-Ug}に注意して以下の図式を考える.
	\begin{center}
		\begin{tikzcd}[row sep=large, column sep=large]
			\gr S(\mathfrak{g})^{\Inn\mathfrak{g}} \ar[d, leftrightarrow]\ar[r, leftrightarrow] & \gr \mathfrak{U}(\mathfrak{g})^{\Inn\mathfrak{g}} \\
			S(\mathfrak{g})^{\Inn\mathfrak{g}} \ar[d, leftrightarrow]\ar[r, "\pi"] & \mathfrak{U}(\mathfrak{g})^{\Inn\mathfrak{g}} = \mathfrak{Z}(\mathfrak{g})\ar[r, "\psi"]\ar[u, "p_\text{G}"] &  \mathfrak{U}(\mathfrak{h})^\mathscr{W} = S(\mathfrak{h})^\mathscr{W}\ar[d, leftrightarrow] \\
			\mathbb{K}[\mathfrak{g}]^{\Inn\mathfrak{g}} \ar[rr, "\theta"']&& \mathbb{K}[\mathfrak{h}]^\mathscr{W}
		\end{tikzcd}
	\end{center}
	ただし,$S(\cdot)$と$\mathbb{K}[\cdot]$の同型性はKilling形式による双対による.\\
	$\pi$が準同型であれば簡単だが,単なる線型写像ではあっても代数の準同型ではない\footnote{$\gr\mathfrak{U}$を考えていることから,$\mathfrak{U}$の最高次係数は変わらないことはわかる.準同型でないのは低次項は変わることがあるため.}.しかし,$\pi$を除いて,$S(\mathfrak{h})^\mathscr{W}$を$S(\mathfrak{h})$にすれば可換図式であり,標準的射映(の拡張)$p_\text{G}$と,$\gr \mathfrak{U}(\mathfrak{g})^{\Inn\mathfrak{g}}$から$S(\mathfrak{h})^\mathscr{W}$まで反時計回りに移す写像は全て準同型で,その合成は$\psi$に等しい\footnote{~\cite{Humphreys1972introduction}ではここを逆向きに追っている.}.群作用の不変性が$\Ker$が$0$の準同型で保存されるから,$\Im\psi \subset S(\mathfrak{h})^\mathscr{W} \subset S(\mathfrak{h})$\footnote{つまり,$S(\mathfrak{h})^\mathscr{W}$のままでも可換図式となる.}.
\end{proof}

\section{指標の諸公式}
この節では有限次元\hyperref[ax:g-module]{$\mathfrak{g}$-加群}の\hyperref[def:character]{指標}や\hyperref[def:mutiplicity]{重複度}に関する公式を示す.
\subsection{いくつかの$\mathfrak{h}^*$の函数}
$\lambda \in \Lambda^+$に対する$L(\lambda)$の\hyperref[def:alg-character]{代数的指標}$\ch_\lambda$は自由$\mathbb{Z}$-加群でもあった.$\lambda \in \mathfrak{h}^*$に対する(任意の)\hyperref[def:highest-weight-module]{最高ウエイト加群}(の直和加群)上に\hyperref[def:alg-character]{代数的指標}を一般化しよう.\\
\hyperref[def:highest-weight-module]{最高ウエイト加群}の直和の\hyperref[def:alg-character]{代数的指標}の展開係数$f(g)$は,定理 6.1.1(2)(3)より
\begin{equation}
	\mathfrak{X} \coloneqq \bigl\{f \colon \mathfrak{h}^* \lto \mathbb{K} \bigm| \supp f = \bigcup_{\text{有限和}}\{\lambda - \sum_{\alpha \succ 0} k_\alpha\alpha, k_\alpha \in \mathbb{Z} \}\bigr\}
\end{equation}
なる集合に含まれる.$\forall f, g \in \mathfrak{X}$に対し,\hyperref[def:highest-weight-module]{最高ウエイト加群}の直和分解の際の\hyperref[def:highest-weight-module]{最高ウエイト}の(有限)集合を$\Lambda_f, \Lambda_g$とする.$\forall \mu \in \supp g$に対し,
\begin{equation}
	\{\nu \in \mathfrak{h}^* | f(\nu)g(\mu - \nu) \neq 0\}
\end{equation}
は$\mathfrak{X}$の定義より離散集合.かつ,
\begin{equation}
	\exists \lambda_f \in \Lambda_f,\ \lambda_g \in \Lambda_g,\quad  \lambda - \lambda_g \prec \nu \prec \lambda_f
\end{equation}
より全ての$(\lambda_f,  \lambda_g)$の合併を考えるとコンパクト集合.よって有限集合だから,\hyperref[def:alg-character]{代数的指標}の積が,任意の\hyperref[def:highest-weight-module]{最高ウエイト加群}のとしてもwell-defined.

ここで,生成子$e^\lambda$は特性函数
\begin{equation}
	e^{\lambda}(\lambda) = 1,\quad  e^{\lambda}(\mathfrak{h}^* \setminus \{\lambda\}) = 0
\end{equation}
と同一視できる.また$\mathfrak{X}$に対するWeyl群$\mathscr{W}$の作用は定義\ref{def:poly-func-ring-modular}のように自然に誘導される.特に,$\sigma(e^\lambda) = e^{\sigma\lambda}$.

$p(\lambda)$を$-\lambda = \sum_{\alpha \prec 0} k_\alpha\alpha$を満たす非負整数の集合$\{k_\alpha\}$の数とする.当然$\lambda$がルート格子上に無い場合は,$p(\lambda) = 0$.




Kostant函数







\begin{mytheo}[label=thm:Kostant]{Kostantの重複度公式}
	優ウエイト$\lambda \in \Lambda^+$に対し,\hyperref[thm:hwmodule-exist]{既約最高ウエイト加群}$L(\lambda)$の\hyperref[def:weight-rep]{ウエイト}$\mu$の\hyperref[def:mutiplicity]{重複度}は以下
	\begin{equation}
		m_\lambda(\mu) = \sum_{\sigma \in \mathscr{W}} sn(\sigma)p(\mu + \delta - \sigma(\lambda + \delta))
	\end{equation}
\end{mytheo}




\begin{mytheo}[label=thm:Kostant]{Weylの指標公式}
	優ウエイト$\lambda \in \Lambda^+$に対し,以下が成り立つ:
	\begin{equation}
		\Bigl(\sum_{\sigma \in \mathscr{W}} sn(\sigma)\varepsilon_{\sigma\delta}\Bigr) * \ch_\lambda = \sum_{\sigma \in \mathscr{W}} sn(\sigma)\varepsilon_{\sigma(\lambda + \delta)}
	\end{equation}
\end{mytheo}



\begin{mytheo}[label=thm:Kostant]{Steinbergの公式}
	優ウエイト$\lambda, \lambda', \lambda^{\prime\prime} \in \Lambda^+$に対し,$L(\lambda') \otimes L(\lambda^{\prime\prime})$の直和分解した際の$L(\lambda)$の個数は以下:
	\begin{equation}
		\sum_{\sigma,\sigma' \in \mathscr{W}} sn(\sigma\tau)p(\lambda + 2\delta - \sigma(\lambda' + \delta) - \sigma'(\lambda^{\prime\prime} + \delta))
	\end{equation}
\end{mytheo}












\end{document}