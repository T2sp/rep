\documentclass[rep_main]{subfiles}

\begin{document}

\setcounter{chapter}{5}

\chapter{表現定理}
この章では,$\mathfrak{g}$は標数$0$の代数閉体$\mathbb{K}$上の半単純Lie代数とし,$\mathfrak{h}$を$\mathfrak{g}$のCartan部分代数,$\Phi$をルート系,$\Delta = \{\alpha_1, \ldots, \alpha_l\}$を$\Phi$の底,$\mathscr{W}$をWeyl群とする.

\section{表現のウエイトと極大ベクトル}
$\lsl{2}{\mathbb{K}}$で考えたウエイトを一般化する.$\lsl{2}{\mathbb{K}}$の場合,\hyperref[def:toral-subLieAlg]{極大トーラス}$\mathfrak{h}$は$1$次元線型空間だったので,ウエイトは固有値として定義された.$\mathfrak{h}$が一般の次元の場合は,ルートのように$\mathfrak{h}^*$の元として定義すれば良い.ルートと異なる点は,表現が随伴表現とは限らないことだけである.
\subsection{ウエイト空間}
\begin{mydef}[label=def:weightspacerep]{表現のウエイト,ウエイト空間,ウェイトの集合}
	$V$を$\mathfrak{g}$-加群とし,\hyperref[def:toral-subLieAlg]{極大トーラス}$\mathfrak{h}$を一つ固定する.$\lambda \in h^*$に対し,
	\begin{align}
		V_\lambda \coloneqq \bigl\{\, v \in V \bigm| h \btr v = \lambda(h) v \,\bigr (\forall h \in \mathfrak{h})\}
	\end{align}
	が定義される.この内$V_\lambda \neq 0$のものを,\textbf{ウエイト空間}と呼び,$\lambda$を$V$の\textbf{ウエイト}(より正確には$\mathfrak{h}$の$V$上のウエイト)と呼ぶ.\\
	また$V$の\textbf{ウェイトの集合}を
	\begin{equation}
		\Pi(V) \coloneqq \bigl\{\, \mu \in \mathfrak{h}^* \bigm| V_\mu \neq 0\}
	\end{equation}
	と定義する.
\end{mydef}
\begin{mylem}[label=lem:weightspacerep]{}
	\begin{enumerate}
		\item $V' \coloneqq \sum_{\lambda \in \mathfrak{h}^*} V_\lambda$は直和で,$V$の部分$\mathfrak{g}$-加群.
		\item $V$が有限次元の場合,$V = V'$.
		\item $\mathfrak{g}_\alpha \btr V_\lambda = V_{\lambda + \alpha},\ (\forall \lambda \in \mathfrak{h}^*,\ \forall \alpha \in \Phi)$.
	\end{enumerate}
\end{mylem}
\begin{proof}
	\begin{enumerate}
		\item ウエイト$\lambda, \lambda'$に対し,$v \in V_\lambda \cap V_\lambda' \setminus \{0\}$とする.任意の$h$に対し,
		\begin{equation}
			h \btr v = \lambda(h)v = \lambda'(h)v  \IFF  (\lambda(h) - \lambda'(h))v = 0 
		\end{equation}
		であるから,$v \neq 0$より$\lambda = \lambda'$となる.よって,直和となる.
		\item $\mathfrak{g}$は半単純なので,系\ref{col:JC}より,$\mathfrak{h}^*$の元を同時対角化できる.
		\item $x \in \mathfrak{g}_\alpha,\ h \in \mathfrak{h},\ v \in V_\lambda$とすると,
		\begin{equation}
			h \btr (x \btr v) = x \btr (h \btr v) + [h, x] \btr v = (\lambda(h) + \alpha(h))(x \btr v)
		\end{equation}
		より,$\mathfrak{g}_\alpha \btr V_\lambda = V_{\lambda + \alpha}$となる.
	\end{enumerate}
\end{proof}
\subsection{最高ウェイト加群}
極大ベクトルも同様に一般化する.
\begin{mydef}[label=def:weightspacerep]{ウエイトの極大ベクトル}
	$\mathfrak{g}$-加群$V$に対し,ルートの底$\Delta$を固定する.$V$のそのウエイト$\lambda$の固有空間の元$v \neq 0$が,
	\begin{align}
		\mathfrak{g}_\alpha \btr v = 0\quad  (\forall \alpha \succ 0)  \IFF  \mathfrak{g}_\alpha \btr v = 0\quad  (\forall \alpha \in \Delta)
	\end{align}
	を満たすとき,$v$を\textbf{極大ベクトル}(maximal vector)と呼ぶ.
\end{mydef}
$\Longleftarrow$は,命題\ref{prop:root-decomp-int,breakable}の$[g_\alpha, g_\beta] = g_{\alpha + \beta}$より言える.\\
無限次元では存在すら保証されない.一方有限次元ではBorel部分代数(極大\hyperref[def:solvable-LieAlg]{可解}部分代数,第4章)$\mathfrak{b}(\Delta) = \mathfrak{h} + \oplus_{\alpha \succ 0} \mathfrak{g}_\alpha$を考えると,\hyperref[thm:Lie]{Lieの定理}と\hyperref[root-decomp-basic1]{$g_\alpha$の冪零性}から,$\mathfrak{g}_\alpha$の作用で$0$になるような共通の固有ベクトルが存在する.

極大ベクトルで生成される$\mathfrak{g}$-加群を考える.
\begin{mydef}[label=def:highest-weight-module]{最高ウェイト加群}
	ウェイト$\lambda$の極大ベクトル$v^+$と,$\mathfrak{g}$の\hyperref[def:univ-env-alg]{普遍包絡代数}$\mathfrak{U}(\mathfrak{g})$に対し,$V = \mathfrak{U}(\mathfrak{g}) \btr v^+$を満たすとき,$\lambda$を$V$の\textbf{最高ウェイト}(highest weight)と呼び,$V$は(ウェイト$\lambda$の)\textbf{最高ウェイト加群}(または標準巡回(stnadard cyclic)加群)と呼ぶ.
\end{mydef}
最高ウェイト加群の構造は単純.命題\ref{prop:root-decomp-ortho}より,正ルート$\alpha$に対し,$x_\alpha \in \mathfrak{g}_\alpha\setminus\{0\}$を選ぶと,$y_\alpha \in \mathfrak{g}_{-\alpha}$が存在し,$[x_\alpha, y_\alpha] = h_\alpha \in \mathfrak{h}$となる.\\
ルート系$(\mathbb{E}, \Phi)$で定義された\hyperref[def:base-root,breakable]{半順序}同様に,Lie代数のウェイトの\textbf{半順序}$\prec \subset \mathfrak{h}^* \times \mathfrak{h}^*$を
\begin{equation}
	\mu \prec \lambda  \DEF  \exists! \Familyset[\big]{k_\alpha}{\alpha \in \Delta} \in \prod_{\alpha \in \Delta} \mathbb{R}_{\ge 0},\; \lambda - \mu = \sum_{\alpha \in \Delta} k_\alpha \alpha
\end{equation}
と定義する.

\begin{mylem}[label=lem:max-sub-quot-mod]{}
	加群$M$の部分加群$N$に対し,
	\begin{equation}
		\text{$N$が極大} \IFF \text{商加群$M/N$は既約}
	\end{equation}
\end{mylem}
\begin{proof}
	$\Leftrightarrow$のどちらも対偶を示す.\\
	($\Rightarrow$)商加群の非自明部分加群$K$が存在したとすると,$N \subsetneq K + N \subsetneq M$より,$N$は極大でない.\\
	($\Leftarrow$)$N \subsetneq K \subsetneq M$を満たす部分加群$K$が存在したとすると,$M / K$は$M / N$の非自明部分加群である.
\end{proof}
\begin{mytheo}[label=thm:hwmodule]{}
	$V$を最高ウェイト$\mathfrak{g}$-加群,$\lambda$を最高ウェイト,$v^+ \in V_\lambda$を極大ベクトル,$\Phi^+ = \{\beta_1, \ldots, \beta_m\}$を正ルートの集合とする.
	\begin{enumerate}
		\item $V = \Span\{y_{\beta_1}^{i_1}\cdots y_{\beta_m}^{i_m} \btr v^+\ (i_j \in \mathbb{Z}_{\geq 0})\}$.特に,ウエイト空間の直和である.
		\item\label{thm:hwmodule-b} $V$の任意のウェイトについて,$\mu \prec \lambda$(最高ウェイトと呼ばれる所以).
		\item\label{thm:hwmodule-c} $\forall \mu \in \mathfrak{h}^*$に対し,$\dim V_\mu < \infty$.また,$\dim V_\lambda = 1$.
		\item\label{thm:hwmodule-d} 任意の$V$の部分加群はウエイト空間の直和.
		\item\label{thm:hwmodule-e} $V$は\hyperref[def:irr]{直既約$\mathfrak{g}$-加群}であり,唯一の極大部分加群と対応する既約な商をもつ.
		\item 全ての非零な準同型な$V$の像もウエイト$\lambda$の巡回加群.
	\end{enumerate}
\end{mytheo}
\begin{proof}
	\begin{description}
		\item[\textbf{(1)}] $\mathfrak{n}^- = \bigoplus_{\alpha \prec 0}\mathfrak{g}_\alpha,\quad  \mathfrak{b} = \mathfrak{b}(\Delta)$とする.PBW定理(の系\ref{col:PBW-D}),\ref{def:univ-env-alg-module}より,
		\begin{equation}
			\mathfrak{U}(\mathfrak{g}) \btr v^+ = \mathfrak{U}(\mathfrak{n}^-)\mathfrak{U}(\mathfrak{b}) \btr v^+ = \mathfrak{U}(\mathfrak{n}^-) \btr (\mathbb{K}v^+)\quad  (\because \text{$B$の共通の固有ベクトル})
		\end{equation}
		$\mathfrak{U}(\mathfrak{n}^-)$の基底は$y_{\beta_1}^{i_1}\cdots y_{\beta_m}^{i_m}$だったから,$V$はこのベクトルの集合で張られる.
		\item[\textbf{(2)}] 補題\ref{lem:weightspacerep}より,
		\begin{equation}
			\label{eq:hwmodule-v}
			v \coloneqq y_{\beta_1}^{i_1}\cdots y_{\beta_m}^{i_m}v^+
		\end{equation}
		のウェイトは$\mu = \lambda - \sum_j i_j\beta_j$であり,$\beta_j \in \Phi^+$は単純ルートの和だったから,$\mu \prec \lambda$.
		\item[\textbf{(3)}] $\sum_j i_j\beta_j = \sum_{i=1}^l k_i\alpha_i$を満たす$(i_j,\ \beta_j)$の組み合わせは高々有限個だから,$\dim V_\mu < \infty$.特に,ウェイト$\lambda$を持つ\eqref{eq:hwmodule-v}の形のベクトルは$v^+$しかないので,$\dim V_\mu = 1$.
		\item[\textbf{(4)}] $W$を$V$の部分加群とする,$\forall w \in W \subset V$は異なるウェイト空間$V_{\mu_i}$の元$v_i$の和で書ける.$\forall v_i \in W$を背理法で示す.\\
		$W$は線型空間なので,$w = v_1 + \cdots + v_n \in W,\ (\forall v_i \notin W)$を(背理法の仮定として)仮定して良い.$n=1$はあり得ないので,$n > 1$とする.ここで,
		\begin{equation}
			\exists h \in \mathfrak{h},\quad  h \btr w = \sum_{i=1}^n \mu_i(h) v_i \in W\ \mu_1(h) \neq \mu_2(h)
		\end{equation}
		とでき,
		\begin{equation}
			(h - \mu_1(h)1) \btr w = \sum_{i=2}^n (\mu_i(h) - \mu_1(h)) v_i \in W \setminus \{0\}
		\end{equation}
		となる.これを繰り返すと,$v_n \in W$となり矛盾する.
		\item[\textbf{(5)}] $V$の部分加群が$V_\lambda$を含むと,(1)より$V$自身となる.(4)と合わせて,$V$の非自明な部分$\mathfrak{g}$-加群は$V_\lambda$以外のウェイト空間の直和でかけるから,これら全ての部分加群の和$W$も部分$\mathfrak{g}$-加群である.よって$V$は,唯一の極大部分加群$W$と唯一の既約な商(補題\ref{lem:max-sub-quot-mod})をもつ.全ての非自明部分加群は$W$に含まれ,$V_\lambda$を含まないから,$V$は部分加群の和で表せない.i.e. 直既約$\mathfrak{g}$-加群である.\\
		\item[\textbf{(6)}] \textbf{(1)}の生成系を実際に準同型で送ったものなので,巡回加群.
	\end{description}
\end{proof}

\begin{mycol}[label=col:highest-weight-uni]{$\mathfrak{g}$-加群の極大ベクトルの唯一性}
	既約な最高ウェイト$\mathfrak{g}$-加群$V$の,(最高ウェイト$\lambda$の)$v^+$は高々スカラー倍の違いを除いて唯一.
\end{mycol}
\begin{proof}
	$w^+$をウェイト$\lambda'$の極大ベクトルとすると,$\mathfrak{U}(\mathfrak{g}) \btr w^+$は$V$の部分加群である.$V$は既約だったので,$\mathfrak{U}(\mathfrak{g}) \btr w^+ = V$.定理\ref{thm:hwmodule}\ref{thm:hwmodule-b}より,$\lambda = \lambda'$.よって定理\ref{thm:hwmodule}\ref{thm:hwmodule-c}より,$w^+$は$v^+$のスカラー倍.
\end{proof}

\subsection{存在と唯一性}
各ウェイト$\lambda \in \mathfrak{h}^*$に対し,(同型を除いて)唯一の既約な最高ウェイト$\mathfrak{g}$-加群が存在する事を示す(無限次元でも成立).唯一性の方の証明は,第4章14.2の定理の証明と似ている上に単純になっているらしい.
\begin{mytheo}[label=thm:hwmodule-unique]{唯一性}
	$V, W$を最高ウエイト$\lambda$の最高ウェイト加群とする.$V, W$が共に既約ならば同型.
\end{mytheo}
\begin{proof}
	$\mathfrak{g}$-加群$X  = V \oplus W$を考える.$v^+, w^+$をそれぞれ$V, W$のウェイト$\lambda$の極大ベクトルとすると,
	\begin{equation}
		X = V \oplus W = (\mathfrak{U}(\mathfrak{g}) \btr v^+) \oplus (\mathfrak{U}(\mathfrak{g}) \btr w^+) = \mathfrak{U}(\mathfrak{g}) \btr (v^+, w^+)
	\end{equation}
	より$x^+ = (v^+, w^+)$は$X$のウェイト$\lambda$の極大ベクトルとなる.$x^+$で生成される$X$の(最高ウェイト)部分加群を$Y$とし,$p:\ Y \to V,\ p':\ Y \to W$をそれぞれ$X$の第一,第二引数を取ってくる射影に誘導される写像とすると,
	\begin{align}
		&p(x^+) = v^+,\quad  p'(x^+) = w^+ \\
		&\Im p = V,\quad  \Im p' = W
	\end{align}
	より,$p,\ p'$は共に$\mathfrak{g}$-加群の全射準同型.
	\begin{center}
		\begin{tikzcd}[row sep=large, column sep=large]
			Y \ar[d, "p"']\ar[r, "p''"] & W \\
			V \ar[ur, "\sim"']&
		\end{tikzcd}
	\end{center}
	よって加群の同型定理と定理\ref{thm:hwmodule}\ref{thm:hwmodule-e}より,$V, W$は最高ウェイト加群$Y$の既約な商加群として同型.
\end{proof}
存在について,まず,\textbf{Verma加群}と呼ばれる最高ウェイト加群$M(\lambda)$を2通りの方法で構成する.

まずは誘導加群の方法.Borel部分代数を$\mathfrak{b} = \mathfrak{b}(\Delta)$とすると,$\mathfrak{b}$-加群としての最高ウェイト加群は極大ベクトル$v^+$で生成される一次元部分加群を含む.よってまずは$\lambda \in \mathfrak{h}^*$を固定し,$v^+$を基底に持つ一次元ベクトル空間$\mathbb{K}_\lambda$に対し,$\mathbb{K}_\lambda$上の$\mathfrak{b}$の作用を
\begin{align}
	h \btr v^+ &= \lambda(h) v^+,\quad  (h \in \mathfrak{h}) \\
	x_\alpha \btr v^+ &= 0\quad  (x_\alpha \in \mathfrak{g}_\alpha,\ \alpha \in \Phi^+)
\end{align}
と定義する.$\mathbb{K}_\lambda$は$\mathfrak{b}$-加群となる.それと同時に$\mathbb{K}_\lambda$は$\mathfrak{U}(\mathfrak{b})$-加群でもあるので,テンソル積
\begin{equation}
	M(\lambda) = \mathfrak{U}(\mathfrak{g}) \otimes_{\mathfrak{U}(\mathfrak{b})} \mathbb{K}_\lambda
\end{equation}
を定義すると,自然な左作用
\begin{equation}
	x \btr (y \otimes v) = (xy) \otimes v,\quad  (x, y \in \mathfrak{U}(\mathfrak{g}),\ v \in \mathbb{K}_\lambda)
\end{equation}
により$\mathfrak{U}(\mathfrak{g})$-加群となる.\\
次に$M(\lambda)$がウェイト$\lambda$の最高ウェイト加群であることを示す.$1 \otimes v^+$は$M(\lambda)$を生成する.また,\hyperref[thm:PBW]{PBW定理}の系\ref{col:PBW-D}より,$\mathfrak{U}(\mathfrak{g})$は$1$および単項式$y_{\beta_1}^{i_1}\cdots y_{\beta_m}^{i_m}$を基底に持つ自由$\mathfrak{U}(\mathfrak{b})$-加群であったから,$1 \otimes v^+$は非零である.よって,$1 \otimes v^+$はウェイト$\lambda$の極大ベクトルである.\\
\hyperref[thm:PBW]{PBW定理}の系\ref{col:PBW-D}:$\mathfrak{U}(\mathfrak{g}) \simeq \mathfrak{U}(\mathfrak{n}^-) \otimes \mathfrak{U}(\mathfrak{b})$より,$M(\lambda)$を$\mathfrak{U}(\mathfrak{n}^-)$-加群と見ることができ,
\begin{equation}
	M(\lambda) \simeq \mathfrak{U}(\mathfrak{n}^-) \otimes \mathbb{K} \simeq  \mathfrak{U}(\mathfrak{n}^-)
\end{equation}

次に生成系と関係式の方法で構成し,同型であることを示す.\\
正ルートの集合$\Phi^+$および,$h_\alpha - \lambda(h_\alpha)1\ (\alpha \in \Phi)$で生成される左イデアルを$I(\lambda)$とする.
\begin{equation}
	I(\lambda) \btr v^+ = 0
\end{equation}
より,$\mathfrak{U}(\mathfrak{g})$-加群の準同型定理から,$\mathfrak{U}(\mathfrak{g}) / I(\lambda) \to M(\lambda)$は$1 + I \mapsto 1\otimes v^+$となる.再び,\hyperref[thm:PBW]{PBW定理}の系\ref{col:PBW-D}より,$\mathfrak{U}(\mathfrak{b}) + I \mapsto \mathbb{K}(1\otimes v^+)$となる.よって,この標準的射影は一対一対応であり,
\begin{equation}
	U(\lambda) / I(\lambda) \simeq M(\lambda)
\end{equation}

\begin{mytheo}[label=thm:hwmodule-exist]{存在}
	$\forall \lambda \in \mathfrak{h}^*$に対し,ウェイト$\lambda$の既約な最高ウェイト加群が存在する.
\end{mytheo}
\begin{proof}
	上で構成された$M(\lambda)$は,ウェイト$\lambda$の最高ウェイト加群で,唯一の極大部分加群$Y(\lambda)$をもつ(定理\ref{thm:hwmodule}\ref{thm:hwmodule-d}).よって,
	\begin{equation}
		L(\lambda) = M(\lambda) / Y(\lambda)
	\end{equation}
	はウェイト$\lambda$の既約最高ウェイト加群(定理\ref{thm:hwmodule}\ref{thm:hwmodule-e}).
\end{proof}
ウェイト$\lambda \in \mathfrak{h}^*$の既約最高ウェイト加群$L(\lambda)$は一意的に定まるから,$L(\lambda)$のウェイトの集合を$\Pi(\lambda)$と書く.
\begin{mytheo}[label=thm:finite-irr-mod]{有限次元既約加群の構造}
	$V$を有限次元既約$\mathfrak{g}$-加群とすると,
	\begin{equation}
		\exists \lambda \in \mathfrak{h}^*,\quad  V \simeq L(\lambda)
	\end{equation}
\end{mytheo}
\begin{proof}
	有限次元には,ウェイト$\lambda$の極大ベクトル$v^+$の存在が保証されていた.$L(\lambda) = \mathfrak{U}(\mathfrak{g}) \btr v^+$は既約加群$V$の部分加群で,非零なので$V \simeq L(\lambda)$.
\end{proof}

\section{有限次元加群}
既約(最高ウェイト)$\mathfrak{g}$-加群$L(\lambda)$が有限次元であるためのウェイト$\lambda$の条件を調べる.
\subsection{有限次元に対する必要条件}
各単純ルート$\alpha_i$に対し,
	\begin{equation}
		\mathfrak{s}_i \coloneqq \mathfrak{g}_{\alpha_i} \oplus \mathfrak{g}_{-\alpha_i} \oplus [\mathfrak{g}_{\alpha_i}, \mathfrak{g}_{-\alpha_i}] \simeq \Lsl(2,\, \mathbb{K})
	\end{equation}
	とする($\simeq$は定理\ref{prop:root-decomp-ortho}).$L(\lambda)$は有限次元$\mathfrak{s}_i$-加群でもあり,$\mathfrak{g}$の極大ベクトル$v^+$は$\mathfrak{s}_i$の極大ベクトルでもある.特に,$\mathfrak{s}_i$の\hyperref[def:toral-subLieAlg]{極大トーラス}$\mathfrak{h}_i = [\mathfrak{g}_{\alpha_i}, \mathfrak{g}_{-\alpha_i}]$に対し,$h_i \in \mathfrak{h}_i$の作用は,固有値$\lambda(h_i)$で完全に決まる.その固有値が非負整数になることは既に知っている.
\begin{mytheo}[label=thm:necessary-for-finite]{ウェイトの整性の必要性}
	有限次元既約$\mathfrak{g}$-加群$V$の最高ウェイトを$\lambda$,単純ルートを$\alpha$,$h_i \in \mathfrak{h}_{\alpha_i} = [\mathfrak{g}_{\alpha_i}, -\mathfrak{g}_{\alpha_i}]$とすると,\\
	$\lambda(\mathfrak{h}_i)$は非負整数.
\end{mytheo}
\begin{proof}
	定理\ref{thm:irr-sl2}.
\end{proof}
この定理の系2.4.2:より,最高ウェイトでない任意の$V$のウェイト$\mu$でも成り立つ:
\begin{equation}
	\label{eq:int-weight}
	\mu(h_i) = \sspair{\mu}{\alpha_i} \in \mathbb{Z},\quad  (1 \leq i \leq l)
\end{equation}
これは,\hyperref[def:root-lattice]{ルート系の整ウェイト}に対応している.これをLie代数の\textbf{整ウェイト}と呼ぶのは自然だろう.\hyperref[def:domweight]{優,強いウェイト},\hyperref[def:fundamental-weight]{基本優ウェイト}も同様に定義され,当然,ルート系の整ウェイトに対する定理は全て成り立つ.\\
ルート系で定義した優ウェイトについては,優かつ整と呼ぶ方が親切かもしれないが,ここでは単に優と呼ぶことにする.
\subsection{有限次元に対する十分条件}
\begin{mytheo}[label=thm:suff-for-finite]{ウェイトの優性の十分性}
	$\lambda \in \mathfrak{h}^*$を優ウェイトとする.\\
	このとき,既約$\mathfrak{g}$-加群$V = L(\lambda)$は有限次元で,$V$のウェイトの集合$\Pi(\lambda)$は,Weyl群の作用によって置換され,$\dim V_\mu = \dim V_{\sigma\mu},\ \forall \sigma \in \mathscr{W}$を満たす.
\end{mytheo}
\begin{mycol}[label=col:suff-for-finite]{}
	$\lambda \mapsto L(\lambda)$は,優ウェイト$\Lambda^*$と有限次元既約$\mathfrak{g}$-加群(の同型の類)の一対一対応を誘導する.
\end{mycol}
\begin{proof}
	優ウェイトは整ウェイトなので,有限次元であるための\hyperref[thm:necessary-for-finite]{必要条件}を満たす.定理\ref{thm:finite-irr-mod}より一対一対応.
\end{proof}
十分条件を証明しよう.
\begin{mylem}[label=lem:suff-for-finite]{}
	半単純Lie代数$\mathfrak{g}$の\hyperref[def:univ-env-alg]{普遍包絡代数}を$\mathfrak{U}(\mathfrak{g})$,単純ルートを$\alpha_1, \ldots, \alpha_l$,$x_i \in \mathfrak{g}_{\alpha_i} \setminus \{0\},\ y_\alpha \in \mathfrak{g}_{-\alpha_i} \setminus \{0\}, h_i = [x_i, y_i] \in \mathfrak{h}$とすると,任意の$k \in \mathbb{Z}_{\geq 0},\ 1 \leq i, j \leq l$に対し以下が成り立つ.
	\begin{enumerate}
		\item $[h_j,\ y_i^{k+1}] = -(k+1)\alpha_i(h_j)y_i^{k+1}$
		\item $[x_j,\ y_i^{k+1}] = -(k+1)y_i^k(k - h_i)\delta_{ij}$
	\end{enumerate}
\end{mylem}
\begin{proof}
	\begin{description}
		\item[\textbf{(a)}] $k$についての数学的帰納法.$k=0$は\hyperref[prop:semisimple-Lie-alg-relation]{半単純Lie代数の関係式}より明らか.ある$k$で成り立つとき,
		\begin{equation}
			[h_j, y_i^{k+1}] = [h_j, y_i]y_i^k + y_i[h_j, y_i^k] = -\alpha_i(h_j)y_i^{k+2} - (k+1)\alpha_i(h_j)y_i^{k+2} = - (k+2)\alpha_i(h_j)y_i^{k+2}
		\end{equation}
		より任意の$k$で成り立つ.
		\item[\textbf{(b)}] $i \neq j$の場合は,補題\ref{lem:base}より,$\alpha_j - \alpha_i$がルートでないことより従う.\\
		$i = j$のとき,$k = 0$は$y_i, h_i$の選び方より明らか.ある$k$で成り立つとき,
		\begin{equation}
			[x_i, y_i^{k+1}] = [x_i, y_i]y_i^k + y_i^k[x_i, y_i] = h_iy_i^k - (k+1)y_i^{k+1}(k - h_i) = -(k+2)y_i^{k+1}(k - h_i)
		\end{equation}
		より任意の$k$で成り立つ.
	\end{description}
\end{proof}
\hyperref[thm:suff-for-finite]{十分条件の証明}のポイントは,$V$のウェイトが$\mathscr{W}$で置換されることから有限次元であることを示すことである(\hyperref[thm:Serre]{Serreの定理}参照).
\begin{proof}
	$g$-加群$V$を表現と見たい場合は$\phi$とする.$V$のウェイト$\lambda$の極大ベクトルを$v^+$とし,$m_i = \lambda(h_i)$とする(仮定より$\lambda$は優ウェイトなので非負整数).
	\begin{description}
		\item[\textbf{Step 1: $w_i \coloneqq (y_i^{m_i + 1} \btr v^+) = 0$}] 
		
		$x_i \btr v^+ = 0$と補題\ref{lem:suff-for-finite}(1)に注意すると,
		\begin{equation}
			x_i \btr w_i = y_i^{m_i + 1} \btr (x_j \btr v^+) - (m_i + 1)y_i^{m_i} \btr (m_i - m_i)v^+ = 0
		\end{equation}
		となる.また,$\alpha_j - \alpha_i$はルートでないから$x_j \btr w_i = 0\ (1 \leq j \leq l)$.もし$w \neq 0$とすると,最高ウェイト$\lambda - (m_i + 1)\alpha_i \neq \lambda$が存在することになり.既約最高ウェイト加群の\hyperref[col:highest-weight-uni]{最高ウェイトの唯一性}に反する.
		\item[\textbf{Step 2: $V$に非零な有限次元$\mathfrak{s}_i$-加群が含まれる}]  
		
		部分空間$V_i = \{v^+,\ y_i \btr v^+,\ \ldots,\ y_i^{m_i}\btr v^+ \}$を考える.\textbf{(Step 1)}と合わせると$y_i$の作用について不変.補題\ref{lem:suff-for-finite}(1)より$x_i, h_i$の作用についても不変.よって,$\mathfrak{s}_i$-加群として,$V_i$は$V$の部分加群.
		\item[\textbf{Step 3: $V$は有限次元部分$\mathfrak{s}_i$-加群の和}]  
		
		$V' = \sum_{i = 1}^l V_i$,$W$を$V$の任意の有限次元$\mathfrak{s}_i$-部分加群とすると,$x_i \btr W,\ y_i \btr W$も有限次元$\mathfrak{s}_i$-部分加群.命題 2.5.4より,任意のルート$\alpha \in \Phi$,$x_\alpha = \mathfrak{g}_\alpha \setminus \{0\}$に対し,$x_\alpha \btr W$は有限次元$\mathfrak{s}_i$-部分加群となり,$W$の部分空間.よって,$V'$は$\mathfrak{g}$-加群.\\
		$V$は既約で,\textbf{(Step 2)}より$V' \subset V$は非零だったから,$V = V'$.
		\item[\textbf{Step 4: $1 \leq i \leq l$に対し,$\phi(x_i),\ \phi(y_i)$は$V$の\hyperref[def:locally-nilpotent]{局所冪零}自己準同型}] 
		
		補題\ref{lem:suff-for-finite},\textbf{(Step 1)}より,$x_i, y_i$の作用は局所冪零.\hyperref[col:JC]{Jordan分解の保存}より,$\phi(x_i),\ \phi(y_i)$も局所冪零.
		\item[\textbf{Step 5: (Step 4)より$s_i = \exp\phi(x_i)\exp\phi(-y_i)\exp\phi(x_i)$はwell-defined}] \hyperref[def:locally-nilpotent]{局所冪零}の定義の後の文章参照のこと
		\item[\textbf{Step 6: $\mu$を$V$のウェイトとすると,$s_i(V_\mu) = V_{\sigma_i\mu}\ (\sigma_i: \alpha_i\text{に関する鏡映})$}] 
		
		\textbf{(Step 3)}より$V_\mu$は有限次元$\mathfrak{s}_i$-部分加群$V' = V$に含まれる.$s_i|_{V'}$は存在しない$\Lsl(2, \mathbb{K})$の既約表現の分類の自己同型$\tau$で,$s_i(V_\mu) = V_{\sigma_i\mu}$が従う.
		\item[\textbf{Step 7: ウェイトの集合$\Pi(\lambda)$は$\mathscr{W}$の作用で不変.また,$\dim V_\mu = \dim V_{\sigma\mu},\ \forall \mu \in \Pi(\lambda),\ \sigma \in \mathscr{W}$}] 
		
		Weyl群$\mathscr{W}$は単純ルートに関する鏡映で生成されたから,\textbf{(Step 6)}より従う.
		\item[\textbf{Step 8: $\Pi(\lambda)$は有限集合.}] 
			
		補題\ref{lem:dom-weight-B}より,$\mu \prec \lambda$を満たす優ウェイト$\mu$の集合は有限なので,$\mathscr{W}$で写した集合も有限.定理\ref{thm:hwmodule}\ref{thm:hwmodule-b}より,$\Pi(\lambda)$はこの部分集合だから,有限集合.
		\item[\textbf{Step 9: $V$は有限次元.}] 
		
		定理\ref{thm:hwmodule}\ref{thm:hwmodule-c}より,$V_\mu\ (\mu \in \Pi(\lambda))$は有限次元.$\mu \in \Pi(\lambda)$上の和は有限和なので,$V$は有限次元.
	\end{description}
\end{proof}

\subsection{ウェイトstringとウェイト図}
ここでも優ウェイト$\lambda \in \Lambda^+$に対する有限次元既約$\mathfrak{g}$-加群$V = L(\lambda)$を考える.\\
補題\ref{lem:weightspacerep}より,部分加群$W = \oplus_{i \in \mathbb{Z}} V_{\mu + i\alpha} \subset V$は$S_\alpha$-不変.\\
$(\mu + \alpha\mathbb{Z}) \cap \Pi(\lambda)$なる整ウェイトの集合を\textbf{$\bm{\alpha}$-string through $\bm{\mu}$}という.$\Pi(\lambda)$は有限集合であったから,
\begin{align}
	r &\coloneqq \max \bigl\{\, i \in \mathbb{Z}_{\ge 0} \bigm| \mu - i \alpha \in \Pi(\lambda) \,\bigr\}, \\
	q &\coloneqq \max \bigl\{\, i \in \mathbb{Z}_{\ge 0} \bigm| \mu + i \alpha \in \Pi(\lambda) \,\bigr\} 
\end{align}
が定義されるので,命題\ref{prop:a-string-basic}と同様のものが成り立つ.
\begin{myprop}[label=prop:weight-diagram]{$\alpha$-string through$\mu$の性質}
	$\alpha \neq \pm \mu$を充たす任意の$\alpha \in \Phi,\ \mu \in \Pi(\lambda)$に対して
	\begin{align}
		r &\coloneqq \max \bigl\{\, i \in \mathbb{Z}_{\ge 0} \bigm| \mu - i \alpha \in \Pi(\lambda) \,\bigr\}, \\
		q &\coloneqq \max \bigl\{\, i \in \mathbb{Z}_{\ge 0} \bigm| \mu + i \alpha \in \Pi(\lambda) \,\bigr\} 
	\end{align}
	とおく.このとき以下が成り立つ:
	\begin{enumerate}
		\item 
		\hyperref[def:a-sting]{$\alpha$-string through $\mu$}は $\mathbb{E}$ の部分集合
		\begin{align}
			\label{eq:a-string}
			\bigl\{\, \mu + i\alpha \in \mathbb{E} \bigm| -r \le \lambda \le q \,\bigr\} 
		\end{align}
		に等しい.i.e. 
		\begin{align}
			i \in \mathbb{Z} \AND \mu + i\alpha \in \Pi(\lambda) \IMP -r \le i \le q
		\end{align}
		である.
		\item $\sigma_{\alpha}(\mu + i\alpha) = \mu - i\alpha$.i.e. \hyperref[def:a-sting]{$\alpha$-string through $\mu$}は鏡映 $\sigma_\alpha$ の作用の下で不変である.
		\item $r-q = \sspair{\mu}{\alpha}$.特に\hyperref[def:a-sting]{$\alpha$-string through $\mu$}の長さは $4$ 以下である.
		\item 優ウェイト$\lambda$に対し,$\Pi(\lambda)$は\hyperref[weight-saturated]{飽和集合}.
		\item 任意のウェイト$\mu$について
		\begin{equation}
			\mu \in \Pi(\lambda)  \IFF  \sigma(\mu) \prec \lambda\quad  (\forall \sigma \in \mathscr{W})
		\end{equation}
	\end{enumerate}
\end{myprop}
\begin{proof}
	(1)-(3): 命題\ref{prop:a-string-basic}の証明の$\Phi$を$\Pi(\lambda)$に直したもの.\\
	(4): 以上から$\Pi(\lambda)$は\hyperref[weight-saturated]{飽和集合}の定義そのものを満たすと言える.\\
	(5): 補題 3.5.1と補題 3.5.6を組み合わせると成り立つ.
\end{proof}
ルートと$\Pi(\lambda)$の元を書いた図を\textbf{ウェイト図}(weight diagram)と呼ぶ.重複度とやらも書き込みたいので,具体例は次節に回す.

\subsection{$L(\lambda)$の生成系と関係式}
$\mathfrak{U}(\mathfrak{g}) \to M(\lambda) \to L(\lambda)$という準同型を$\lambda$が優ウェイトの場合にさらに考察する.\\
$M(\lambda) = \mathfrak{U}(\mathfrak{g}) / I(\lambda)$であり,$M(\lambda)$の極大ベクトル$1$に対し,$I(\lambda) \btr 1 = 0$となる.\\
$\mathfrak{U}(\mathfrak{g})$のイデアルで,$L(\lambda)$の極大ベクトルに作用すると$0$となるようなものを$J(\lambda)$とする.$I(\lambda) \subset J(\lambda)$より,準同型
\begin{equation}
	M(\lambda) = \mathfrak{U}(\mathfrak{g}) / I(\lambda) \lto L(\lambda) \simeq \mathfrak{U}(\mathfrak{g}) / J(\lambda)
\end{equation}
を誘導する.また,定理\ref{thm:suff-for-finite}の証明\textbf{(Step 1)}より,$y_i^{m_i + 1} \in J(\lambda)$である.
\begin{mylem}[label=lem:Lie-bracket-power]{}
	$A$を標数$0$の体$\mathbb{K}$上の結合代数,$\ad y(z) = yz - zy\ (y, z \in A)$とすると,\\
	$\forall y, z \in A,\ \forall k \in \mathbb{N}$に対し,以下が成り立つ.
	\begin{equation}
		(\ad y^k)(z) = \sum_{i=1}^k \binom{k}{i}(\ad y)^i(z)y^{k-i}
	\end{equation}
\end{mylem}
\begin{proof}
	$(\ad y^1)(z) = (\ad y)^1(z)$.ある$k$で成り立つとき,以下より$k+1$でも成り立つ.
	\begin{align}
		(\ad y^{k+1})(z) &= \bigl(\ad y(z)y^k + \ad y^k(\ad y(z))\bigr) + \ad y^k(z)y = \sum_{i=1}^{k+1} \Biggl(\binom{k}{i-1} + \binom{k}{i}\Biggr)(\ad y)^i(z)y^{k+1-i} \\
		&= \sum_{i=1}^{k+1} \binom{k+1}{i}(\ad y)^i(z)y^{k+1-i}
	\end{align}
	
\end{proof}
\begin{mytheo}[label=thm:generator-of-J]{$J(\lambda)$の生成系}
	$\lambda \in \Lambda^+,\ m_i = \sspair{\lambda}{\alpha_i}\ (1 \leq i \leq l)$とすると,$J(\lambda)$は$I(\lambda)$と$y_i^{m_i + 1}\  (1 \leq i \leq l)$で生成される.
\end{mytheo}
\begin{proof}
	$I(\lambda)$と$y_i^{m_i + 1}\  (1 \leq i \leq l)$で生成されるイデアルを$J'(\lambda)$とする.まず,$L'(\lambda) = \mathfrak{U}(\mathfrak{g}) / J'(\lambda)$が有限次元であることを示す.\\
	$I(\lambda) \subset J'(\lambda)$より,$L'(\lambda)$は$M(\lambda)$の極大ベクトル$1$に関する最高ウェイト加群.定理\ref{thm:suff-for-finite}の証明\textbf{(Step 3)}「有限次元$\mathfrak{s}_i$-部分加群の和」を示せば,\textbf{(Step 4)}以降はそのまま使えて$L'(\lambda)$は有限次元となる.その\textbf{(Step 3)}の証明の中でも,$x_i,\ y_i$が$L'(\lambda)$上\hyperref[def:locally-nilpotent]{局所冪零}であることを示せば,有限次元となる.\\
	$x_i^k$はウェイトを$k\alpha_i$増やすので,ある$k$で最高ウェイトを越えるため,\hyperref[def:locally-nilpotent]{局所冪零}.$y_i^k$については,まず定理 6.1.1(1)より,
	\begin{equation}
		V'(\lambda) = \Span\bigl\{y_{i_1}\cdots y_{i_t} \bigm| 0 \leq i_j \leq l,\ t \in \mathbb{Z}_{\geq 0} \bigr\} + J'(\lambda) 
	\end{equation}
	である.ここで,補題\ref{lem:Lie-bracket-power}を普遍包絡代数に対して考えると$\alpha$-stringの長さは高々4だから,和は$i=3$で止まる.つまり,$[y^{k+3}, z] = (\text{多項式})\times y_i^k$.よって,
	\begin{equation}
		y_i^ky_{i_1}\cdots y_{i_t} \in J'(\lambda) \IMP  y_i^{k+3}y_{i_0}y_{i_1}\cdots y_{i_t} \in J'(\lambda)
	\end{equation}
	となる.$J'(\lambda)$の定義より$y_i^{m_i + 1}1 = y_i^{m_i + 1} \in J'(\lambda)$だから,単項式の長さ$t$に関する帰納法より,$y_i$は局所冪零.
	
	よって,$L'(\lambda)$は有限次元最高ウェイト加群(または$0$)なので,定理\ref{thm:hwmodule}\ref{thm:hwmodule-e}より直既約で,\hyperref[thm:Weyl]{完全可約性に関するWeylの定理}より既約(または$0$).一方定義より$J'(\lambda) \subset J(\lambda)$なので,$L(\lambda) \subset L'(\lambda)$(より$0$でない).同型でないと,$L'(\lambda)$の既約性に矛盾するから,
	\begin{equation}
		L(\lambda) \simeq L'(\lambda)  \IFF  J(\lambda) \simeq J'(\lambda)
	\end{equation}
\end{proof}

\section{重複度公式}
この節では有限次元加群を考える.
\begin{mydef}[label=def:mutiplicity]{ウエイトの重複度}
	$\mu \in \mathfrak{h}^*$を整ウェイト,$\lambda \in \Lambda^+$とする.有限次元既約$\mathfrak{g}$-加群$L(\lambda)$に対し,ウェイト$\mu$のウェイト空間の次元
	\begin{equation}
		m_\lambda(\mu) \coloneqq \dim L(\lambda)_\mu
	\end{equation}
	のことを$L(\lambda)$における$\mu$の\textbf{重複度}(multiplicity)と呼ぶ($L(\lambda)$のウェイトでなければ$0$とする).
\end{mydef}
目標は,ウェイトの重複度$m_\lambda(\mu)$に対する再帰的公式である,Freudenthalの公式を示すこと.

\subsection{普遍Casmir演算子}
\hyperref[thm:Weyl]{完全可約性に関するWeylの定理}の証明で出てきた,半単純Lie代数$\mathfrak{g}$の表現の元であるCasimir演算子$c_\phi(\beta)$を思い出そう.$c_\phi(\beta)$を普遍包絡代数$\mathfrak{U}(\mathfrak{g})$の元とみると,
\begin{align}
	c_\phi(\beta) &\coloneqq \sum_{\mu=1}^{\dim\mathfrak{g}} \phi(e_\mu)\phi(e^\mu) = \sum_{\mu=1}^{\dim\mathfrak{g}} \phi(e_\mu e^\mu) \\
	\label{eq:univ-Casmir-beta}
	&= \phi(c_\mathfrak{g}(\beta))\quad  \biggl(c_\mathfrak{g}(\beta) \coloneqq \sum_{\mu=1}^{\dim\mathfrak{g}} e_\mu e^\mu \biggr)
\end{align}

ここで,$\mathfrak{g}$の随伴表現を考える.その跡形式をKilling形式$\kappa$と呼んでいた.半単純Lie代数のKilling形式は非退化であったから,$\kappa$に関する双対基底を構成できる.\\
命題2.5.1より,$\alpha,\ \beta \in \mathfrak{h}^*$に対し,$\alpha \neq -\beta$であれば$g_\alpha$と$g_\beta$は直交する.よって,$\mathfrak{h} = \mathfrak{g}_0$及び$\mathfrak{g}_\alpha \oplus \mathfrak{g}_{-\alpha}$それぞれでKilling形式を考えれば良い.\\
$\mathfrak{h}$の基底を$\{h_i\}_{i=1}^l$,その双対を$\{h^i\}_{i=1}^l$とする.一方,$x_\alpha \in \mathfrak{g}_\alpha$の双対は$x^\alpha \in \mathfrak{g}_{-\alpha}$となる.特に,命題 2.5.3より
\begin{equation}
	[x_\alpha, x^\alpha] = t_\alpha = \frac{(\alpha, \alpha)}{2}h_\alpha
\end{equation}
である.このときの\eqref{eq:univ-Casmir-beta}で定義した$c_\mathfrak{g}(\beta)$に注目する.
\begin{mydef}[label=def:univ-Casmir-op]{普遍Casmir演算子}
	半単純Lie代数$\mathfrak{g}$の普遍包絡代数を$\mathfrak{U}(\mathfrak{g})$,極大トーラス$\mathfrak{h}$の基底を$\{h_i\}_{i=1}^l$,$x_\alpha \in \mathfrak{g}_\alpha \setminus \{0\}$,$h_i, x_\alpha$をKilling形式$\kappa$に関して双対をとったものをそれぞれ$h^i, x^\alpha$とする.
	\begin{equation}
		c_\mathfrak{g} \coloneqq \sum_{i=1}^l h_ih^i + \sum_{\alpha \in \Phi} x_\alpha x^\alpha \in \mathfrak{U}(\mathfrak{g})
	\end{equation}
	を(Killing形式に関する)\textbf{普遍Casmir演算子}と呼ぶ.
\end{mydef}
縮約をとっているので,$c_\mathfrak{g}$は基底によらない.$c_\phi = \phi(c_\mathfrak{g})$と$\phi(\mathfrak{g})$は可換なので,$\phi$が既約なら$\phi(c_\mathfrak{g})$はスカラーとして作用する.\\
$c_\phi$と$\phi(c_\mathfrak{g})$の関係を調べよう.
\begin{mylem}[label=lem:]{単純Lie代数の非退化対称結合双線型形式の同型性}
	$\mathfrak{g}$を単純Lie代数とし,$f(x, y),\ g(x, y)$を$\mathfrak{g}$上の非退化で対称な双線型形式とし,$\beta = f, g$について
	\begin{equation}
		\label{eq:associative-bilinear}
		\beta(x, [y, z]) = \beta([x, y], z)
	\end{equation}
	を満たすとすると,
	\begin{equation}
		\exists a \in \mathbb{K},\quad  f = ag
	\end{equation}
\end{mylem}
\begin{proof}
	各非退化双線型形式$f, g$で定まる同型$\pi_f, \pi_g\colon \mathfrak{g} \lto \mathfrak{g}^*,\ x \mapsto s_f, s_g$を
	\begin{equation}
		s_f(y) = f(x, y),\quad  s_g(y) = g(x, y)
	\end{equation}
	で定義する.\eqref{eq:associative-bilinear}より,$\pi_f, \pi_g$はそれぞれ$\mathfrak{g}$-加群として同型???\\
	$\mathfrak{g}$-加群$\pi \coloneqq \pi_f^{-1}\pi_g\colon \mathfrak{g} \to \mathfrak{g}$は$\mathfrak{g}$上の同型写像.$\mathfrak{g}$は単純なので,既約$\mathfrak{g}$-加群となり,\hyperref[col:Schur-closed]{代数閉体上のSchurの補題}より,$\pi$はスカラー倍.i.e.
	\begin{equation}
		\forall y \in \mathfrak{g},\quad  f(x, y) = g(z, y)  \IFF  \exists a \in \mathbb{K},\ z = ax
	\end{equation}
	となり,$f, g$の双線型性より$f = ag$.
\end{proof}
Lie代数$\mathfrak{g}$の非零の忠実な表現を$\phi$とする.$\Ker\phi$は$\mathfrak{g}$のイデアルとなる.\\
$\mathfrak{g}$が単純のとき,$\Ker\phi = \{0\}$だから,$\beta(x, y) \coloneqq \Tr(\phi(x)\phi(y))$は非退化で\eqref{eq:associative-bilinear}を満たす.$\phi = \ad$の場合のKilling形式$\kappa$はこれを満たすから,補題\ref{}より,$\kappa = a\beta\ (a \in \mathbb{K} \setminus \{0\})$と書ける.よって$\beta$に関する双対ベクトルは,$\kappa$に関する双対ベクトルの$1/a$倍となるから,\eqref{eq:univ-Casmir-beta}より,
\begin{equation}
	\phi(c_\mathfrak{g}) = \phi(c_\mathfrak{g}(a\beta)) = \frac{1}{a}\phi(c_\mathfrak{g}(\beta)) = \frac{1}{a}c_\phi
\end{equation}

次に,$\mathfrak{g}$が半単純Lie代数のとき.$\mathfrak{g}$は単純Lie代数$g_i\ (1 \leq i \leq t)$の直和
\begin{equation}
	\mathfrak{g} = \bigoplus_{i=1}^t\mathfrak{g}_i
\end{equation}
で書ける.基底は各$\mathfrak{g}_i$の基底の和集合にとれるので,$\mathfrak{g}$の普遍Casimir演算子は,$\mathfrak{g}_i$の普遍Casimir演算子$c_{\mathfrak{g}_i}$の和で$c_\mathfrak{g} = \sum_{i=1}^t c_{\mathfrak{g}_i}$と書ける.定理 2.2.2より,各$g_i$のKilling形式は$\kappa|_{\kappa_i \times \kappa_i}$に等しい.よって,
\begin{equation}
	\phi(c_{\mathfrak{g}_i}) = \sum_{i=1}^t \frac{1}{a_i}c_{\phi|_{\mathfrak{g}_i}}\quad  (a_i \in \mathbb{K})
\end{equation}
と書ける.次に,この$a_i$の値の求め方を示す.
\subsection{Freudenthalの公式}
$V$のウェイト$\mu \in \Pi(\lambda)$に対するウェイト空間$V_\mu$に対し,$\phi(x_\alpha)\phi(z_\alpha)$は$V_\mu \lto V_{\mu - \alpha} \lto V_\mu$より$V_\mu$の準同型.まずは,$V_\mu$上の跡を調べる.
\begin{mylem}[label=lem:Freudenthal]{ウェイト空間上の既約表現の跡形式}
	優ウェイト$\lambda \in \Lambda^+$の既約な$\mathfrak{g}$-加群を$V = L(\lambda)$(表現と見たものを$\phi$と書く),$V$のウェイトを$\mu \in \Lambda$($\Pi(\lambda)$でないことに注意),その重複度を$m(\mu)$とすると,
	\begin{align}
		\Tr_{V_\mu} \phi(c_\mathfrak{g}) &= (\mu, \mu)m(\mu) + \sum_{\alpha \in \Phi} \sum_{i=1}^\infty m(\mu + i\alpha)(\mu + i\alpha, \alpha) \\
		&= (\mu, \mu + 2\delta)m(\mu) + 2\sum_{\alpha \succ 0} \sum_{i=1}^\infty m(\mu + i\alpha)(\mu + i\alpha, \alpha)
	\end{align}
\end{mylem}
\begin{proof}
	上述のように,$\mathfrak{h}$の基底$\{h_i\}_{i=1}^l$,$\mathfrak{g}_\alpha$の基底$\{x_\alpha\}$を固定し,Killing形式$\kappa$に関する双対基底を取る.\\
	まずは,$\mu \in \Pi(\lambda)$として,$V_\mu$上の$x_\alpha x^\alpha,\ k_ik^i$の作用を直接見る.まずは$x_\alpha x^\alpha$.
	
	まず,$\mu + \alpha \notin \Pi(\lambda)$とすると,$\alpha$-string through $\mu$は,$r = \sspair{\mu}{\alpha}$に対し,$\{\mu,\ \mu - \alpha,\ \ldots,\ \mu - r\alpha\}$となる.Weylの定理より
	\begin{equation}
		W \coloneqq V_\mu \oplus \cdots \oplus V_{\mu - r\alpha}
	\end{equation}
	は既約$\mathfrak{s}_\alpha$-加群の直和となる.$w^0 \in V_\mu$を$\mathfrak{s}_\alpha$-加群としての極大ベクトルとすると,
	\begin{equation}
		w_i \coloneqq \left\{\begin{aligned}
			 &0 &&(i = -1) \\
			 &\frac{(\alpha, \alpha)^i}{2^i}y^i \btr w_0  &&(i \geq 0)
		\end{aligned} \right.
	\end{equation}
	に対し,$x^\alpha = \frac{(\alpha, \alpha)}{2}y_\alpha,\ t_\alpha = \frac{(\alpha, \alpha)}{2}h_\alpha$と補題\ref{lem:sl2-2}より,
	\begin{enumerate}
		\item $t_\alpha \btr w_i = (r - 2i)\frac{(\alpha, \alpha)}{2}w_i$
		\item $x^\alpha \btr w_i = w_{i+1}$
		\item $x_\alpha \btr w_i = i(r - i+1)\frac{(\alpha, \alpha)}{2}w_{i-1}  \WHERE i \ge 0$
	\end{enumerate}
	を満たす.特に,$w_m \in V_{\mu - m\alpha}$.よって,
	\begin{equation}
		\label{eq:Freudenthal-proof-1}
		x_\alpha x^\alpha \btr w_i = (r - i)(i + 1)\frac{(\alpha, \alpha)}{2}w_i
	\end{equation}
	鏡映$\sigma_\alpha$に対し,$\sigma_\alpha(\mu - i\alpha) = \mu + (r - i)$と定理 6.2.2より,
	\begin{equation}
		\dim V_{\mu - i\alpha} = \dim V_{\mu - (r - i)\alpha}\quad  (0 \leq i \leq r)
	\end{equation}
	となる.最高ウェイト$r - 2i = (\mu - i\alpha)(h_\alpha)$をもつベクトルの数を$n_i\ (0 \leq i \leq r/2)$とすると,
	\begin{equation}
		m(\mu - i\alpha) = \sum_{j = 0}^i n_i  \IMP  n_i = m(\mu - i\alpha) - m(\mu - (i - 1)\alpha)
	\end{equation}
	となる.$0 \leq j \leq i \leq r/2$とする.最高ウェイト$r - 2j$のウェイト空間は,\eqref{eq:Freudenthal-proof-1}の$r \mapsto r - 2j,\ i \mapsto j - i$より,
	\begin{equation}
		\phi(x_\alpha)\phi(x^\alpha)w_{i - j} = (r - j - i)(i - j + 1)\frac{(\alpha, \alpha)}{2}w_{i - j}
	\end{equation}
	よって,
	\begin{align}
		\Tr_{V_{\mu - i\alpha}} \phi(x_\alpha)\phi(x^\alpha) &= \sum_{j = 0}^i n_j(r - j - i)(i - j + 1)\frac{(\alpha, \alpha)}{2} \\
		&= \sum_{j = 0}^i (m(\mu - j\alpha) - m(\mu - (i - 1)\alpha))(r - j - i)(i - j + 1)\frac{(\alpha, \alpha)}{2} \\
		&= \sum_{j = 0}^i m(\mu - j\alpha)((r - j - i)(i - j + 1) - (r - j - i - 1)(i - j))\frac{(\alpha, \alpha)}{2} \\
		&= \sum_{j = 0}^i m(\mu - j\alpha)(r - 2j)\frac{(\alpha, \alpha)}{2} \\
		&= \sum_{j = 0}^i m(\mu - j\alpha)((\mu, \alpha) - j(\alpha, \alpha))\quad  \Biggl(r = \sspair{\mu}{\alpha} = 2\frac{(\mu, \alpha)}{(\alpha, \alpha)}\Biggr) \\
		\label{eq:Freudenthal-proof-2}
		&= \sum_{j=0}^i m(\mu - j\alpha)(\mu - j\alpha, \alpha)  \Bigl(0 \leq i \leq \frac{r}{2}\Bigr)
	\end{align}
	次に,$\sigma_\alpha$の下で不変だから,$m(\mu - j\alpha) = m(\mu - (r - j)\alpha)$より、
	\begin{equation}
		\Tr_{V_{\mu - i\alpha}} \phi(x_\alpha)\phi(x^\alpha) = \sum_{j=0}^{r - i - 1} m(\mu - j\alpha)(\mu - j\alpha, \alpha)\quad  \Bigl(\frac{r}{2} < i \leq r\Bigr)
	\end{equation}
	となる(右辺の和は本来$r - i$までだが,$x^\alpha$で)ここで,$(\mu - j\alpha, \alpha) + (\mu - (r-j)\alpha, \alpha) = 2(\mu, \alpha) - r(\alpha, \alpha) = 0$より,\ref{eq:Freudenthal-proof-2}は任意の$i$で成り立つ.
	
	任意の$\nu \in \Pi(\lambda)$は,$\nu + j\alpha = \mu$の形にできる.$m(\nu + (j + i)\alpha) = 0\ (i > 0)$より,
	\begin{equation}
		\Tr_{V_\mu} \phi(x_\alpha)\phi(z_\alpha) = \sum_{i=0}^\infty m(\mu + i\alpha)(\mu + i\alpha)
	\end{equation}
	
	次に,$h_ih^i$.系 2.5.2より$\kappa|_{\mathfrak{h}\times\mathfrak{h}}$は非退化だったから,
	\begin{equation}
		\exists t_\mu \in \mathfrak{h},\ \forall h \in \mathfrak{h},\quad  \mu(h) = \kappa(t_\mu, h)
	\end{equation}
	を満たす.$t_\mu = \sum_{i=1}^l a^ih_i$と展開すると,
	\begin{equation}
		\mu(h_i) = \sum_{j=1}^l a^j\kappa(h_j, h_i),\quad  \mu(h^i) = a^i
	\end{equation}
	より,
	\begin{equation}
		\sum_{i=1}^l\Tr_{V_\mu} \phi(h_i)\phi(h^i) = m(\mu)\sum_{i=1}^l\mu(h_i)\mu(h^i) = m(\mu)\sum_{i,j=1}^l a^ia^j\kappa(h_i, h_j) = m(\mu)\kappa(t_\mu, t_\mu) = m(\mu)(\mu, \mu)
	\end{equation}
	となる.以上より,
	\begin{equation}
		\Tr_{V_\mu} \phi(c_\mathfrak{g}) = (\mu, \mu)m(\mu) + \sum_{\alpha \in \Phi} \sum_{i=0}^\infty m(\mu + i\alpha)(\mu + i\alpha, \alpha)
	\end{equation}
	$\forall \alpha \in \Phi,\ -\alpha \in \Phi$だったから,和の$i = 0$の項は$0$.また,実は
	\begin{equation}
		\sum_{i=-\infty}^\infty m(\mu + i\alpha)(\mu + i\alpha, \alpha) = 0
	\end{equation}
	となる(よって,$\mu \notin \Pi(\lambda)$でも成立).よって,
	\begin{align}
		\Tr_{V_\mu} \phi(c_\mathfrak{g}) &= (\mu, \mu)m(\mu) + \sum_{\alpha \succ 0} m(\mu)(\mu, \alpha) + 2\sum_{\alpha \succ 0} \sum_{i=1}^\infty m(\mu + i\alpha)(\mu + i\alpha, \alpha) \\
		&= (\mu, \mu + 2\delta)m(\mu) + 2\sum_{\alpha \succ 0} \sum_{i=1}^\infty m(\mu + i\alpha)(\mu + i\alpha, \alpha)
	\end{align}
\end{proof}

\begin{mytheo}[label=thm:Freudenthal]{Freudenthalの公式}
	最高ウェイト$\lambda \in \Lambda^+$の非零で忠実な既約$\mathfrak{g}$-加群を$L(\lambda)$,整ウェイト$\mu \in \Lambda$の重複度を$m(\lambda)$とすると,以下を満たす.
	\begin{equation}
		\big((\lambda + \delta, \lambda + \delta) - (\mu + \delta, \mu + \delta)\big)m(\mu) = 2\sum_{\alpha \succ 0}\sum_{i=1}^\infty m(\mu + i\alpha)(\mu + i\alpha, \alpha)
	\end{equation}
\end{mytheo}
\begin{proof}
	続き.まず,$\phi(c_\mathfrak{g})$はスカラー倍だったから,それを$c$と置くと,補題\ref{lem:Freudenthal}の左辺は
	\begin{equation}
		\Tr_{V_\mu} \phi(c_\mathfrak{g}) = cm(\mu)
	\end{equation}
	となる.補題\ref{lem:Freudenthal}を最高ウェイト$\mu = \lambda$の場合で考えると,
	\begin{equation}
		c = (\lambda, \lambda + 2\delta) = (\lambda + \delta, \lambda + \delta) - (\delta, \delta)
	\end{equation}
	より,Freudenthalの公式が成り立つ.
\end{proof}



\subsection{具体例}
\subsection{代数的指標}

\begin{mydef}[label=def:group-ring]{群環}
	$G$を群,$R$を環とする.$G$で生成される自由$R$-加群$R[G]$に対し,その積を
	\begin{equation}
		\biggl(\sum_{g \in G} a(g)g \biggr)\biggl(\sum_{h \in G} b(h)h \biggr) \coloneqq \sum_{g, h \in G} a(g)b(h)(gh) = \biggl(\sum_{g \in G} \biggl(\sum_{h \in G} a(h)b(h^{-1}g)g \biggr) \biggr)
	\end{equation}
	と定めた(和は実際は有限和なのでwell-defined)$R$上の代数を\textbf{群環}(group ring)と呼ぶ.
\end{mydef}
ウェイト格子$\Lambda \subset \mathfrak{h}^*$は基本ウェイトで生成される$\mathbb{Z}$-加群だったので,スカラー倍の構造を忘れば加法群となる.よって,$\{e^\lambda | \lambda \in \Lambda\}$で生成される群環$Z[\Lambda]$が定義される(生成系を別の文字にしたのは,$Z[\Lambda]$の和$+$と$\Lambda$の積$+$を区別するため.$e^\lambda e^\mu = e^{\lambda + \mu}$).\\
$Z[\Lambda]$上のWeyl群$\mathscr{W}$の作用を$\sigma e^\lambda = e^{\sigma\lambda}\ (\sigma \in \mathscr{W})$で定義する.\\
一般の有限次元$\mathfrak{g}$-加群$V$は,補題 6.1.1よりウェイト空間の直和で書けるから,$V$のウェイトの集合$\Pi(V)$は有限集合.これに注意して指標を定義する.
\begin{mydef}[label=def:alg-character]{代数的指標}
	半単純Lie代数$\mathfrak{g}$のウェイト格子を$\Lambda \subset \mathfrak{h}^*$,有限次元$\mathfrak{g}$-加群を$V$,ウェイト$\mu \in \Pi(V)$の重複度を$m_\lambda(\mu)$とする.群環$\mathbb{Z}[\Lambda]$の元
	\begin{equation}
		\ch_V \coloneqq \sum_{\mu \in \Pi(V)} m_\lambda(\mu)e^\mu = \sum_{\mu \in \Lambda} m_\lambda(\mu)e^\mu
	\end{equation}
	を\textbf{代数的指標}(algebraic character)や\textbf{形式的指標}(formal character),または単に\textbf{指標}と呼ぶ.\\
	特に既約加群$L(\lambda)$に対しては,$\ch_{L(\lambda)} = \ch_\lambda$とも書く.
\end{mydef}
\begin{mylem}[label=lem:Weyl-stable-irr-ch]{既約最高ウェイト加群の指標の$\mathscr{W}$-不変性}
	$\ch_\lambda$はWeyl群$\mathscr{W}$の作用の下で不変.
\end{mylem}
\begin{proof}
	定理6.2.1より,$m_\lambda(\mu) = m_\lambda(\sigma\mu)\ (\forall \sigma \in \mathscr{W})$なので$\mathscr{W}$-不変.
\end{proof}
有限次元加群$V$について,完全可約性より,直和分解$V = \bigoplus_{i=1}^t V(\lambda_i)$に対し,
\begin{equation}
	\ch_V = \sum_{i=1}^t \ch_{\lambda_i}
\end{equation}
と一意的に書けるはず.特に補題\ref{lem:Weyl-stable-irr-ch}より,この和も$\mathscr{W}$の作用の下で不変.このことに注意して指標の和の性質を確認しよう.
\begin{myprop}[label=prop:alg-character-add]{指標の和}
	$f \in \mathbb{Z}[\Lambda]$がWeyl群$\mathscr{W}$の作用の下で不変とすると,\\
	$f$は$\ch_\lambda\ (\lambda \in \Lambda^+)$の$\mathbb{Z}$-係数線型結合で一意的に書ける.特に,
	\begin{equation}
		\ch_\lambda = \sum_{\sigma\lambda} e^{\sigma\lambda}
	\end{equation}
\end{myprop}
\begin{proof}
	\begin{description}
		\item[(存在)] $f = \sum_{\lambda \in \Lambda} c(\lambda)e^\lambda\ (c(\lambda) \in \mathbb{Z})$と書くと,$\mathscr{W}$の作用で不変だから,
		\begin{equation}
			f = \sum_{\lambda \in \Pi} c(\lambda) \sum_{\sigma\lambda} e^{\sigma\lambda}\quad  \biggl(\Pi = \bigl\{\lambda \in \Lambda^+ \bigm| c(\lambda) \neq 0\bigr\}\biggr)
		\end{equation}
		と書ける.$M_f = \bigcup_{\lambda \in \Pi} \{\mu \in \Lambda^+ | \mu \prec \lambda\}$とすると,補題13.3LemmaBより有限集合.\\
		ここで,$\lambda \in M_f$が極大のとき,補題\ref{lem:Weyl-stable-irr-ch}より,$f' = f - c(\lambda)\ch_\lambda$も$\mathscr{W}$の作用の下で不変.$L(\lambda)$のウェイトの集合$\Pi(\lambda)$は飽和集合だから,$\mu \prec \lambda$なる優ウェイトを全て含み,かつ$\lambda \notin M_{f'}$より,$M_{f'} \subsetneq M_f$.\\
		よって,$\Card(M_f)$についての帰納法で,$\Card(M_f') = 0$になるまでこれを繰り返せば,$\mathscr{W}$-不変性より$f' = 0$になる.特に$\Card(M_f) = 1$の場合,
		\begin{equation}
			f = \sum_{\lambda \in \Lambda^+} c(\lambda) \sum_{\sigma\lambda} e^{\sigma\lambda}
		\end{equation}
		\item[(一意性)] $f = \sum_{\lambda \in \Pi} c(\lambda)\ch_\lambda = \sum_{\lambda' \in \Pi'} c'(\lambda')\ch_{\lambda'}\ \bigl(0 \notin c(\Pi), c'(\Pi')\bigr)$と書けたとする.\\
		$\lambda \in \Pi$のうち,極大なものを$\lambda_0$とすると,$\lambda_0 \prec \lambda'_0$を満たす$\lambda'_0 \in \Pi'$が存在する.すると,$\lambda'_0 \prec \lambda$となる$\lambda \in \Pi$が存在するが,それは$\lambda_0$しかないから,
		\begin{equation}
			\lambda_0 \prec \lambda'_0 \prec \lambda_0  \IFF  \lambda_0 = \lambda'_0
		\end{equation}
		次に,$f - c(\lambda_0)\ch_{\lambda_0}$を考えると,同様の議論から,$c(\lambda_0) - c'(\lambda'_0) = 0$となる.これを繰り返せば,$\Pi = \Pi', c = c'$が言えるので,$f$の展開は一意.
	\end{description}
\end{proof}
次に指標の積の性質を確認しよう.
\begin{myprop}[label=prop:alg-character-multiply]{指標の積}
	$V, W$を有限次元$\mathfrak{g}$-加群とすると,
	\begin{equation}
		\ch_{V \otimes W} = \ch_V\ch_W
	\end{equation}
\end{myprop}
\begin{proof}
	g-加群のテンソル積の定義より,ウェイト空間どうしのテンソル積$V_\mu \otimes W_\nu$はウェイトは$\mu + \nu$のウェイト空間.つまり,
	\begin{equation}
		m_{V \otimes W}(\mu + \nu) = \sum_{\pi + \pi' = \mu + \nu} m_V(\pi)m_W(\pi')
	\end{equation}
	となる.右辺は$\ch_V\ch_W$の$e^{\mu + \nu}$の係数に等しい.
\end{proof}
特に,$V, W$の最高ウェイトをそれぞれ$\lambda_V, \lambda_W$とすると,$V \otimes W$の既約分解には,$L(\lambda_V + \lambda_W)$が含まれる.

\section{指標}



\subsection{不変式論}

\begin{mydef}[label=def:polynomial-function-ring]{多項式函数環}
	体$\mathbb{K}$上の線型空間$V$に対し,$V^*$に対する対称代数
	\begin{equation}
		\mathbb{K}[V] \coloneqq S(V^*)
	\end{equation}
	を$V$上の\textbf{多項式函数環}(ring of polynomial functions)と呼ぶ.
\end{mydef}
$V$を有限次元とし,$V^*$の基底を$(f^1,\ldots , f^n)$と取ると,$n$変数多項式代数$\mathbb{K}[f^1, \ldots, f^n]$と同型である.\\
ウェイト格子$\Lambda$は$\mathfrak{h}^*$を張るから,$\lambda \in \Lambda$の多項式は$\mathbb{K}[V]$を張る.
\begin{mydef}[label=def:poly-func-ring-modular]{誘導される多項式函数環の加群}
	群を$G$,体$\mathbb{K}$上の線型空間を$V$とする.\\
	$V$が$G$-加群のとき,対応する表現$\phi$に対し,
	\begin{equation}
		g \btr f = f \circ \phi
	\end{equation}
	により,多項式函数環上の加群が誘導される.\\
	また,$V^*$が$G$-加群のとき,
	\begin{equation}
		\mathbb{K}[V]^G = \bigl\{f \in \mathbb{K}[V] \bigm| g \btr f = f\ (\forall g \in G) \bigr\}
	\end{equation}
	を$V$上の$G$-\textbf{不変多項式}($G$-invariant polynomials on $V$)あるいは単に\textbf{不変式}と呼ぶ.
\end{mydef}
\begin{mydef}[label=def:invariant-polynomial]{不変式}
	群を$G$,体$\mathbb{K}$上の線型空間を$V$とする.$\mathbb{K}[V]$が(定義\ref{def:poly-func-ring-modular}により誘導された)$G$-加群のとき,この作用で不変な集合
	\begin{equation}
		\mathbb{K}[V]^G = \bigl\{f \in \mathbb{K}[V] \bigm| g \btr f = f\ (\forall g \in G) \bigr\}
	\end{equation}
	を$V$上の$G$-\textbf{不変多項式}($G$-invariant polynomials on $V$)あるいは単に\textbf{不変式}と呼ぶ.
\end{mydef}
Weyl群$\mathscr{W}$は$\mathfrak{h}^*$に作用するから,$\mathscr{W}$-不変式$\mathbb{K}[\mathfrak{h}]^\mathscr{W}$が定義される.また,$\mathfrak{g}$の内部自己同型$\Int\mathfrak{g}$は$\mathfrak{g}$に作用するから,$\Int\mathfrak{g}$-不変式$\mathbb{K}[\mathfrak{g}]^{\Int\mathfrak{g}}$が定義される.








\begin{mytheo}[label=thm:Lie-poly-func-ring]{(Chevalley)}
	$\mathfrak{g}$を半単純Lie代数,$\mathfrak{h}$をその極大トーラス,$\mathscr{W}$をWeyl群とする.不変式の代数準同型
	\begin{equation}
		\theta\colon \mathbb{K}[\mathfrak{g}]^{\Int\mathfrak{g}} \lto \mathbb{K}[\mathfrak{h}]^\mathscr{W}
	\end{equation}
	は全射.
\end{mytheo}
\begin{mytheo}[label=thm:]{}
	$\mathfrak{g}$を半単純Lie代数,$\mathfrak{U}(\mathfrak{g})$をその普遍包絡代数,$Z(\mathfrak{U}(\mathfrak{g}))$をその中心とすると,
	\begin{equation}
		\mathbb{K}[\mathfrak{U}(\mathfrak{g})]^{\Int\mathfrak{g}} = Z(\mathfrak{U}(\mathfrak{g}))
	\end{equation}
\end{mytheo}





















\end{document}