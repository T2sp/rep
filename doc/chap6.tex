\documentclass[rep_main]{subfiles}

\begin{document}

\setcounter{chapter}{5}

\chapter{表現定理}
この章では,$\mathfrak{g}$は標数$0$の代数閉体上の半単純Lie代数し,$\mathfrak{h}$を$\mathfrak{g}$のCartan部分代数,$\Phi$をルート系,$\Delta = \{\alpha_1, \ldots, \alpha_l\}$を$\Phi$の底,$\mathscr{W}$をWeyl群とする.

\section{表現のウエイトと極大ベクトル}
$\lsl{2}{\mathbb{K}}$で考えたウエイトを一般化する.$\lsl{2}{\mathbb{K}}$の場合,\hyperref[def:toral-subLieAlg]{極大トーラス}$\mathfrak{h}$は$1$次元線型空間だったので,ウエイトは固有値として定義された.$\mathfrak{h}$が一般の次元の場合は,ルートのように$\mathfrak{h}^*$の元として定義すれば良い.ルートと異なる点は,表現が随伴表現とは限らないことだけである.
\subsection{ウエイト空間}
\begin{mydef}[label=def:weightspacerep]{表現のウエイト,ウエイト空間}
	$V$を$\mathfrak{g}$-加群とし,\hyperref[def:toral-subLieAlg]{極大トーラス}$\mathfrak{h}$を一つ固定する.$\lambda \in h^*$に対し,
	\begin{align}
		V_\lambda \coloneqq \bigl\{\, v \in V \bigm| h \btr v = \lambda(h) v \,\bigr (\forall h \in \mathfrak{h})\} 
	\end{align}
	が定義される.この内$V_\lambda \neq 0$のものを,\textbf{ウエイト空間}と呼び,$\lambda$を$V$の\textbf{ウエイト}(より正確には$\mathfrak{h}$の$V$上のウエイト)と呼ぶ.
\end{mydef}
\begin{mylem}[label=lem:weightspacerep]{}
	\begin{enumerate}
		\item $V' = \sum_{\lambda \in \mathfrak{h}^*} V_\lambda$は直和で,$V$の部分$\mathfrak{g}$-加群.
		\item $V$が有限次元の場合,$V = V'$.
		\item $\mathfrak{g}_\alpha \btr V_\lambda = V_{\lambda + \alpha},\ (\forall \lambda \in \mathfrak{h}^*,\ \forall \alpha \in \Phi)$.
	\end{enumerate}
\end{mylem}
\begin{proof}
	\begin{enumerate}
		\item ウエイト$\lambda, \lambda'$に対し,$v \in V_\lambda \cap V_\lambda' \setminus \{0\}$とする.任意の$h$に対し,
		\begin{equation}
			h \btr v = \lambda(h)v = \lambda'(h)v  \IFF  (\lambda(h) - \lambda'(h))v = 0 
		\end{equation}
		であるから,$v \neq 0$より$\lambda = \lambda'$となる.よって,直和となる.
		\item $\mathfrak{g}$は半単純なので,系\ref{col:JC}より,$\mathfrak{h}^*$の元を同時対角化できる.
		\item $x \in \mathfrak{g}_\alpha,\ h \in \mathfrak{h},\ v \in V_\lambda$とすると,
		\begin{equation}
			h \btr (x \btr v) = x \btr (h \btr v) + [h, x] \btr v = (\lambda(h) + \alpha(h))(x \btr v)
		\end{equation}
		より,$\mathfrak{g}_\alpha \btr V_\lambda = V_{\lambda + \alpha}$となる.
	\end{enumerate}
\end{proof}
\subsection{標準巡回加群}
極大ベクトルも同様に一般化する.
\begin{mydef}[label=def:weightspacerep]{ウエイトの極大ベクトル}
	$\mathfrak{g}$-加群$V$に対し,ルートの底$\Delta$を固定する.$V$のそのウエイト$\lambda$の固有空間の元$v \neq 0$が,
	\begin{align}
		\mathfrak{g}_\alpha \btr v = 0\quad  (\forall \alpha \succ 0)  \IFF  \mathfrak{g}_\alpha \btr v = 0\quad  (\forall \alpha \in \Delta)
	\end{align}
	を満たすとき,$v$を\textbf{極大ベクトル}(maximal vector)と呼ぶ.
\end{mydef}
$\Longleftarrow$は,命題\ref{prop:root-decomp-int,breakable}の$[g_\alpha, g_\beta] = g_{\alpha + \beta}$より言える.\\
無限次元では存在すら保証されない.一方有限次元ではBorel部分代数(極大\hyperref[def:solvable-LieAlg]{可解}部分代数,第4章)$\mathfrak{B}(\Delta) = \mathfrak{h} + \oplus_{\alpha \succ 0} \mathfrak{g}_\alpha$を考えると,\hyperref[thm:Lie]{Lieの定理}と\hyperref[root-decomp-basic1]{$g_\alpha$の冪零性}から,$\mathfrak{g}_\alpha$の作用で$0$になるような共通の固有ベクトルが存在する.

極大ベクトルで生成される$\mathfrak{g}$-加群を考える.
\begin{mydef}[label=def:standard-cyclic]{標準巡回,最高ウェイト}
	ウェイト$\lambda$の極大ベクトル$v^+$と,$\mathfrak{g}$の\hyperref[def:univ-env-alg]{普遍包絡代数}$\mathfrak{U}(\mathfrak{g})$に対し,$V = \mathfrak{U}(\mathfrak{g}) \btr v^+$を満たすとき,$V$は(ウェイト$\lambda$の)\textbf{標準巡回}(stnadard cyclic)加群と呼び,$\lambda$を$V$の\textbf{最高ウェイト}(highest weight)と呼ぶ.
\end{mydef}
以下では,$\mathfrak{g}$-加群を有限次元とする.

標準巡回加群の構造は単純.命題\ref{prop:root-decomp-ortho}より,正ルート$\alpha$に対し,$x_\alpha \in \mathfrak{g}_\alpha\setminus\{0\}$を選ぶと,$y_\alpha \in \mathfrak{g}_{-\alpha}$が存在し,$[x_\alpha, y_\alpha] = h_\alpha \in \mathfrak{h}$となる.\\
ルート系$(\mathbb{E}, \Phi)$で定義された\hyperref[def:base-root,breakable]{半順序}同様に,Lie代数のウェイトの\textbf{半順序}$\prec \subset \mathfrak{h}^* \times \mathfrak{h}^*$を
\begin{equation}
	\mu \prec \lambda  \DEF  \exists! \Familyset[\big]{k_\alpha}{\alpha \in \Delta} \in \prod_{\alpha \in \Delta} \mathbb{R}_{\ge 0},\; \lambda - \mu = \sum_{\alpha \in \Delta} k_\alpha \alpha
\end{equation}
と定義する.

\begin{mylem}[label=lem:max-sub-quot-mod]{}
	加群$M$の部分加群$N$に対し,
	\begin{equation}
		\text{$N$が極大} \IFF \text{商加群$M/N$は既約}
	\end{equation}
\end{mylem}
\begin{proof}
	$\Leftrightarrow$のどちらも対偶を示す.\\
	($\Rightarrow$)商加群の非自明部分加群$K$が存在したとすると,$N \subsetneq K + N \subsetneq M$より,$N$は極大でない.\\
	($\Leftarrow$)$N \subsetneq K \subsetneq M$を満たす部分加群$K$が存在したとすると,$M / K$は$M / N$の非自明部分加群である.
\end{proof}
\begin{mytheo}[label=thm:standardcm]{}
	$V$を標準巡回$\mathfrak{g}$-加群,$\lambda$を最高ウェイト,$v^+ \in V_\lambda$を極大ベクトル,$\Phi^+ = \{\beta_1, \ldots, \beta_m\}$を正ルートの集合とする.
	\begin{enumerate}
		\item $V = \Span\{y_{\beta_1}^{i_1}\cdots y_{\beta_m}^{i_m} \btr v^+\ (i_j \in \mathbb{Z}_{\geq 0})\}$.特に,ウエイト空間の直和である.
		\item\label{thm:standardcm-b} $V$の任意のウェイトについて,$\mu \prec \lambda$(最高ウェイトと呼ばれる所以).
		\item\label{thm:standardcm-c} $\forall \mu \in \mathfrak{h}^*$に対し,$\dim V_\mu < \infty$.また,$\dim V_\lambda = 1$.
		\item\label{thm:standardcm-d} 任意の$V$の部分加群はウエイト空間の直和.
		\item\label{thm:standardcm-e} $V$は\hyperref[def:irr]{直既約$\mathfrak{g}$-加群}であり,唯一の極大部分加群と対応する既約な商をもつ.
		\item 全ての非零な準同型な$V$の像もウエイト$\lambda$の巡回加群.
	\end{enumerate}
\end{mytheo}
\begin{proof}
	\begin{description}
		\item[\textbf{(1)}] $\mathfrak{n}^- = \bigoplus_{\alpha \prec 0}\mathfrak{g}_\alpha,\quad  \mathfrak{b} = \mathfrak{b}(\Delta)$とする.PBW定理(の系\ref{col:PBW-D}),\ref{def:univ-env-alg-modular}より,
		\begin{equation}
			\mathfrak{U}(\mathfrak{g}) \btr v^+ = \mathfrak{U}(\mathfrak{n}^-)\mathfrak{U}(\mathfrak{b}) \btr v^+ = \mathfrak{U}(N^-) \btr (\mathbb{K}v^+)\quad  (\because \text{$B$の共通の固有ベクトル})
		\end{equation}
		よって,$\mathfrak{U}(N^-)$の基底は$y_{\beta_1}^{i_1}\cdots y_{\beta_m}^{i_m}$だったから,$V$はこのベクトルの集合で張られる.
		\item[\textbf{(2)}] 補題\ref{lem:weightspacerep}より,
		\begin{equation}
			\label{eq:standardcm-v}
			v \coloneqq y_{\beta_1}^{i_1}\cdots y_{\beta_m}^{i_m}v^+
		\end{equation}
		のウェイトは$\mu = \lambda - \sum_j i_j\beta_j$であり,$\beta_j \in \Phi^+$は単純ルートの和だったから,$\mu \prec \lambda$.
		\item[\textbf{(3)}] $\sum_j i_j\beta_j = \sum_{i=1}^l k_i\alpha_i$を満たす$(i_j,\ \beta_j)$の組み合わせは高々有限個だから,$\dim V_\mu < \infty$.特に,ウェイト$\lambda$を持つ\eqref{eq:standardcm-v}の形のベクトルは$v^+$しかないので,$\dim V_\mu = 1$.
		\item[\textbf{(4)}] $W$を$V$の部分加群とする,$\forall w \in W \subset V$は異なるウェイト空間$V_{\mu_i}$の元$v_i$の和で書ける.$\forall v_i \in W$を背理法で示す.\\
		$W$は線型空間なので,$w = v_1 + \cdots + v_n \in W,\ (\forall v_i \notin W)$を(背理法の仮定として)仮定して良い.$n=1$はあり得ないので,$n > 1$とする.ここで,
		\begin{equation}
			\exists h \in \mathfrak{h},\quad  h \btr w = \sum_{i=1}^n \mu_i(h) v_i \in W\ \mu_1(h) \neq \mu_2(h)
		\end{equation}
		とでき,
		\begin{equation}
			(h - \mu_1(h)1) \btr w = \sum_{i=2}^n (\mu_i(h) - \mu_1(h)) v_i \in W \setminus \{0\}
		\end{equation}
		となる.これを繰り返すと,$v_n \in W$となり矛盾する.
		\item[\textbf{(5)}] $V$の部分加群が$V_\lambda$を含むと,(1)より$V$自身となる.(4)と合わせて,$V$の非自明な部分$\mathfrak{g}$-加群は$V_\lambda$以外のウェイト空間の直和でかけるから,これら全ての部分加群の和$W$も部分$\mathfrak{g}$-加群である.よって$V$は,唯一の極大部分加群$W$と唯一の既約な商(補題\ref{lem:max-sub-quot-mod})をもつ.全ての非自明部分加群は$W$に含まれ,$V_\lambda$を含まないから,$V$は部分加群の和で表せない.i.e. 直既約$\mathfrak{g}$-加群である.\\
		\item[\textbf{(6)}] 明らか.
	\end{description}
\end{proof}

\begin{mycol}[label=col:standardcm]{$\mathfrak{g}$-加群の極大ベクトルの唯一性}
	標準巡回$\mathfrak{g}$-加群$V$が既約でもあれば,(最高ウェイト$\lambda$の)$v^+$は高々スカラー倍の違いを除いて唯一.
\end{mycol}
\begin{proof}
	$w^+$をウェイト$\lambda'$の極大ベクトルとすると,$\mathfrak{U}(\mathfrak{g}) \btr w^+$は$V$の部分加群である.$V$は既約だったので,$\mathfrak{U}(\mathfrak{g}) \btr w^+ = V$.定理\ref{thm:standardcm}\ref{thm:standardcm-b}より,$\lambda = \lambda'$.よって定理\ref{thm:standardcm}\ref{thm:standardcm-c}より,$w^+$は$v^+$のスカラー倍.
\end{proof}

\subsection{存在と唯一性}
各ウェイト$\lambda \in \mathfrak{h}^*$に対し,(同型を除いて)唯一の既約な標準巡回$\mathfrak{g}$-加群が存在する事を示す(無限次元でも成立).唯一性の方の証明は,第4章14.2の定理の証明と似ている上に単純になっているらしい.
\begin{mytheo}[label=thm:standardcm-unique]{唯一性}
	$V, W$を最高ウエイト$\lambda$の標準巡回加群とする.$V, W$が共に既約ならば同型.
\end{mytheo}
\begin{proof}
	$\mathfrak{g}$-加群$X  = V \oplus W$を考える.$v^+, w^+$をそれぞれ$V, W$のウェイト$\lambda$の極大ベクトルとすると,
	\begin{equation}
		X = V \oplus W = (\mathfrak{U}(\mathfrak{g}) \btr v^+) \oplus (\mathfrak{U}(\mathfrak{g}) \btr w^+) = \mathfrak{U}(\mathfrak{g}) \btr (v^+, w^+)
	\end{equation}
	より$x^+ = (v^+, w^+)$は$X$のウェイト$\lambda$の極大ベクトルとなる.$x^+$で生成される$X$の(標準巡回)部分加群を$Y$とし,$p:\ Y \to V,\ p':\ Y \to W$をそれぞれ$X$の第一,第二引数を取ってくる射影に誘導される写像とすると,
	\begin{align}
		&p(x^+) = v^+,\quad  p'(x^+) = w^+ \\
		&\Im p = V,\quad  \Im p' = W
	\end{align}
	より,$p,\ p'$は共に$\mathfrak{g}$-加群の全射準同型.よって加群の同型定理と定理\ref{thm:standardcm}\ref{thm:standardcm-e}より,$V, W$は標準巡回加群$Y$の既約な商加群として同型.
\end{proof}
存在について,2通りの方法で構成する.

まずは誘導加群の方法.Borel部分代数を$\mathfrak{b} = \mathfrak{b}(\Delta)$とすると,$\mathfrak{b}$-加群としての標準巡回加群は極大ベクトル$v^+$で生成される一次元部分加群を含む.よってまずは$\lambda \in \mathfrak{h}^*$を固定し,$v^+$を基底に持つ一次元ベクトル空間$D_\lambda$に対し,$D_\lambda$上の$\mathfrak{b}$の作用を
\begin{equation}
	\Biggl(h + \sum_{\alpha \succ 0}x_\alpha \Biggr) \btr v^+ = h \btr v^+ = \lambda(h) v^+
\end{equation}
と定義する.$D_\lambda$は$\mathfrak{b}$-加群となる.それと同時に$D_\lambda$は$\mathfrak{U}(\mathfrak{b})$-加群でもあるので,テンソル積
\begin{equation}
	Z(\lambda) = \mathfrak{U}(\mathfrak{g}) \otimes_{\mathfrak{U}(\mathfrak{b})} D_\lambda
\end{equation}
を定義すると,自然な左作用により$\mathfrak{U}(\mathfrak{g})$-加群となる.\\
次に$Z(\lambda)$がウェイト$\lambda$の標準巡回加群であることを示す.$1 \otimes v^+$は$Z(\lambda)$を生成する.また,\hyperref[thm:PBW]{PBW定理}の系\ref{col:PBW-D}より,$\mathfrak{U}(\mathfrak{g})$は$1$および単項式$y_{\beta_1}^{i_1}\cdots y_{\beta_m}^{i_m}$を基底に持つ自由$\mathfrak{U}(\mathfrak{b})$-加群であったから,$1 \otimes v^+$非零である.よって,$1 \otimes v^+$はウェイト$\lambda$の極大ベクトルである.\\
\hyperref[thm:PBW]{PBW定理}の系\ref{col:PBW-D}$\mathfrak{U}(\mathfrak{g}) \simeq \mathfrak{U}(\mathfrak{n}^-) \otimes \mathfrak{U}(\mathfrak{b})$より,$Z(\lambda)$を$\mathfrak{U}(\mathfrak{n}^-)$-加群と見ると,$Z(\lambda) \simeq \mathfrak{U}(N^-) \otimes \mathbb{K}$である.よって,$Z(\lambda)$は$\mathfrak{U}(\mathfrak{n}^-)$と同型.

次に生成系と関係の方法で構成し,同型であることを示す.\\
正ルートの集合$\Phi^+$および,$h_\alpha - \lambda(h_\alpha)1\ (\alpha \in \Phi)$で生成される左イデアルを$I(\lambda)$とする.
\begin{equation}
	I(\lambda) \btr v^+ = 0
\end{equation}
より,$\mathfrak{U}(\mathfrak{g})$-加群の標準的射影$\mathfrak{U}(\mathfrak{g}) / I(\lambda) \to Z(\lambda)$は$1 + I \mapsto 1\otimes v^+$となる.再び,\hyperref[thm:PBW]{PBW定理}の系\ref{col:PBW-D}より,$\mathfrak{U}(\mathfrak{b}) + I \mapsto \mathfrak{K}(1\otimes v^+)$となる.よって,この標準的射影は一対一対応であり,
\begin{equation}
	U(\lambda) / I(\lambda) \simeq Z(\lambda)
\end{equation}

\begin{mytheo}[label=thm:standardcm-exist]{存在}
	$\forall \lambda \in \mathfrak{h}^*$に対し,ウェイト$\lambda$の既約な標準巡回加群が存在する.
\end{mytheo}
\begin{proof}
	上で構成された$Z(\lambda)$は,ウェイト$\lambda$の標準巡回加群で,唯一の極大部分加群$Y(\lambda)$をもつ(定理\ref{thm:standardcm}\ref{thm:standardcm-d}).よって,$V(\lambda) = Z(\lambda) / Y(\lambda)$はウェイト$\lambda$の既約標準巡回加群(定理\ref{thm:standardcm}\ref{thm:standardcm-e}).
\end{proof}
\begin{mytheo}[label=thm:finite-irr-mod]{有限次元既約加群の構造}
	$V$を有限次元既約$\mathfrak{g}$-加群とすると,
	\begin{equation}
		\exists \lambda \in \mathfrak{h}^*,\quad  V \simeq V(\lambda)
	\end{equation}
\end{mytheo}
\begin{proof}
	有限次元には,ウェイト$\lambda$の極大ベクトル$v^+$の存在が保証されていた.$V(\lambda) = \mathfrak{U}(\mathfrak{g}) \btr v^+$は既約加群$V$の部分加群で,非零なので$V \simeq V(\lambda)$.
\end{proof}

\section{有限次元加群}
既約標準巡回加群$V(\lambda)$が有限次元であるためのウェイト$\lambda$の条件を調べる.
\subsection{有限次元に対する必要条件}
各単純ルート$\alpha_i$に対し,
	\begin{equation}
		\mathfrak{s}_i \coloneqq \mathfrak{g}_{\alpha_i} \oplus \mathfrak{g}_{-\alpha_i} \oplus [\mathfrak{g}_{\alpha_i}, \mathfrak{g}_{-\alpha_i}] \simeq \Lsl(2,\, \mathbb{K})
	\end{equation}
	とする($\simeq$は定理\ref{prop:root-decomp-ortho}).$V(\lambda)|_{\mathfrak{s}_i}$も有限次元加群で,$\mathfrak{g}$の極大ベクトル$v^+$は$\mathfrak{s}_i$の極大ベクトルでもある.特に,$\mathfrak{s}_i$の\hyperref[def:toral-subLieAlg]{極大トーラス}$\mathfrak{h}_i$に対し,$h_i \in \mathfrak{h}_i$の作用は,固有値$\lambda(h_i)$で完全に決まる.その固有値が非負整数になることは既に知っている.
\begin{mytheo}[label=thm:necessary-for-finite]{ウェイトの整性の必要性}
	有限次元既約$\mathfrak{g}$-加群$V$の最高ウェイトを$\lambda$,単純ルートを$\alpha$,$h_i \in h_{\alpha_i} = [\mathfrak{g}_{\alpha_i}, -\mathfrak{g}_{\alpha_i}]$とすると,$\lambda(\mathfrak{h}_i)$は非負整数.
\end{mytheo}
\begin{proof}
	定理\ref{thm:irr-sl2}.
\end{proof}
これは,最高ウェイトでない任意の$V$のウェイト$\mu$でも成り立つ:
\begin{equation}
	\label{eq:int-weight}
	\mu(h_i) = \sspair{\mu}{\alpha_i} \in \mathbb{Z},\quad  (1 \leq i \leq l)
\end{equation}
これは,\hyperref[def:root-lattice]{ルート系の整ウェイト}に対応している.これをLie代数の\textbf{整ウェイト}と呼ぶのは自然だろう.\hyperref[def:domweight]{優,強いウェイト},\hyperref[def:fundamental-weight]{基本優ウェイト}も同様に定義され,当然,ルート系の整ウェイトに対する定理は全て成り立つ.\\
ルート系で定義した優ウェイトについては,優かつ整と呼ぶ方が親切かもしれないが,ここでは単に優と呼ぶことにする.

また,一般の$\mathfrak{g}$-加群$V$に対し,$V$のウェイトの集合を$\Pi(V)$と表す.特に,$V = V(\lambda)$のとき,$\Pi(\lambda)$と書く.
\subsection{有限次元に対する十分条件}
\begin{mytheo}[label=thm:suff-for-finite]{優ウェイトが十分条件}
	$\lambda \in \mathfrak{h}^*$を優ウェイトとする.\\
	このとき,既約$\mathfrak{g}$-加群$V(\lambda)$は有限次元で,$V$のウェイトの集合$\Pi(\lambda)$は,Weyl群の作用によって置換され,$\dim V_\mu = \dim_{\sigma\mu},\ \forall \sigma \in \mathscr{W}$を満たす.
\end{mytheo}
\begin{mycol}[label=col:suff-for-finite]{}
	$\lambda \mapsto V(\lambda)$は,優ウェイト$\Lambda^*$と有限次元既約$\mathfrak{g}$-加群(の同型の類)の一対一対応を誘導する.
\end{mycol}
\begin{proof}
	優ウェイトは整ウェイトなので,\hyperref[thm:necessary-for-finite]{必要条件}を満たす.定理\ref{thm:finite-irr-mod}より一対一対応.
\end{proof}
十分条件を証明しよう.$\mathfrak{g}$の生成系$\{x_i, y_i\}$を固定する.
\begin{mylem}[label=lem:suff-for-finite]{}
	Lie代数$\mathfrak{g}$の\hyperref[def:univ-env-alg]{普遍包絡代数}$\mathfrak{U}(\mathfrak{g})$とし,$x_\alpha \in \mathfrak{g}_\alpha,\ y_\alpha \in \mathfrak{g}_{-\alpha}$とすると,任意の$k \in \mathbb{Z}_{\geq 0},\ 1 \leq i, j \leq l$に対し以下が成り立つ.
	\begin{enumerate}
		\item $[x_j,\ y_i^{k+1}] = -(k+1)\alpha_i(h_j)y_i^{k+1}$
		\item $[x_j,\ y_i^{k+1}] = -(k+1)y_i^k(k - h_i)$
	\end{enumerate}
\end{mylem}
\begin{proof}
	\begin{description}
		\item[\textbf{(a)}] $k$についての数学的帰納法.$k=0$は\hyperref[prop:semisimple-Lie-alg-relation]{半単純Lie代数の関係}より明らか.ある$k$で成り立つとき,
		\begin{equation}
			[h_j, y_i^{k+1}] = [h_j, y_i]y_i^k + y_i[h_j, y_i^k] = -\alpha_i(h_j)y_i^{k+2} - (k+1)\alpha_i(h_j)y_i^{k+2} = - (k+2)\alpha_i(h_j)y_i^{k+2}
		\end{equation}
		より任意の$k$で成り立つ.
		\item[\textbf{(b)}] $i \neq j$の場合は,補題\ref{lem:base}より,$\alpha_j - \alpha_i$がルートでないことより従う.\\
		$i = j$のとき,$k = 0$は$y_\alpha, h_\alpha$の選び方より明らか.ある$k$で成り立つとき,
		\begin{equation}
			[x_i, y_i^{k+1}] = [x_i, y_i]y_i^k + y_i^k[x_i, y_i] = h_iy_i^k - (k+1)y_i^{k+1}(k - h_i) = -(k+2)y_i^{k+1}(k - h_i)
		\end{equation}
		より任意の$k$で成り立つ.
	\end{description}
\end{proof}
\hyperref[thm:suff-for-finite]{十分条件の証明}のポイントは,$V$のウェイトが$\mathscr{W}$で置換されることから有限次元であることを示すことである(\hyperref[thm:Serre]{Serreの定理}参照).
\begin{proof}
	$g$-加群$V$に誘導される表現を$\phi$とする.ウェイト$\lambda$の$V$の極大ベクトルを$v^+$とし,$m_i = \lambda(h_i)$とする(仮定より$\lambda$は優ウェイトなので整数).
	\begin{description}
		\item[\textbf{Step 1: $w_i \coloneqq y_i^{m_i + 1} \btr v^+ = 0$}] 
		
		$x_i \btr v^+ = 0$と補題\ref{lem:suff-for-finite}(1)に注意すると,
		\begin{equation}
			x_i \btr w_i = y_i^{m_i + 1} \btr (x_j \btr v^+) - (m_i + 1)y_i^{m_i} \btr (m_i - m_i)v^+ = 0
		\end{equation}
		となる.また,$\alpha_j - \alpha_i$はルートでないから$x_j \btr w_i = 0\ (1 \leq j \leq l)$.もし$w \neq 0$とすると,最高ウェイト$\lambda - (m_i + 1)\alpha_i \neq \lambda$が存在することになり.既約標準巡回加群の\hyperref[col:standardcm]{最高ウェイトの唯一性}に反する.
		\item[\textbf{Step 2: $V$に非零な有限次元$S_i$-加群が含まれる}]  
		
		部分空間$V_i = \{v^+,\ y_i \btr v^+,\ \ldots,\ y_i^{m_i}\btr v^+ \}$を考える.\textbf{(Step 1)}と合わせると$y_i$の作用について不変.補題\ref{lem:suff-for-finite}(1)より$x_i, h_i$の作用についても不変.よって,$S_i$-加群として,$V_i$は$V$の部分加群.
		\item[\textbf{Step 3: $V$は有限次元部分$S_i$-加群の和}]  
		
		$V' = \sum_{i = 1}^l V_i$,$W$を$V$の任意の有限次元$S_i$-部分加群,$\alpha \in \Phi$をルートとする.$x_\alpha \btr W$も有限次元で,$S_i$-部分加群だから$W$の部分空間.よって,$V'$は$\mathfrak{g}$-加群.\\
		$V$は既約で,\textbf{(Step 2)}より$V'$は非零だったから,$V = V'$.
		\item[\textbf{Step 4: $1 \leq i \leq l$に対し,$\phi(x_i),\ \phi(y_i)$は$V$の\hyperref[def:locally-nilpotent]{局所冪零}自己準同型}] 
		
		補題\ref{lem:suff-for-finite},\textbf{(Step 1)}より,$x_i, y_i$の作用は局所冪零.\hyperref[col:JC]{Jordan分解の保存}より,$\phi(x_i),\ \phi(y_i)$も局所冪零.
		\item[\textbf{Step 5: (Step 4)より$s_i = \exp\phi(x_i)\exp\phi(-y_i)\exp\phi(x_i)$はwell-defined}] \hyperref[def:locally-nilpotent]{局所冪零}の定義の後の文章参照のこと
		\item[\textbf{Step 6: $\mu$を$V$のウェイトとすると,$s_i(V_\mu) = V_{\sigma_i\mu}\ (\sigma_i: \alpha_i\text{に関する鏡映})$}] 
		
		\textbf{(Step 3)}より$V_\mu$は有限次元$S_i$部分加群$V' = V$に含まれる.$s_i|_{V'}$は存在しない$\Lsl(2, \mathbb{K})$の既約表現の分類の自己同型$\tau$で,$s_i(V_\mu) = V_{\sigma_i\mu}$が従う.
		\item[\textbf{Step 7: ウェイトの集合$\Pi(\lambda)$は$\mathscr{W}$の作用で不変.また,$\dim V_\mu = \dim_{\sigma\mu},\ \forall \mu \in \Pi(\lambda),\ \sigma \in \mathscr{W}$}] 
		
		Weyl群$\mathscr{W}$は単純ルートに関する鏡映で生成されたから,\textbf{(Step 6)}より従う.
		\item[\textbf{Step 8: $\Pi(\lambda)$は有限集合.}] 
			
		補題\ref{lem:dom-weight-B}より,$\mu \prec \lambda$を満たす優ウェイト$\mu$は有限なので,$\mathscr{W}$で写した集合も有限.定理\ref{thm:standardcm}\ref{thm:standardcm-c}より,$\Pi(\lambda)$はこれに含まれるから,有限集合.
		\item[\textbf{Step 9: $V$は有限次元.}] 
		
		定理\ref{thm:standardcm}\ref{thm:standardcm-c}より,$V_\mu\ (\mu \in \Pi(\lambda))$は有限次元.$\mu \in \Pi(\lambda)$上の和は有限和なので,$V$は有限次元.
	\end{description}
\end{proof}

\subsection{ウェイトstringとウェイト図}
ここでも優ウェイト$\lambda$に対する有限次元既約$\mathfrak{g}$-加群$V = V(\lambda)$を考える.\\
補題\ref{lem:weightspacerep}より,部分加群$W = \oplus_{i \in \mathbb{Z}} V_{\mu + i\alpha} \subset V$は$S_\alpha$-不変.\\
\hyperref[thm:Weyl]{完全可約性に関するWeylの定理}より,
$\mu + \alpha\mathbb{Z} \cap \Pi(\lambda)$なるウェイトの集合を\textbf{$\bm{\alpha}$-string through $\bm{\mu}$}という.$\Pi(\lambda)$は有限集合であったから,
\begin{align}
	r &\coloneqq \max \bigl\{\, i \in \mathbb{Z}_{\ge 0} \bigm| \beta - i \alpha \in \Pi(\lambda) \,\bigr\}, \\
	q &\coloneqq \max \bigl\{\, i \in \mathbb{Z}_{\ge 0} \bigm| \beta + i \alpha \in \Pi(\lambda) \,\bigr\} 
\end{align}
が定義されるので,命題\ref{prop:a-string-basic}と同様のものが成り立つ.
\begin{myprop}[label=prop:weight-diagram]{$\alpha$-string through$\mu$の性質}
	$\alpha \neq \pm \mu$を充たす任意の$\alpha \in \Phi,\ \mu \in \Pi(\lambda)$に対して
	\begin{align}
		r &\coloneqq \max \bigl\{\, i \in \mathbb{Z}_{\ge 0} \bigm| \beta - i \alpha \in \Phi \,\bigr\}, \\
		q &\coloneqq \max \bigl\{\, i \in \mathbb{Z}_{\ge 0} \bigm| \beta + i \alpha \in \Phi \,\bigr\} 
	\end{align}
	とおく.このとき以下が成り立つ:
	\begin{enumerate}
		\item 
		\hyperref[def:a-sting]{$\alpha$-string through $\mu$}は $\mathbb{E}$ の部分集合
		\begin{align}
			\label{eq:a-string}
			\bigl\{\, \mu + i\alpha \in \mathbb{E} \bigm| -r \le \lambda \le q \,\bigr\} 
		\end{align}
		に等しい.i.e. 
		\begin{align}
			i \in \mathbb{Z} \AND \mu + i\alpha \in \Phi \IMP -r \le i \le q
		\end{align}
		である.
		\item $\sigma_{\alpha}(\mu + i\alpha) = \mu - i\alpha$.i.e. \hyperref[def:a-sting]{$\alpha$-string through $\mu$}は鏡映 $\sigma_\alpha$ の作用の下で不変である.
		\item $r-q = \sspair{\mu}{\alpha}$.特に\hyperref[def:a-sting]{$\alpha$-string through $\mu$}の長さは $4$ 以下である.
		\item 優ウェイト$\lambda$に対し,$\Pi(\lambda)$は\hyperref[weight-saturated]{飽和集合}.
		\item 任意のウェイト$\mu$について
		\begin{equation}
			\mu \in \Pi(\lambda)  \IFF  \mathscr{W}(\mu) \prec \lambda
		\end{equation}
	\end{enumerate}
\end{myprop}
\begin{proof}
	(1)~(3)は命題\ref{prop:a-string-basic}の証明の$\Phi$を$\Pi(\lambda)$に直したもの.\\
	(4)は\hyperref[weight-saturated]{飽和集合}の定義より明らか.\\
	(5)は????????
\end{proof}
ルートと$\Pi(\lambda)$の元を書いた図を\textbf{ウェイト図}と呼ぶ.重複度とやらも書き込みたいので,具体例は次節に回す.

\subsection{$V(\lambda)$の生成系と関係}
$\mathfrak{U}(\mathfrak{g}) \to Z(\lambda) \to V(\lambda)$という準同型をより正確に考察する.\\
$Z(\lambda) = \mathfrak{U}(\mathfrak{g}) / I(\lambda)$であった.同じことだが,$Z(\lambda)$の極大ベクトル$v^+$に対し,$I(\lambda) v^+ = 0$となる.\\
ここで,優ウェイト$\lambda$を固定し,$\mathfrak{U}(\mathfrak{g})$のイデアルで,$V(\lambda)$の極大ベクトルに作用すると$0$となるようなものを$J(\lambda)$とする.$I(\lambda) \subset J(\lambda)$は準同型
\begin{equation}
	Z(\lambda) = \mathfrak{U}(\mathfrak{g}) / I(\lambda) \to V(\lambda) \simeq \mathfrak{U} / J(\lambda)
\end{equation}
を誘導する.定理\ref{thm:suff-for-finite}より,$y_i^{m_i + 1} \in J(\lambda)$であった.
\begin{mytheo}[label=thm:generator-of-J]{$J(\lambda)$の生成系}
	$\lambda \in \Lambda^+,\ m_i = \sspair{\lambda}{\alpha_i}\ (1 \leq i \leq l)$とすると,
	\begin{equation}
		\label{eq:generator-of-J}
		J(\lambda) = I(\lambda) \cup \bigl\{y_i^{m_i + 1} \bigm| 1 \leq i \leq l \bigr\}
	\end{equation}
\end{mytheo}
\begin{proof}
	\eqref{eq:generator-of-J}の右辺を$J'(\lambda)$とする.まず,$V'(\lambda) = \mathfrak{U}(\mathfrak{g}) / J'(\lambda)$が有限次元であることを示す.定理\ref{thm:suff-for-finite}の証明\textbf{(Step 3)}: 有限次元$S_i$-部分加群の和であることを示してしまえば.\textbf{(Step 4)}以降はそのまま使えて$V'(\lambda)$は有限次元となる.つまり,$V'(\lambda)$上$x_i,\ y_i$が\hyperref[def:locally-nilpotent]{局所冪零}であることを示すのだが,$x_i^k$はウェイトを$k\alpha_i$増やすので,ある$k$で最高ウェイトを越えるため,\hyperref[def:locally-nilpotent]{局所冪零}.仮定より,$y_i^{m_i + 1} \btr (1 + J'(\lambda)) = 0$.??????
	
	$V'(\lambda)$は標準巡回加群(または$0$)なので,定理\ref{thm:standardcm}\ref{thm:standardcm-e}より直既約で,\hyperref[thm:Weyl]{完全可約性に関するWeylの定理}より既約(または$0$).
	一方,定義より$J'(\lambda) \subset J(\lambda)$なので,$V(\lambda) \subset V'(\lambda)$.同型でないと,$V'(\lambda)$の既約性に矛盾するから,$V(\lambda) \simeq V'(\lambda)$.i.e. $J(\lambda) \subset J'(\lambda)$.
\end{proof}


\begin{mylem}[label=lem:Lie-bracket-power]{}
	$A$を標数$0$の体$\mathbb{K}$上の結合代数に対し,$y, z \in A,\ k \in \mathbb{Z}$に対し,以下が成り立つ.
	\begin{equation}
		[y^k, z] = \binom{k}{1}[y,z]y^{k-1} + \binom{k}{2}[y,z]y^{k-2} + \cdots + [y,[y, [\cdots [y, z]\cdots ]\cdots ]]
	\end{equation}
\end{mylem}
\begin{proof}
	$[y^1, z] = [y, z]$.ある$k$で成り立つとき,$[y^{k+1}, z]$でも成り立つ.
\end{proof}

\section{重複度公式}
この節では有限次元加群を考える.目標は,$m_\lambda(\mu)$に対するFreudenthalの再帰的公式を示すこと.
\begin{mydef}[label=def:mutiplicity]{ウエイトの重複度}
	$\mu \in \mathfrak{h}^*$を整ウェイト,$\lambda \in \Lambda^+$とする.既約標準巡回$\mathfrak{g}$-加群$V(\lambda)$に対し,
	\begin{equation}
		m_\lambda(\mu) \coloneqq \dim V(\lambda)_\mu
	\end{equation}
	のことを$V(\lambda)$における$\mu$の重複度と呼ぶ($V(\lambda)$のウェイトでなければ$0$とする).
\end{mydef}

\subsection{普遍Casmir演算子}
\hyperref[thm:Weyl]{完全可約性に関するWeylの定理}の証明で出てきた,半単純Lie代数$\mathfrak{g}$の表現のCasimir元$c_\phi$を思い出そう.今はさらに$\mathfrak{U}(\mathfrak{g})$が使えるので,普遍的なCasimir演算子を構成できる.

まずは$\mathfrak{g}$の随伴表現を考える.その跡形式をKilling形式$\kappa$と呼んでいた.半単純Lie代数のKilling形式は非退化であったから,$\kappa$に関する双対基底を構成できる.






随伴表現に対するCasimir演算子
\begin{equation}
	c_{\ad} = \sum_{i=1}^l \ad h_i\ad k_i + \sum_{\alpha \in \Phi} \ad x_\alpha\ad z_\alpha
\end{equation}
は$\mathfrak{g}$,定義より自己準同型.これと同様なものを普遍包絡代数上に定義する.
\begin{mydef}[label=def:univ-Casmir-op]{普遍Casmir演算子}
	半単純Lie代数$\mathfrak{g}$の普遍包絡代数を$\mathfrak{U}(\mathfrak{g})$し,極大トーラス$\mathfrak{h}$の基底を$(h_i)_{i=1}^l$,$x_\alpha \in \mathfrak{g}_\alpha \setminus \{0\}$とする.
	\begin{equation}
		c_\mathfrak{g} = \sum_{i=1}^l h_ih^i + \sum_{\alpha \in \Phi} x_\alpha x^\alpha
	\end{equation}
	を(Killing形式に関する)\textbf{普遍Casmir演算子}と呼ぶ.
\end{mydef}
縮約の取り方から,$c_\mathfrak{g}$は基底によらない.\\
$\ad$は$\mathfrak{U}(\mathfrak{g})$上に拡張すると,$\ad c_\mathfrak{g} = c_{\ad}$となる.?????\\
また,以前示した\hyperref[prop:Casimir-basic]{Casimir演算子の性質}も成り立つ.特に,任意の$\mathfrak{g}$の表現$\phi$に対し,$\phi(c_\mathfrak{g})$と$\phi(\mathfrak{g})$は可換であった.\\
$\phi(c_\mathfrak{g})$と$c_\phi$の関係を見る.
\begin{mylem}[label=lem:]{}
	$\mathfrak{g}$を単純Lie代数とする.$f, g$を$\mathfrak{g}$上の非退化,対称,結合的双線型形式とすると,$0$でないスカラー$a$で$f = ag$となる.
\end{mylem}
\begin{proof}
	各非退化双線型形式$f, g$で定まる$\mathfrak{g}$から$\mathfrak{g}^*$への同型を$x \mapsto s$とすると,
	\begin{equation}
		s_f(y) = f(x, y),\quad  s_g(y) = g(x, y)
	\end{equation}
	となる.
	\hyperref[col:Schur-closed]{代数閉体上のSchurの補題}より,
\end{proof}

\begin{mytheo}[label=thm:]{}
\end{mytheo}





\subsection{ウェイト空間上の跡}
\subsection{Freudenthalの公式}
\subsection{具体例}
\subsection{Formal character}


























\end{document}