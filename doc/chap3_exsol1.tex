\documentclass{ltjsarticle}

%%%%%%%%%%packages%%%%%%%%%%
%% colors and links
\usepackage[svgnames]{xcolor}
\usepackage[colorlinks,citecolor=DarkGreen,linkcolor=DeepPink,linktocpage,unicode]{hyperref} 

%% equations
%%%% math
\usepackage{amsmath,amsfonts,amssymb,amsthm}
\usepackage{mathtools}
\usepackage{mathrsfs}
\usepackage{bm}
\usepackage{cancel}
\usepackage{dsfont}
\usepackage{stmaryrd}
%%%% physics
\usepackage{siunitx}
\usepackage{physics}
%%%% chemistry
\usepackage[version=3]{mhchem}

%% positioning
\usepackage{array}
\usepackage{arydshln}
\usepackage{float}

%% table
\usepackage{booktabs}
\usepackage{multirow}
\usepackage{hhline}
\usepackage{caption}
\captionsetup{format=hang}
\usepackage{subcaption}

%% figure
\usepackage{graphicx}
\usepackage{tikz}
\usetikzlibrary{intersections,calc,patterns,through,positioning,arrows,backgrounds,cd,tqft,braids}
\usepackage{tikz-feynman}
\usepackage{pgfplots}
\usepgfplotslibrary{patchplots}
\pgfplotsset{compat=1.15}

%% decorations
\usepackage{titlesec}
\usepackage{picture}

%% framing
\usepackage{fancybox}
\usepackage{boites}
\usepackage{tcolorbox}
\tcbuselibrary{skins,theorems,breakable}

%% citation
\usepackage{cite}

%% miscellaneous
%%%% comment
\usepackage{comment}
%%%% Japanese
\usepackage{ascmac}
\usepackage{listings} %日本語のコメントアウトをする場合jlisting(もしくはjvlisting)が必要
\lstset{
    basicstyle={\ttfamily},
    identifierstyle={\small},
    commentstyle={\smallitshape},
    keywordstyle={\small\bfseries},
    ndkeywordstyle={\small},
    stringstyle={\small	tfamily},
    frame={tb},
    breaklines=true,
    columns=[l]{fullflexible},
    numbers=left,
    xrightmargin=0pt,
    xleftmargin=27pt,
    numberstyle={\scriptsize},
    stepnumber=1,
    lineskip=-0.5ex
}

%%macros
\usepackage{mylogics}
\usepackage{myalgebra}
\usepackage{mygeometry}
\usepackage{myQM}

%%%%%%%%%%optional settings%%%%%%%%%%

%%%%%図表並列%%%%%
\makeatletter
\newcommand{\figcaption}[1]{\def\@captype{figure}\caption{#1}}
\newcommand{\tblcaption}[1]{\def\@captype{table}\caption{#1}}
\makeatother

%%%%%itemization%%%%%
\renewcommand{\labelenumi}{\theenumi}
\renewcommand{\theenumi}{(\arabic{enumi})} % 箇条書きをローマ数字に

%%%%%%theorem environments%%%%%
\newtheoremstyle{mystyle}%   % スタイル名
    {}%                      % 上部スペース
    {}%                      % 下部スペース
    {\normalfont}%          % 本文フォント
    {}%                      % インデント量
    {\bf}%                  % 見出しフォント
    {.}%                      % 見出し後の句読点
    {\newline}%                     % 見出し後のスペース
    {\underline{\thmname{#1}\thmnumber{#2}\thmnote{(#3)}}}%
    % 見出しの書式 (can be left empty, meaning `normal')
\theoremstyle{mystyle} % スタイルの適用

\newtheorem{theorem}{定理}[section]
\newtheorem{definition}{定義}[section]
\newtheorem{proposition}[definition]{命題}
\newtheorem{corollary}[theorem]{系}
\renewcommand{\proofname}{証明}

%proof
\makeatletter % use at mark
\renewcommand{\qedsymbol}{$\blacksquare$}
\renewenvironment{proof}[1][\proofname]{\par
    \pushQED{\qed}%
    \normalfont \topsep6\p@\@plus6\p@\relax
    \trivlist
    \item[\hskip\labelsep
        \itshape
    \textbf{\underline{#1}}]\ignorespaces
    % {\bf\underline{#1}\@addpunct{.}}]\ignorespaces % ピリオドあり
}{%
    \popQED\endtrivlist\@endpefalse
}
\makeatother % end at mark

%%%%%%mathtools%%%%%
\mathtoolsset{showonlyrefs=true} % 被参照数式のみ数式番号割り振り
\numberwithin{equation}{section}

\makeatletter
\@addtoreset{equation}{section}
\makeatother

%%%%%%framing%%%%%

\newtcolorbox[auto counter,number within=section]{myaxiom}[2][]{%
colback=red!5!white,colframe=red!75!black,fonttitle=\bfseries,
title=公理~\thetcbcounter: #2,#1}

\newtcolorbox[auto counter,number within=section]{mytheo}[2][]{%
colback=blue!5!white,colframe=blue!75!black,fonttitle=\bfseries,
title=定理~\thetcbcounter: #2,#1}

\newtcolorbox[auto counter,number within=section]{myprop}[2][]{%
colback=blue!5!white,colframe=blue!75!black,fonttitle=\bfseries,
title=命題~\thetcbcounter: #2,#1}

\newtcolorbox[auto counter,number within=section]{mylem}[2][]{%
colback=blue!5!white,colframe=blue!75!black,fonttitle=\bfseries,
title=補題~\thetcbcounter: #2,#1}

\newtcolorbox[auto counter,number within=section]{mydef}[2][]{%
colback=green!5!white,colframe=green!75!black,fonttitle=\bfseries,
title=定義~\thetcbcounter: #2,#1}

\newtcolorbox[use counter from=mytheo]{mycol}[2][]{%
fonttitle=\bfseries,enhanced,
attach boxed title to top left=
{xshift=-2mm,yshift=-2mm},
title=系~\thetcbcounter: #2,#1}

\newtcolorbox{marker}[1][breakable]{enhanced,
  	before skip=2mm,after skip=3mm,
  	boxrule=0.4pt,left=5mm,right=2mm,top=1mm,bottom=1mm,
  	colback=yellow!50,
  	colframe=yellow!20!black,
  	sharp corners,rounded corners=southeast,arc is angular,arc=3mm,
  	underlay={%
  	  	\path[fill=tcbcolback!80!black] ([yshift=3mm]interior.south east)--++(-0.4,-0.1)--++(0.1,-0.2);
  	  	\path[draw=tcbcolframe,shorten <=-0.05mm,shorten >=-0.05mm] ([yshift=3mm]interior.south east)--++(-0.4,-0.1)--++(0.1,-0.2);
  	  	\path[fill=yellow!50!black,draw=none] (interior.south west) rectangle node[white]{\Huge\bfseries !} ([xshift=4mm]interior.north west);
  	  	},
  	drop fuzzy shadow,#1}


\newtcolorbox[auto counter, number within=section]{myproblem}[2][]{
    breakable,
    % empty,
    outer arc=0pt,
    arc=0pt,
    enhanced,
    attach boxed title to top left={
    %     % empty,
        xshift=-2mm,
        yshift=-2mm
    },
    coltitle=black,
    colbacktitle=white,
    colframe=black,
    colback=white,
    fonttitle=\bfseries,
    title=【問題~\thetcbcounter】#2,#1
}
\newcommand{\probref}[1]{\textbf{【問題\ref{#1}】}}


%%%%%title decorations%%%%%

%% section
% \titleformat{\section}[block]
% {}{}{0pt}
% {
%     \colorbox{teal}{\begin{picture}(0,20)\end{picture}}
%     \hspace{0pt}
%     \normalfont \Huge\bfseries \thesection
%     \hspace{0.5em}
% }
% [
% \begin{picture}(100,0)
%     \put(3,30){\color{teal}\line(1,0){400}}
% \end{picture}
% \
% \vspace{-50pt}
% ]

% % subsection
% \titleformat{\subsection}[block]
% {}{}{0pt}
% {
%     \colorbox{blue}{\begin{picture}(0,10)\end{picture}}
%     \hspace{0pt}
%     \normalfont \Large\bfseries \thesubsection
%     \hspace{0.5em}
% }
% [
% \begin{picture}(100,0)
%     \put(3,18){\color{blue}\line(1,0){300}}
% \end{picture}
% \
% \vspace{-30pt}
% ]

% % subsubsection
% \titleformat{\subsubsection}[block]
% {}{}{0pt}
% {
%     \normalfont \large\bfseries \thesubsubsection
%     \hspace{0.5em}
% }
% [
% \begin{picture}(100,0)
%     \put(3,15){\color{black}\line(1,0){200}}
% \end{picture}
% \
% \vspace{-25pt}
% ]
\newcommand{\lto}{\longrightarrow}
\newcommand{\lmto}{\longmapsto}
\newcommand{\btl}{\blacktriangleleft}
\newcommand{\btr}{\blacktriangleright}
\usepackage{blkarray}
\DeclarePairedDelimiterX{\sspair}[2]{\llbracket}{\rrbracket}{#1, #2}
\newcommand{\Euc}[1]{\bigl(\, #1,\, (\;,\,)_{#1}\,\bigr)}
\newcommand{\Weyl}[2]{\mathscr{W}_{#1}(#2)}

\begin{document}

\title{Humphreys Chapter III Exercises \\ 問題(2024/1/15 実施分)}
\author{高間俊至}
\date{\today}
\maketitle

\setcounter{section}{6}

\cite[p.63, Exercise1, 3]{Humphreys1972introduction}
% と\cite[p.14, Exercise 2]{Humphreys1972introduction}
の解答例です.

何の断りもない場合,体 $\mathbb{K}$ は代数閉体でかつ $\character \mathbb{K} = 0$ であるとする.
% また,$\mathbb{E}$ を有限次元Euclid空間であるとする.
% また,$\mathfrak{g}$ は\underline{零でない}体 $\mathbb{K}$ 上の\underline{有限次元} Lie代数とする.

% \section{$\Lsl(2)$ の表現論}

% \footnote{Euclid空間と言って位相空間のことを指す場合があるが,そのときは双線型形式 $\rpair{\;}{\,}_{\mathbb{E}}$ を使って $\mathbb{E}$ 上の距離関数を $d_{\mathbb{E}} \colon \mathbb{E} \times \mathbb{E} \lto \mathbb{R}_{\ge 0},\; (x,\, y) \lmto \rpair{x-y}{x-y}_{\mathbb{E}}$ と定義し(これは通常\textbf{Euclid距離}と呼ばれる),$\mathbb{E}$ に $d_{\mathbb{E}}$ による距離位相を入れる.}.
Euclid空間 $\Euc{\mathbb{E}}$ の任意の元 $\alpha \in \mathbb{E}$ に対して,
\begin{itemize}
    \item \textbf{鏡映面} (reflecting hyperplane)\footnote{余次元 $1$ の部分 $\mathbb{R}$-ベクトル空間.最右辺は\hyperref[def:radical-bilinear]{対称かつ非退化な双線型形式 $(\;,\,)_{\mathbb{E}}$ による直交補空間}の意味である.}
    \begin{align}
        \label{def:hyperplane}
        \bm{P_\alpha} \coloneqq \bigl\{\, \beta \in \mathbb{E} \bigm| \rpair{\beta}{\alpha}_{\mathbb{E}} = 0 \,\bigr\} = (\mathbb{R}\alpha)^\perp
    \end{align}
    \item 鏡映面 $P_\alpha$ に関する\textbf{鏡映} (reflecting)
    \begin{align}
        \bm{\sigma_\alpha} \colon \mathbb{E} \lto \mathbb{E},\; \beta \lmto \beta - 2 \frac{\rpair{\beta}{\alpha}_{\mathbb{E}}}{\rpair{\alpha}{\alpha}_{\mathbb{E}}} \alpha
    \end{align}
\end{itemize}
を考える.

\begin{marker}
    $2 \frac{(\beta,\, \alpha)}{(\alpha,\, \alpha)} \in \mathbb{R}$ が頻繁に登場するので,
    \begin{align}
		\sspair{\beta}{\alpha} \coloneqq 2\frac{(\beta,\, \alpha)_{\mathbb{E}}}{(\alpha,\, \alpha)_{\mathbb{E}}}
	\end{align}
	と略記することにする.写像 $\sspair{\;}{\,} \colon \mathbb{E} \times \mathbb{E} \lto \mathbb{R}$ は記号的には内積のように見えるかもしれないが,あくまで第一引数についてのみ線型なのであって,\underline{対称でも双線型でもない}ことに注意.
\end{marker}

\begin{myaxiom}[label=ax:root-system,breakable]{ルート系}
	\begin{itemize}
		\item 有限次元Euclid空間 $\Euc{\mathbb{E}}$ 
		\item $\mathbb{E}$ の部分集合 $\Phi \subset \mathbb{E}$
	\end{itemize}
	の組 $(\mathbb{E},\, \Phi)$ が\textbf{ルート系} (root system) であるとは,以下の条件を充たすことを言う:
	\begin{description}
		\item[\textbf{(Root-1)}] $\Phi$ は $0$ を含まない有限集合で,かつ $\mathbb{E} = \Span_{\mathbb{R}} \Phi$ を充たす.
		\item[\textbf{(Root-2)}] $\lambda \alpha \in \Phi \IMP \lambda = \pm 1$
		\item[\textbf{(Root-3)}] $\alpha,\, \beta \in \Phi \IMP \sigma_\alpha(\beta) \in \Phi$
		\item[\textbf{(Root-4)}] $\alpha,\, \beta \in \Phi \IMP \sspair{\beta}{\alpha} \in \mathbb{Z}$
	\end{description}
	\tcblower
	$\Phi$ の元のことを\textbf{ルート} (root) と呼ぶ.
\end{myaxiom}


\begin{mydef}[label=def:Weylgroup]{Weyl群}
	$(\mathbb{E},\, \Phi)$ を\hyperref[ax:root-system]{ルート系}とする.
	$\LGL (\mathbb{E})$ の部分集合 $\bigl\{\, \sigma_\alpha \in \LGL(\mathbb{E}) \bigm| \alpha \in \Phi \,\bigr\}$ が生成する $\LGL(\mathbb{E})$ の部分群のことをルート系 $(\mathbb{E},\, \Phi)$ の\textbf{Weyl群} (Weyl group) と呼び,$\bm{\Weyl{\mathbb{E}}{\Phi}}$ と書く.
	
	% \tcblower
	
	% $\forall \sigma \in \mathscr{W}$ に対して,
	% \begin{align}
	% 	\bm{\len (\sigma)} \coloneqq \min \bigl\{\, t \in \mathbb{Z}_{\ge 0} \bigm| \sigma = \sigma_{\alpha_1} \circ \cdots\circ \sigma_{\alpha_t} \WHERE \alpha_i \in \Phi \,\bigr\} 
	% \end{align}
	% を $\sigma$ の\textbf{長さ} (length) と呼ぶ.ただし $\len (\mathrm{id}_{\mathbb{E}}) \coloneqq 0$ と定義する.
\end{mydef}

\begin{mydef}[label=def:dual-root]{双対ルート系}
	\hyperref[ax:root-system]{ルート系} $(\mathbb{E},\, \Phi)$ に対して
	\begin{align}
		\bm{\Phi^{\vee}} \coloneqq \biggl\{\, \frac{2}{(\alpha,\, \alpha)} \alpha \in \mathbb{E} \biggm| \alpha \in \Phi \,\biggr\} 
	\end{align}
	とおき,組 $\bm{(\mathbb{E},\, \Phi^\vee)}$ のことを $(\mathbb{E},\, \Phi)$ の\textbf{双対ルート系} (dual root system) と呼ぶ.
	
	\tcblower
	
	$\alpha \in \Phi$ に対して
	\begin{align}
		\bm{\alpha^\vee} \coloneqq \frac{2}{(\alpha,\, \alpha)} \alpha \in \Phi^\vee
	\end{align}
	と書く.
\end{mydef}

\begin{myproblem}[label=ex:3-9-2]{p.46 の Exercise 2}
    % \setcounter{\tcbcounter}{7}
    \begin{enumerate}
        \item ルート系 $\Phi$ を与えたとき,$\Phi$ の\hyperref[def:dual-root]{双対ルート系} $\Phi^\vee$ もまたルート系であることを示せ.
        \item $\Phi^\vee$ のWeyl群は $\Phi$ のWeyl群と同型であることを示せ.
        \item $\sspair{\alpha^\vee}{\beta^\vee} = \sspair{\beta}{\alpha}$ を示せ.
    \end{enumerate}
\end{myproblem}


\begin{myproblem}[label=ex:3-sp]{}
    ノートでは分類定理の $A_l,\, B_l,\, C_l,\, D_l$ 型の $l$ に条件をつけたが,これは $l$ が小さいところでは同型があるからである.
    この同型をできるだけ多く見つけてみよう.
\end{myproblem}

\begin{myproblem}[label=ex:3-11-1]{p.63 の Exercise 1}
    Dynkin図形からCartan行列を復元せよ.
\end{myproblem}

\begin{myproblem}[label=ex:3-11-3]{p.63 の Exercise 3}
    % \setcounter{\tcbcounter}{7}
    $G_2$ のCartan行列から $G_2$ のルート系を復元し,\cite[p.44]{Humphreys1972introduction}の図と整合しているかどうか確認せよ.
\end{myproblem}

\begin{myproblem}[label=ex:3-11-5]{p.63 の Exercise 5}
    % \setcounter{\tcbcounter}{7}
    \begin{enumerate}
        \item 
        $B_l,\, C_l$ 以外の既約なルート系はその\hyperref[def:dual-root]{双対ルート系}とルート系として同型であることを示せ.
        \item 
        $B_l,\, C_l$ は互いに双対ルート系であることを示せ.
    \end{enumerate}
\end{myproblem}

位数 $2n$ の\textbf{2面体群} (dihedral group) とは\footnote{抽象代数の分野だと $\mathrm{D}_{\textcolor{red}{2n}}$ と書く場合があるので注意.},
\begin{align}
    \mathrm{D}_n \coloneqq \braket{s,\, t}{s^2 = t^2 = 1_{D_n},\; (st)^n = 1_{D_n}}
\end{align}
の生成元と関係式によって定義される有限群のことである.

\begin{myproblem}[label=ex:3-9-4]{p.63 の Exercise 4}
    % \setcounter{\tcbcounter}{7}
    \begin{enumerate}
        \item 
        $A_1 \oplus A_1,\; A_2,\, B_2,\, G_2$ のルート系を図示せよ.
        \item 
        2次元Euclid空間 $\mathbb{E}^2$ および\hyperref[def:Weylgroup]{Weyl群}について,
        \begin{align}
            \Weyl{\mathbb{E}^2}{A_1 \oplus A_1} &\cong \mathrm{D}_2, \\
            \Weyl{\mathbb{E}^2}{A_2} &\cong \mathrm{D}_3, \\
            \Weyl{\mathbb{E}^2}{B_2} &\cong \mathrm{D}_4, \\
            \Weyl{\mathbb{E}^2}{G_2} &\cong \mathrm{D}_6
        \end{align}
        を示せ.
        % \item 任意のランク $2$ のルート系のWeyl群は $\mathrm{D}_2,\, \mathrm{D}_3,\, \mathrm{D}_4,\, \mathrm{D}_6$ のいずれかであることを示せ.
    \end{enumerate}
\end{myproblem}

% $1 \le \forall i \le l$ に対して定義できる写像
% \begin{align}
%     \varepsilon^i \colon \mathfrak{h} \lto \mathbb{K},\; \mqty[\dmat{\lambda_1,\ddots,\lambda_{l},-\lambda_{1},\ddots,-\lambda_{l}}] \lmto \lambda_i
% \end{align}
% は $\mathbb{K}$-線型写像なので,$\varepsilon^i \in \mathfrak{h}^*$ である.

% $\forall h = \mqty[\dmat{h_1,\ddots,h_{l},-h_{1},\ddots,-h_{l}}] \in \mathfrak{h}$ に対して,
% $1 \le i \neq j \le l$ ならば
% \begin{align}
%     \ad(h)(e_{l+i,j}) &= -(h_i + h_j) e_{l+i,j} = -\bigl(\varepsilon^i(h) + \varepsilon^j(h)\bigr)\, e_{l+i,j}, \\
%     \ad(h)(e_{i,l+j}) &= (h_i + h_j) e_{i,l+j} = \bigl(\varepsilon^i(h) + \varepsilon^j(h)\bigr)\, e_{i,l+j}, \\
%     \ad(h)(e_{i,j}) &= - \ad(h)(e_{l+i,l+j}) = (h_i - h_j) e_{i,j} = \bigl(\varepsilon^i(h) - \varepsilon^j(h)\bigr)\,e_{i,j}
% \end{align}
% が成り立つので,
% \begin{align}
%     \ad(h)(e_{l+i,j} - e_{l+j,i}) &= -(\varepsilon^i + \varepsilon^j)(h)\, (e_{l+i,j} - e_{l+j,i}), \\
%     \ad(h)(e_{i,l+j} - e_{j,l+i}) &= (\varepsilon^i + \varepsilon^j)(h)\, (e_{i,l+j} - e_{j,l+i}), \\
%     \ad(h)(e_{i,j} - e_{l+j,l+i}) &= (\varepsilon^i - \varepsilon^j)(h)\, (e_{i,j} - e_{l+j,l+i})
% \end{align}
% がわかった.故に
% % $\alpha^{\pm}_{ij} \in \mathfrak{h}^*$ を
% % \begin{align}
% %     \alpha^+_{ij}(h) &\coloneqq h_i + h_j \\
% %     \alpha^-_{ij}(h) &\coloneqq h_i - h_j
% % \end{align}
% % と定義すれば
% \begin{align}
%     D_l
%     &\coloneqq 
%     \bigl\{\, \varepsilon^i + \varepsilon^j \in \mathfrak{h}^* \setminus \{0\} \bigm| 1 \le i \neq j \le l \,\bigr\} \\
%     &\quad \cup \bigl\{\, \varepsilon^i - \varepsilon^j \in \mathfrak{h}^* \setminus \{0\} \bigm| 1 \le i \neq j \le l \,\bigr\} \\
%     &= \bigl\{\, \pm \varepsilon^i \pm \varepsilon^j \in \mathfrak{h}^* \setminus \{0\} \bigm| 1 \le i < j \le l \,\bigr\}
% \end{align}
% がルート系となる.ただし最右辺は複号任意である.
% 実際 $i < j$ ならば
% \begin{align}
%     \mathfrak{g}_{-\varepsilon^i-\varepsilon^j} &= \mathbb{K}(e_{l+i,j} - e_{l+j,i}), \\
%     \mathfrak{g}_{\varepsilon^i+\varepsilon^j} &= \mathbb{K}(e_{i,l+j} - e_{j,l+i}) \\
%     \mathfrak{g}_{\varepsilon^i-\varepsilon^j} &= \mathbb{K}(e_{i,j} - e_{l+j,l+i}) \\
%     \mathfrak{g}_{-\varepsilon^i+\varepsilon^j} &= \mathbb{K}(e_{j,i} - e_{l+i,l+j})
% \end{align}
% なので,$D_l$ 型の基底の取り方からルート空間分解
% \begin{align}
%     \mathfrak{g} = \mathfrak{h} \oplus \bigoplus_{\alpha \in D_l} \mathfrak{g}_{\alpha}
% \end{align}
% が再現される.

% \begin{myproblem}[label=ex:2-8-2C]{p.40 の Exercise 2}
%     % \setcounter{\tcbcounter}{7}
%     $\mathfrak{g}$ を $C_l$ 型Lie代数とする.
%     極大トーラス
%     \begin{align}
%         \mathfrak{h} \coloneqq \Biggl\{\, \mqty[\dmat{\lambda_1,\ddots,\lambda_{l},-\lambda_{1},\ddots,-\lambda_{l}}] \in \Mat{2l}{\mathbb{K}} \,\Biggr\} 
%     \end{align}
%     に付随するルート系を求めよ.
% \end{myproblem}


% $1 \le \forall i \le l$ に対して定義できる写像
% \begin{align}
%     \varepsilon^i \colon \mathfrak{h} \lto \mathbb{K},\; \mqty[\dmat{\lambda_1,\ddots,\lambda_{l},-\lambda_{1},\ddots,-\lambda_{l}}] \lmto \lambda_i
% \end{align}
% は $\mathbb{K}$-線型写像なので,$\varepsilon^i \in \mathfrak{h}^*$ である.
% $\forall h = \mqty[\dmat{h_1,\ddots,h_{l},-h_{1},\ddots,-h_{l}}] \in \mathfrak{h}$ に対して,
% $1 \le i \neq j \le l$ ならば
% \begin{align}
%     \ad(h)(e_{l+i,i}) &= -2h_i e_{l+i,i} = -2\varepsilon^i(h)\, e_{l+i,i}, \\
%     \ad(h)(e_{l+i,j}) &= -(h_i + h_j) e_{l+i,j} = - \bigl( \varepsilon^i(h) + \varepsilon^j(h) \bigr)\, e_{l+i,j} , \\
%     \ad(h)(e_{i,l+i}) &= 2h_i e_{i,l+i} = 2\varepsilon^i(h)\, e_{i,l+i}, \\
%     \ad(h)(e_{i,l+j}) &= (h_i + h_j) e_{i,l+j} = \bigl( \varepsilon^i(h) + \varepsilon^j(h) \bigr)\, e_{i,l+j}, \\
%     \ad(h)(e_{i,j}) &= - \ad(h)(e_{l+i,l+j}) = (h_i - h_j) e_{i,j} = \bigl( \varepsilon^i(h) - \varepsilon^j(h) \bigr)\, e_{i,j}
% \end{align}
% が成り立つので,
% \begin{align}
%     \ad(h)(e_{l+i,i}) &= -2\varepsilon^i(h)\, e_{l+i,i}, \\
%     \ad(h)(e_{l+i,j} + e_{l+j,i}) &= -(\varepsilon^i + \varepsilon^j)(h)\, (e_{l+i,j} + e_{l+j,i}), \\
%     \ad(h)(e_{i,l+j} + e_{j,l+i}) &= (\varepsilon^i + \varepsilon^j)(h)\, (e_{i,l+j} + e_{j,l+i}), \\
%     \ad(h)(e_{i,j} - e_{l+j,l+i}) &= (\varepsilon^i - \varepsilon^j)(h)\, (e_{i,j} - e_{l+j,l+i})
% \end{align}
% がわかった.故に
% % $\alpha_{i} \in \mathfrak{h}^*$ を
% % \begin{align}
% %     \alpha_{i}(h) &\coloneqq 2h_i \\
% % \end{align}
% % と定義すれば
% \begin{align}
%     C_l
%     &\coloneqq D_l \cup \bigl\{\, \pm 2\varepsilon^i \in \mathfrak{h}^* \setminus \{0\}\bigm| 1 \le i \le l \,\bigr\}
% \end{align}
% がルート系となる.

% \begin{myproblem}[label=ex:2-8-2B]{p.40 の Exercise 2}
%     % \setcounter{\tcbcounter}{7}
%     $\mathfrak{g}$ を $B_l$ 型Lie代数とする.
%     極大トーラス
%     \begin{align}
%         \mathfrak{h} \coloneqq \Biggl\{\, \mqty[\dmat{0,\lambda_1,\ddots,\lambda_{l},-\lambda_{1},\ddots,-\lambda_{l}}] \in \Mat{2l+1}{\mathbb{K}} \,\Biggr\} 
%     \end{align}
%     に付随するルート系を求めよ.
% \end{myproblem}

% $1 \le \forall i \le l$ に対して定義できる写像
% \begin{align}
%     \varepsilon^i \colon \mathfrak{h} \lto \mathbb{K},\; \mqty[\dmat{0,\lambda_1,\ddots,\lambda_{l},-\lambda_{1},\ddots,-\lambda_{l}}] \lmto \lambda_i
% \end{align}
% は $\mathbb{K}$-線型写像なので,$\varepsilon^i \in \mathfrak{h}^*$ である.
% $\forall h = \mqty[\dmat{0,h_1,\ddots,h_{l},-h_{1},\ddots,-h_{l}}] \in \mathfrak{h}$ に対して,
% $1 \le i \neq j \le l$ ならば
% \begin{align}
%     \ad(h)(e_{1+i,1}) &= h_i e_{1,1+i} = \varepsilon^i(h)\, e_{1,1+i}, \\
%     \ad(h)(e_{1,1+i}) &= -h_i e_{1,1+i} = -\varepsilon^i(h)\, e_{1,1+i}, \\
%     \ad(h)(e_{1+l+i,1}) &= -h_i e_{1+l+i,1} = -\varepsilon^i(h)\, e_{1+l+i,1}, \\
%     \ad(h)(e_{1,1+l+i}) &= h_i e_{1,1+l+i} = \varepsilon^i(h)\, e_{1,1+l+i}, \\
%     \ad(h)(e_{1+l+i,1+j}) &= -(h_i + h_j) e_{1+l+i,1+j} = -\bigl( \varepsilon^i(h) + \varepsilon^j(h) \bigr)\, e_{1+l+i,1+j} , \\
%     \ad(h)(e_{1+i,1+l+j}) &= (h_i + h_j) e_{1+i,1+l+j} = \bigl( \varepsilon^i(h) + \varepsilon^j(h) \bigr)\,  e_{1+i,1+l+j}, \\
%     \ad(h)(e_{1+i,1+j}) &= - \ad(h)(e_{1+l+i,1+l+j}) = (h_i - h_j) e_{i,j} = \bigl( \varepsilon^i(h) - \varepsilon^j(h) \bigr)\, e_{i,j}
% \end{align}
% が成り立つので,
% \begin{align}
%     \ad(h)(e_{1,1+i} - e_{1+l+i,1}) &= -\varepsilon^i(h)\, (e_{1,1+i} - e_{1+l+i,1}), \\
%     \ad(h)(e_{1+i,1} - e_{1,1+l+i}) &= \varepsilon^i(h)\,(e_{1+i,1} - e_{1,1+l+i}), \\
%     \ad(h)(e_{1+l+i,1+j} + e_{1+l+j,1+i}) &= -(\varepsilon^i + \varepsilon^j)(h)\, (e_{1+l+i,1+j} + e_{1+l+j,1+i}), \\
%     \ad(h)(e_{1+i,1+l+j} + e_{1+j,1+l+i}) &= (\varepsilon^i + \varepsilon^j)(h)\, (e_{1+i,1+l+j} + e_{1+j,1+l+i}), \\
%     \ad(h)(e_{1+i,1+j} - e_{1+l+j,1+l+i}) &= (\varepsilon^i - \varepsilon^j)(h)\, (e_{1+i,1+j} - e_{1+l+j,1+l+i})
% \end{align}
% がわかった.故に
% % $\beta_{i} \in \mathfrak{h}^*$ を
% % \begin{align}
% %     \beta_{i}(h) &\coloneqq h_i \\
% % \end{align}
% % と定義すれば
% \begin{align}
%     B_l
%     &\coloneqq D_l \cup \bigl\{\, \pm \varepsilon^i \in \mathfrak{h}^* \setminus \{0\}\bigm| 1 \le i \le l \,\bigr\}
% \end{align}
% がルート系となる.

% \begin{myproblem}[label=ex:2-8-10B]{p.40 の Exercise 10}
%     % \setcounter{\tcbcounter}{7}
%     $4,\, 5,\, 7$ 次元の半単純Lie代数が存在しないことを示せ
% \end{myproblem}

% \begin{proof}
%     半単純Lie代数 $\mathfrak{g}$ とその極大トーラス $\mathfrak{h} \subset \mathfrak{g}$ をとる.
%     $\mathfrak{h} \neq 0$ なので $\dim \mathfrak{h} > 0$ である.
%     このときルート空間分解
%     \begin{align}
%         \mathfrak{g} = \mathfrak{h} \oplus \bigoplus_{\alpha \in \Phi} \mathfrak{g}_\alpha
%     \end{align}
%     が成り立つ.

%     ところで,$\alpha \in \Phi \IMP -\alpha \in \Phi$ かつ $\forall \alpha \in \Phi$ に対して $\dim \mathfrak{g}_\alpha = 1$ なのである整数 $k$ が存在して
%     \begin{align}
%         0 < \dim \mathfrak{h} = \dim \mathfrak{g} - \sum_{\alpha \in \Phi} \dim \mathfrak{g}_\alpha = \dim \mathfrak{g} - 2k
%     \end{align}
%     と書ける.一方で $\mathfrak{h}^* = \Span_{\mathbb{K}} \Phi = \Span_{\mathbb{K}} \{\pm \alpha_1,\, \dots,\, \pm\alpha_k\}$ であるから $\dim \mathfrak{h} = \dim \mathfrak{h}^* \le k$ である.よって
%     \begin{align}
%         \frac{1}{3} \dim \mathfrak{g} \le k < \frac{1}{2} \dim \mathfrak{g}
%     \end{align}
%     が成り立たねばならない.
%     \begin{itemize}
%         \item $\dim \mathfrak{g} = 4$ だとすると $k \notin \mathbb{Z}$ となり矛盾.
%         \item $\dim \mathfrak{g} = 5$ だとすると $k=2$ に確定する.このとき $\pm \alpha_1 \neq \pm \alpha_2$(複合任意)を充たす $\alpha_1,\, \alpha_2 \in \Phi$ が取れるが,$\dim \mathfrak{h} = \dim \mathfrak{h}^* = 1$ となるのである $\lambda \in \mathbb{K}$ が存在して $\alpha_2 = \lambda \alpha_1$ と書ける.
%         すると $\lambda \alpha_1 \in \Phi$ なので $\lambda = \pm 1$ のどちらかでなくてはいけず,$\pm \alpha_1 \neq \pm\alpha_2$ に矛盾する.
%         \item $\dim \mathfrak{g} = 7$ だとすると $k=3$ に確定する.すると $\dim \mathfrak{h} = \dim \mathfrak{h}^* = 1$ となり $\dim \mathfrak{g} = 5$ のときと同様の議論から矛盾する.
%     \end{itemize}
%     よって背理法から $\dim \mathfrak{g} \neq 4,\, 5,\, 7$ が示された.
% \end{proof}


%参考文献のスタイル(style.bstを参照)
\bibliographystyle{myh-physrev}
%参考文献(reference.bibを参照)
\bibliography{repref}

\end{document}