\documentclass{ltjsarticle}

%%%%%%%%%%packages%%%%%%%%%%
%% colors and links
\usepackage[svgnames]{xcolor}
\usepackage[colorlinks,citecolor=DarkGreen,linkcolor=DeepPink,linktocpage,unicode]{hyperref} 

%% equations
%%%% math
\usepackage{amsmath,amsfonts,amssymb,amsthm}
\usepackage{mathtools}
\usepackage{mathrsfs}
\usepackage{bm}
\usepackage{cancel}
\usepackage{dsfont}
%%%% physics
\usepackage{siunitx}
\usepackage{physics}
%%%% chemistry
\usepackage[version=3]{mhchem}

%% positioning
\usepackage{array}
\usepackage{arydshln}
\usepackage{float}

%% table
\usepackage{booktabs}
\usepackage{multirow}
\usepackage{hhline}
\usepackage{caption}
\captionsetup{format=hang}
\usepackage{subcaption}

%% figure
\usepackage{graphicx}
\usepackage{tikz}
\usetikzlibrary{intersections,calc,patterns,through,positioning,arrows,backgrounds,cd,tqft,braids}
\usepackage{tikz-feynman}
\usepackage{pgfplots}
\usepgfplotslibrary{patchplots}
\pgfplotsset{compat=1.15}

%% decorations
\usepackage{titlesec}
\usepackage{picture}

%% framing
\usepackage{fancybox}
\usepackage{boites}
\usepackage{tcolorbox}
\tcbuselibrary{skins,theorems,breakable}

%% citation
\usepackage{cite}

%% miscellaneous
%%%% comment
\usepackage{comment}
%%%% Japanese
\usepackage{ascmac}
\usepackage{listings} %日本語のコメントアウトをする場合jlisting(もしくはjvlisting)が必要
\lstset{
    basicstyle={\ttfamily},
    identifierstyle={\small},
    commentstyle={\smallitshape},
    keywordstyle={\small\bfseries},
    ndkeywordstyle={\small},
    stringstyle={\small	tfamily},
    frame={tb},
    breaklines=true,
    columns=[l]{fullflexible},
    numbers=left,
    xrightmargin=0pt,
    xleftmargin=27pt,
    numberstyle={\scriptsize},
    stepnumber=1,
    lineskip=-0.5ex
}

%%macros
\usepackage{mylogics}
\usepackage{myalgebra}
\usepackage{mygeometry}
\usepackage{myQM}

%%%%%%%%%%optional settings%%%%%%%%%%

%%%%%図表並列%%%%%
\makeatletter
\newcommand{\figcaption}[1]{\def\@captype{figure}\caption{#1}}
\newcommand{\tblcaption}[1]{\def\@captype{table}\caption{#1}}
\makeatother

%%%%%itemization%%%%%
\renewcommand{\labelenumi}{\theenumi}
\renewcommand{\theenumi}{(\arabic{enumi})} % 箇条書きをローマ数字に

%%%%%%theorem environments%%%%%
\newtheoremstyle{mystyle}%   % スタイル名
    {}%                      % 上部スペース
    {}%                      % 下部スペース
    {\normalfont}%          % 本文フォント
    {}%                      % インデント量
    {\bf}%                  % 見出しフォント
    {.}%                      % 見出し後の句読点
    {\newline}%                     % 見出し後のスペース
    {\underline{\thmname{#1}\thmnumber{#2}\thmnote{(#3)}}}%
    % 見出しの書式 (can be left empty, meaning `normal')
\theoremstyle{mystyle} % スタイルの適用

\newtheorem{theorem}{定理}[section]
\newtheorem{definition}{定義}[section]
\newtheorem{proposition}[definition]{命題}
\newtheorem{corollary}[theorem]{系}
\renewcommand{\proofname}{証明}

%proof
\makeatletter % use at mark
\renewcommand{\qedsymbol}{$\blacksquare$}
\renewenvironment{proof}[1][\proofname]{\par
    \pushQED{\qed}%
    \normalfont \topsep6\p@\@plus6\p@\relax
    \trivlist
    \item[\hskip\labelsep
        \itshape
    \textbf{\underline{#1}}]\ignorespaces
    % {\bf\underline{#1}\@addpunct{.}}]\ignorespaces % ピリオドあり
}{%
    \popQED\endtrivlist\@endpefalse
}
\makeatother % end at mark

%%%%%%mathtools%%%%%
\mathtoolsset{showonlyrefs=true} % 被参照数式のみ数式番号割り振り
\numberwithin{equation}{section}

\makeatletter
\@addtoreset{equation}{section}
\makeatother

%%%%%%framing%%%%%

\newtcolorbox[auto counter,number within=section]{myaxiom}[2][]{%
colback=red!5!white,colframe=red!75!black,fonttitle=\bfseries,
title=公理~\thetcbcounter: #2,#1}

\newtcolorbox[auto counter,number within=section]{mytheo}[2][]{%
colback=blue!5!white,colframe=blue!75!black,fonttitle=\bfseries,
title=定理~\thetcbcounter: #2,#1}

\newtcolorbox[auto counter,number within=section]{myprop}[2][]{%
colback=blue!5!white,colframe=blue!75!black,fonttitle=\bfseries,
title=命題~\thetcbcounter: #2,#1}

\newtcolorbox[auto counter,number within=section]{mylem}[2][]{%
colback=blue!5!white,colframe=blue!75!black,fonttitle=\bfseries,
title=補題~\thetcbcounter: #2,#1}

\newtcolorbox[auto counter,number within=section]{mydef}[2][]{%
colback=green!5!white,colframe=green!75!black,fonttitle=\bfseries,
title=定義~\thetcbcounter: #2,#1}

\newtcolorbox[use counter from=mytheo]{mycol}[2][]{%
fonttitle=\bfseries,enhanced,
attach boxed title to top left=
{xshift=-2mm,yshift=-2mm},
title=系~\thetcbcounter: #2,#1}

\newtcolorbox{marker}[1][breakable]{enhanced,
  	before skip=2mm,after skip=3mm,
  	boxrule=0.4pt,left=5mm,right=2mm,top=1mm,bottom=1mm,
  	colback=yellow!50,
  	colframe=yellow!20!black,
  	sharp corners,rounded corners=southeast,arc is angular,arc=3mm,
  	underlay={%
  	  	\path[fill=tcbcolback!80!black] ([yshift=3mm]interior.south east)--++(-0.4,-0.1)--++(0.1,-0.2);
  	  	\path[draw=tcbcolframe,shorten <=-0.05mm,shorten >=-0.05mm] ([yshift=3mm]interior.south east)--++(-0.4,-0.1)--++(0.1,-0.2);
  	  	\path[fill=yellow!50!black,draw=none] (interior.south west) rectangle node[white]{\Huge\bfseries !} ([xshift=4mm]interior.north west);
  	  	},
  	drop fuzzy shadow,#1}


\newtcolorbox[auto counter, number within=section]{myproblem}[2][]{
    breakable,
    % empty,
    outer arc=0pt,
    arc=0pt,
    enhanced,
    attach boxed title to top left={
    %     % empty,
        xshift=-2mm,
        yshift=-2mm
    },
    coltitle=black,
    colbacktitle=white,
    colframe=black,
    colback=white,
    fonttitle=\bfseries,
    title=【問題~\thetcbcounter】#2,#1
}
\newcommand{\probref}[1]{\textbf{【問題\ref{#1}】}}


%%%%%title decorations%%%%%

%% section
% \titleformat{\section}[block]
% {}{}{0pt}
% {
%     \colorbox{teal}{\begin{picture}(0,20)\end{picture}}
%     \hspace{0pt}
%     \normalfont \Huge\bfseries \thesection
%     \hspace{0.5em}
% }
% [
% \begin{picture}(100,0)
%     \put(3,30){\color{teal}\line(1,0){400}}
% \end{picture}
% \
% \vspace{-50pt}
% ]

% % subsection
% \titleformat{\subsection}[block]
% {}{}{0pt}
% {
%     \colorbox{blue}{\begin{picture}(0,10)\end{picture}}
%     \hspace{0pt}
%     \normalfont \Large\bfseries \thesubsection
%     \hspace{0.5em}
% }
% [
% \begin{picture}(100,0)
%     \put(3,18){\color{blue}\line(1,0){300}}
% \end{picture}
% \
% \vspace{-30pt}
% ]

% % subsubsection
% \titleformat{\subsubsection}[block]
% {}{}{0pt}
% {
%     \normalfont \large\bfseries \thesubsubsection
%     \hspace{0.5em}
% }
% [
% \begin{picture}(100,0)
%     \put(3,15){\color{black}\line(1,0){200}}
% \end{picture}
% \
% \vspace{-25pt}
% ]
\newcommand{\lto}{\longrightarrow}
\newcommand{\lmto}{\longmapsto}
\newcommand{\btl}{\blacktriangleleft}
\newcommand{\btr}{\blacktriangleright}
\usepackage{blkarray}

\begin{document}

\title{Humphreys Chapter II Exercises \\ 解答例(2023/12/11 実施分)}
\author{高間俊至}
\date{\today}
\maketitle

\setcounter{section}{6}

\cite[p.34, Exercise1, 2; p.40, Exercise1, 2, 10]{Humphreys1972introduction}
% と\cite[p.14, Exercise 2]{Humphreys1972introduction}
の解答例です.

何の断りもない場合,体 $\mathbb{K}$ は代数閉体でかつ $\character \mathbb{K} = 0$ であるとする.
また,$\mathfrak{g}$ は\underline{零でない}体 $\mathbb{K}$ 上の\underline{有限次元} Lie代数とする.

\section{$\Lsl(2)$ の表現論}

$\Lsl(2,\, \mathbb{K})$ の基底として
\begin{align}
    x &\coloneqq \mqty[0 & 1 \\ 0&0], & y &\coloneqq \mqty[0&0 \\ 1 &0], & h &\coloneqq \mqty[1 & 0 \\ 0 & -1]
\end{align}
をとることができる.この基底同士のLieブラケットは
\begin{align}
    \label{eq:comm-sl2}
    \comm{x}{y} &= h, & \comm{h}{x} &= 2x, & \comm{h}{y} &= -2y
\end{align}
と計算できる.

$\Lsl(2,\, \mathbb{K})$ の任意の有限次元表現 $\phi \colon \Lsl(2,\, \mathbb{K}) \lto \Lgl(V)$ 
(同じことだが,有限次元 $\Lsl(2,\, \mathbb{K})$-加群 $V$\footnote{$\Lsl(2,\, \mathbb{K})$ の $V$ への左作用を $\btr \colon \Lsl(2,\, \mathbb{K}) \times V \lto V,\; (x,\, v) \lmto x \btr v \coloneqq \phi(x)(v)$ と書く.})において,
\begin{itemize}
    \item $\mathbb{K}$-ベクトル空間 $V$ のことを\textbf{表現空間}
    \item 表現空間 $V$ の部分 $\mathbb{K}$-ベクトル空間 
    \begin{align}
        \bm{V_\lambda} \coloneqq \bigl\{\, v \in V \bigm| \phi(h)(v)\; (= h \btr v) = \lambda v \,\bigr\} 
    \end{align}
    が $0$ でないとき,\textbf{ウエイト $\bm{\lambda}$ のウエイト空間}
    \item ウエイト $\lambda$ のウエイト空間 $V_\lambda$ の非零な元 $v \in V_\lambda \setminus \{0\}$ であって,
    \begin{align}
        \phi(x)(v) \; (= x \btr v) = 0 \in V_{\lambda+2}
    \end{align}
    を充たすもののことを\textbf{ウエイト $\bm{\lambda}$ の極大ベクトル}
\end{itemize}
と呼ぶのだった.以下の補題は\underline{任意の}\footnote{既約表現でなくとも良い} $\Lsl(2,\, \mathbb{K})$ の有限次元に対して成り立つ:

\begin{mylem}[label=lem:sl2-1]{}
	$v \in V_\lambda \IMP x \btr v \in V_{\lambda +2},\; y \btr v \in V_{\lambda - 2}$
\end{mylem}

特に,$\Lsl(2,\, \mathbb{K})$ の $m+1$ 次元\underline{既約}表現は構造が完全に分かっている:

\begin{mylem}[label=lem:sl2-2]{}
    \underline{既約} $\Lsl(2,\, \mathbb{K})$ 加群 $V$ において
    ウエイト $\lambda$ の極大ベクトル $v_0 \in V_\lambda$ をとり\footnote{極大ベクトルの存在証明が\probref{ex:2-7-1}である.},
    \begin{align}
        v_i \coloneqq 
        \begin{cases}
            0, & i = -1 \\
            & \\
            \displaystyle \frac{1}{i!} y^i \btr v_0, & i \ge 0 \\
        \end{cases}
    \end{align}
    とおく.このとき以下が成り立つ:
	\begin{enumerate}
		\item $h \btr v_i = (\lambda - 2i) v_i$
		\item $y \btr v_i = (i+1) v_{i+1}$
		\item $x \btr v_i = (\lambda - i+1) v_{i-1}\WHERE i \ge 0$
	\end{enumerate}
\end{mylem}

\begin{mytheo}[label=thm:irr-sl2,breakable]{}
	$V$ を\underline{既約} $\Lsl(2,\, \mathbb{K})$ 加群とする.
	\begin{enumerate}
		\item $m \coloneqq \dim V - 1$ とおくと \underline{$\mathbb{K}$-ベクトル空間として}
		\begin{align}
			V = \bigoplus_{\mu = 0}^m V_{m-2\mu}\quad \WHERE 0 \le \forall \mu \le m,\; \dim V_{m - 2\mu} = 1
		\end{align}
		が成り立つ.
		\item $V$ の極大ベクトルは零でないスカラー倍を除いて一意に決まり,そのウエイトは $m$ である.
		\item $\Lsl(2,\, \mathbb{K})$ の $V$ への左作用は補題\ref{lem:sl2-2}によって完全に決まる.
	\end{enumerate}
\end{mytheo}

定理\ref{thm:irr-sl2}を鑑みて,$\Lsl(2,\, \mathbb{K})$ の $m+1$ 次元既約表現を常に $\bm{V(m)}$ と書くことにする.これはLie代数の準同型と $\mathbb{K}$-ベクトル空間の組
\begin{align}
    \Bigl(\, \bm{\phi_m} \colon \Lsl(2,\, \mathbb{K}) \lto \Lgl \bigl( \bm{V(m)} \bigr),\, \bm{V(m)}\, \Bigr)
\end{align}
の略記である.

\begin{myproblem}[label=ex:2-7-1]{p.34 の Exercise 1}
    % \setcounter{\tcbcounter}{7}
    任意の有限次元 $\Lsl(2,\, \mathbb{K})$-加群 $V$ に対して極大ベクトルが存在することを示せ.
\end{myproblem}

\begin{proof}
    $\phi \colon \Lsl(2,\, \mathbb{K}) \lto \Lgl(V)$ を与えられた $\Lsl(2,\, \mathbb{K})$ の表現とする.
    $\mathfrak{b} \coloneqq \Span_{\mathbb{K}}\{x,\, h\}\subset \Lsl(2,\, \mathbb{K})$ とおくと,\eqref{eq:comm-sl2}よりこれは $\Lsl(2,\, \mathbb{K})$ の部分Lie代数になる.
    さらに $\comm{\comm{\mathfrak{b}}{\mathfrak{b}}}{\comm{\mathfrak{b}}{\mathfrak{b}}} = \comm{\mathbb{K}x}{\mathbb{K}x} = 0$ が成り立つので $\mathfrak{b}$ は可解である.よって部分Lie代数 $\phi(\mathfrak{b}) \subset \Lgl(V)$ もまた可解である.
    よって定理2.1.1(資料 p.25)を使うことができ,$\phi(\mathfrak{b})$ の任意の元は共通の固有ベクトル $v \in V\setminus \{0\}$ を持つ.
    $x$ は冪零なので $\phi(x)$ もまた冪零であり,$\phi(x)(v) = x \btr v = 0$ が言える.
    i.e. $\phi(h)(v) = h \btr v = \lambda v$ とおくと $v \in V_\lambda \setminus \{0\}$ かつ $0 = x \btr v$ が成り立ち,$v$ はウエイト $\lambda$ の極大ベクトルである.
\end{proof}


\begin{myproblem}[label=ex:2-7-6]{p.34 の Exercise 6}
    $\forall 0 \le n \le m$ に対して,Weylの定理より
    % \footnote{$\Lsl(2,\, \mathbb{K})$ は単純なので半単純である.}
    $\Lsl(2,\, \mathbb{K})$-加群のテンソル積 $V(m) \otimes V(n)$ は既約部分 $\Lsl(2,\, \mathbb{K})$-加群の直和に分解する.この直和分解を求めよ.
\end{myproblem}

テンソル積表現 $\phi\colon \Lsl(2,\, \mathbb{K}) \lto \Lgl\bigl(V(m) \otimes V(n)\bigr)$ のウエイト $\lambda$ のウエイト空間を $\bigl(V(m) \otimes V(n)\bigr)_\lambda$ と書く代わりに $V_\lambda$ と略記する.
線型変換 $\phi(h) \in \Lgl\bigl(V(m) \otimes V(n)\bigr)$ の固有空間分解を考えることで,ある $\Phi \subset \mathbb{K}$ が存在して
\begin{align}
    V(m) \otimes V(n) = \bigoplus_{\lambda \in \Phi} V_\lambda
\end{align}
が成り立つことがわかる.
一方,Weylの定理と定理\ref{thm:irr-sl2}-(1)より添字集合 $I \subset \bm{\mathbb{Z}}$ および $V(m) \otimes V(n)$ の\underline{既約}部分 $\Lsl(2,\, \mathbb{K})$-加群の族 $\Familyset[\big]{V(k)}{k \in I}$ が存在して
\begin{align}
    V(m) \otimes V(n) = \bigoplus_{k \in I} V(k) = \bigoplus_{k \in I} \bigoplus_{\mu = 0}^{k} V(k)_{k - 2\mu}
\end{align}
と書けるので,
\begin{align}
    \label{eq:sl2-Weyl}
    V(m) \otimes V(n) &= \bigoplus_{\lambda \in \Phi} V_\lambda
    % =  \bigoplus_{k \in I} V(k)
    =  \bigoplus_{k \in I} \bigoplus_{\mu = 0}^{k} V(k)_{k - 2\mu}
\end{align}
が成り立つ\footnote{従って $V(k)_{k-2\mu} = V_{k-2\mu} \cap V(k)$ と書くこともできる.}.このことから即座に $\Phi \subset \mathbb{Z}$ がわかる.
求めるべきなのは集合 $I \subset \mathbb{Z}$ であるが,まず手始めに $\Phi$ を求め,次に $\Phi$ と $I$ の関係を調べることにする.


% 系2.4.2より,ある $\Phi \subset \mathbb{Z}$ が存在して
% \begin{align}
%     \label{eq:tensor-decomp}
%     V(m) \otimes V(n) = \bigoplus_{\lambda \in \Phi} V_\lambda
% \end{align}
% と書ける.この $\Phi$ を求めよう.

$V(m),\, V(n)$ の定理\ref{thm:irr-sl2}に基づく基底をそれぞれ $\{e_\mu\}_{0 \le \mu \le m},\; \{f_\nu\}_{0 \le \nu \le n}$ とおく.
このとき補題\ref{lem:sl2-2}から
\begin{align}
    h \btr_{V(m)} e_\mu &= (m - 2\mu)e_\mu, \\
    h \btr_{V(n)} f_\nu &= (n - 2\nu)f_\nu, \\
    x \btr_{V(n)} e_{\mu} &= (m - \mu + 1)e_{\mu-1}, \\
    x \btr_{V(n)} f_\nu &= (n - \nu + 1)f_{\nu-1}, \\
\end{align}
が成り立つ.
よって $\Lsl(2,\, \mathbb{K})$-加群のテンソル積の定義から
\begin{align}
    h \btr (e_\mu \otimes f_\nu) &= ( h \btr_{V(m)} e_\mu) \otimes f_\nu + e_\mu \otimes ( h \btr_{V(n)} f_\nu) \\
    &= \bigl(m+n - 2(\mu + \nu)\bigr) e_\mu \otimes f_\nu
\end{align}
が成り立つ.i.e. $e_\mu \otimes f_\nu \in V_{m+n-2(\mu+\nu)}$ である. 
$\Lsl(2,\, \mathbb{K})$-加群 $V(m) \otimes V(n)$ の\underline{$\mathbb{K}$-ベクトル空間としての}基底は $\{e_\mu \otimes f_\nu\}_{0 \le \mu \le m,\, 0 \le \nu \le n}$ なので,
\begin{align}
    V_{m+n-2k} 
    &= \Span_{\mathbb{K}} \bigl\{\, e_\mu \otimes f_{\nu} \bigm| 0 \le \mu \le m,\, 0 \le \nu \le n,\, \mu + \nu = k \,\bigr\}
\end{align}
であり,かつ式\eqref{eq:sl2-Weyl}において $\Phi = \bigl\{\, m+n-2k \in \mathbb{Z} \bigm| 0 \le k \le m+n\,\bigr\}$ であること,
および
\begin{align}
    \dim V_{m+n-k} 
    =
    \begin{cases}
        k+1, &0 \le k < n \\
        n+1, &n \le k \le m \\
        m+n-k+1, &m < k \le m+n
    \end{cases}
\end{align}
がわかった\footnote{次元を求めると
\begin{align}
    \dim V(m) \otimes V(n) &= (m+1)(n+1),\\
    \dim \bigoplus_{\lambda \in \Phi} V_\lambda &= \sum_{k=0}^{n-1} (k+1) + \sum_{k=n}^{m} (n+1) + \sum_{k=m+1}^{m+n} (m+n-k+1) = (m+1)(n+1)
\end{align}
となり,整合的である.
}.

 次に,式\eqref{eq:sl2-Weyl}の添字集合 $I \subset \mathbb{Z}$ を求める.
$0 \le k \le n$ に対して $v = \sum_{\mu=0}^k \lambda_\mu e_\mu \otimes f_{k - \mu} \in V_{m+n-2k}$ がウエイト $m+n-2k$ の極大ベクトルであるとする\footnote{当たり前だが,極大ベクトルのウエイトとしてあり得るのは $\Phi$ の元だけである.}.このとき補題\ref{lem:sl2-2}-(3)より
\begin{align}
    0 &= x \btr v \\
    &= \sum_{\mu=0}^k \lambda_\mu \bigl( x \btr_{V(m)} e_\mu \otimes f_{k-\mu} + e_\mu \otimes (x \btr_{V(n)} f_{k-\mu}) \bigr) \\
    &= \sum_{\mu=1}^k \lambda_\mu (m-\mu+1) e_{\mu-1} \otimes f_{k-\mu} + \sum_{\mu=0}^{k-1} \lambda_\mu (n-k+\mu+1) e_\mu \otimes f_{k-\mu-1} \\
    &= \sum_{\mu=1}^k \bigl( \lambda_\mu(m-\mu+1) + \lambda_{\mu-1}(n-k+\mu) \bigr) e_{\mu-1} \otimes f_{k-\mu}
\end{align}
が成り立つので
\begin{align}
    1 \le \forall \mu \le k,\; \lambda_\mu(m-\mu+1) + \lambda_{\mu-1}(n-k+\mu) = 0
\end{align}
がわかった.
今 $k \le n$ なので $m-\mu+1 \ge m-n+1 > 0,\; n-k+\mu \ge \mu > 0$ であり,
\begin{align}
    \lambda_\mu = (-1)^\mu \frac{(n-k+\mu)! (m-\mu)!}{(n-k)! m!} \lambda_0
\end{align}
が言える.よって $\lambda_0 \neq 0$ ならば $v \neq 0$ であり,ウエイト $m+n-2k$ の極大ベクトルがスカラー倍を除いて一意的に存在することが分かった.
定理\ref{thm:irr-sl2}-(2) から,ウエイト $m+n-2k$ の極大ベクトル $v$ は既約な $\Lsl(2,\, \mathbb{K})$-加群 $V(m+n-2k)$ を作る.
よって
\begin{align}
    \bigoplus_{k=0}^n V(m+n-2k) \subset V(m) \otimes V(n)
\end{align}
が言えた.左辺の次元を計算すると
\begin{align}
    \sum_{k=0}^n (m+n-2k + 1) = (m+n+1)(n+1) - n(n+1) = (m+1)(n+1)
\end{align}
であり右辺の次元と一致するので
\begin{align}
    \bigoplus_{k=0}^n V(m+n-2k) = V(m) \otimes V(n)
\end{align}
が得られた.i.e. $I = \bigl\{\, m + n -2k \in \mathbb{Z} \bigm| 0 \le k \le n \,\bigr\} \subset \Phi$ と求まった.

\section{ルート空間分解}

一回目の演習回で扱った通り,
$A_l$ 型Lie代数の基底は
\begin{align}
    &\bigl\{\, e_{ij} \bigm| 1 \le i \neq j \le l+1 \,\bigr\} \\
    \cup\, &\bigl\{\, e_{ii} - e_{i+1,\, i+1} \bigm| 1 \le i \le l \,\bigr\} 
\end{align}
$B_l$ 型Lie代数の基底は
\begin{align}
    &\bigl\{\, e_{1,\, 1 + \textcolor{red}{i}} - e_{1 + l + \textcolor{red}{i},\, 1} \bigm| 1 \le \textcolor{red}{i} \le l \,\bigr\} \\
    \cup\, &\bigl\{\, e_{1,\, 1 + l + \textcolor{red}{i}} - e_{1 + \textcolor{red}{i},\, 1} \bigm| 1 \le \textcolor{red}{i} \le l \,\bigr\} \\
    \cup\, &\bigl\{\, e_{1 + l + \textcolor{red}{i},\, 1 + \textcolor{blue}{j}} - e_{1 + l + \textcolor{blue}{j},\, 1 + \textcolor{red}{i}} \bigm| 1 \le \textcolor{red}{i} < \textcolor{blue}{j} \le l \,\bigr\} \\
    \cup\, &\bigl\{\, e_{1 + \textcolor{blue}{j},\, 1 + \textcolor{red}{i}} - e_{1 + l + \textcolor{red}{i},\, 1 + l + \textcolor{blue}{j}} \bigm| 1 \le \textcolor{red}{i},\, \textcolor{blue}{j} \le l \,\bigr\}  \\
    \cup\, &\bigl\{\, e_{1 + \textcolor{red}{i},\, 1 + l + \textcolor{blue}{j}} - e_{1 + \textcolor{blue}{j},\, 1 + l + \textcolor{red}{i}} \bigm| 1 \le \textcolor{red}{i} < \textcolor{blue}{j} \le l \,\bigr\} 
\end{align}
$C_l$ 型Lie代数の基底は
\begin{align}
    &\bigl\{\, e_{l + \textcolor{red}{i},\, \textcolor{red}{i}} \bigm| 1 \le \textcolor{red}{i} \le l \,\bigr\} \cup \bigl\{\, e_{l + \textcolor{red}{i},\, \textcolor{blue}{j}} + e_{l + \textcolor{blue}{j},\, \textcolor{red}{i}} \bigm| 1 \le \textcolor{red}{i} < \textcolor{blue}{j} \le l \,\bigr\} \\
    \cup\, &\bigl\{\, e_{\textcolor{blue}{j},\, \textcolor{red}{i}} - e_{l + \textcolor{red}{i},\, l + \textcolor{blue}{j}} \bigm| 1 \le \textcolor{red}{i},\, \textcolor{blue}{j} \le l \,\bigr\}  \\
    \cup\, &\bigl\{\, e_{\textcolor{red}{i},\, l + \textcolor{red}{i}} \bigm| 1 \le \textcolor{red}{i} \le l \,\bigr\} \cup \bigl\{\, e_{\textcolor{red}{i},\, l + \textcolor{blue}{j}} + e_{1 + \textcolor{blue}{j},\, l + \textcolor{red}{i}} \bigm| 1 \le \textcolor{red}{i} < \textcolor{blue}{j} \le l \,\bigr\} 
\end{align}
$D_l$ 型Lie代数の基底は
\begin{align}
    &\bigl\{\, e_{l + \textcolor{red}{i},\, \textcolor{blue}{j}} - e_{l + \textcolor{blue}{j},\, \textcolor{red}{i}} \bigm| 1 \le \textcolor{red}{i} < \textcolor{blue}{j} \le l \,\bigr\} \\
    \cup\, &\bigl\{\, e_{\textcolor{blue}{j},\, \textcolor{red}{i}} - e_{l + \textcolor{red}{i},\, l + \textcolor{blue}{j}} \bigm| 1 \le \textcolor{red}{i},\, \textcolor{blue}{j} \le l \,\bigr\}  \\
    \cup\, &\bigl\{\, e_{\textcolor{red}{i},\, l + \textcolor{blue}{j}} - e_{\textcolor{blue}{j},\, l + \textcolor{red}{i}} \bigm| 1 \le \textcolor{red}{i} < \textcolor{blue}{j} \le l \,\bigr\} 
\end{align}
にとることができる.

\begin{myproblem}[label=ex:2-8-1]{p.40 の Exercise 1}
    % \setcounter{\tcbcounter}{7}
    $\mathfrak{g}$ を $A_l,\, B_l,\, C_l,\, D_l$ 型Lie代数のどれかとする.
    このとき,$\mathfrak{g}$ の対角行列全体が成す部分Lie代数 $\mathfrak{h}$ は次元 $l$ の極大トーラスであることを示せ.
\end{myproblem}

\begin{proof}
    $\mathfrak{h}$ はLieブラケットについて可換である.

    $\mathfrak{g}$ のトーラス $\mathfrak{k} \subsetneq \mathfrak{g}$ が $\mathfrak{h} \subset \mathfrak{k}$ を充たすとする.
    このとき $\forall x \in \mathfrak{k}$ は同時対角化可能なので結局 $x \in \mathfrak{h}$,i.e. $\mathfrak{h} = \mathfrak{k}$ である.
    上記の基底の対角成分の個数を数えることで,全ての場合に次元が $l$ だとわかる.
\end{proof}

\begin{myproblem}[label=ex:2-8-2A]{p.40 の Exercise 2}
    % \setcounter{\tcbcounter}{7}
    $\mathfrak{g}$ を $A_l$ 型Lie代数とする.
    極大トーラス
    \begin{align}
        \mathfrak{h} \coloneqq \Biggl\{\, \mqty[\dmat{\lambda_1,\ddots,\lambda_{l+1}}] \in \Mat{l+1}{\mathbb{K}}\Biggm| \sum_{\mu=1}^{l+1} \lambda_\mu = 0 \,\Biggr\} 
    \end{align}
    に付随するルート系を求めよ.
\end{myproblem}

$1 \le \forall i \le l+1$ に対して定義できる写像
\begin{align}
    \varepsilon^i \colon \mathfrak{h} \lto \mathbb{K},\; \mqty[\dmat{\lambda_1,\ddots,\lambda_{l+1}}] \lmto \lambda_i
\end{align}
は $\mathbb{K}$-線型写像なので,$\varepsilon^i \in \mathfrak{h}^*$ である.
$\forall h = \mqty[\dmat{h_1,\ddots,h_{l+1}}] \in \mathfrak{h}$ に対して
(Einsteinの規約を使わない)
\begin{align}
    \ad(h)(e_{ij}) &= he_{ij} - e_{ij}h = (h_i - h_j) e_{ij} = \bigl(\varepsilon^i(h) - \varepsilon^j(h)\bigr)\, e_{ij}
\end{align}
が成り立つので,
\begin{align}
    A_l \coloneqq \bigl\{\, \varepsilon^i - \varepsilon^j \in \mathfrak{h}^* \setminus \{0\} \bigm| 1 \le i\neq j \le l+1\,\bigr\} 
\end{align}
がルート全体の集合となる.実際,上述の $A_l$ 型の基底の取り方から
\begin{align}
    \label{eq:Al}
    \mathfrak{g} = \Span_{\mathbb{K}} \bigl\{\, e_{ii} - e_{i+1,i+1} \bigm| 1\le i \le l \,\bigr\} \oplus \bigoplus_{1 \le i \neq j \le l+1} \mathbb{K}e_{ij}
\end{align}
が言えるが,明らかに $\Span_{\mathbb{K}} \bigl\{\, e_{ii} - e_{i+1,i+1} \bigm| 1\le i \le l \,\bigr\} = \mathfrak{h}$ なので
\begin{align}
    \mathfrak{g}_{\varepsilon^i - \varepsilon^j} \coloneqq \bigl\{\, x \in \mathfrak{g} \bigm| \forall h \in \mathfrak{h},\; \ad(h)(x) = (\varepsilon^{i} - \varepsilon^{j})(h)\, x \,\bigr\} = \mathbb{K}e_{ij}
\end{align}
と併せると\eqref{eq:Al}がルート空間分解
\begin{align}
    \mathfrak{g} = \mathfrak{h} \oplus \bigoplus_{\alpha \in A_l} \mathfrak{g}_{\alpha}
\end{align}
になっていることがわかった.

\begin{myproblem}[label=ex:2-8-2D]{p.40 の Exercise 2}
    % \setcounter{\tcbcounter}{7}
    $\mathfrak{g}$ を $D_l$ 型Lie代数とする.
    極大トーラス
    \begin{align}
        \mathfrak{h} \coloneqq \Biggl\{\, \mqty[\dmat{\lambda_1,\ddots,\lambda_{l},-\lambda_{1},\ddots,-\lambda_{l}}] \in \Mat{2l}{\mathbb{K}} \,\Biggr\} 
    \end{align}
    に付随するルート系を求めよ.
\end{myproblem}


$1 \le \forall i \le l$ に対して定義できる写像
\begin{align}
    \varepsilon^i \colon \mathfrak{h} \lto \mathbb{K},\; \mqty[\dmat{\lambda_1,\ddots,\lambda_{l},-\lambda_{1},\ddots,-\lambda_{l}}] \lmto \lambda_i
\end{align}
は $\mathbb{K}$-線型写像なので,$\varepsilon^i \in \mathfrak{h}^*$ である.

$\forall h = \mqty[\dmat{h_1,\ddots,h_{l},-h_{1},\ddots,-h_{l}}] \in \mathfrak{h}$ に対して,
$1 \le i \neq j \le l$ ならば
\begin{align}
    \ad(h)(e_{l+i,j}) &= -(h_i + h_j) e_{l+i,j} = -\bigl(\varepsilon^i(h) + \varepsilon^j(h)\bigr)\, e_{l+i,j}, \\
    \ad(h)(e_{i,l+j}) &= (h_i + h_j) e_{i,l+j} = \bigl(\varepsilon^i(h) + \varepsilon^j(h)\bigr)\, e_{i,l+j}, \\
    \ad(h)(e_{i,j}) &= - \ad(h)(e_{l+i,l+j}) = (h_i - h_j) e_{i,j} = \bigl(\varepsilon^i(h) - \varepsilon^j(h)\bigr)\,e_{i,j}
\end{align}
が成り立つので,
\begin{align}
    \ad(h)(e_{l+i,j} - e_{l+j,i}) &= -(\varepsilon^i + \varepsilon^j)(h)\, (e_{l+i,j} - e_{l+j,i}), \\
    \ad(h)(e_{i,l+j} - e_{j,l+i}) &= (\varepsilon^i + \varepsilon^j)(h)\, (e_{i,l+j} - e_{j,l+i}), \\
    \ad(h)(e_{i,j} - e_{l+j,l+i}) &= (\varepsilon^i - \varepsilon^j)(h)\, (e_{i,j} - e_{l+j,l+i})
\end{align}
がわかった.故に
% $\alpha^{\pm}_{ij} \in \mathfrak{h}^*$ を
% \begin{align}
%     \alpha^+_{ij}(h) &\coloneqq h_i + h_j \\
%     \alpha^-_{ij}(h) &\coloneqq h_i - h_j
% \end{align}
% と定義すれば
\begin{align}
    D_l
    &\coloneqq 
    \bigl\{\, \varepsilon^i + \varepsilon^j \in \mathfrak{h}^* \setminus \{0\} \bigm| 1 \le i \neq j \le l \,\bigr\} \\
    &\quad \cup \bigl\{\, \varepsilon^i - \varepsilon^j \in \mathfrak{h}^* \setminus \{0\} \bigm| 1 \le i \neq j \le l \,\bigr\} \\
    &= \bigl\{\, \pm \varepsilon^i \pm \varepsilon^j \in \mathfrak{h}^* \setminus \{0\} \bigm| 1 \le i < j \le l \,\bigr\}
\end{align}
がルート系となる.ただし最右辺は複号任意である.
実際 $i < j$ ならば
\begin{align}
    \mathfrak{g}_{-\varepsilon^i-\varepsilon^j} &= \mathbb{K}(e_{l+i,j} - e_{l+j,i}), \\
    \mathfrak{g}_{\varepsilon^i+\varepsilon^j} &= \mathbb{K}(e_{i,l+j} - e_{j,l+i}) \\
    \mathfrak{g}_{\varepsilon^i-\varepsilon^j} &= \mathbb{K}(e_{i,j} - e_{l+j,l+i}) \\
    \mathfrak{g}_{-\varepsilon^i+\varepsilon^j} &= \mathbb{K}(e_{j,i} - e_{l+i,l+j})
\end{align}
なので,$D_l$ 型の基底の取り方からルート空間分解
\begin{align}
    \mathfrak{g} = \mathfrak{h} \oplus \bigoplus_{\alpha \in D_l} \mathfrak{g}_{\alpha}
\end{align}
が再現される.

\begin{myproblem}[label=ex:2-8-2C]{p.40 の Exercise 2}
    % \setcounter{\tcbcounter}{7}
    $\mathfrak{g}$ を $C_l$ 型Lie代数とする.
    極大トーラス
    \begin{align}
        \mathfrak{h} \coloneqq \Biggl\{\, \mqty[\dmat{\lambda_1,\ddots,\lambda_{l},-\lambda_{1},\ddots,-\lambda_{l}}] \in \Mat{2l}{\mathbb{K}} \,\Biggr\} 
    \end{align}
    に付随するルート系を求めよ.
\end{myproblem}


$1 \le \forall i \le l$ に対して定義できる写像
\begin{align}
    \varepsilon^i \colon \mathfrak{h} \lto \mathbb{K},\; \mqty[\dmat{\lambda_1,\ddots,\lambda_{l},-\lambda_{1},\ddots,-\lambda_{l}}] \lmto \lambda_i
\end{align}
は $\mathbb{K}$-線型写像なので,$\varepsilon^i \in \mathfrak{h}^*$ である.
$\forall h = \mqty[\dmat{h_1,\ddots,h_{l},-h_{1},\ddots,-h_{l}}] \in \mathfrak{h}$ に対して,
$1 \le i \neq j \le l$ ならば
\begin{align}
    \ad(h)(e_{l+i,i}) &= -2h_i e_{l+i,i} = -2\varepsilon^i(h)\, e_{l+i,i}, \\
    \ad(h)(e_{l+i,j}) &= -(h_i + h_j) e_{l+i,j} = - \bigl( \varepsilon^i(h) + \varepsilon^j(h) \bigr)\, e_{l+i,j} , \\
    \ad(h)(e_{i,l+i}) &= 2h_i e_{i,l+i} = 2\varepsilon^i(h)\, e_{i,l+i}, \\
    \ad(h)(e_{i,l+j}) &= (h_i + h_j) e_{i,l+j} = \bigl( \varepsilon^i(h) + \varepsilon^j(h) \bigr)\, e_{i,l+j}, \\
    \ad(h)(e_{i,j}) &= - \ad(h)(e_{l+i,l+j}) = (h_i - h_j) e_{i,j} = \bigl( \varepsilon^i(h) - \varepsilon^j(h) \bigr)\, e_{i,j}
\end{align}
が成り立つので,
\begin{align}
    \ad(h)(e_{l+i,i}) &= -2\varepsilon^i(h)\, e_{l+i,i}, \\
    \ad(h)(e_{l+i,j} + e_{l+j,i}) &= -(\varepsilon^i + \varepsilon^j)(h)\, (e_{l+i,j} + e_{l+j,i}), \\
    \ad(h)(e_{i,l+j} + e_{j,l+i}) &= (\varepsilon^i + \varepsilon^j)(h)\, (e_{i,l+j} + e_{j,l+i}), \\
    \ad(h)(e_{i,j} - e_{l+j,l+i}) &= (\varepsilon^i - \varepsilon^j)(h)\, (e_{i,j} - e_{l+j,l+i})
\end{align}
がわかった.故に
% $\alpha_{i} \in \mathfrak{h}^*$ を
% \begin{align}
%     \alpha_{i}(h) &\coloneqq 2h_i \\
% \end{align}
% と定義すれば
\begin{align}
    C_l
    &\coloneqq D_l \cup \bigl\{\, \pm 2\varepsilon^i \in \mathfrak{h}^* \setminus \{0\}\bigm| 1 \le i \le l \,\bigr\}
\end{align}
がルート系となる.

\begin{myproblem}[label=ex:2-8-2B]{p.40 の Exercise 2}
    % \setcounter{\tcbcounter}{7}
    $\mathfrak{g}$ を $B_l$ 型Lie代数とする.
    極大トーラス
    \begin{align}
        \mathfrak{h} \coloneqq \Biggl\{\, \mqty[\dmat{0,\lambda_1,\ddots,\lambda_{l},-\lambda_{1},\ddots,-\lambda_{l}}] \in \Mat{2l+1}{\mathbb{K}} \,\Biggr\} 
    \end{align}
    に付随するルート系を求めよ.
\end{myproblem}

$1 \le \forall i \le l$ に対して定義できる写像
\begin{align}
    \varepsilon^i \colon \mathfrak{h} \lto \mathbb{K},\; \mqty[\dmat{0,\lambda_1,\ddots,\lambda_{l},-\lambda_{1},\ddots,-\lambda_{l}}] \lmto \lambda_i
\end{align}
は $\mathbb{K}$-線型写像なので,$\varepsilon^i \in \mathfrak{h}^*$ である.
$\forall h = \mqty[\dmat{0,h_1,\ddots,h_{l},-h_{1},\ddots,-h_{l}}] \in \mathfrak{h}$ に対して,
$1 \le i \neq j \le l$ ならば
\begin{align}
    \ad(h)(e_{1+i,1}) &= h_i e_{1,1+i} = \varepsilon^i(h)\, e_{1,1+i}, \\
    \ad(h)(e_{1,1+i}) &= -h_i e_{1,1+i} = -\varepsilon^i(h)\, e_{1,1+i}, \\
    \ad(h)(e_{1+l+i,1}) &= -h_i e_{1+l+i,1} = -\varepsilon^i(h)\, e_{1+l+i,1}, \\
    \ad(h)(e_{1,1+l+i}) &= h_i e_{1,1+l+i} = \varepsilon^i(h)\, e_{1,1+l+i}, \\
    \ad(h)(e_{1+l+i,1+j}) &= -(h_i + h_j) e_{1+l+i,1+j} = -\bigl( \varepsilon^i(h) + \varepsilon^j(h) \bigr)\, e_{1+l+i,1+j} , \\
    \ad(h)(e_{1+i,1+l+j}) &= (h_i + h_j) e_{1+i,1+l+j} = \bigl( \varepsilon^i(h) + \varepsilon^j(h) \bigr)\,  e_{1+i,1+l+j}, \\
    \ad(h)(e_{1+i,1+j}) &= - \ad(h)(e_{1+l+i,1+l+j}) = (h_i - h_j) e_{i,j} = \bigl( \varepsilon^i(h) - \varepsilon^j(h) \bigr)\, e_{i,j}
\end{align}
が成り立つので,
\begin{align}
    \ad(h)(e_{1,1+i} - e_{1+l+i,1}) &= -\varepsilon^i(h)\, (e_{1,1+i} - e_{1+l+i,1}), \\
    \ad(h)(e_{1+i,1} - e_{1,1+l+i}) &= \varepsilon^i(h)\,(e_{1+i,1} - e_{1,1+l+i}), \\
    \ad(h)(e_{1+l+i,1+j} + e_{1+l+j,1+i}) &= -(\varepsilon^i + \varepsilon^j)(h)\, (e_{1+l+i,1+j} + e_{1+l+j,1+i}), \\
    \ad(h)(e_{1+i,1+l+j} + e_{1+j,1+l+i}) &= (\varepsilon^i + \varepsilon^j)(h)\, (e_{1+i,1+l+j} + e_{1+j,1+l+i}), \\
    \ad(h)(e_{1+i,1+j} - e_{1+l+j,1+l+i}) &= (\varepsilon^i - \varepsilon^j)(h)\, (e_{1+i,1+j} - e_{1+l+j,1+l+i})
\end{align}
がわかった.故に
% $\beta_{i} \in \mathfrak{h}^*$ を
% \begin{align}
%     \beta_{i}(h) &\coloneqq h_i \\
% \end{align}
% と定義すれば
\begin{align}
    B_l
    &\coloneqq D_l \cup \bigl\{\, \pm \varepsilon^i \in \mathfrak{h}^* \setminus \{0\}\bigm| 1 \le i \le l \,\bigr\}
\end{align}
がルート系となる.

\begin{myproblem}[label=ex:2-8-10B]{p.40 の Exercise 10}
    % \setcounter{\tcbcounter}{7}
    $4,\, 5,\, 7$ 次元の半単純Lie代数が存在しないことを示せ
\end{myproblem}

\begin{proof}
    半単純Lie代数 $\mathfrak{g}$ とその極大トーラス $\mathfrak{h} \subset \mathfrak{g}$ をとる.
    $\mathfrak{h} \neq 0$ なので $\dim \mathfrak{h} > 0$ である.
    このときルート空間分解
    \begin{align}
        \mathfrak{g} = \mathfrak{h} \oplus \bigoplus_{\alpha \in \Phi} \mathfrak{g}_\alpha
    \end{align}
    が成り立つ.

    ところで,$\alpha \in \Phi \IMP -\alpha \in \Phi$ かつ $\forall \alpha \in \Phi$ に対して $\dim \mathfrak{g}_\alpha = 1$ なのである整数 $k$ が存在して
    \begin{align}
        0 < \dim \mathfrak{h} = \dim \mathfrak{g} - \sum_{\alpha \in \Phi} \dim \mathfrak{g}_\alpha = \dim \mathfrak{g} - 2k
    \end{align}
    と書ける.一方で $\mathfrak{h}^* = \Span_{\mathbb{K}} \Phi = \Span_{\mathbb{K}} \{\pm \alpha_1,\, \dots,\, \pm\alpha_k\}$ であるから $\dim \mathfrak{h} = \dim \mathfrak{h}^* \le k$ である.よって
    \begin{align}
        \frac{1}{3} \dim \mathfrak{g} \le k < \frac{1}{2} \dim \mathfrak{g}
    \end{align}
    が成り立たねばならない.
    \begin{itemize}
        \item $\dim \mathfrak{g} = 4$ だとすると $k \notin \mathbb{Z}$ となり矛盾.
        \item $\dim \mathfrak{g} = 5$ だとすると $k=2$ に確定する.このとき $\pm \alpha_1 \neq \pm \alpha_2$(複合任意)を充たす $\alpha_1,\, \alpha_2 \in \Phi$ が取れるが,$\dim \mathfrak{h} = \dim \mathfrak{h}^* = 1$ となるのである $\lambda \in \mathbb{K}$ が存在して $\alpha_2 = \lambda \alpha_1$ と書ける.
        すると $\lambda \alpha_1 \in \Phi$ なので $\lambda = \pm 1$ のどちらかでなくてはいけず,$\pm \alpha_1 \neq \pm\alpha_2$ に矛盾する.
        \item $\dim \mathfrak{g} = 7$ だとすると $k=3$ に確定する.すると $\dim \mathfrak{h} = \dim \mathfrak{h}^* = 1$ となり $\dim \mathfrak{g} = 5$ のときと同様の議論から矛盾する.
    \end{itemize}
    よって背理法から $\dim \mathfrak{g} \neq 4,\, 5,\, 7$ が示された.
\end{proof}


%参考文献のスタイル(style.bstを参照)
\bibliographystyle{myh-physrev}
%参考文献(reference.bibを参照)
\bibliography{repref}

\end{document}