\documentclass[rep_main]{subfiles}

\begin{document}

% \setcounter{}{}
\chapter{Lie群とLie代数}

この章は\cite[Chapter I]{Humphreys1972introduction}, \cite[第1-3章]{Satake1987LieAlg}に相当する.

本章に限ってはベクトルを $x \in L$ のように英小文字で表記し,係数体の元は $\lambda \in \mathbb{K}$ のようにギリシャ文字で表記する.零ベクトルは $o \in L$ と書き\footnote{$0$ の濫用を回避するための苦肉の策です... 普通に不便なので次章以降では零ベクトルも $0$ と書きます.},$0 \in \mathbb{K}$ を係数体の加法単位元,$1 \in \mathbb{K}$ を係数体の乗法単位元とする.
ベクトル空間の加法を $+$ と書き,スカラー乗法は $\lambda x$ のように係数を左に書く.
また,特に断りがなければEinsteinの規約を使う.

\section{公理的Lie代数}

この節では,特に断りがなければ $\mathbb{K}$ を\underline{任意の体}とする.Lie代数を純粋に代数学的な対象として扱うことを考える.
% 幾何学的側面については後述する.

\begin{marker}
    Lie代数を表す記号はFrakturという字体で書く慣例がある.例えば「Fraktur」という文字列は $\mathfrak{Fraktur}$ のようになる.
    \LaTeX コマンドは数式モード中で \texttt{\textbackslash mathfrak\{\}} とする.
\end{marker}

\begin{myaxiom}[label=ax:LieAlg, breakable]{Lie代数の公理}
    体 $\mathbb{K}$ 上のベクトル空間 $\mathfrak{g}$ の上に二項演算\footnote{ベクトル空間に備わっている加法とスカラー乗法の他に,追加で $\comm{\;}{\,}$ が定義されているという状況である.この付加的な二項演算はしばしば\textbf{括弧積} (bracket) とか\textbf{交換子} (commutator) とか\textbf{Lieブラケット} (Lie bracket) とか呼ばれる.}
    \begin{align}
        \comm{\;}{\,} \colon \mathfrak{g} \times \mathfrak{g} \lto \mathfrak{g},\; (x,\, y) \lmto \comm{x}{y}
    \end{align}
    が定義されていて,かつ以下の条件を充たすとき,$\mathfrak{g}$ は\textbf{Lie代数} (Lie algebra) と呼ばれる:
    \begin{description}
        \item[\textbf{(L-1)}] $\comm{\;}{\,}$ は双線型写像である.i.e. $\forall x,\, x_i,\, y,\, y_i \in \mathfrak{g},\; \forall \lambda_i,\, \mu_i \in \mathbb{K}\; (i = 1,\, 2)$ に対して
        \begin{align}
            \comm{\lambda_1 x_1 + \lambda_2 x_2}{y} &= \lambda_1 \comm{x_1}{y} + \lambda_2 \comm{x_2}{y}, \\
            \comm{x}{\mu_1 y_1 + \mu_2 y_2} &= \mu_1 \comm{x}{y_1} + \mu_2 \comm{x}{y_2}
        \end{align}
        が成り立つ.
        \item[\textbf{(L-2)}] $\forall x \in \mathfrak{g}$ に対して
        \begin{align}
            \comm{x}{x} = o
        \end{align}
        が成り立つ.
        \item[\textbf{(L-3)}] $\forall x,\, y,\, z \in \mathfrak{g}$ に対して
        \begin{align}
            \label{eq:Jacobi-identity}
            \comm{x}{\comm{y}{z}} + \comm{y}{\comm{z}{x}} + \comm{z}{\comm{x}{y}} = o
        \end{align}
        が成り立つ\footnote{\underline{結合律ではない!}}(\textbf{Jacobi}恒等式).
    \end{description}
    
\end{myaxiom}

\hyperref[ax:LieAlg]{公理\textsf{\textbf{(L-1)}}, \textsf{\textbf{(L-2)}}}から
\begin{align}
    o = \comm{x+y}{x+y} = \comm{x}{x} + \comm{x}{y} + \comm{y}{x} + \comm{y}{y} = \comm{x}{y} + \comm{y}{x}
\end{align}
が従う.i.e. $\comm{x}{y}$ は反交換 (anticommute) する:
\begin{description}
    \item[\textbf{(L'-2)}] $\forall x,\, y \in \mathfrak{g}$ に対して
    \begin{align}
        \comm{x}{y} = -\comm{y}{x}
    \end{align}
    が成り立つ.
\end{description}
逆に\textsf{\textbf{(L'-2)}}を仮定すると
\begin{align}
    o = \comm{x}{x} + \comm{x}{x} = (1 + 1)\comm{x}{x}
\end{align}
が成り立つ\footnote{2つ目の等号ではスカラー乗法の分配律(ベクトル空間の公理である)を使った.}ので,\underline{体 $\mathbb{K}$ において $1 + 1 \neq 0$ ならば} $\comm{x}{x} = o$ が言える.
i.e. $\character \mathbb{K} \neq 2$ ならば\footnote{体 $\mathbb{K}$ の\textbf{標数} (characteristic) を $\character \mathbb{K}$ と書いた.} \textsf{\textbf{(L'-2)}} と \textsf{\textbf{(L-2)}} は同値である.

\begin{myexample}[label=ex:R3-LieAlg]{3次元実ベクトル空間}
    $\mathbb{R}^3$ を $\mathbb{R}$-ベクトル空間と見做す.$\mathbb{R}^3$ 上のLieブラケットを
    \begin{align}
        \comm{\;}{\,} \colon \mathbb{R}^3 \times \mathbb{R}^3 \lto \mathbb{R}^3,\; (x,\, y) \lmto x \times y
    \end{align}
    によって定義する.ただし右辺の $\times$ はベクトル積である.このとき $\mathbb{R}^3$ が\hyperref[ax:LieAlg]{Lie代数の公理}を充たすことを確認しよう:
    \begin{proof}
        $\mathbb{R}^3$ の元の成分を $x = (\, x_\mu \,)_{1 \le \mu \le 3}$ のように書く.ひたすら成分計算をする.
        \begin{description}
            \item[\textbf{(L-1)}] ベクトル積の定義から
            \begin{align}
                (\comm{\lambda_1 x_1 + \lambda_2 x_2}{y} )_\mu 
                &= \epsilon_{\mu\nu\lambda} (\lambda_1 x_1{}_\nu + \lambda_2 x_2{}_\nu) y_\lambda \\
                &= \lambda_1 \epsilon_{\mu\nu\lambda} x_1{}_\mu y_\lambda + \lambda_2 \epsilon_{\mu\nu\lambda} x_2{}_\nu y_\lambda \\
                &= \lambda_1 \comm{x_1}{y} + \lambda_2 \comm{x_2}{y}
            \end{align}
            が成り立つ.全く同様に
            \begin{align}
                \comm{x}{\mu_1 y_1 + \mu_2 y_2} = \mu_1 \comm{x}{y_1} + \mu_2 \comm{x}{y_2}
            \end{align}
            が示される.
            \item[\textbf{(L-2)}] Levi-Civita記号の添字に関する反対称性から
            \begin{align}
                (\comm{x}{x})_\mu = \epsilon_{\mu\nu\lambda} x_\nu x_\lambda = \epsilon_{\mu\nu\lambda} x_\lambda x_\nu = - \epsilon_{\mu\lambda\nu} x_\lambda x_\nu = (- \comm{x}{x})_\mu
            \end{align}
            $\character \mathbb{R} = 0 \neq 2$ なので,ここから $\comm{x}{x} = o$ が従う.
            \item[\textbf{(L-3)}] 行と列をそれぞれ適切に入れ替えた単位行列の行列式を考えることで,恒等式 $\epsilon_{\mu\nu\lambda} \epsilon_{\rho\sigma \lambda} = \delta_{\mu\rho} \delta_{\nu\sigma} - \delta_{\mu\sigma} \delta_{\nu\rho}$ が成り立つことがわかる.従って
            \begin{align}
                (\comm{x}{\comm{y}{z}})_\mu &= \epsilon_{\mu\nu\lambda} x_\nu \epsilon_{\lambda\rho\sigma} y_\rho z_\sigma \\
                &= (\delta_{\mu\rho}\delta_{\nu \sigma} - \delta_{\mu\sigma} \delta_{\nu\rho}) x_\nu y_\rho z_\sigma \\
                &= x_\nu y_\mu z_\nu - x_\nu y_\nu z_\mu
            \end{align}
            であるから,
            \begin{align}
                (\comm{x}{\comm{y}{z}} + \comm{y}{\comm{z}{x}} + \comm{z}{\comm{x}{y}})_\mu 
                &= x_\nu y_\mu z_\nu - x_\nu y_\nu z_\mu + y_\nu z_\mu x_\nu - y_\nu z_\nu x_\mu + z_\nu x_\mu y_\nu - z_\nu x_\nu y_\mu \\
                &= 0.
            \end{align}
        \end{description}
    \end{proof}
    $\mathbb{R}^3$ の標準的な基底
    \begin{align}
        e_1 \coloneqq \mqty[1 \\ 0 \\ 0],\quad e_2 \coloneqq \mqty[0 \\ 1 \\ 0],\quad e_3 \coloneqq \mqty[0 \\ 0 \\ 1]
    \end{align}
    に対して
    \begin{align}
        \comm{e_i}{e_j} = \epsilon_{ijk} e_k
    \end{align}
    が成り立つ.$\mathbb{R}^3$ の任意の2つの元に対するLieブラケットがこの恒等式によって完全に決まることを含意して,$3^3 = 27$ 個の実定数 $\epsilon_{ijk}$ のことを $\mathbb{R}^3$ の\textbf{構造定数} (structure constant) と呼ぶ.
\end{myexample}


\begin{myexample}[label=def:gl-alg]{一般線形代数 $\Lgl (V)$}
     $V$ を体 $\mathbb{K}$ 上のベクトル空間とする.$V$ から $V$ への線型写像全体が成す集合を $\End V$ と書く\footnote{\textbf{自己準同型} (endomorphism) の略である.}.
    $\End V$ の加法とスカラー乗法をそれぞれ
    \begin{align}
        + \colon \End V \times \End V &\lto \End V,\; (x,\, y) \lmto \bigl( v \mapsto x(v) + y(v) \bigr) \\
        \cdot\; \colon \mathbb{K} \times \End V &\lto \End V,\; (\lambda,\, x) \lmto \bigl( v \mapsto \lambda x(v) \bigr)
    \end{align}
    として定義すると,組 $(\End V,\, +,\, \cdot\; )$ は体 $\mathbb{K}$ 上のベクトル空間になる.以降では常に $\End V$ をこの方法でベクトル空間と見做す.

     $\End V$ の上のLieブラケットを
    \begin{align}
        \label{eq:Liebracket-gl}
        \comm{\;}{\,} \colon \End V \times \End V \lto \End V,\; (x,\, y) \lmto xy - yx
    \end{align}
    と定義する.ただし右辺の $xy$ などは写像の合成 $x \circ y$ の略記である.このとき組 $(\End V,\, +,\, \cdot\;,\, \comm{\;}{\,})$ が\hyperref[ax:LieAlg]{Lie代数の公理}を充たすことを確認しよう:
    \begin{proof}
        \begin{description}
            \item[\textbf{(L-1)}] $\forall v \in V$ を1つとる.定義に従ってとても丁寧に計算すると
            \begin{align}
                \comm{\lambda_1 x_1 + \lambda_2 x_2}{y}(v) 
                &= \bigl( (\lambda_1 x_1 + \lambda_2 x_2) y - y (\lambda_1 x_1 + \lambda_2 x_2) \bigr)(v) \\
                &= (\lambda_1 x_1 + \lambda_2 x_2) \bigl( y(v) \bigr) - y \bigl((\lambda_1 x_1 + \lambda_2 x_2)(v)\bigr) \\
                &= (\lambda_1 x_1) \bigl(y (v) \bigr) + (\lambda_2 x_2) \bigl( y(v) \bigr) - y \bigl( (\lambda_1 x_1)(v) + (\lambda_2 x_2)(v)  \bigr) \\
                &= \lambda_1 x_1 \bigl(y (v) \bigr) + \lambda_2 x_2 \bigl( y(v) \bigr) - \lambda_1 y \bigl( x_1(v) \bigr) - \lambda_2 y \bigl( x_2(v) \bigr) \\
                &= \lambda_1 \Bigl( x_1 \bigl(y (v) \bigr) - y \bigl( x_1(v) \bigr) \Bigr) + \lambda_2 \Bigl(x_2 \bigl( y(v) \bigr) - y \bigl( x_2(v) \bigr) \Bigr) \\
                &= \lambda_1 \comm{x_1}{y} (v) + \lambda_2 \comm{x_2}{y} (v) \\
                &= \bigl( \lambda_1 \comm{x_1}{y} + \lambda_2 \comm{x_2}{y}  \bigr) (v)
            \end{align}
            となる.ただし4つ目の等号で $y \in \End V$ が線型写像であることを使った.全く同様にして
            \begin{align}
                \comm{x}{\mu_1 y_1 + \mu_2 y_2}(v) = \mu_1 \comm{x}{y_1} + \mu_2 \comm{x}{y_2}
            \end{align}
            を示すこともできる.
            \item[\textbf{(L-2)}] 明らかに $\comm{x}{x} = xx - xx = o$ なのでよい.
            \item[\textbf{(L-3)}] $\comm{\;}{\,}$ の双線型(\textsf{\textbf{(L-1)}})から
            \begin{align}
                &\comm{x}{\comm{y}{z}} + \comm{y}{\comm{z}{x}} + \comm{z}{\comm{x}{y}} \\
                &= \comm{x}{yz} - \comm{x}{zy} + \comm{y}{zx} - \comm{y}{xz} + \comm{z}{xy} - \comm{z}{yx} \\
                &= xyz - yzx - xzy + zyx + yzx - zxy - yxz + xzy + zxy - xyz - zyx + yxz \\
                &= o.
            \end{align}
        \end{description}
    \end{proof}
    
    このLie代数 $(\End V,\, +,\, \cdot\;,\, \comm{\;}{\,})$ は\textbf{一般線形代数} (general linear algebra) と呼ばれ,記号として $\bm{\Lgl (V)}$ と書かれる.

     $\dim V \eqqcolon n < \infty$ のとき,$\End V$ は $n \times n$ $\mathbb{K}$-行列全体が成す $\mathbb{K}$ ベクトル空間 $\Mat{n}{\mathbb{K}}$ と同型である\footnote{$V$ の基底 $e_1,\, \dots,\, e_n$ を1つ固定する.このとき同型写像 $\varphi \colon V \lto \mathbb{K}^n,\; v^\mu e_\mu \lmto (v^\mu)_{1 \le \mu \le n}$ を使って定義される線型写像 $\phi \colon \End V \lto \Mat{n}{\mathbb{K}},\; f \lmto \varphi \circ f \circ \varphi^{-1}$ が所望の同型写像である.}.
    $\Mat{n}{\mathbb{K}}$ をLieブラケット $\comm{x}{y} \coloneqq xy - yx$ によってLie代数と見做す\footnote{右辺の $xy$ は行列の積である.}ときは,この同型を意識して $\lgl{n}{\mathbb{K}}$ と書く.さて,$\lgl{n}{\mathbb{K}}$ の標準的な基底は所謂\textbf{行列単位}
    \begin{align}
        e_{ij} \coloneqq \bigl[\, \delta_{i\mu} \delta_{j\nu} \bigr]_{1 \le \mu,\, \nu \le n} 
        = 
        \begin{blockarray}{cccccc}
            & & j & & \\
            \begin{block}{(ccccc)c}
                0 & \cdots & 0 & \cdots & 0 & \\
                \vdots & \ddots & \vdots & \ddots & \vdots & \\
                0 & \cdots & 1 & \cdots & 0 & i \\
                \vdots & \ddots & \vdots & \ddots & \vdots & \\
                0 & \cdots & 0 & \cdots & 0 & \\
            \end{block}
        \end{blockarray}
    \end{align}
    である.Einsteinの規約を使って $e_{ij} e_{kl} = \bigl[\, \delta_{i\mu} \delta_{j\lambda} \delta_{k \lambda} \delta_{l \nu} \bigr]_{1 \le \mu,\, \nu \le n} = \delta_{jk} \bigl[\, \delta_{i\mu} \delta_{l\nu} \bigr]_{1 \le \mu,\, \nu \le n} = \delta_{jk} E_{il}$ と計算できるので,$\lgl{n}{\mathbb{K}}$ の構造定数は
    \begin{align}
        \comm{e_{ij}}{e_{kl}} = \delta_{jk} e_{il} - \delta_{li} e_{kj}
    \end{align}
    となる\footnote{少し紛らわしいかもしれないが,右辺において $e_{il}$ は行列(行列の成分\underline{ではない}!)で $\delta_{jk}$ はただの $\mathbb{K}$ の元である.以下では行列を指定する添字を $a,\,b,\, c,\, \dots,\, i,\, j,\, k,\, \dots$ で,行列の成分の添字を $\mu,\, \nu,\, \lambda,\, \dots$ で書くことにする.つまり,例えば $x_{ij}$ はある一つの行列を表す一方,$x_{ij}{}_{\mu\nu}$ は行列 $x_{ij}$ の第 $(\mu,\, \nu)$ 成分を表す.あまりにも紛らわしい場合には,行列の成分を表す際に $[x_{ij}]_{\mu\nu}$ のように角括弧で括ることにする.}.
\end{myexample}

\subsection{線型Lie代数}

\begin{mydef}[label=def:subLieAlg]{部分Lie代数}
    \hyperref[ax:LieAlg]{Lie代数} $\mathfrak{g}$ の\underline{部分ベクトル空間} $\mathfrak{h} \subset \mathfrak{g}$ が\textbf{部分Lie代数}であるとは,$\mathfrak{h}$ がLieブラケットについても閉じていることを言う.
    i.e. $\forall x,\, y \in \mathfrak{h},\; \forall \lambda \in \mathbb{K}$ に対して
    \begin{align}
        x + y,\; \lambda x,\; \comm{x}{y} \in \mathfrak{h}
    \end{align}
    が成り立つこと.
\end{mydef}

この小節では以降,$V$ を体 $\mathbb{K}$ 上の\underline{有限次元}ベクトル空間とする.

\begin{mydef}[label=def:linearLieAlg]{線型Lie代数}
    \hyperref[def:gl-alg]{一般線形代数} $\Lgl (V)$ の\hyperref[def:subLieAlg]{部分Lie代数}のことを\textbf{線型Lie代数} (linear Lie algebra) と呼ぶ.
\end{mydef}

線型Lie代数として有名なものは\textbf{古典代数} (classical algebra) である.これは $A_l,\, B_l,\, C_l,\, D_l$ と呼ばれる4つの無限系列からなる.
以下,$\character \mathbb{K} \neq 2$ とする.

\begin{myexample}[label=def:typeA]{線型Lie代数:$A_l$ 型}
     $\dim V = l+1$ とする.\textbf{特殊線形代数} $\Lsl (V)$(または $\Lsl (l+1,\, \mathbb{K})$)は次のように定義される:
    \begin{align}
        \Lsl (V) \coloneqq \bigl\{\, x \in \Lgl (V) \bigm| \Tr (x) = 0 \,\bigr\} 
    \end{align}
    $\Lsl (V)$ が本当に\hyperref[def:linearLieAlg]{線型Lie代数}かどうか確認しよう.
    \begin{proof}
        $\forall x,\, y \in \Lsl (V)$ をとる.トレースの線形性および $\Tr (xy) = \Tr (yx)$ から
        \begin{align}
            \Tr (x + y) &= \Tr (x) + \Tr (y) = 0 + 0 = 0, \\
            \Tr (\lambda x) &= \lambda \Tr (x) = \lambda 0 = 0, \\
            \Tr (\comm{x}{y}) &= \Tr (xy) - \Tr (yx) = \Tr (xy) - \Tr (xy) = 0
        \end{align}
        が言えるので,$x + y,\, \lambda x,\, \comm{x}{y} \in \Lsl (V)$ が言えた.
    \end{proof}
     \exref{def:gl-alg}で使った $\lgl{l+1}{\mathbb{K}}$ の標準的な基底で $\forall x \in \lsl{l+1}{\mathbb{K}}$ を $x = x^{ij} e_{ij}$ と展開すると,トレースレスであることから
    \begin{align}
        h_{ij} \coloneqq 
        \begin{cases}
            e_{ij} &i \neq j,\; 1 \le i,\, j \le l+1 \\
            e_{ii} - e_{i+1,\, i+1} &1 \le i=j \le l
        \end{cases}
    \end{align}
    の $(l+1)^2 - 1$ 個の行列が $\lsl{l+1}{\mathbb{K}}$ の基底を成すことがわかる.従って $\dim \lsl{l+1}{\mathbb{K}} = (l+1)^2 -1$ である.
\end{myexample}

\begin{myexample}[label=def:typeB]{線型Lie代数:$B_l$ 型}
    $\dim V = 2l + 1$ とする.行列
    \begin{align}
        s \coloneqq \mqty[
            1 & 0 & 0 \\
            0 & 0 & \unity_l \\
            0 & \unity_l & 0
        ] \in \Mat{2l+1}{\mathbb{K}}
    \end{align}
    により定まる $V$ 上の非退化\footnote{「$v \in V$ が $\forall w \in V$ に対して $f(v,\, w) = 0$ を充たす $\IMP v = o$」かつ「$w \in V$ が $\forall v \in V$ に対して $f(v,\, w) = 0$ を充たす $\IMP w = o$」}かつ対称な双線型形式を $f$ と書く\footnote{$f \colon V \times V \lto \mathbb{K},\; (v,\, w) \lmto v^{\mathsf{T}} s w$ のこと.}.
    このとき,\textbf{直交代数} (orthogonal algebra) $\Lo (V)$(または $\lo{2l+1,\, \mathbb{K}}$)が以下のように定義される:
    \begin{align}
        \Lo (V) \coloneqq \bigl\{\, \textcolor{red}{x} \in \Lgl (V) \bigm| f \bigl( \textcolor{red}{x}(v),\, w \bigr) = -f \bigl( v,\, \textcolor{red}{x}(w) \bigr),\; \forall v,\, w \in V \,\bigr\} 
    \end{align}
    $\Lo (V)$ が本当に\hyperref[def:linearLieAlg]{線型Lie代数}かどうか確認しよう.
    \begin{proof}
        $\forall x,\, y \in \Lo (V)$ をとる.$f$ の双線型性から $\forall v,\, w \in V$ に対して
        \begin{align}
            f \bigl( (x+y)(v),\, w \bigr) 
            &= f \bigl( x(v),\, w \bigr) + f \bigl( y(v),\, w \bigr) \\
            &= - f \bigl( v,\, x(w) \bigr) - f \bigl( v,\, y(w) \bigr) \\
            &= - f \bigl( v,\, x(w) + y(w) \bigr) \\
            &= - f \bigl( v,\, (x+y)(w) \bigr) \\
            f \bigl( (\lambda x)(v),\, w \bigr)  
            &= \lambda f \bigl( x(v),\, w \bigr) \\
            &= - \lambda f \bigl( v,\, x(w) \bigr) \\
            &= - f \bigl( v,\, \lambda x(w) \bigr) \\
            &= - f \bigl( v,\, (\lambda x)(w) \bigr) \\
            f \bigl(\comm{x}{y}(v),\, w \bigr) 
            &= f \bigl(x(y(v)),\, w \bigr) - f \bigl(y(x(v)),\, w \bigr) \\
            &= - f \bigl(y(v),\, x(w) \bigr) + f \bigl(x(v),\, y(w) \bigr) \\
            &= f \bigl( v,\, y(x(w)) \bigr) - f \bigl( v,\, x(y(w)) \bigr) \\
            &= - f \bigl( v,\, \comm{x}{y}(w) \bigr) 
        \end{align}
        が言えるので,$x + y,\, \lambda x,\, \comm{x}{y} \in \Lo (V)$ が言えた.
    \end{proof}
\end{myexample}

\begin{myexample}[label=def:typeC]{線型Lie代数:$C_l$ 型}
    $\dim V = 2l$ とする.行列
    \begin{align}
        s \coloneqq \mqty[
            0 & \unity_l \\
            -\unity_l & 0
        ] \in \Mat{2l}{\mathbb{K}}
    \end{align}
    により定まる $V$ 上の非退化かつ歪対称な双線型形式を $f$ と書く.
    このとき,\textbf{シンプレクティック代数} (symplectic algebra) $\Lsp (V)$(または $\lsp{2l,\, \mathbb{K}}$)が以下のように定義される:
    \begin{align}
        \Lsp (V) \coloneqq \bigl\{\, \textcolor{red}{x} \in \Lgl (V) \bigm| f \bigl( \textcolor{red}{x}(v),\, w \bigr) = -f \bigl( v,\, \textcolor{red}{x}(w) \bigr),\; \forall v,\, w \in V \,\bigr\} 
    \end{align}
    \exref{def:typeB}と全く同様にして $\Lsp (V)$ が\hyperref[def:linearLieAlg]{線型Lie代数}であることを証明できる.
\end{myexample}

\begin{myexample}[label=def:typeD]{線型Lie代数:$D_l$ 型}
    $\dim V = 2l$ とし,行列
    \begin{align}
        s \coloneqq \mqty[
            0 & \unity_l \\
            \unity_l & 0
        ] \in \Mat{2l}{\mathbb{K}}
    \end{align}
    により定まる $V$ 上の非退化かつ対称な双線型形式を $f$ と書く.
    このとき,\exref{def:typeB}と全く同様に\textbf{直交代数} (orthogonal algebra) $\Lo (V)$(または $\lo{2l,\, \mathbb{K}}$)が定義される.
\end{myexample}

さらに,古典代数以外の線型Lie代数をいくつか導入する.

\begin{myexample}[label=def:t]{線型Lie代数:$\mathfrak{t}$}
    $n \times n$ 上三角行列全体の集合を $\mathfrak{t}(n,\, \mathbb{K}) \subset \lgl{n}{\mathbb{K}}$ と書く\footnote{$\mathfrak{t}$ の\LaTeX コマンドは \texttt{\textbackslash mathfrak\{t\}}}:
    \begin{align}
        \mathfrak{t}(n,\, \mathbb{K}) \coloneqq \Bigl\{\, \bigl[\, a_{\mu\nu}\, \bigr]_{1 \le \mu,\, \nu \le n} \in \lgl{n}{\mathbb{K}} \Bigm| \mu > \nu \IMP a_{\mu\nu} = 0 \,\Bigr\} 
    \end{align}
    $\mathfrak{t}(n,\, \mathbb{K})$ が\hyperref[def:linearLieAlg]{線型Lie代数}であることを確認しよう.
    \begin{proof}
        $\forall x = \bigl[\, x_{\mu\nu}\, \bigr]_{1\le \mu,\, \nu \le n},\, y = \bigl[\, y_{\mu\nu}\, \bigr]_{1\le \mu,\, \nu \le n} \in \mathfrak{t}(n,\, \mathbb{K})$ をとる.このとき $1 \le \nu < \mu \le n$ なる全ての $(\mu,\, \nu)$ に対して
        \begin{align}
            [x + y]_{\mu\nu} &= x_{\mu\nu} + y_{\mu\nu} = 0 + 0 = 0 \\
            [\lambda x]_{\mu\nu} &= \lambda x_{\mu\nu} = \lambda 0 = 0 \\
            [xy]_{\mu\nu}
            &= \sum_{\lambda=1}^n x_{\mu \lambda} y_{\lambda \nu}
            = \sum_{\lambda=1}^{\nu} 0 y_{\lambda\nu} + \sum_{\lambda=\nu+1}^n x_{\mu\lambda} 0
            = 0
        \end{align}
        が成り立つので $x + y,\, \lambda x,\, \comm{x}{y} \in \mathfrak{t}(n,\, \mathbb{K})$ が言えた.
    \end{proof}
\end{myexample}


\begin{myexample}[label=def:n]{線型Lie代数:$\mathfrak{n}$}
    $n \times n$ 上三角行列のうち対角成分が全て $0$ であるもの全体がなす集合を $\mathfrak{n}(n,\, \mathbb{K}) \subset \lgl{n}{\mathbb{K}}$ と書く\footnote{$\mathfrak{n}$ の\LaTeX コマンドは \texttt{\textbackslash mathfrak\{n\}}}:
    \begin{align}
        \mathfrak{n}(n,\, \mathbb{K}) \coloneqq \Bigl\{\, \bigl[\, a_{\mu\nu}\, \bigr]_{1 \le \mu,\, \nu \le n} \in \lgl{n}{\mathbb{K}} \Bigm| \mu \;\textcolor{red}{\ge}\; \nu \IMP a_{\mu\nu} = 0 \,\Bigr\} 
    \end{align}
    $\mathfrak{n}(n,\, \mathbb{K})$ が\hyperref[def:linearLieAlg]{線型Lie代数}であることは\exref{def:t}と全く同様に証明できる.
\end{myexample}


\begin{myexample}[label=def:d]{線型Lie代数:$\mathfrak{d}$}
    $n \times n$ 対角行列全体がなす集合を $\mathfrak{d}(n,\, \mathbb{K}) \subset \lgl{n}{\mathbb{K}}$ と書く\footnote{$\mathfrak{d}$ の\LaTeX コマンドは \texttt{\textbackslash mathfrak\{d\}}}:
    \begin{align}
        \mathfrak{d}(n,\, \mathbb{K}) \coloneqq \Bigl\{\, \bigl[\, a_{\mu\nu}\, \bigr]_{1 \le \mu,\, \nu \le n} \in \lgl{n}{\mathbb{K}} \Bigm| \mu \neq \nu \IMP a_{\mu\nu} = 0 \,\Bigr\} 
    \end{align}
    $\mathfrak{d}(n,\, \mathbb{K})$ は明らかに\hyperref[def:linearLieAlg]{線型Lie代数}である.
\end{myexample}

\begin{myaxiom}[label=ax:Alg]{代数}
    体 $\mathbb{K}$ 上のベクトル空間 $\mathfrak{U}$\footnote{\texttt{\textbackslash mathfrak\{U\}}}の上に二項演算
    \begin{align}
        * \colon \mathfrak{U} \times \mathfrak{U} \lto \mathfrak{U},\; (x,\, y) \lmto x * y
    \end{align}
    が定義されていて,かつ以下の条件を充たすとき,$\mathfrak{U}$ は $\mathbb{K}$-\textbf{代数} ($\mathbb{K}$-algebra) と呼ばれる\footnote{結合律は\underline{要請しない}!$*$ に関する結合律を公理に含める場合は\textbf{結合代数} (associative algebra) と言う.文献によっては代数と言って結合代数のことを指す場合があるので注意.}:
    \begin{description}
        \item[\textbf{(A-1)}] $*$ は双線型写像である.i.e. $\forall x,\, x_i,\, y,\, y_i \in \mathfrak{g},\; \forall \lambda_i,\, \mu_i \in \mathbb{K}\; (i = 1,\, 2)$ に対して
        \begin{align}
            (\lambda_1 x_1 + \lambda_2 x_2)*y &= \lambda_1 x_1*y + \lambda_2 x_2*y, \\
            x*(\mu_1 y_1 + \mu_2 y_2) &= \mu_1 x*y_1 + \mu_2 x*y_2
        \end{align}
        が成り立つ.
        % \item[\textbf{(L-2)}] $\forall x \in \mathfrak{g}$ に対して
        % \begin{align}
        %     \comm{x}{x} = o
        % \end{align}
        % が成り立つ.
        % \item[\textbf{(L-3)}] $\forall x,\, y,\, z \in \mathfrak{g}$ に対して
        % \begin{align}
        %     \label{eq:Jacobi-identity}
        %     \comm{x}{\comm{y}{z}} + \comm{y}{\comm{z}{x}} + \comm{z}{\comm{x}{y}} = o
        % \end{align}
        % が成り立つ\footnote{\underline{結合律ではない!}}(\textbf{Jacobi}恒等式).
    \end{description}
\end{myaxiom}

\hyperref[ax:LieAlg]{Lie代数}は $\mathbb{K}$-\hyperref[ax:Alg]{代数}である.

\begin{mydef}[label=def:derivation-Alg]{代数の微分}
    $\mathbb{K}$-\hyperref[ax:Alg]{代数} $\mathfrak{U}$ の\textbf{微分} (derivation) とは,
    線型写像
    \begin{align}
        d \colon \mathfrak{U} \lto \mathfrak{U}
    \end{align}
    であってLeibniz則
    \begin{align}
        d (x * y) = x * d (y) + d(x) * y
    \end{align}
    を充たすもののこと.
\end{mydef}

全ての $\mathfrak{U}$ の微分がなす集合を $\Der \mathfrak{U}$ と書く.

\begin{myprop}[label=prop:Der-LieAlg]{代数の微分がなすLie代数}
    $\Der \mathfrak{U}$ の上の加法,スカラー乗法,Lieブラケットをそれぞれ
    \begin{align}
        +\colon \Der \mathfrak{U} \times \Der \mathfrak{U} \lto \Der \mathfrak{U},\; &(d,\, e) \lmto \bigl( x \mapsto d(x) + e(x) \bigr) \\
        \cdot\;\colon \mathbb{K} \times \Der \mathfrak{U} \lto \Der \mathfrak{U},\; &(\lambda,\, d) \lmto \bigl( x \mapsto \lambda d(x) \bigr) \\
        \comm{\;}{\,}\colon \Der \mathfrak{U} \times \Der \mathfrak{U} \lto \Der \mathfrak{U},\; &(d,\, e) \lmto \bigl( x \mapsto d(e(x)) - e(d(x)) \bigr)
    \end{align}
    と定義すると,組 $(\Der \mathfrak{U},\, +,\, \cdot\;,\, \comm{\;}{\,})$ は $\Lgl (\mathfrak{U})$ の\hyperref[def:subLieAlg]{部分Lie代数}である.
    Lie代数としての $\Der \mathfrak{U}$ のことを\textbf{微分代数} (derivation algebra) と呼ぶ\footnote{\url{https://mathworld.wolfram.com/DerivationAlgebra.html}}.
\end{myprop}

\begin{proof}
    $\forall x,\, y \in \mathfrak{U}$ をとる.
    \hyperref[def:derivation-Alg]{代数の微分}は線型写像なので,$*$ の双線型性および微分のLeibniz則から
    \begin{align}
        (d + e)(x * y) &= x * d(y) + d(x) * y + x * e(y) + e(x) * y = x * (d + e)(y) + (d + e)(x) * y \\
        (\lambda d)(x * y) &= \lambda \bigl( d(x) * y + x * d(y) \bigr) = \bigl(\lambda d(x) \bigr) * y + x * \bigl(\lambda d(y)\bigr) = (\lambda d)(x) * y + x * (\lambda d)(y)
    \end{align}
    が成り立つ.i.e. $d + e,\, \lambda d \in \Der \mathfrak{U}$ が言えた.
    Lieブラケットに関しては,
    \begin{align}
        d \bigl(e (x * y)\bigr) = d \bigl( x * e(y) + e(x) * y \bigr) = x * d\bigl(e(y)\bigr) + d(x) * e(y) + e(x) * d(y) + d\bigl(e(x)\bigr) * y
    \end{align}
    に注意すると
    \begin{align}
        \comm{d}{e}(x*y) 
        &= x * d\bigl(e(y)\bigr) + \cancel{d(x) * e(y) + e(x) * d(y)} + d\bigl(e(x)\bigr) * y \\
        &\quad - x * e\bigl(d(y)\bigr) \cancel{- e(x) * d(y) - d(x) * e(y)} - e\bigl(d(x)\bigr) * y \\
        &=x * \comm{d}{e}(y) + \comm{d}{e}(x) * y
    \end{align}
    が成り立つ.i.e. $\comm{d}{e} \in \Der \mathfrak{U}$ が言えた.
\end{proof}

\begin{myexample}[label=def:Der-inner]{内部微分}
    \hyperref[ax:LieAlg]{Lie代数} $\mathfrak{g}$ は $\mathbb{K}$-\hyperref[ax:Alg]{代数}なので,その\hyperref[prop:Der-LieAlg]{微分代数} $\Der \mathfrak{g}$ を考えることができる.
    
    ここで $\textcolor{blue}{x} \in \mathfrak{g}$ を任意に取ろう.このとき,写像
    \begin{align}
        \ad \textcolor{blue}{x} \colon \mathfrak{g} \lto \mathfrak{g},\; z \lmto \comm{\textcolor{blue}{x}}{z}
    \end{align}
    は線型写像で,\hyperref[ax:LieAlg]{Jacobi恒等式}から
    \begin{align}
        \ad \textcolor{blue}{x} (\comm{z}{w}) 
        &= \comm{\textcolor{blue}{x}}{\comm{z}{w}} \\
        &= \comm{\comm{\textcolor{blue}{x}}{z}}{w} + \comm{z}{\comm{\textcolor{blue}{x}}{w}} \\
        &= \comm{\ad \textcolor{blue}{x} (z)}{w} + \comm{z}{\ad \textcolor{blue}{x} (w)}
    \end{align}
    を充たすことがわかる.これはまさにLeibniz則なので,$\ad \textcolor{blue}{x} \in \Der (\mathfrak{g})$ である.
    $\ad \textcolor{blue}{x}$ の形で書ける $\Der (\mathfrak{g})$ の元のことを\textbf{内部微分} (inner derivation) と呼ぶ\footnote{$\Der (\mathfrak{g})$ の元が内部微分でないとき,\textbf{外部微分} (outer derivation) であると言う.}.
\end{myexample}


\subsection{イデアル・商代数}

\begin{mydef}[label=def:ideal-LieAlg]{Lie代数のイデアル}
    \hyperref[ax:LieAlg]{Lie代数} $\mathfrak{g}$ の部分集合 $\mathfrak{i}$ \footnote{\texttt{\textbackslash mathfrak\{i\}}}が $\mathfrak{g}$ の\textbf{イデアル} (ideal) であるとは,以下の2条件を充たすことを言う:
    \begin{description}
        \item[\textbf{(LI-1)}]  $\mathfrak{i}$ は $\mathfrak{g}$ の部分ベクトル空間である.
        \item[\textbf{(LI-2)}]  $\forall \textcolor{blue}{x} \in \textcolor{blue}{\mathfrak{g}},\, \forall y \in \mathfrak{i}$ に対して $\comm{\textcolor{blue}{x}}{y} \in \mathfrak{i}$
    \end{description}
\end{mydef}

\begin{myexample}[label=def:trivialIdeal-LieAlg]{自明なイデアル}
    $\forall x \in \mathfrak{g}$ に対して $\comm{x}{o} = \comm{x}{x + (-x)} = \comm{x}{x} - \comm{x}{x} = o$ が成り立つので,$\{o\}$ は $\mathfrak{g}$ の\hyperref[def:ideal-LieAlg]{イデアル}である.
    また,Lieブラケットについて閉じているので $\mathfrak{g}$ 自身もイデアルである.この2つを\textbf{自明なイデアル} (trivial ideal) と呼ぶ.
\end{myexample}


\begin{myexample}[label=def:center-LieAlg]{中心}
    \hyperref[ax:LieAlg]{Lie代数} $\mathfrak{g}$ の\textbf{中心} (center) は
    \begin{align}
        Z(\mathfrak{g}) \coloneqq \bigl\{\, z \in \mathfrak{g} \bigm| \forall \textcolor{blue}{x} \in \mathfrak{g},\; \comm{\textcolor{blue}{x}}{z} = o \,\bigr\} 
    \end{align}
    と定義される.$o \in Z(\mathfrak{g})$ であることから $Z(\mathfrak{g})$ は\hyperref[def:ideal-LieAlg]{イデアル}である.
    $\mathfrak{g}$ がLieブラケットに関して可換である必要十分条件は $Z(\mathfrak{g}) = \mathfrak{g}$ が成り立つことである\footnote{\hyperref[ax:LieAlg]{Lie代数の公理} \textsf{\textbf{(L'-1)}} より,$\comm{x}{y} = \comm{y}{x}$ ならば $\comm{x}{y} = - \comm{x}{y} \IFF \comm{x}{y} = o$ である.}.
\end{myexample}



\begin{mylem}[label=lem:ideal-LieAlg]{イデアル同士の演算}
    \hyperref[ax:LieAlg]{Lie代数} $\mathfrak{g}$ とその\hyperref[def:ideal-LieAlg]{イデアル} $\mathfrak{i},\, \mathfrak{j}$ を与える.このとき以下の3つが成り立つ:
    \begin{enumerate}
        \item 集合 $\mathfrak{i} \cap \mathfrak{j}$ は $\mathfrak{g}$ のイデアルである.
        \item 集合 $\mathfrak{i} + \mathfrak{j} \coloneqq \bigl\{\, x + y \bigm| x \in \mathfrak{i},\, y \in \mathfrak{j} \,\bigr\}$ は $\mathfrak{g}$ のイデアルである.
        \item 集合 $\comm{\mathfrak{i}}{\mathfrak{j}} \coloneqq \bigl\{\, \sum_{i=1}^n \comm{x_i}{y_i} \bigm| x_i \in \mathfrak{i},\, y_i \in \mathfrak{j};\, \textcolor{red}{n < \infty} \,\bigr\}$ は $\mathfrak{g}$ のイデアルである.
    \end{enumerate}
\end{mylem}

\begin{proof}
    \begin{enumerate}
        \item ほぼ自明.
        \item $\mathfrak{i} + \mathfrak{j}$ の勝手な元 $z + w$ と $\forall \textcolor{blue}{x} \in \mathfrak{g}$ をとる.Lieブラケットの双線型性から $\comm{\textcolor{blue}{x}}{z + w} = \comm{\textcolor{blue}{x}}{z} + \comm{\textcolor{blue}{x}}{w}$ が言えるが,$\mathfrak{i},\, \mathfrak{j}$ が\hyperref[def:ideal-LieAlg]{イデアル}であることにより $\comm{\textcolor{blue}{x}}{z} \in \mathfrak{i},\, \comm{\textcolor{blue}{x}}{w} \in \mathfrak{j}$ なので右辺は $\mathfrak{i} + \mathfrak{j}$ に属する.
        \item $\forall \textcolor{blue}{x} \in \mathfrak{g}$ を1つとる.Lieブラケットの双線型性から,このとき $\forall z \in \mathfrak{i},\, \forall w \in \mathfrak{j}$ に対して $\comm{\textcolor{blue}{x}}{\comm{z}{w}} \in \comm{\mathfrak{i}}{\mathfrak{j}}$ が成り立つことを示せば良い.
        実際,\hyperref[ax:LieAlg]{Lie代数の公理} \textsf{\textbf{(L-2)}}, \textsf{\textbf{(L-3)}} より
        \begin{align}
            \comm{\textcolor{blue}{x}}{\comm{z}{w}} = \comm{\comm{w}{\textcolor{blue}{x}}}{z} + \comm{\comm{\textcolor{blue}{x}}{z}}{w} = \comm{z}{\comm{\textcolor{blue}{x}}{w}} + \comm{\comm{\textcolor{blue}{x}}{z}}{w}
        \end{align}
        が言えるが,$\mathfrak{i},\, \mathfrak{j}$ が\hyperref[def:ideal-LieAlg]{イデアル}であることにより $\comm{\textcolor{blue}{x}}{w} \in \mathfrak{j},\, \comm{\textcolor{blue}{x}}{z} \in \mathfrak{i}$ なので右辺は $\comm{\mathfrak{i}}{\mathfrak{j}}$ に属する.
    \end{enumerate}
    
\end{proof}

\begin{myexample}[label=def:derived-LieAlg]{導来Lie代数}
    \hyperref[ax:LieAlg]{Lie代数} $\mathfrak{g}$ の\textbf{導来代数} (derived algebra) とは,$\mathfrak{g}$ の\hyperref[def:ideal-LieAlg]{イデアル} $\comm{\mathfrak{g}}{\mathfrak{g}}$ のこと.
\end{myexample}

\begin{mydef}[label=def:simple-LieAlg]{単純Lie代数}
    \hyperref[ax:LieAlg]{Lie代数} $\mathfrak{g}$ が\textbf{単純} (simple) であるとは,
    $\mathfrak{g}$ が\hyperref[def:trivialIdeal-LieAlg]{自明なイデアル}以外の\hyperref[def:ideal-LieAlg]{イデアル}を持たず,かつ $\comm{\mathfrak{g}}{\mathfrak{g}} \neq \{o\}$ を充たすことを言う.
\end{mydef}

\begin{myexample}[label=ex:sl2]{単純Lie代数:$\Lsl (2)$}
    
\end{myexample}

\hyperref[ax:LieAlg]{Lie代数} $\mathfrak{g}$ の\hyperref[def:ideal-LieAlg]{イデアル} $\mathfrak{i}$ が与えられたとき,商群のときと同じ要領で商代数を構成できる.このことを復習しよう:

$\mathfrak{g}$ 上の同値関係を
\begin{align}
    \sim \; \coloneqq \bigl\{\, (x,\, y) \in \mathfrak{g} \times \mathfrak{g} \bigm| x - y \in \mathfrak{i} \,\bigr\} 
\end{align}
と定義する\footnote{\hyperref[def:ideal-LieAlg]{\textsf{\textbf{(LI-1)}}}より $\mathfrak{i}$ は部分ベクトル空間なので反射律と対称律が言える.$x \sim y \AND y \sim z$ ならば $x - z = (x - y) + (y - z) \in \mathfrak{i}$ なので $x \sim z$ であり,推移律が成り立つことがわかる.}.
$\sim$ による $x \in \mathfrak{g}$ の同値類のことを $x + \mathfrak{i}$ と書き,商集合 $\mathfrak{g}/{\sim}$ のことを $\bm{\mathfrak{g}/\mathfrak{i}}$ と書く.
全射 $p \colon \mathfrak{g} \lto \mathfrak{g}/\mathfrak{i}$ を\textbf{標準的射影} (canonical projection) と呼ぶ.

\begin{mydef}[label=def:quotient-LieAlg]{商代数}
    集合 $\mathfrak{g} / \mathfrak{i}$ の上には次のようにしてwell-definedな加法,スカラー乗法,Lieブラケットが定義できる:
    \begin{align}
        + \colon \mathfrak{g} / \mathfrak{i} \times \mathfrak{g} / \mathfrak{i} \lto \mathfrak{g} / \mathfrak{i},\; &(x + \mathfrak{i},\, y + \mathfrak{i}) \lmto (x + y) + \mathfrak{i} \\
        \cdot\; \colon \mathbb{K} \times \mathfrak{g} / \mathfrak{i} \lto \mathfrak{g} / \mathfrak{i},\; &(\lambda,\, x + \mathfrak{i}) \lmto (\lambda x) + \mathfrak{i} \\
        \comm{\;}{\,} \colon \mathfrak{g} / \mathfrak{i} \times \mathfrak{g} / \mathfrak{i} \lto \mathfrak{g} / \mathfrak{i},\; &(x + \mathfrak{i},\, y + \mathfrak{i}) \lmto \comm{x}{y} + \mathfrak{i}
    \end{align}
    \hyperref[ax:LieAlg]{Lie代数} $(\mathfrak{g} / \mathfrak{i},\, + ,\, \cdot\; ,\, \comm{\;}{\,})$ のことを\textbf{商代数} (quotient algebra) と呼ぶ.
\end{mydef}

\begin{proof}
    well-definednessを示す.$x + \mathfrak{i} = x' + \mathfrak{i},\, y + \mathfrak{i} = y' + \mathfrak{i}$ ならば $x'-x,\, y' - y \in \mathfrak{i}$ であるから
    \begin{align}
        x' + y' &= (x + y) + (x' - x) + (y' - y) \in (x + y) + \mathfrak{i}, \\
        \lambda x' &= \lambda x + \lambda (x' - x) \in (\lambda x) + \mathfrak{i}
    \end{align}
    が成り立つ.また,$\textcolor{red}{\mathfrak{i}}$ \textcolor{red}{が}\hyperref[def:ideal-LieAlg]{イデアル}\textcolor{red}{であることにより}
    \begin{align}
        \comm{x'}{y'} &= \comm{x + (x' - x)}{y + (y' - y)} \\
        &= \comm{x}{y} + \comm{x}{y' - y} - \comm{y}{x' - x} + \comm{x'-x}{y'-y} \\
        &\in \comm{x}{y} + \mathfrak{i}
    \end{align}
    が言える.
\end{proof}

\begin{marker}
    定義\ref{def:quotient-LieAlg}による商代数の構成によって,標準的射影は自動的にLie代数の準同型写像になる:
    \begin{align}
        p (x + y) &= (x + y) + \mathfrak{i} \eqqcolon (x + \mathfrak{i}) + (y + \mathfrak{i}) = p(x) + p(y), \\
        p (\lambda x) &= (\lambda x) + \mathfrak{i} \eqqcolon\lambda(x + \mathfrak{i}) = \lambda p(x), \\
        p (\comm{x}{y}) &= \comm{x}{y} + \mathfrak{i} \eqqcolon \comm{(x + \mathfrak{i})}{(y + \mathfrak{i})} = \comm{p(x)}{p(y)}.
    \end{align}
\end{marker}


\begin{mydef}[label=def:normalizer-LieAlg]{正規化代数・中心化代数}
    \begin{itemize}
        \item \hyperref[ax:LieAlg]{Lie代数} $\mathfrak{g}$ とその\hyperref[def:subLieAlg]{部分Lie代数}(もしくは部分ベクトル空間) $\mathfrak{h}$ を与える.
        このとき $\mathfrak{h}$ の\textbf{正規化代数} (normalizer) を
        \begin{align}
            N_{\mathfrak{g}} (\mathfrak{h}) \coloneqq \bigl\{\, x \in \mathfrak{g} \bigm| \forall \textcolor{blue}{h} \in \mathfrak{h},\; \comm{x}{\textcolor{blue}{h}} \in \mathfrak{h} \,\bigr\} 
        \end{align}
        で定義する
        \footnote{
            $N_{\mathfrak{g}} (\mathfrak{h})$ は $\mathfrak{g}$ の部分Lie代数である:$\forall x,\, y \in N_{\mathfrak{g}}(\mathfrak{h})$ に対して,Jacobi恒等式から $\forall z \in \mathfrak{h},\; \comm{\comm{x}{y}}{z} = \comm{x}{\comm{y}{z}} - \comm{y}{\comm{x}{z}} \in \mathfrak{h}$ が言える.
            さらに,もし $\mathfrak{h}$ が部分Lie代数ならば,定義から $\mathfrak{h}$ を $N_{\mathfrak{g}} (\mathfrak{h})$ の部分ベクトル空間と見做したときに $\mathfrak{h}$ は自動的に\hyperref[def:ideal-LieAlg]{イデアル}になる.
            特に,$N_{\mathfrak{g}} (\mathfrak{h})$ は $\mathfrak{h} \subset N_{\mathfrak{g}} (\mathfrak{h})$ をイデアルとして持つ最大の部分Lie代数である.
        }.特に $N_{\mathfrak{g}} (\mathfrak{h})  = \mathfrak{h}$ のとき,$\mathfrak{h}$ は\textbf{self-normalizing}であると言う.
        \item \hyperref[ax:LieAlg]{Lie代数} $\mathfrak{g}$ とその\underline{部分集合} $X$ を与える.
        このとき $X$ の\textbf{中心化代数} (centralizer) を
        \begin{align}
            C_{\mathfrak{g}} (X) \coloneqq \bigl\{\, x \in \mathfrak{g} \bigm| \forall \textcolor{blue}{z} \in X,\; \comm{x}{\textcolor{blue}{z}} = o \,\bigr\} 
        \end{align}
        と定義する
        \footnote{
            $C_{\mathfrak{g}} (X)$ が $\mathfrak{g}$ の部分Lie代数であることは,$N_{\mathfrak{g}} (\mathfrak{h})$ のときと全く同様にして示される.
        }
        .
    \end{itemize}
\end{mydef}

\exref{def:center-LieAlg}を思い出すと,$Z (\mathfrak{g}) = C_{\mathfrak{g}} (\mathfrak{g})$ である.

\subsection{準同型・表現}

\begin{mydef}[label=def:hom-LieAlg]{Lie代数の準同型}
    2つの\hyperref[ax:LieAlg]{Lie代数} $\mathfrak{g},\, \mathfrak{h}$ を与える.写像 $f \colon \mathfrak{g},\, \mathfrak{h}$ が\textbf{Lie代数の準同型} (homomorphism) であるとは,
    $f$ が和,スカラー乗法,Lieブラケットの全てを保存することを言う.i.e. $\forall x,\, y \in \mathfrak{g},\; \forall \lambda \in \mathbb{K}$ に対して
    \begin{align}
        f (x + y) &= f(x) + f(y), \\
        f (\lambda x) &= \lambda f(x), \\
        f (\comm{x}{y}) &= \comm{f(x)}{f(y)}
    \end{align}
    が成り立つこと.
    \tcblower
    全単射なLie代数の準同型のことをLie代数の\textbf{同型} (isomorphism) と呼ぶ.
\end{mydef}

Lie代数の準同型 $f \colon \mathfrak{g} \lto \mathfrak{h}$ の\textbf{核} (kernel), \textbf{像} (image) をそれぞれ
\begin{align}
    \Ker f &\coloneqq \bigl\{\, x \in \mathfrak{g} \bigm| f(x) = o \,\bigr\}, \\
    \Im f &\coloneqq \bigl\{\, f(x) \in \mathfrak{h} \bigm| x \in \mathfrak{g} \,\bigr\}
\end{align}
と定義すると $\Ker f$ は\hyperref[def:ideal-LieAlg]{イデアル}であり
\footnote{
    $\forall x,\, y \in \Ker f$ に対して $f(x + y) = f(x) + f(y) = o + o = o,\; f(\lambda x) = \lambda f(x) = \lambda o = o$ なので部分ベクトル空間,かつ $\forall z \in \mathfrak{g}$ に対して $f(\comm{z}{x}) = \comm{f(z)}{f(x)} = \comm{f(z)}{o} = o$.
},
$\Im f$ は\hyperref[def:subLieAlg]{部分Lie代数}である
\footnote{
    $\forall f(x),\, f(y) \in \Im f$ に対して $f(x) + f(y) = f(x+y),\; \lambda f(x) = f(\lambda x);\, \comm{f(x)}{f(y)} = f(\comm{x}{y})$
}
.

\begin{myprop}[label=prop:homo]{準同型定理}
    Lie代数 $\mathfrak{g}$ およびその\hyperref[def:ideal-LieAlg]{イデアル} $\mathfrak{i},\, \mathfrak{j}$ を与える.
    このとき,\underline{任意の}Lie代数の準同型 $\textcolor{blue}{f} \colon \mathfrak{g} \textcolor{blue}{\lto \mathfrak{h}}$ に対して\footnote{わざわざこのような記述をしたのは,(1) が商代数の\textbf{普遍性} (universal property) を意味していることを明示するためである.}以下が成り立つ:
    \begin{enumerate}
        \item $\mathfrak{g}/\Ker \textcolor{blue}{f} \cong \Im \textcolor{blue}{f}$.
        また,$\mathfrak{i} \subset \Ker \textcolor{blue}{f}$ が成り立つならば,以下の図式を可換にするLie代数の準同型 $\textcolor{red}{\psi} \colon \mathfrak{g}/\mathfrak{i} \lto \textcolor{blue}{\mathfrak{h}}$ が一意的に存在する:
        \begin{figure}[H]
            \centering
            \begin{tikzcd}[row sep=large, column sep=large]
                &\mathfrak{g} \ar[rr, blue, "f"]\ar[dr, "p"'] & &\forall \textcolor{blue}{\mathfrak{h}} \ar[from=dl,red, dashed, "\exists! \psi"'] \\
                & &\mathfrak{g}/\mathfrak{i} &
            \end{tikzcd}
            \caption{\hyperref[def:quotient-LieAlg]{商代数}の普遍性.$p$ は標準的射影 $p \colon x \lmto x + \mathfrak{i}$ を表す.}
            \label{cmtd:1st-homo}
        \end{figure}%
        \item $\mathfrak{i} \subset \mathfrak{j}$ が成り立つとする.
        このとき $\mathfrak{j}/\mathfrak{i}$ は $\mathfrak{g}/\mathfrak{i}$ のイデアルであり,自然な同型 $(\mathfrak{g}/\mathfrak{i})/(\mathfrak{j}/\mathfrak{i}) \cong \mathfrak{g} / \mathfrak{j}$ が成り立つ.
        \item 自然な同型 $(\mathfrak{i} + \mathfrak{j}) / \mathfrak{j} \cong \mathfrak{i} / (\mathfrak{i} \cap \mathfrak{j})$ が成り立つ.
    \end{enumerate}
\end{myprop}

\begin{proof}
    \begin{enumerate}
        \item 
        % $i \colon \Ker f \hookrightarrow \mathfrak{g}$ を包含写像とすると,Lie代数とLie代数の準同型からなる図式
        % \begin{center}
        %     \begin{tikzcd}[row sep=large, column sep=large]
        %         \Ker f \ar[r, "i"] &\mathfrak{g} \ar[r, "f"] &\Im f \ar[r] &\{o\}
        %     \end{tikzcd}
        % \end{center}
        % は完全列である.
         まず,$\iota \subset \Ker f$ ならば写像 $\psi \colon \mathfrak{g} / \mathfrak{i} \lto \Im f,\; x + \mathfrak{i} \lmto f(x)$ がwell-definedなLie代数の準同型であることを示す.実際 $x + \mathfrak{i} = x' + \mathfrak{i}$ ならば $x' - x \in \mathfrak{i} \subset \Ker f$ であり, 
        $\psi (x' + \mathfrak{i}) = f(x') = f\bigl(x + (x' - x)\bigr) = f(x) + f(x'-x) = f(x) = \psi (x + \mathfrak{i})$ が言えた.$\psi$ がLie代数の準同型であることは $f$ がLie代数の準同型であることから明らか.
        定義から $\psi \circ p = f$ である.

         次に $\psi$ の一意性を示す.別の $\phi \colon \mathfrak{g} / \mathfrak{i} \lto \mathfrak{h}$ が存在して $\phi \circ p = f$ が成り立つとする.このとき $\forall x \in \mathfrak{g}$ に対して $\phi (x + \iota) = f(x)$ が成り立つので $\phi = \psi$ である.

         特に $\mathfrak{i} = \Ker f$ ならば,$\Ker \psi = \{\Ker f\} = \{ o \}$ なので $\psi$ は単射であり $\mathfrak{g} / \Ker f \cong \Im f$ が言えた.
        \item 
         $\mathfrak{i} \subset \mathfrak{j}$ のとき,\hyperref[def:ideal-LieAlg]{イデアルの定義}より $\mathfrak{i}$ は $\mathfrak{j}$ のイデアルでもある.よって商代数 $\mathfrak{j}/\mathfrak{i}$ はwell-defined.明らかに $\mathfrak{j} / \mathfrak{i} \subset \mathfrak{g} / \mathfrak{i}$ なので $\mathfrak{j} / \mathfrak{i}$ は $\mathfrak{g} / \mathfrak{i}$ の\hyperref[def:subLieAlg]{部分Lie代数}だと分かった.
        ここで $\forall x + \mathfrak{i} \in \mathfrak{g}/\mathfrak{i},\; \forall y + \mathfrak{i} \in \mathfrak{j}/ \mathfrak{i}$ をとると,$\mathfrak{j}$ が $\mathfrak{g}$ のイデアルであることから $\comm{x + \mathfrak{i}}{y + \mathfrak{i}} = \comm{x}{y} + \mathfrak{i} \in \mathfrak{j} / \mathfrak{i}$ が従う\footnote{$\comm{x}{y} \in \mathfrak{j}$ なので.}.以上の議論から $\mathfrak{j}/\mathfrak{i}$ が $\mathfrak{g} / \mathfrak{i}$ のイデアルであることが示された.

         さて,Lie代数の全射準同型 $f \colon \mathfrak{g}/\mathfrak{i} \lto \mathfrak{g}/\mathfrak{j},\; x + \mathfrak{i} \lmto x + \mathfrak{j}$ は仮定よりwell-definedで\footnote{$x + \mathfrak{i} = x' + \mathfrak{i} \IMP x' - x \in \mathfrak{i} \subset \mathfrak{j} \IMP f(x' + \mathfrak{i}) = (x + (x' - x)) + \mathfrak{j} = x + \mathfrak{j} = f(x + \mathfrak{i})$} ,かつ $\Ker f = \mathfrak{j} / \mathfrak{i}$ である.従って (1) より $ \mathfrak{g} / \mathfrak{j} \cong (\mathfrak{g}/ \mathfrak{i}) / (\mathfrak{j} / \mathfrak{i})$ が言える.
        \item Lie代数の全射準同型 $f \colon \mathfrak{i} \lto (\mathfrak{i} + \mathfrak{j}) / \mathfrak{j},\; x \lmto x + \mathfrak{j}$ に対して $\Ker f = \mathfrak{i} \cap \mathfrak{j}$ である.従って (1) より $(\mathfrak{i} + \mathfrak{j}) / \mathfrak{j} \cong \mathfrak{i} / (\mathfrak{i} \cap \mathfrak{j})$ が言える.
    \end{enumerate}
\end{proof}


\begin{mydef}[label=def:rep-LieAlg]{Lie代数の表現}
    $V$ を $\mathbb{K}$-ベクトル空間とする.Lie代数の準同型
    \begin{align}
        \phi \colon \mathfrak{g} \lto \Lgl (V)
    \end{align}
    と $V$ の組 $(\phi,\, V)$ のことを\textbf{Lie代数の表現} (representation) と呼ぶ\footnote{「代数の表現」と言われたとき,\hyperref[ax:Alg]{代数}\textbf{の表現}なのか\textbf{結合代数の表現}なのか\textbf{Lie代数の表現}なのか文脈で判断しなくてはならないかもしれない.}.
\end{mydef}

\begin{myexample}[label=def:adj-LieAlg]{随伴表現}
     \exref{def:Der-inner}で定義した $\ad (x)$ は,写像
    \begin{align}
        \ad \colon \mathfrak{g} \lto \Der(\mathfrak{g}) \subset \Lgl (\mathfrak{g}),\; \textcolor{blue}{x} \lmto \bigl( z \mapsto \comm{\textcolor{blue}{x}}{z} \bigr) 
    \end{align}
    だと思うことができる.$\ad$ は明らかに線型写像で,かつ $\forall z \in \mathfrak{g}$ に対して
    \begin{align}
        \comm{\ad x}{\ad y}(z) &= \ad x (\comm{y}{z}) - \ad y (\comm{x}{z}) \\
        &= \comm{x}{\comm{y}{z}} - \comm{y}{\comm{x}{z}} \\
        &= \comm{x}{\comm{y}{z}} + \comm{y}{\comm{z}{x}} \\
        &= \comm{\comm{x}{y}}{z} \\
        &= \ad \comm{x}{y} (z)
    \end{align}
    が成り立つのでLie代数の準同型である.
    故に組 $(\ad,\, \mathfrak{g})$ は\hyperref[def:rep-LieAlg]{Lie代数の表現}である.
    \begin{align}
        x \in \Ker \ad &\IFF \forall y \in \mathfrak{g},\; \comm{x}{y} = o
    \end{align}
    より,\exref{def:center-LieAlg}を思い出すと $\Ker \ad = Z(\mathfrak{g})$ と分かる.特に\hyperref[def:simple-LieAlg]{単純Lie代数} $\mathfrak{g}$ に対して $Z(\mathfrak{g}) = \{o\}$ なので $\ad$ は単射準同型である.i.e. \textcolor{red}{任意の単純Lie代数は\hyperref[def:linearLieAlg]{線形Lie代数}と同型である}\footnote{単射準同型は包含準同型だと見做せる.}.
\end{myexample}

\subsection{自己同型}

\begin{mydef}[label=def:auto-LieAlg]{自己同型}

\end{mydef}

\subsection{可解Lie代数}

\hyperref[ax:LieAlg]{Lie代数} $\mathfrak{g}$ を与える.$D \mathfrak{g} \coloneqq \comm{\mathfrak{g}}{\mathfrak{g}}$ とおくと,\exref{def:derived-LieAlg}より $D \mathfrak{g}$ は $\mathfrak{g}$ の\hyperref[def:ideal-LieAlg]{イデアル}である.

\begin{mydef}[label=def:solvable-LieAlg]{可解Lie代数}
    \textbf{導来列} (derived series) とは,\hyperref[def:ideal-LieAlg]{イデアル}の減少列
    \begin{align}
        \mathfrak{g} = D^0 \mathfrak{g} \supset D^1 \mathfrak{g} \supset D^2 \mathfrak{g} \supset \cdots
    \end{align}
    のこと.
    ある $n > 0$ に対して導来列が初めてゼロになって止まるとき,i.e. $D^n \mathfrak{g} = \{o\}$ が成り立つとき,Lie代数 $\mathfrak{g}$ は\textbf{可解} (solvable) であると言われる.
\end{mydef}

\begin{myprop}[label=prop:solvable-basic]{}
    Lie代数 $\mathfrak{g}$ およびLie代数の準同型 $f \colon \mathfrak{g} \lto \mathfrak{h}$ を与える.このとき以下が成り立つ:
    \begin{enumerate}
        \item $\mathfrak{g}$ が\hyperref[def:solvable-LieAlg]{可解}ならば,
        $\mathfrak{g}$ の任意の\hyperref[def:subLieAlg]{部分Lie代数} $\mathfrak{a}$ は可解である.また,$\Im f$ も可解である.
        \item $\mathfrak{g}$ が\hyperref[def:solvable-LieAlg]{可解}ならば,$\mathfrak{g}$ の任意の\hyperref[def:ideal-LieAlg]{イデアル} $\mathfrak{a}$ について\hyperref[def:quotient-LieAlg]{商代数} $\mathfrak{g} / \mathfrak{a}$ も可解である.
        \item $\mathfrak{g}$ のイデアル $\mathfrak{i}$ と\hyperref[def:quotient-LieAlg]{商代数} $\mathfrak{g}/\mathfrak{i}$ がどちらも可解ならば,$\mathfrak{g}$ 自身も可解である.
        \item $\mathfrak{g}$ のイデアル $\mathfrak{i},\, \mathfrak{j}$ が可解ならば,イデアル $\mathfrak{i} + \mathfrak{j}$ も可解である.
    \end{enumerate}
\end{myprop}

\begin{proof}
    \begin{enumerate}
        \item 仮定より $\mathfrak{g}$ が可解なので,ある $n > 0$ に対して $D^n \mathfrak{g} = \{o\}$ となる.$\mathfrak{a} \subset \mathfrak{g}$ ならば $D \mathfrak{a} \subset D \mathfrak{g}$ であるから,帰納的に $D^i \mathfrak{a} \subset D^i \mathfrak{g}\; (i = 0,\, 1,\, \dots)$ が分かる.よって $D^n \mathfrak{a} = \{o\}$ である.
        また,$f$ がLie代数の準同型であることから $f(D \mathfrak{g}) = \comm{f(\mathfrak{g})}{f(\mathfrak{g})} = \comm{\Im f}{\Im f} = D(\Im f)$ であり,帰納的に $f(D^i \mathfrak{g}) = D^i(\Im f)$ が分かる.故に $D^n (\Im f) = f(\{o\}) = \{o\}$ である.
        \item 標準的射影 $p \colon \mathfrak{g} \lto \mathfrak{g}/\mathfrak{a},\; x \lmto x + \mathfrak{a}$ に対して (1) を適用すれば良い.
        \item 仮定より $\mathfrak{g} / \mathfrak{i}$ が可解なので,ある $n > 0$ に対して $D^n (\mathfrak{g}/\mathfrak{i}) = \{o\}$ と成る.標準的射影 $p \colon \mathfrak{g} \lto \mathfrak{g}/\mathfrak{i}$ について,(1) の証明から $\{o\} = D^n (\mathfrak{g}/\mathfrak{i}) = D^n (\Im p) = p (D^n \mathfrak{g})$ がわかる.従って $D^n \mathfrak{g} \subset \mathfrak{i}$ である.
        仮定より $\mathfrak{i}$ も可解だからある $m > 0$ に対して $D^m\mathfrak{i} = \{o\}$ となる.故に帰納的に $D^{n+m} \mathfrak{g} \subset \mathfrak{i}^m = \{o\}$ が示される.
        \item \hyperref[prop:homo]{準同型定理}-(3) より
        \begin{align}
            (\mathfrak{i} + \mathfrak{j})/\mathfrak{j} \cong \mathfrak{i}/(\mathfrak{i} \cap \mathfrak{j})
        \end{align}
        が成り立つ.仮定より $\mathfrak{i}$ は可解なので,(2) より右辺も可解.よって (3) より $\mathfrak{i} + \mathfrak{j}$ も可解である.
    \end{enumerate}
\end{proof}

$\mathfrak{g}$ を\underline{有限次元}Lie代数とし,
$\mathfrak{r}$ \footnote{\texttt{\textbackslash mathfrak\{r\}}}を $\mathfrak{g}$ の極大\footnote{$\mathfrak{r} \subsetneq \mathfrak{i} \subset \mathfrak{g}$ を充たす $\mathfrak{g}$ の可解イデアル $\mathfrak{i}$ が $\mathfrak{g}$ 以外に存在しない.}\hyperref[def:solvable-LieAlg]{可解}イデアルとしよう\footnote{$\mathfrak{r}$ は少なくとも1つ存在する.$\mathfrak{g}$ が有限次元なので,$\mathfrak{g}$ の可解イデアルのうち次元が最大のものを取れば良い.}

\begin{mylem}[label=lem:maximal]{}
    $\mathfrak{r}$ は $\mathfrak{g}$ の全ての可解イデアルを含む.i.e. 最大 (maximal) の可解イデアルである.
\end{mylem}

\begin{proof}
    任意の $\mathfrak{g}$ の可解イデアル $\mathfrak{i}$ をとる.\hyperref[prop:homo]{準同型定理}-(3) より
    \begin{align}
        (\mathfrak{i} + \mathfrak{r})/\mathfrak{i} \cong \mathfrak{r} / (\mathfrak{i} \cap \mathfrak{r})
    \end{align}
    が成り立つ.$\mathfrak{r}$ は可解なので命題\ref{prop:solvable-basic}-(1) より $\mathfrak{r} / (\mathfrak{i} \cap \mathfrak{r})$ は可解であり,従って $(\mathfrak{i} + \mathfrak{r})/\mathfrak{i}$ も可解である.$\mathfrak{i}$ も可解なので命題\ref{prop:solvable-basic}-(3) より $\mathfrak{i} + \mathfrak{r}$ も可解である.
    然るに,$\mathfrak{r}$ の極大性により $\mathfrak{i} + \mathfrak{r} = \mathfrak{r}$ であるから $\mathfrak{i} \subset \mathfrak{r}$ が言えた.
\end{proof}

\begin{mydef}[label=def:rad-LieAlg]{根基}
    $\mathfrak{g}$ の最大可解イデアル\footnote{補題\ref{lem:maximal}によりこれは一意的に存在する.} (maximal solvable ideal) $\mathfrak{r}$ のことを $\mathfrak{g}$ の\textbf{根基} (radical) と呼び,$\rad \mathfrak{g}$ と書く.
\end{mydef}

\begin{mydef}[label=def:semisimple-LieAlg]{半単純Lie代数}
    Lie代数 $\mathfrak{g}$ が\textbf{半単純} (semisimple) であるとは,$\rad \mathfrak{g} = \{o\}$ であることを言う.
\end{mydef}

\subsection{冪零Lie代数}

Lie代数 $\mathfrak{g}$ に対して
\begin{align}
    C^0 \mathfrak{g} &\coloneqq \mathfrak{g}, \\
    C^n \mathfrak{g} &\coloneqq \comm{\mathfrak{g}}{C^{n-1} \mathfrak{g}}
\end{align}
のように帰納的に\hyperref[def:ideal-LieAlg]{イデアル}の減少列
\begin{align}
    \label{eq:descendingCentralSeries}
    \mathfrak{g} = C^0 \mathfrak{g} \supset C^1 \mathfrak{g} \supset C^2 \mathfrak{g} \supset \cdots
\end{align}
を定義する.

\begin{mydef}[label=def:nilpotent-LieAlg]{冪零Lie代数}
    イデアルの減少列\eqref{eq:descendingCentralSeries}がある $n > 0$ に対して初めてゼロになって止まるとき,i.e. $C^n \mathfrak{g} = \{o\}$ が成り立つとき,Lie代数 $\mathfrak{g}$ は\textbf{冪零} (nilpotent) であると言われる.
\end{mydef}

\begin{marker}
    線型写像 $x \colon \mathfrak{g} \lto \mathfrak{g}$ の冪零性との混同を避けるため,定義\ref{def:nilpotent-LieAlg}の意味で冪零と言う場合は冪零Lie代数と言うことにする.
\end{marker}

\begin{myprop}[label=prop:nilpo-basic]{}
    Lie代数 $\mathfrak{g}$ およびLie代数の準同型 $f \colon \mathfrak{g} \lto \mathfrak{h}$ を与える.このとき以下が成り立つ:
    \begin{enumerate}
        \item $\mathfrak{g}$ が\hyperref[def:nilpotent-LieAlg]{冪零Lie代数}ならば,
        $\mathfrak{g}$ の任意の\hyperref[def:subLieAlg]{部分Lie代数} $\mathfrak{a}$ は冪零Lie代数である.
        また,$\Im f$ も冪零Lie代数である.
        \item $\mathfrak{g}$ が\hyperref[def:nilpotent-LieAlg]{冪零Lie代数}ならば,$\mathfrak{g}$ の任意の\hyperref[def:ideal-LieAlg]{イデアル} $\mathfrak{a}$ について\hyperref[def:quotient-LieAlg]{商代数} $\mathfrak{g} / \mathfrak{a}$ も冪零Lie代数である.
        \item $\mathfrak{g}$ の\hyperref[def:center-LieAlg]{中心}による商代数 $\mathfrak{g} / Z (\mathfrak{g})$ が冪零Lie代数ならば $\mathfrak{g}$ は冪零Lie代数である.
        \item $\mathfrak{g}$ が冪零Lie代数かつ $\mathfrak{g} \neq \{o\}$ ならば,$Z (\mathfrak{g}) \neq \{o\}$ である.
        \item $\mathfrak{g}$ が冪零Lie代数ならば,$\forall x \in \mathfrak{g}$ に対して $\ad x$ は冪零である.
    \end{enumerate}
\end{myprop}

\begin{proof}
    \begin{enumerate}
        \item 仮定より $\mathfrak{g}$ が冪零Lie代数なので,ある $n > 0$ に対して $C^n \mathfrak{g} = \{o\}$ となる.$\mathfrak{a} \subset \mathfrak{g}$ ならば $C \mathfrak{a} \subset C \mathfrak{g}$ であるから,帰納的に $C^i \mathfrak{a} \subset C^i \mathfrak{g}\; (i = 0,\, 1,\, \dots)$ が分かる.よって $C^n \mathfrak{a} = \{o\}$ である.
        また,$f$ がLie代数の準同型であることから $f(C \mathfrak{g}) = \comm{f(\mathfrak{g})}{f(\mathfrak{g})} = \comm{\Im f}{\Im f} = C(\Im f)$ であり,帰納的に $f(C^i \mathfrak{g}) = C^i(\Im f)$ が分かる.故に $C^n (\Im f) = f(\{o\}) = \{o\}$ である.
        \item 標準的射影 $p \colon \mathfrak{g} \lto \mathfrak{g}/\mathfrak{a},\; x \lmto x + \mathfrak{a}$ に対して (1) を適用すれば良い.
        \item 仮定より $\mathfrak{g} / Z(\mathfrak{g})$ が冪零Lie代数なので,ある $n > 0$ に対して $C^n (\mathfrak{g}/Z(\mathfrak{g})) = \{o\}$ と成る.標準的射影 $p \colon \mathfrak{g} \lto \mathfrak{g}/Z(\mathfrak{g})$ について,(1) の証明から $\{o\} = C^n (\mathfrak{g}/Z(\mathfrak{g})) = C^n (\Im p) = p (C^n \mathfrak{g})$ がわかる.従って $C^n \mathfrak{g} \subset Z(\mathfrak{g})$ である.
        故に\hyperref[def:center-LieAlg]{中心の定義}から $C^{n+1} \mathfrak{g} = \comm{\mathfrak{g}}{C^n \mathfrak{g}} \subset \comm{\mathfrak{g}}{Z(\mathfrak{g})} = \{o\}$ が示された.
        \item 仮定より $\mathfrak{g}$ が冪零Lie代数なので,ある $n > 0$ に対して $C^{n-1} \mathfrak{g} \neq \{o\}$ かつ $C^n \mathfrak{g} = \{o\}$ となる.このとき $\{o\} = C^{n} \mathfrak{g} = \comm{\mathfrak{g}}{C^{n-1}\mathfrak{g}}$ なので中心の定義から $\{o\} \neq C^{n-1}\mathfrak{g} \subset Z(\mathfrak{g})$ が言える.
        \item 仮定より $\mathfrak{g}$ が冪零Lie代数なので,ある $n > 0$ に対して $C^n \mathfrak{g} = \{o\}$ となる.このとき $\forall x \in \mathfrak{g}$ に対して\hyperref[def:adj-LieAlg]{随伴表現} $\ad (x) \colon \mathfrak{g} \lto \mathfrak{g}$ を考える.$\ad(x)(C^i \mathfrak{g}) = \comm{x}{C^i\mathfrak{g}} \subset \comm{\mathfrak{g}}{C^i\mathfrak{g}} = C^{i+1} \mathfrak{g}$ なので,イデアルの減少列\eqref{eq:descendingCentralSeries}に $\ad(x)$ を作用させると
        \begin{align}
            \mathfrak{g} \xrightarrow{\ad(x)} C^1 \mathfrak{g} \xrightarrow{\ad(x)} \cdots \xrightarrow{\ad(x)} C^n \mathfrak{g} = \{o\}
        \end{align}
        となる.i.e. $\forall y \in \mathfrak{g}$ に対して $\bigl(\ad(x)\bigr)^n = 0$ である.
    \end{enumerate}
\end{proof}

\subsection{Engelの定理}

\begin{mylem}[label=lem:ad-nilpo]{$\ad$ の冪零性}
    $x \in \Lgl (V)$ が冪零ならば $\ad x \in \Lgl (\Lgl (V))$ も冪零である.
\end{mylem}

\begin{proof}
    $x$ が冪零であるという仮定から,$x^n = 0$ を充たす $n \in \mathbb{N}$ が存在する. 
    
    さて,$x$ による左移動と右移動をそれぞれ
    \begin{align}
        \lambda_x \colon \Lgl (V) \lto \Lgl (V),\; y \lmto \bigl(\, v \mapsto x (y(v))\,\bigr), \\
        \rho_x \colon \Lgl (V) \lto \Lgl (V),\; y \lmto \bigl(\, v \mapsto y(x(v)) \bigr)
    \end{align}
    と定義しよう.$\forall y \in \Lgl (V)$ を1つとると
    \begin{align}
        \label{eq:ad-nilpo-lr}
        (\lambda_x)^n (y) &= x^n y = 0, &
        (\rho_x)^n (y) &= y x^n = 0
    \end{align}
    が成り立つ.さらに $\rho_x (\lambda_x (y)) = xyx = \lambda_x (\rho_x (y))$ が成り立つ.i.e. $\rho_x \lambda_x = \lambda_x \rho_x$ である.
    従って二項定理が使えて\footnote{写像の合成の結合律を暗に使っている.},\eqref{eq:ad-nilpo-lr}から
    \begin{align}
        (\ad x)^{2n-1} &= (\lambda_x - \rho_x)^{2n-1} \\
        &= \sum_{k=0}^{2n-1} \mqty(2n-1 \\ k) (\rho_x)^{2n-k-1} (\lambda_x)^{k} \\ 
        &= \sum_{k=0}^{n-1} \mqty(2n-1 \\ k) 0 (\lambda_x)^k + \sum_{k=n}^{2n}-1 \mqty(2n-1 \\ k) (\rho_x)^{2n - k - 1} 0 \\
        &= 0
    \end{align}
    が言えた.
\end{proof}


\begin{mylem}[label=lem:eigen]{}
    $V \neq \{o\}$ を\underline{有限次元}ベクトル空間とし,$\Lgl (V)$ の任意の\hyperref[def:subLieAlg]{部分Lie代数} $\mathfrak{g}$ をとる.

    このとき $\forall x \in \mathfrak{g}$ が冪零ならば,ある $v \in V \setminus \{o\}$ が存在して $\forall x \in \mathfrak{g}$ に対して $x(v) = o$ が成り立つ.
\end{mylem}

\begin{proof}
    $\dim \mathfrak{g}$ に関する数学的帰納法により示す.
    $\dim \mathfrak{g} = 0$ のときは $\mathfrak{g} = \{0\}$ なので自明.

    $\dim \mathfrak{g} > 0$ とし,$\mathfrak{g}$ の\hyperref[def:subLieAlg]{部分Lie代数} $\mathfrak{h} \subsetneq \mathfrak{g}$ を任意に1つとる.
    
    \begin{description}
        \item[$\bm{\mathfrak{h} \subsetneq N_{\mathfrak{g}} (\mathfrak{h})}$ \textbf{であること}] 
        
             補題\ref{lem:ad-nilpo}より,$\ad|_{\mathfrak{h}} \colon \mathfrak{h} \lto \Lgl(\mathfrak{g})$ の\footnote{$\ad \colon \mathfrak{g} \lto \Lgl (\mathfrak{g})$ の $\mathfrak{h} \subset \mathfrak{g}$ への制限.}像の全ての元は冪零である\footnote{仮定より $\forall x \in \mathfrak{h} \subset \mathfrak{g}$ が冪零なので.}.
            ここで $\mathfrak{g},\, \mathfrak{h}$ のLieブラケットを忘れて $\mathbb{K}$ 上のベクトル空間と見做したとき,商ベクトル空間 $\mathfrak{g} / \mathfrak{h}$ を構成できる\footnote{\hyperref[def:quotient-LieAlg]{商代数の構成}においてLieブラケットを忘れることで商ベクトル空間が構成できる.この際 $\mathfrak{h}$ は $\mathfrak{g}$ の部分ベクトル空間でありさえすれば良い.}.
            このとき,$\forall x \in \mathfrak{h}$ に対して写像
            \begin{align}
                \overline{\ad|_{\mathfrak{h}}}(x) \colon \mathfrak{g} / \mathfrak{h} \lto \mathfrak{g} / \mathfrak{h},\; y + \mathfrak{h} \lmto \ad (x) (y) + \mathfrak{h}
            \end{align}
            がwell-definedな線型写像であることを示す.実際 $y + \mathfrak{h} = y' + \mathfrak{h} \IFF y' - y \in \mathfrak{h}$ ならば
            \begin{align}
                \overline{\ad|_{\mathfrak{h}}}(x) (y' + \mathfrak{h}) &= \ad (x) (y')  + \mathfrak{h} \\
                &= \bigl(\ad (x) (y) + \ad (x) (y' - y)\bigr) + \mathfrak{h} \\
                &= \bigl(\ad (x) (y) + \comm{x}{y' - y}\bigr) + \mathfrak{h} \\
                &= \ad (x) (y) + \mathfrak{h} \\
                &= \overline{\ad|_{\mathfrak{h}}}(x) (y + \mathfrak{h})
            \end{align}
            が成り立つので $\overline{\ad|_{\mathfrak{h}}}(x)$ はwell-definedであり,かつ $\ad (x)$ が線型写像なので線型写像である.
            以上の考察から,写像 $\overline{\ad|_{\mathfrak{h}}} \colon \mathfrak{h} \lto \Lgl (\mathfrak{g} / \mathfrak{h}),\; x \lmto \overline{\ad|_{\mathfrak{h}}} (x)$ はwell-definedなLie代数の準同型で,$\Im \overline{\ad|_{\mathfrak{h}}}$ (もちろんこれは $\Lgl (\mathfrak{g}/\mathfrak{h})$ の部分Lie代数である)の元は全て冪零である.

             さて,$\mathfrak{h} \neq \mathfrak{g}$ より $\dim \mathfrak{h} < \dim \mathfrak{g}$ であるから,帰納法の仮定よりある $v + \mathfrak{h} \in (\mathfrak{g} / \mathfrak{h}) \setminus \{\mathfrak{h}\}$ が存在して,$\forall \overline{\ad|_{\mathfrak{h}}} (\textcolor{blue}{x}) \in \Im \overline{\ad|_{\mathfrak{h}}}$ に対して $\overline{\ad|_{\mathfrak{h}}} (\textcolor{blue}{x}) (v + \mathfrak{h}) = \mathfrak{h}$ が成り立つ.
            i.e. ある $v \in \mathfrak{g} \setminus \mathfrak{h}$ が存在して $\forall \textcolor{blue}{x} \in \mathfrak{h}$ に対して $\ad|_{\mathfrak{h}} (\textcolor{blue}{x}) (v) = \comm{\textcolor{blue}{x}}{v} \in \mathfrak{h}$ が成り立つ.i.e. $\mathfrak{h}$ は $\mathfrak{h}$ の\hyperref[def:normalizer-LieAlg]{正規化代数} $N_{\mathfrak{g}} (\mathfrak{h})$ の真部分集合である\footnote{$\exists v \in N_{\mathfrak{g}}(\mathfrak{h}) \setminus \mathfrak{h}$ なので}.
    \end{description}

    $\mathfrak{h}$ を,$\mathfrak{h} \subsetneq \mathfrak{g}$ を充たす極大\footnote{$\mathfrak{h} \subsetneq \mathfrak{h}'$ でかつ $\mathfrak{h}' \subsetneq \mathfrak{g}$ を充たす $\mathfrak{h}'$ が存在しない.}部分Lie代数とする.$\mathfrak{h}$ の極大性と先ほど証明した事実から $N_{\mathfrak{g}} (\mathfrak{h}) = \mathfrak{g}$ でなくてはならない.従って\hyperref[def:normalizer-LieAlg]{正規化代数の定義の脚注}から $\mathfrak{h}$ は $\mathfrak{g}$ の\hyperref[def:ideal-LieAlg]{イデアル}であり,商代数 $\mathfrak{g} / \mathfrak{h}$ がwell-definedである.
    もし $\dim \mathfrak{g} / \mathfrak{h} > 1$ だとすると $\mathfrak{g} / \mathfrak{h}$ の1次元部分Lie代数 $\mathbb{K} (x + \mathfrak{h})\; (\WHERE x \in \mathfrak{g} \setminus \mathfrak{h})$ の,標準的射影による逆像が $\mathfrak{h}$ を真部分集合にもつ $\mathfrak{g}$ の真部分代数ということになり,$\mathfrak{h}$ の極大性に矛盾する.故に背理法から $\dim \mathfrak{g} / \mathfrak{h} = 1$ であり,$\mathfrak{g} = \mathfrak{h} + \mathbb{K} z\; (\WHERE z \in \mathfrak{g} \setminus \mathfrak{h})$ だと分かった.

    ここで $\mathfrak{g}$ の部分ベクトル空間 $W \coloneqq \bigl\{\, v \in V \bigm| \forall y \in \mathfrak{h},\; y(v) = o \,\bigr\}$ を考える.$\dim \mathfrak{h} = \dim \mathfrak{g} - 1 < \dim \mathfrak{g}$ なので,帰納法の仮定より $W \neq \{o\}$ だと分かる\footnote{仮定より $\mathfrak{h}$ の全ての元は冪零.}.
    % あとはある $v \in W \setminus \{o\}$ が存在して $\forall z \in \mathfrak{g} \setminus \mathfrak{h}$ に対しても $z(v) = o$ を充たすことを示せば証明が完了する.
    $\mathfrak{h}$ はイデアルなので,$\forall \textcolor{blue}{x} \in \mathfrak{g},\, \forall y \in \mathfrak{h},\, \forall w \in W$ に対して $y (\textcolor{blue}{x}(w)) = \textcolor{blue}{x} (y(w)) - \comm{\textcolor{blue}{x}}{y}(w) = o$ が成り立つ.i.e. $\forall w \in W$ に対して $\forall \textcolor{blue}{x} \in \mathfrak{g},\; \textcolor{blue}{x} (w) \in W$ が言える($W$ は $\textcolor{blue}{x}$ 不変).
    故に $\forall z \in \mathfrak{g} \setminus \mathfrak{h}$ の $W$ における固有ベクトル $v \in W \setminus \{o\}$ が存在する\footnote{$\forall z \in \mathfrak{g}/\mathfrak{h}$ はスカラー倍の違いしかないので,固有ベクトルは共通のものを取れる.}.$z$ は冪零なのでその固有値は $0$ であり,$z (v) = o$ が成り立つ.


\end{proof}


\begin{mytheo}[label=thm:Engel]{Engelの定理}
    $\mathfrak{g}$ を\underline{有限次元}Lie代数とする.
    $\forall x \in \mathfrak{g}$ に対して $\ad x$ が冪零ならば,$\mathfrak{g}$ は\hyperref[def:nilpotent-LieAlg]{冪零Lie代数}である.
\end{mytheo}

\begin{marker}
    命題\ref{prop:nilpo-basic}-(5) と併せて以下を得る:

    \underline{有限次元}Lie代数 $\mathfrak{g}$ が\hyperref[def:nilpotent-LieAlg]{冪零Lie代数} $\IFF$ $\forall x \in \mathfrak{g}$ に対して $\ad x$ が冪零
\end{marker}

\begin{proof}
    $\dim \mathfrak{g}$ に関する数学的帰納法により示す.
    $\dim \mathfrak{g} = 0,\, 1$ のときは自明.

    $\dim \mathfrak{g} > 1$ とする.
    仮定より,$\Lgl (\mathfrak{g})$ の部分Lie代数 $\Im \ad$ に対して補題\ref{lem:eigen}を使うことができる.すなわちある $v \in \mathfrak{g} \setminus \{o\}$ が存在して,$\forall x \in \mathfrak{g}$ に対して $\ad(x)(v) = \comm{x}{v} = o$ を充たす.i.e. $\mathfrak{g}$ の\hyperref[def:center-LieAlg]{中心} $Z(\mathfrak{g})$ は非零である.
    このとき $\dim \mathfrak{g} / Z(\mathfrak{g}) < \dim \mathfrak{g}$ で,かつ $\forall x + Z(\mathfrak{g}) \in \mathfrak{g} / Z(\mathfrak{g})$ に対して $\ad (x + Z(\mathfrak{g}))$ は冪零である.故に帰納法の仮定から $\mathfrak{g} / Z(\mathfrak{g})$ は\hyperref[def:nilpotent-LieAlg]{冪零Lie代数}である.従って命題\ref{prop:nilpo-basic}-(3)より $\mathfrak{g}$ は冪零Lie代数である.
\end{proof}

\section{Lie群とLie代数の関係}

この節は~\cite[第5章]{Kobayashi2005Lie}による.

\end{document}