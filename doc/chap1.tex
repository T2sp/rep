\documentclass[rep_main]{subfiles}

\begin{document}

% \setcounter{}{}
\chapter{Lie群とLie代数}

本資料ではベクトル空間を英大文字で表記し,係数体をblackboardbold\footnote{\LaTeX コマンドは\texttt{\textbackslash mathbb}}で表記する(e.g. 体 $\mathbb{K}$ 上のベクトル空間 $L$).
本章に限ってはベクトルを $x \in L$ のように英小文字で表記し,係数体の元は $\lambda \in \mathbb{K}$ のようにギリシャ文字で表記する.零ベクトルは $o \in L$ と書き\footnote{$0$ の濫用を回避するための苦肉の策です... 普通に不便なので次章以降では零ベクトルも $0$ と書きます.},$0 \in \mathbb{K}$ を係数体の加法単位元,$1 \in \mathbb{K}$ を係数体の乗法単位元とする.
ベクトル空間の加法を $+$ と書き,スカラー乗法は $\lambda x$ のように係数を左に書く.

\section{公理的Lie代数}

この節では $\mathbb{K}$ を任意の体とする.

\begin{myaxiom}[label=ax:LieAlg, breakable]{Lie代数の公理}
    体 $\mathbb{K}$ 上のベクトル空間 $L$ の上に二項演算\footnote{ベクトル空間に備わっている加法とスカラー乗法の他に,追加で $\comm{\;}{\,}$ が定義されているという状況である.この付加的な二項演算はしばしば\textbf{括弧積} (bracket) とか\textbf{交換子} (commutator) とか\textbf{Lieブラケット} (Lie bracket) とか呼ばれる.}
    \begin{align}
        \comm{\;}{\,} \colon L \times L \lto L,\; (x,\, y) \lmto \comm{x}{y}
    \end{align}
    が定義されていて,かつ以下の条件を充たすとき,$L$ は\textbf{Lie代数} (Lie algebra) と呼ばれる:
    \begin{description}
        \item[\textbf{(L-1)}] $\comm{\;}{\,}$ は双線型写像である.i.e. $\forall x,\, x_i,\, y,\, y_i \in L,\; \forall \lambda_i,\, \mu_i \in \mathbb{K}\; (i = 1,\, 2)$ に対して
        \begin{align}
            \comm{\lambda_1 x_1 + \lambda_2 x_2}{y} &= \lambda_1 \comm{x_1}{y} + \lambda_2 \comm{x_2}{y}, \\
            \comm{x}{\mu_1 y_1 + \mu_2 y_2} &= \mu_1 \comm{x}{y_1} + \mu_2 \comm{x}{y_2}
        \end{align}
        が成り立つ.
        \item[\textbf{(L-2)}] $\forall x \in L$ に対して
        \begin{align}
            \comm{x}{x} = o
        \end{align}
        が成り立つ.
        \item[\textbf{(L-3)}] $\forall x,\, y,\, z \in L$ に対して
        \begin{align}
            \label{eq:Jacobi-identity}
            \comm{x}{\comm{y}{z}} + \comm{y}{\comm{z}{x}} + \comm{z}{\comm{x}{y}} = o
        \end{align}
        が成り立つ\footnote{\underline{結合律ではない!}}(\textbf{Jacobi}恒等式).
    \end{description}
    
\end{myaxiom}

\hyperref[ax:LieAlg]{公理\textsf{\textbf{(L-1)}}, \textsf{\textbf{(L-2)}}}から
\begin{align}
    o = \comm{x+y}{x+y} = \comm{x}{x} + \comm{x}{y} + \comm{y}{x} + \comm{y}{y} = \comm{x}{y} + \comm{y}{x}
\end{align}
が従う.i.e. $\comm{x}{y}$ は反交換 (anticommute) する:
\begin{description}
    \item[\textbf{(L'-2)}] $\forall x,\, y \in L$ に対して
    \begin{align}
        \comm{x}{y} = -\comm{y}{x}
    \end{align}
    が成り立つ.
\end{description}
逆に\textsf{\textbf{(L'-2)}}を仮定すると
\begin{align}
    o = \comm{x}{x} + \comm{x}{x} = (1 + 1)\comm{x}{x}
\end{align}
が成り立つ\footnote{2つ目の等号ではスカラー乗法の分配律(ベクトル空間の公理である)を使った.}ので,\underline{体 $\mathbb{K}$ において $1 + 1 \neq 0$ ならば} $\comm{x}{x} = o$ が言える.
i.e. $\character \mathbb{K} \neq 2$ ならば\footnote{体 $\mathbb{K}$ の\textbf{標数} (characteristic) を $\character \mathbb{K}$ と書いた.} \textsf{\textbf{(L'-2)}} と \textsf{\textbf{(L-2)}} は同値である.

\begin{myexample}[label=def:gl-alg]{一般線形代数 $\Lgl (V)$}
     $V$ を体 $\mathbb{K}$ 上のベクトル空間とする.$V$ から $V$ への線型写像全体が成す集合を $\End V$ と書く\footnote{\textbf{自己準同型} (endomorphism) の略である.}.
    $\End V$ の加法とスカラー乗法をそれぞれ
    \begin{align}
        + \colon \End V \times \End V &\lto \End V,\; (f,\, g) \lmto \bigl( v \mapsto f(v) + g(v) \bigr) \\
        \cdot\; \colon \mathbb{K} \times \End V &\lto \End V,\; (\lambda,\, f) \lmto \bigl( v \mapsto \lambda f(v) \bigr)
    \end{align}
    として定義すると,組 $(\End V,\, +,\, \cdot\; )$ は体 $\mathbb{K}$ 上のベクトル空間になる.以降では常に $\End V$ をこの方法でベクトル空間と見做す.

     $\End V$ の上のLieブラケットを
    \begin{align}
        \label{eq:Liebracket-gl}
        \comm{\;}{\,} \colon \End V \times \End V \lto \End V,\; (f,\, g) \lmto fg - gf
    \end{align}
    と定義する.ただし右辺の $fg$ は写像の合成 $f \circ g$ の略記である.このとき組 $(\End V,\, +,\, \cdot\;,\, \comm{\;}{\,})$ が\hyperref[ax:LieAlg]{Lie代数の公理}を充たすことを確認しよう:
    \begin{description}
        \item[\textbf{(L-1)}] $\forall v \in V$ を1つとる.定義に従ってとても丁寧に計算すると
        \begin{align}
            \comm{\lambda_1 f_1 + \lambda_2 f_2}{g}(v) 
            &= \bigl( (\lambda_1 f_1 + \lambda_2 f_2) g - g (\lambda_1 f_1 + \lambda_2 f_2) \bigr)(v) \\
            &= (\lambda_1 f_1 + \lambda_2 f_2) \bigl( g(v) \bigr) - g \bigl((\lambda_1 f_1 + \lambda_2 f_2)(v)\bigr) \\
            &= (\lambda_1 f_1) \bigl(g (v) \bigr) + (\lambda_2 f_2) \bigl( g(v) \bigr) - g \bigl( (\lambda_1 f_1)(v) + (\lambda_2 f_2)(v)  \bigr) \\
            &= \lambda_1 f_1 \bigl(g (v) \bigr) + \lambda_2 f_2 \bigl( g(v) \bigr) - \lambda_1 g \bigl( f_1(v) \bigr) - \lambda_2 g \bigl( f_2(v) \bigr) \\
            &= \lambda_1 \Bigl( f_1 \bigl(g (v) \bigr) - g \bigl( f_1(v) \bigr) \Bigr) + \lambda_2 \Bigl(f_2 \bigl( g(v) \bigr) - g \bigl( f_2(v) \bigr) \Bigr) \\
            &= \lambda_1 \comm{f_1}{g} (v) + \lambda_2 \comm{f_2}{g} (v) \\
            &= \bigl( \lambda_1 \comm{f_1}{g} + \lambda_2 \comm{f_2}{g}  \bigr) (v)
        \end{align}
        となる.ただし4つ目の等号で $g \in \End V$ が線型写像であることを使った.全く同様にして
        \begin{align}
            \comm{f}{\mu_1 g_1 + \mu_2 g_2}(v) = \mu_1 \comm{f}{g_1} + \mu_2 \comm{f}{g_2}
        \end{align}
        を示すこともできる.
        \item[\textbf{(L-2)}] 明らかに $\comm{f}{f} = ff - ff = o$ なのでよい.
        \item[\textbf{(L-3)}] $\comm{\;}{\,}$ の双線型(\textsf{\textbf{(L-1)}})から
        \begin{align}
            &\comm{f}{\comm{g}{h}} + \comm{g}{\comm{h}{f}} + \comm{h}{\comm{f}{g}} \\
            &= \comm{f}{gh} - \comm{f}{hg} + \comm{g}{hf} - \comm{g}{fh} + \comm{h}{fg} - \comm{h}{gf} \\
            &= fgh - ghf - fhg + hgf + ghf - hfg - gfh + fhg + hfg - fgh - hgf + gfh \\
            &= o.
        \end{align}
    \end{description}
    このLie代数 $(\End V,\, +,\, \cdot\;,\, \comm{\;}{\,})$ は\textbf{一般線形代数} (general linear algebra) と呼ばれ,記号として $\bm{\Lgl (V)}$ と書かれる.

     $\dim V \eqqcolon n < \infty$ のとき,$\End V$ は $n \times n$ $\mathbb{K}$-行列全体が成す $\mathbb{K}$ ベクトル空間 $\Mat{n}{\mathbb{K}}$ と同型である\footnote{$V$ の基底 $e_1,\, \dots,\, e_n$ を1つ固定する.このとき同型写像 $\varphi \colon V \lto \mathbb{K}^n,\; v^\mu e_\mu \lmto (v^\mu)_{1 \le \mu \le n}$ を使って定義される線型写像 $\phi \colon \End V \lto \Mat{n}{\mathbb{K}},\; f \lmto \varphi \circ f \circ \varphi^{-1}$ が所望の同型写像である.}.
    $\Mat{n}{\mathbb{K}}$ をLieブラケット $\comm{x}{y} \coloneqq xy - yx$ によってLie代数と見做す\footnote{右辺の $xy$ は行列の積である.}ときは,この同型を意識して $\lgl{n}{\mathbb{K}}$ と書く.さて,$\lgl{n}{\mathbb{K}}$ の標準的な基底は所謂\textbf{行列単位}
    \begin{align}
        e_{ij} \coloneqq \bigl[\, \delta_{i\mu} \delta_{j\nu} \bigr]_{1 \le \mu,\, \nu \le n} 
        = 
        \begin{blockarray}{cccccc}
            & & j & & \\
            \begin{block}{(ccccc)c}
                0 & \cdots & 0 & \cdots & 0 & \\
                \vdots & \ddots & \vdots & \ddots & \vdots & \\
                0 & \cdots & 1 & \cdots & 0 & i \\
                \vdots & \ddots & \vdots & \ddots & \vdots & \\
                0 & \cdots & 0 & \cdots & 0 & \\
            \end{block}
        \end{blockarray}
    \end{align}
    である.Einsteinの規約を使って $e_{ij} e_{kl} = \bigl[\, \delta_{i\mu} \delta_{j\lambda} \delta_{k \lambda} \delta_{l \nu} \bigr]_{1 \le \mu,\, \nu \le n} = \delta_{jk} \bigl[\, \delta_{i\mu} \delta_{l\nu} \bigr]_{1 \le \mu,\, \nu \le n} = \delta_{jk} E_{il}$ と計算できるので,$\lgl{n}{\mathbb{K}}$ のLieブラケットは
    \begin{align}
        \comm{e_{ij}}{e_{kl}} = \delta_{jk} e_{il} - \delta_{li} e_{kj}
    \end{align}
    によって完全に決まる.
\end{myexample}

\subsection{線型Lie群}

\begin{mydef}[label=def:subLieAlg]{部分Lie代数}
    \hyperref[ax:LieAlg]{Lie代数} $L$ の\underline{部分ベクトル空間} $M \subset L$ が\textbf{部分Lie代数}であるとは,$M$ がLieブラケットについても閉じていることを言う.
    i.e. $\forall x,\, y \in M,\; \forall \lambda \in \mathbb{K}$ に対して
    \begin{align}
        x + y,\; \lambda x,\; \comm{x}{y} \in M
    \end{align}
    が成り立つこと.
\end{mydef}

この小節では以降,$V$ を体 $\mathbb{K}$ 上の有限次元ベクトル空間とする.

\begin{mydef}[label=def:linearLieAlg]{線型Lie群}
    \hyperref[def:gl-alg]{一般線形代数} $\Lgl (V)$ の\hyperref[def:subLieAlg]{部分Lie代数}のことを\textbf{線型Lie群} (linear Lie algebra) と呼ぶ.
\end{mydef}

線型Lie群として有名なものは\textbf{古典代数} (classical algebra) である.これは $A_l,\, B_l,\, C_l,\, D_l$ と呼ばれる4つの無限系列からなる.
以下,$\character \mathbb{K} \neq 2$ とする.

\begin{myexample}[label=def:typeA]{線型Lie群:$A_l$ 型}
     $\dim V = l+1$ とする.\textbf{特殊線形代数} $\Lsl (V)$(または $\Lsl (l+1,\, \mathbb{K})$)は次のように定義される:
    \begin{align}
        \Lsl (V) \coloneqq \bigl\{\, x \in \Lgl (V) \bigm| \Tr (x) = 0 \,\bigr\} 
    \end{align}
    $\Lsl (V)$ が本当に\hyperref[def:linearLieAlg]{線型Lie群}かどうか確認しよう.
    \begin{proof}
        $\forall x,\, y \in \Lsl (V)$ をとる.トレースの線形性および $\Tr (xy) = \Tr (yx)$ から
        \begin{align}
            \Tr (x + y) &= \Tr (x) + \Tr (y) = 0 + 0 = 0, \\
            \Tr (\lambda x) &= \lambda \Tr (x) = \lambda 0 = 0, \\
            \Tr (\comm{x}{y}) &= \Tr (xy) - \Tr (yx) = \Tr (xy) - \Tr (xy) = 0
        \end{align}
        が言えるので,$x + y,\, \lambda x,\, \comm{x}{y} \in \Lsl (V)$ が言えた.
    \end{proof}
     \exref{def:gl-alg}で使った $\lgl{l+1}{\mathbb{K}}$ の標準的な基底で $\forall x \in \lsl{l+1}{\mathbb{K}}$ を $x = x^{ij} e_{ij}$ と展開すると,トレースレスであることから
    \begin{align}
        h_{ij} \coloneqq 
        \begin{cases}
            e_{ij} &i \neq j,\; 1 \le i,\, j \le l+1 \\
            e_{ii} - e_{i+1,\, i+1} &1 \le i=j \le l
        \end{cases}
    \end{align}
    の $(l+1)^2 - 1$ 個の行列が $\lsl{l+1}{\mathbb{K}}$ の基底を成すことがわかる.従って $\dim \lsl{l+1}{\mathbb{K}} = (l+1)^2 -1$ である.
\end{myexample}

\begin{myexample}[label=def:typeB]{線型Lie群:$B_l$ 型}
    $\dim V = 2l + 1$ とする.行列
    \begin{align}
        s \coloneqq \mqty[
            1 & 0 & 0 \\
            0 & 0 & \unity_l \\
            0 & \unity_l & 0
        ] \in \Mat{2l+1}{\mathbb{K}}
    \end{align}
    により定まる $V$ 上の非退化かつ対称な双線型形式を $f$ と書く\footnote{$f \colon V \times V \lto \mathbb{K},\; (v,\, w) \lmto v^{\mathsf{T}} s w$ のこと.}.
    このとき,\textbf{直交代数} (orthogonal algebra) $\Lo (V)$(または $\lo{2l+1,\, \mathbb{K}}$)が以下のように定義される:
    \begin{align}
        \Lo (V) \coloneqq \bigl\{\, x \in \Lgl (V) \bigm| f \bigl( x(v),\, w \bigr) = -f \bigl( v,\, x(w) \bigr),\; \forall v,\, w \in V \,\bigr\} 
    \end{align}
    $\Lo (V)$ が本当に\hyperref[def:linearLieAlg]{線型Lie群}かどうか確認しよう.
    \begin{proof}
        $\forall x,\, y \in \Lo (V)$ をとる.$f$ の双線型性から $\forall v,\, w \in V$ に対して
        \begin{align}
            f \bigl( (x+y)(v),\, w \bigr) 
            &= f \bigl( x(v),\, w \bigr) + f \bigl( y(v),\, w \bigr) \\
            &= - f \bigl( v,\, x(w) \bigr) - f \bigl( v,\, y(w) \bigr) \\
            &= - f \bigl( v,\, x(w) + y(w) \bigr) \\
            &= - f \bigl( v,\, (x+y)(w) \bigr) \\
            f \bigl( (\lambda x)(v),\, w \bigr)  
            &= \lambda f \bigl( x(v),\, w \bigr) \\
            &= - \lambda f \bigl( v,\, x(w) \bigr) \\
            &= - f \bigl( v,\, \lambda x(w) \bigr) \\
            &= - f \bigl( v,\, (\lambda x)(w) \bigr) \\
            f \bigl(\comm{x}{y}(v),\, w \bigr) 
            &= f \bigl(x(y(v)),\, w \bigr) - f \bigl(y(x(v)),\, w \bigr) \\
            &= - f \bigl(y(v),\, x(w) \bigr) + f \bigl(x(v),\, y(w) \bigr) \\
            &= f \bigl( v,\, y(x(w)) \bigr) - f \bigl( v,\, x(y(w)) \bigr) \\
            &= - f \bigl( v,\, \comm{x}{y}(w) \bigr) 
        \end{align}
        が言えるので,$x + y,\, \lambda x,\, \comm{x}{y} \in \Lo (V)$ が言えた.
    \end{proof}
\end{myexample}



\end{document}