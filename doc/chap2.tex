\documentclass[rep_main]{subfiles}

\begin{document}

\setcounter{chapter}{1}

\chapter{半単純Lie代数}

この章以降,$\mathbb{K}$-ベクトル空間 $V$ の零ベクトルを $0 \in V$ と書き,零ベクトル空間 $\{0\}$ のことも $0$ と表記する\footnote{記号の濫用だが,広く普及している慣習である.}.
この章において,特に断らない限り体 $\mathbb{K}$ は代数閉体\footnote{つまり,定数でない任意の1変数多項式 $f(x) \in \mathbb{K}[x]$ に対してある $\alpha \in \mathbb{K}$ が存在して $f(\alpha) = 0$ を充たす.}で,かつ $\character \mathbb{K} = 0$ であるとする.
また,\hyperref[ax:LieAlg]{Lie代数} $\mathfrak{g}$ は常に\underline{有限次元}であるとする.
\section{Lieの定理・Cartanの判定条件}

\subsection{Lieの定理}

\begin{mytheo}[label=thm:eigen-Lie]{}
	$V$ を\underline{有限次元} $\mathbb{K}$-ベクトル空間とし,$\Lgl (V)$ の部分Lie代数 $\mathfrak{g} \subset \Lgl (V)$ が\hyperref[def:solvable-LieAlg]{可解}であるとする.

	このとき $V \neq 0$ ならば,$\forall x \in \Lgl (V)$ は共通の固有ベクトルを持つ.
\end{mytheo}

\begin{mycol}[label=thm:Lie]{Lieの定理}
	$V$ を\underline{有限次元} $\mathbb{K}$-ベクトル空間とし,$\Lgl (V)$ の部分Lie代数 $\mathfrak{g} \subset \Lgl (V)$ が\hyperref[def:solvable-LieAlg]{可解}であるとする.

	このとき $\forall x \in \mathfrak{g}$ は $V$ のある\underline{共通の}\hyperref[def:flag]{旗を安定化}する.i.e. $\forall x \in \mathfrak{g}$ の表現行列を同時に上三角行列にするような $V$ の基底が存在する.
\end{mycol}

\begin{mycol}[label=col:Lie-1]{}
	$\mathfrak{g}$ を\hyperref[def:solvable-LieAlg]{可解Lie代数}とする.このとき $\mathfrak{g}$ の\hyperref[def:ideal-LieAlg]{イデアル}の上昇列
	\begin{align}
		0 = \mathfrak{g}_0 \subset \mathfrak{g}_1 \subset \cdots \subset \mathfrak{g}_n = \mathfrak{g}
	\end{align}
	であって,$0 \le \forall i \le n$ に対して $\dim \mathfrak{g}_i = i$ を充たすようなものが存在する.
\end{mycol}


\begin{mycol}[label=col:Lie-2]{}
	$\mathfrak{g}$ を\hyperref[def:solvable-LieAlg]{可解Lie代数}とする.
	このとき
	\begin{align}
		x \in \comm{\mathfrak{g}}{\mathfrak{g}} \IMP \ad(x) \in \Lgl(\mathfrak{g})\; \text{が冪零}
	\end{align}
	が成り立つ.特に,$\comm{\mathfrak{g}}{\mathfrak{g}}$ は\hyperref[def:nilpotent-LieAlg]{冪零Lie代数}である.
\end{mycol}

\subsection{Jordan-Chevalley分解}

この小節において体 $\mathbb{K}$ は代数閉体で,かつ標数は任意とする.

\begin{mydef}[label=def:semisimple-end]{線型変換の半単純性}
	$V$ を\underline{有限次元} $\mathbb{K}$-ベクトル空間とし,$x \in \End V$ とする.
	$x$ が\textbf{半単純} (semisimple) であるとは,$x$ の\hyperref[def:minimal-poly]{最小多項式}が $\mathbb{K}$ 上で重根を持たないことを言う\footnote{$x$ が対角化可能であることと同値である.}.
\end{mydef}


\begin{myprop}[label=prop:Jordan-Chevalley]{Jordan-Chevalley分解}
	$V$ を\underline{有限次元} $\mathbb{K}$-ベクトル空間とし,$x \in \End V$ を与える.
	このとき以下が成り立つ:
	\begin{enumerate}
		\item \hyperref[def:semisimple-end]{半単純}な $x_s \in \End V$ と冪零な $x_n \in \End V$ が一意的に存在して,
		\begin{align}
			x = x_s + x_n
		\end{align}
		と書ける.
		\item 定数項を持たない $p(t),\, q(t) \in \mathbb{K}[t]$ であって,$x_s = p(x),\; x_n = q(x)$ を充たすものが存在する.特に $\comm{x}{x_s} = \comm{x}{x_n} = \comm{x_s}{x_n} = 0$ が成り立つ.
		\item 部分ベクトル空間 $A \subset B \subset V$ があって $x|_B \colon B \lto A$ であるとき,$x_s|_B,\, x_n|_B$ の値域もまた $A$ に収まる.
	\end{enumerate}
\end{myprop}

\begin{mylem}[label=lem:JC-1]{}
	$V$ を\underline{有限次元} $\mathbb{K}$-ベクトル空間とし,$x \in \End V$ とその\hyperref[prop:Jordan-Chevalley]{Jordan-Chevalley分解} $x = x_s + x_n$ を与える.

	このとき $\ad (x) \in \End (\Lgl (V))$ のJordan-Chevalley分解 $\ad(x) = \ad(x)_s + \ad(x)_n$ は $\ad(x)_s = \ad(x_s),\; \ad(x)_n = \ad(x_n)$ を充たす.
\end{mylem}


\begin{mylem}[label=lem:JC-2]{}
	$\mathfrak{U}$ を\underline{有限次元} $\mathbb{K}$-\hyperref[ax:Alg]{代数}とする.このとき\hyperref[prop:Der-LieAlg]{微分代数} $\Der \mathfrak{U} \subset \End \mathfrak{U}$ の任意の元 $x \in \Der \mathfrak{U}$ の\hyperref[prop:Jordan-Chevalley]{Jordan-Chevalley分解} $x = x_s + x_n$ に対して $x_s,\, x_n \in \Der \mathfrak{U}$ が成り立つ.
\end{mylem}

\subsection{Cartanの判定条件}

\begin{mytheo}[label=thm:Cartan-crit]{Cartanの判定条件}
	$V$ を\underline{有限次元} $\mathbb{K}$-ベクトル空間とし,$\Lgl (V)$ の部分Lie代数 $\mathfrak{g} \subset \Lgl (V)$ を与える.

	このとき以下の2つは同値である:
	\begin{enumerate}
		\item $\mathfrak{g}$ が\hyperref[def:solvable-LieAlg]{可解}
		\item $\forall x \in \comm{\mathfrak{g}}{\mathfrak{g}},\; \forall y \in \mathfrak{g}$ に対して $\Tr (x \circ y) = 0$ が成り立つ
	\end{enumerate}
	
	% ならば,$\mathfrak{g}$ は\hyperref[def:solvable-LieAlg]{可解}である.
\end{mytheo}

\begin{mycol}[label=col:Cartan-crit]{}
	Lie代数 $\mathfrak{g}$ を与える.
	
	このとき $\forall x \in \comm{\mathfrak{g}}{\mathfrak{g}},\; \forall y \in \mathfrak{g}$ に対して $\Tr \bigl(\ad (x) \circ \ad (y)\bigr) = 0$ が成り立つならば,$\mathfrak{g}$ は\hyperref[def:solvable-LieAlg]{可解}である.
\end{mycol}


\section{Killing形式}

\subsection{半単純性の判定条件}


\begin{mydef}[label=def:Killing-form]{Killing形式}
	体 $\mathbb{K}$ 上のLie代数 $\mathfrak{g}$ の上の対称な双線型形式
	\begin{align}
		\kappa \colon \mathfrak{g} \times \mathfrak{g} \lto \mathbb{K},\; (x,\, y) \lmto \Tr \bigl( \ad(x) \circ \ad(y) \bigr) 
	\end{align}
	のことを $\mathfrak{g}$ の\textbf{Killing形式} (Killing form) と呼ぶ.
\end{mydef}

\begin{mydef}[label=def:radical-bilinear]{双線型形式のradical}
	体 $\mathbb{K}$ 上のLie代数 $\mathfrak{g}$ の上の対称な双線型形式
	\begin{align}
		\beta \colon \mathfrak{g} \times \mathfrak{g} \lto \mathbb{K}
	\end{align}
	および任意の\underline{部分集合} $\textcolor{blue}{W} \subset \mathfrak{g}$ を与える.
	\begin{itemize}
		\item $\mathfrak{g}$ の\underline{部分ベクトル空間}
		\begin{align}
			\textcolor{blue}{W}^\perp \coloneqq \bigl\{\, x \in \mathfrak{g} \bigm| \forall \textcolor{blue}{w} \in \textcolor{blue}{W},\; \beta(x,\, \textcolor{blue}{w}) = 0 \,\bigr\} 
		\end{align}
		のことを $\textcolor{blue}{W}$ の $\beta$ に関する\textbf{直交補空間} (orthogonal complement) と呼ぶ.
		\item $\mathfrak{g}$ の部分ベクトル空間
		\begin{align}
			S_\beta \coloneqq \mathfrak{g}^\perp = \bigl\{\, x \in \mathfrak{g} \bigm| \forall \textcolor{blue}{y} \in \mathfrak{g},\; \beta(x,\, \textcolor{blue}{y}) = 0 \,\bigr\} 
		\end{align}
		のことを $\beta$ の\textbf{radical}と呼ぶ.
		\item $\beta$ が\textbf{非退化} (nondegenerate) であるとは,$S_\beta = 0$ であることを言う
		\footnote{$\textcolor{blue}{W}$ が $\mathfrak{g}$ の\underline{部分ベクトル空間}のとき,$\beta$ の $\textcolor{blue}{W}$ への制限 $\beta|_{\textcolor{blue}{W} \times \textcolor{blue}{W}}$ が\textbf{非退化} (nondegenerate) であるとは,$\textcolor{blue}{W} \cap \textcolor{blue}{W}^\perp = 0$ であることを言う.}
		.
	\end{itemize}
	
\end{mydef}

\begin{mytheo}[label=thm:semisimple-LieAlg-iff]{Lie代数の半単純性とKilling形式の非退化性}
	$\mathfrak{g}$ が\hyperref[def:semisimple-LieAlg]{半単純Lie代数}
	$\IFF$ $\mathfrak{g}$ の\hyperref[def:Killing-form]{Killing形式}が\hyperref[def:radical-bilinear]{非退化}
\end{mytheo}


\subsection{単純イデアル}

Lie代数 $\mathfrak{g}$ と,その\hyperref[def:ideal-LieAlg]{イデアル}の族 $\Familyset[\big]{\mathfrak{i}_i}{i \in I}$ を与える.
$\mathfrak{g}$ が $\Familyset[\big]{\mathfrak{i}_i}{i \in I}$ の\textbf{直和} (direct sum)\footnote{厳密には,命題\ref{prop:subvec-directsum}の意味で\textbf{内部直和} (internal direct sum) と呼ぶべきだと思う.} 
であるとは,\hyperref[prop:subvec-directsum]{部分ベクトル空間の内部直和}として
\begin{align}
	\mathfrak{g} = \bigoplus_{i \in I} \mathfrak{i}_i
\end{align}
が成り立つことを言う.

\begin{mytheo}[label=thm:semisimple-decomp]{半単純Lie代数の直和分解}
	$\mathfrak{g}$ を\hyperref[def:semisimple-LieAlg]{半単純Lie代数}とする.
	このとき $\mathfrak{g}$ の\hyperref[def:simple-LieAlg]{単純}イデアル $\mathfrak{i}_1,\, \dots,\, \mathfrak{i}_t$ が存在して以下を充たす:
	\begin{enumerate}
		\item \begin{align}
			\mathfrak{g} = \bigoplus_{i = 1}^t \mathfrak{i}_i
		\end{align}
		\item $\mathfrak{g}$ の任意の単純イデアルは $\mathfrak{i}_i$ のどれか1つと一致する.i.e. (1) の直和分解は一意である.
		\item $\mathfrak{i}_i$ 上の\hyperref[def:Killing-form]{Killing形式} $\kappa_{\mathfrak{i}_i}$ は $\kappa_{\mathfrak{i}_i \times \mathfrak{i}_i}$ に等しい.
	\end{enumerate}
	
\end{mytheo}

\begin{mycol}[label=col:semisimple-decomp]{}
	$\mathfrak{g}$ が\hyperref[def:semisimple-LieAlg]{半単純Lie代数}ならば以下が成り立つ:
	\begin{enumerate}
		\item $\mathfrak{g} = \comm{\mathfrak{g}}{\mathfrak{g}}$
		\item $\mathfrak{g}$ の任意の\hyperref[def:ideal-LieAlg]{イデアル}は半単純である.
		\item 任意の\hyperref[def:hom-LieAlg]{Lie代数の準同型} $f \colon \mathfrak{g} \lto \mathfrak{h}$ について,$\Im (\mathfrak{h})$ は半単純である.
		\item $\mathfrak{g}$ の任意の\hyperref[def:ideal-LieAlg]{イデアル}は $\mathfrak{g}$ の\hyperref[def:simple-LieAlg]{単純イデアル}の\hyperref[prop:subvec-directsum]{直和}である.
	\end{enumerate}
	
\end{mycol}

\subsection{内部微分}

\begin{mytheo}[label=thm:semisimple-innerdv]{}
	\hyperref[def:semisimple-LieAlg]{半単純Lie代数} $\mathfrak{g}$ に対して,
	\begin{align}
		\ad(\mathfrak{g}) = \Der \mathfrak{g} \subset \Lgl(\mathfrak{g})
	\end{align}
\end{mytheo}


\subsection{抽象Jordan分解}

$\mathfrak{g}$ を\hyperref[def:semisimple-LieAlg]{半単純Lie代数}とする.
このとき $\mathfrak{g}$ の\hyperref[def:center-LieAlg]{中心} $Z(\mathfrak{g})$ は $\mathfrak{g}$ の\hyperref[def:solvable-LieAlg]{可解}イデアルなので $0$ となる.i.e. $\ad \colon \mathfrak{g} \lto \Lgl (\mathfrak{g})$ に関して $\Ker \ad = Z(\mathfrak{g}) = 0$ が言えるので $\ad$ は単射である.
一方で定理\ref{thm:semisimple-innerdv}より $\ad (\mathfrak{g}) = \Der \mathfrak{g} \subset \Lgl (\mathfrak{g})$ なので,補題\ref{lem:JC-2}を合わせると
$\forall x \in \mathfrak{g}$ に対して $\ad(x) \in \Lgl(\mathfrak{g})$ の\hyperref[prop:Jordan-Chevalley]{Jordan-Chevalley分解}
$\ad(x) = \ad(x)_s + \ad(x)_n\;\WHERE \ad(x)_s,\, \ad(x)_n \in \ad(\mathfrak{g})$ が存在する.

\begin{mydef}[label=def:abstruct-JC]{抽象Jordan分解}
	$\forall x \in \mathfrak{g}$ の\textbf{抽象Jordan分解}とは,
	$\ad(s_x) = \ad(x)_s,\; \ad(n_x) = \ad(x)_n$ を充たす $s_x,\, n_x \in \mathfrak{g}$ のこと.
\end{mydef}

$\ad$ が単射なので,$\Lgl(\mathfrak{g})$ における通常のJordan-Chevalley分解の一意性から $s_x,\, n_x$ は一意に決まる.さらに $\ad(x) = \ad(s_x) + \ad(n_x) = \ad(s_x+n_x)$ ということなので $x = s_x + n_x \in \mathfrak{g}$ が成り立つ.
さらに命題\ref{prop:Jordan-Chevalley}-(2) より $0 = \comm{\ad(s_x)}{\ad(n_x)} = \ad(\comm{s_x}{n_x})$ なので $\comm{s_x}{n_x} = 0$ もわかる.

\section{表現の完全可約性}

\subsection{$\mathfrak{g}$-加群と表現}

この小節では $\mathbb{K}$ を\underline{任意の体}とする.
まず,環上の加群の定義を復習する:

\begin{myaxiom}[label=ax:R-module,breakable]{環上の加群の公理}
	\begin{itemize}
		\item $R$ を環とする.\textbf{左 $\bm{R}$ 加群} (left $R$-module) とは,可換群 $(M,\, +,\, 0)$ と写像\footnote{この写像 $\cdot$ は\textbf{スカラー乗法} (scalar multiplication) と呼ばれる.}
		\begin{align}
			\cdot \; \colon R \times M \to M,\; (a,\, x) \mapsto a \cdot x
		\end{align}
		の組 $(M,\, +,\,\cdot\mathrel{})$ であって, $\forall x,\, x_1,\, x_2 \in M,\; \forall a,\, b \in R$ に対して以下を充たすもののことを言う:
		\begin{description}
			\item[\textbf{(LM1)}] $a \cdot (b \cdot x) = (\textcolor{red}{ab}) \cdot x$
			\item[\textbf{(LM2)}] $(a+b) \cdot x = a \cdot x + b \cdot x$
			\item[\textbf{(LM3)}] $a \cdot (x_1 + x_2) = a \cdot x_1 + a\cdot x_2$
			\item[\textbf{(LM4)}] $1 \cdot x = x$
		\end{description}
		ただし,$1 \in R$ は\underline{環 $R$ の}乗法単位元である.
		\item $R$ を環とする.\textbf{右 $\bm{R}$ 加群} (left $R$-module) とは,可換群 $(M,\, +,\, 0)$ と写像
		\begin{align}
			\cdot \; \colon M \times R \to M,\; (x,\, a) \mapsto x \cdot a
		\end{align}
		の組 $(M,\, +,\,\cdot\mathrel{})$ であって, $\forall x,\, x_1,\, x_2 \in M,\; \forall a,\, b \in R$ に対して以下を充たすもののことを言う:
		\begin{description}
			\item[\textbf{(RM1)}] $(x \cdot b) \cdot a = x \cdot (\textcolor{red}{ba})$
			\item[\textbf{(RM2)}] $x \cdot (a+b) = x \cdot a + x \cdot b$
			\item[\textbf{(RM3)}] $(x_1 + x_2) \cdot a = x_1 \cdot a + x_2 \cdot a$
			\item[\textbf{(RM4)}] $x \cdot 1 = x$
		\end{description}
		\item $R,\, S$ を環とする.\textbf{$\bm{(R,\, S)}$ 両側加群} ($(R,\, S)$-bimodule) とは,可換群 $(M,\, +,\, 0)$ と写像
		\begin{align}
			\irm{\cdot}{L} \mathrel{} &\colon R \times M \to M,\; (a,\, x) \mapsto a \irm{\cdot}{L} x \\
			\irm{\cdot}{R} \mathrel{} &\colon M \times R \to M,\; (x,\, a) \mapsto x \irm{\cdot}{R} a
		\end{align}
		の組  $(M,\, +,\, \irm{\cdot}{L}\mathrel{},\, \irm{\cdot}{R}\mathrel{})$ であって, 
		$\forall x\in M,\; \forall a\in R,\; \forall b \in S$ に対して以下を充たすもののことを言う:
		\begin{description}
			\item[\textbf{(BM1)}] 左スカラー乗法 $\irm{\cdot}{L}$ に関して $M$ は左 $R$ 加群になる
			\item[\textbf{(BM2)}] 右スカラー乗法 $\irm{\cdot}{R}$ に関して $M$ は右 $S$ 加群になる
			\item[\textbf{(BM3)}] $(a \irm{\cdot}{L} x) \irm{\cdot}{R} b = a \irm{\cdot}{L} (x \irm{\cdot}{R} b)$
		\end{description}
	\end{itemize}
\end{myaxiom}

$R$ が\textbf{可換環}の場合,\textsf{\textbf{(LM1)}} と \textsf{\textbf{(RM1)}} が同値になるので,左 $R$ 加群と右 $R$ 加群の概念は同値になる.これを単に\textbf{$\bm{R}$ 加群} ($R$-module) と呼ぶ.

$R$ が\textbf{体}の場合,$R$ 加群のことを \textbf{$\bm{R}$-ベクトル空間}と呼ぶ.

\begin{marker}
	以下では,なんの断りもなければ $R$ 加群と言って左 $R$ 加群を意味する.
\end{marker}

$\mathfrak{g}$ を体 $\mathbb{K}$ 上の\hyperref[ax:LieAlg]{Lie代数}とする.このとき,\hyperref[ax:R-module]{環上の加群の公理}を少し修正することでLie代数 $\mathfrak{g}$ 上の加群の概念を得る:

\begin{myaxiom}[label=ax:g-module]{Lie代数上の加群}
    $\mathfrak{g}$ を体 $\mathbb{K}$ 上の\hyperref[ax:LieAlg]{Lie代数}とする.$\bm{\mathfrak{g}}$\textbf{-加群}とは,$\mathbb{K}$-ベクトル空間 $(V,\, +,\, \cdot\;)$ と写像
    \begin{align}
        \btr \colon \mathfrak{g} \times V \lto V,\; (x,\, v) \lmto x \btr v
    \end{align}
    の4つ組 $(V,\, +,\, \cdot\;,\, \btr)$ であって,$\forall x,\, x_1,\, x_2 \in \mathfrak{g},\; \forall v,\, v_1,\, v_2 \in V,\; \forall \lambda,\, \mu \in \mathbb{K}$ に対して以下を充たすもののことを言う:
    \begin{description}
        \item[\textbf{(M1)}] $(\lambda \cdot x_1 + \mu \cdot x_2) \btr v = \lambda \cdot (x_1 \btr v) + \mu \cdot (x_2 \btr v)$
        \item[\textbf{(M2)}] $x \btr (\lambda \cdot v_1 + \mu \cdot v_2) = \lambda \cdot (x \btr v_1) + \mu \cdot (x \btr v_2)$
        \item[\textbf{(M3)}] $\comm{x}{y} \btr v = x \btr (y \btr v) - y \btr (x \btr v)$
    \end{description}
    \tcblower
    同値な定義として,\hyperref[def:rep-LieAlg]{Lie代数の表現}
    \begin{align}
        \phi \colon \mathfrak{g} \lto \Lgl (V),\; x \lmto \bigl( v \mapsto \phi(x)(v) \bigr) 
    \end{align}
    について
    \begin{align}
        \btr \colon \mathfrak{g} \times V \lto V,\; (x,\, v) \lmto \phi(x)(v)
    \end{align}
    とおくことで得られる4つ組 $(V,\, +,\, \cdot\;,\, \btr)$ のことである\footnote{\exref{def:gl-alg}より $\Lgl (V)$ のLieブラケットは交換子だったので $\comm{x}{y} \btr \mhyphen = \phi(\comm{x}{y}) = \comm{\phi(x)}{\phi(y)} = \phi(x) \circ \phi(y) - \phi(y) \circ \phi(x) = x \btr (y\btr \mhyphen) - y \btr (x \btr \mhyphen)$ となる.}.
\end{myaxiom}

\begin{marker}
    $\mathfrak{g}$-加群に備わっている3つの演算(加法,スカラー乗法,左作用)をいちいち全て明記するのは面倒なので $(V,\, +,\, \cdot\;,\, \btr)$ のことを「$\bm{\mathfrak{g}}$\textbf{-加群} $\bm{V}$」と略記する.この略記において,今まで通りスカラー乗法 $\cdot$ は省略して $\lambda v$ の様に書き,左作用はなんの断りもなく $x \btr v$ の様に書くことにする.
\end{marker}

全く同様に\hyperref[def:Alg]{代数}上の加群,\hyperref[def:Alg]{結合代数}上の加群を定義することもできるが,本章では以降 $\mathfrak{g}$-加群と言ったら\hyperref[ax:g-module]{Lie代数上の加群}を指すことにする.
\hyperref[def:rep-LieAlg]{Lie代数の表現}を考えることは $\mathfrak{g}$-加群を考えることと同値なのである.

\begin{mydef}[label=def:g-module-hom]{$\mathfrak{g}$-加群の準同型}
    $\mathfrak{g}$ を\hyperref[ax:LieAlg]{Lie代数},$(V,\, +,\, \cdot\;,\, \btr_1),\, (W,\, +,\, \cdot\;,\, \btr_2)$ を \hyperref[ax:g-module]{ $\mathfrak{g}$-加群}とする.
    \begin{itemize}
		\item 線型写像 $f \colon V \lto W$ が $\bm{\mathfrak{g}}$\textbf{-加群の準同型} (homomorphism of $\mathfrak{g}$-module)
		\footnote{
			\textbf{同変写像} (equivalent map) と言うこともある.
			\textbf{絡作用素} (intertwining operator),\textbf{インタートウィナー} (intertwiner) と言う場合もあるが,そこまで普及していない気がする.
		} であるとは,$\forall x \in \mathfrak{g},\; \forall v \in V$ に対して
		\begin{align}
			f(x \btr_1 v) = x \btr_2 f(v)
		\end{align}
		が成り立つこと\footnote{スカラー乗法についての線型性の定義を $\btr$ について拡張しただけ.}.	
        \item $\mathfrak{g}$-加群の準同型 $f \colon V \lto W$ が\textbf{同型} (isomorphism) であるとは,
        $f$ が\underline{ベクトル空間の同型写像}であることを言う.
        \item 同型な $\mathfrak{g}$-加群のことを,\textbf{同値な} $\bm{\mathfrak{g}}$ \textbf{の表現} (equivalent representation of $\mathfrak{g}$)とも言う.
    \end{itemize}

    \tcblower

	同値な定義だが,線型写像 $f \colon V \lto W$ が $\mathfrak{g}$-加群の準同型であるとは,\hyperref[def:rep-LieAlg]{Lie代数の表現}
	\begin{align}
		\phi_1 \colon \mathfrak{g} &\lto \Lgl (V),\; x \lmto \bigl( v \mapsto x \btr_1 v \bigr) \\
		\phi_2 \colon \mathfrak{g} &\lto \Lgl (W),\; x \lmto \bigl( v \mapsto x \btr_2 v \bigr) 
	\end{align}
	に関して
	\begin{align}
		\forall x \in \mathfrak{g},\; f \circ \phi_1(x) = \phi_2(x) \circ f
	\end{align}
	が成り立つことを言う.
\end{mydef}


\begin{mydef}[label=def:sub-g-module,breakable]{部分 $\mathfrak{g}$-加群}
	\hyperref[ax:g-module]{ $\mathfrak{g}$-加群} $V$ を与える.\underline{部分集合} $W \subset V$ が\textbf{部分 $\bm{\mathfrak{g}}$-加群}であるとは,$W$ が和,スカラー乗法,$\mathfrak{g}$ の左作用の全てについて閉じていること.
	i.e. $\forall w,\, w_1,\, w_2 \in W,\; \forall \lambda \in \mathbb{K},\; \forall x \in \mathfrak{g}$ に対して
	\begin{align}
		w_1 + w_2 &\in W \\
		\lambda w &\in W \\
		x \btr w &\in W
	\end{align}
	が成り立つことを言う.
	\tcblower
	同値な定義として,以下の2つの条件が充たされることを言う:
	\begin{description}
		\item[\textbf{(sub-M1)}] $W$ が $V$ の\underline{部分ベクトル空間}
		\item[\textbf{(sub-M2)}] \hyperref[def:rep-LieAlg]{Lie代数の表現}
		\begin{align}
			\phi \colon \mathfrak{g} \lto \mathfrak{gl}(V),\; x \lmto (v \lmto x \btr v)
		\end{align}
		に関して
		\begin{align}
			\forall x \in \mathfrak{g},\; \phi(x)(W) \subset W
		\end{align}
		が成り立つ.i.e. $\forall x \in \mathfrak{g}$ に対して $W$ は $\phi(x)$-不変である.
	\end{description}
\end{mydef}

\begin{myexample}[label=ex:gmod-Ker-Im]{$\mathfrak{g}$-加群の準同型の核と像}
	\hyperref[ax:g-module]{$\mathfrak{g}$-加群} $V,\, W$ とその間の\hyperref[def:g-module-hom]{$\mathfrak{g}$-加群の準同型} $f \colon V \lto W$ を与える.このとき
	\begin{align}
		v \in \Ker f &\IMP \forall x \in \mathfrak{g},\; f(x \btr v) = x \btr f(v) = x \btr 0 = 0 \IFF \forall x \in \mathfrak{g},\; x \btr v \in \Ker f, \\
		w \in \Im f &\IFF \exists v \in V,\; w = f(v) \IMP \forall x \in \mathfrak{g},\; x \btr w = x \btr f(v) = f(x \btr v) \\
		&\IFF \forall x \in \mathfrak{g},\; x \btr w \in \Im f
	\end{align}
	が言えるので $\Ker f,\; \Im f$ はそれぞれ $V,\, W$ の\hyperref[def:sub-g-module]{部分 $\mathfrak{g}$-加群}である.
\end{myexample}


\subsection{$\mathfrak{g}$-加群の直和と既約性}

この小節でも $\mathbb{K}$ を\underline{任意の体}とする.

\begin{mydef}[label=def:gmod-directsum,breakable]{$\mathfrak{g}$-加群の直和}
	\hyperref[ax:g-module]{ $\mathfrak{g}$-加群}の族 $\Familyset[\big]{(V_i,\, +,\, \cdot\;,\, \btr_i)}{i \in I}$ を与える.このとき
	\begin{itemize}
		\item \hyperref[def:univ-vec-sum]{直和ベクトル空間} $\bigoplus_{i \in I} V_i$
		\item $\bigoplus_{i \in I} V_i$ への $\mathfrak{g}$ の左作用
		\begin{align}
			\btr \colon \mathfrak{g} \times \bigoplus_{i \in I} V_i \lto \bigoplus_{i \in I} V_i,\; \bigl(x,\, (v_i)_{i \in I}\bigr) \lmto (x \btr_i v_i)_{i \in I}
		\end{align}
	\end{itemize}
	の組として得られる $\mathfrak{g}$-加群 $(\bigoplus_{i \in I} V_i,\, +,\, \cdot\;,\, \btr)$ を\textbf{$\mathfrak{\bm{g}}$-加群の直和} (direct sum) と呼び\footnote{系\ref{col:subvec-directsum}の注と同様に,この定義は厳密には\textbf{外部直和} (external direct sum) と呼ぶべきだと思う.},$\bm{\bigoplus_{i \in I} V_i}$ と略記する.
\end{mydef}

\begin{mydef}[label=def:irr]{Lie代数の表現の既約性}
    \begin{itemize}
		\item \hyperref[ax:g-module]{ $\mathfrak{g}$-加群} $V$ が\textbf{既約} (irreducible)\footnote{i.e. Lie代数 $\mathfrak{g}$ の表現 $(\phi,\, V)$ が\textbf{既約表現} (irreducible representation; irrep) だ,と言っても良い.}であるとは,$V$ の\hyperref[def:sub-g-module]{部分 $\mathfrak{g}$-加群}が $0,\, V$ のちょうど\underline{2つ}\footnote{つまり,零ベクトル空間 $0$ は既約な $\mathfrak{g}$-加群とは呼ばない.}だけであることを言う.
		\item \hyperref[ax:g-module]{ $\mathfrak{g}$-加群} $V$ が\textbf{完全可約} (completely reducible)であるとは,$V$ が既約な\hyperref[def:gmod-directsum]{部分 $\mathfrak{g}$-加群の直和}
		\footnote{こちらの場合,厳密には\textbf{内部直和} (internal direct sum) と呼ぶべきだと思う.}
		であることを言う.
	\end{itemize}
\end{mydef}

次の補題は証明が少し厄介である:
\begin{mylem}[label=lem:semisimple-split]{完全可約の全射}
	\hyperref[def:g-module-hom]{$\mathfrak{g}$-加群の準同型} $p \colon V \lto W$ を与える.
	このとき $p$ が全射かつ $V$ が\hyperref[def:irr]{完全可約}ならば,$\mathfrak{g}$-加群の短完全列
	\begin{align}
		0 \lto \Ker p \hookrightarrow V \xrightarrow{p} W \lto 0
	\end{align}
	は\hyperref[lem:splitting]{分裂}する.
\end{mylem}

\begin{proof}
	% \hyperref[def:g-module-hom]{$\mathfrak{g}$-加群の準同型} $i \colon W \lto V$ であって $p \circ i = \mathrm{id}_W$ を充たすものが存在することを示す.
	$V$ が完全可約という仮定から,\hyperref[def:irr]{既約}な\hyperref[def:sub-g-module]{部分 $\mathfrak{g}$-加群}の族 $\Familyset[\big]{V_i}{i \in I}$ が存在して,\hyperref[col:subvec-directsum]{内部直和}の意味で
	\begin{align}
		V = \bigoplus_{i \in I} V_i
	\end{align}
	と書ける.ここで $\mathcal{S}$ を以下の条件を充たす組 $(J,\, V_J)$ 全体の集合とする:
	\begin{itemize}
		\item $J \subset I,\; V_J = \bigoplus_{j \in J} V_j \subset V$
		\item $\Ker p \cap V_J = 0$
	\end{itemize}
	$\mathcal{S}$ の上の2項関係を
	\begin{align}
		(J,\, V_J) \le (K,\, V_K) \DEF J \subset K
	\end{align}
	と定義すると組 $(\mathcal{S},\, \le)$ は順序集合になる.
	また $\mathcal{S}' = \Familyset[\big]{(J_a,\, V_{J_a})}{a \in A}$ を $\mathcal{S}$ の任意の全順序部分集合とすると $\Bigl(\bigcup_{a\in A} J_a,\, V_{\bigcup_{a \in A} J_a}\Bigr) \in \mathcal{S}$ であり\footnote{$V_{\bigcup_{a \in A} J_a} = \bigoplus_{j \in \bigcup_{a \in A} J_a} V_j = \bigcup_{a \in A} V_{J_a}$ なので,$\cap$ の分配律から $\Ker p \cap V_{\bigcup_{a \in A} J_a} = \Ker p \cap \bigcup_{a \in A} V_{J_a} = \bigcup_{a \in A} (\Ker p \cap V_{J_a}) = 0$ が言える.},これが $\mathcal{S}'$ の上界を与える.i.e. $\mathcal{S}$ は帰納的順序集合である.
	したがってZornの補題を使うことができ,$\mathcal{S}$ は極大元 $(J_0,\, V_{J_0}) \in \mathcal{S}$ を持つ.

	次に $V = \Ker p \oplus V_{J_0}$ を示す.$\mathcal{S}$ の定義から $\Ker p \cap V_{J_0} = 0$ なので,命題\ref{prop:subvec-directsum}より $V = \Ker p + V_{J_0}$ を示せば良い.
	$V \neq \Ker p + V_{J_0}$ を仮定すると,ある $k \in I \setminus J_0$ が存在して $V_k \not\subset \Ker p + V_{J_0}$ を充たす.$V_k$ は既約なので $V_k \cap (\Ker p + V_{J_0}) = 0$ が成り立つが,このことは $(J_0,\, V_{J_0})$ の極大性に矛盾.よって背理法から $V = \Ker p + V_{J_0}$ が言えた.

	以上より $W \cong V / \Ker p \cong V_{J_0}$ が言える.このとき包含準同型 $i \colon V_{J_0} \hookrightarrow V$ が $p \circ i = \mathrm{id}_{V_{J_0}}$ を充たすので証明が完了した.
\end{proof}


\begin{myprop}[label=prop:reducible-1]{完全可約性の特徴付け}
	以下の2つは同値である:
	\begin{enumerate}
		\item \hyperref[ax:g-module]{ $\mathfrak{g}$-加群} $V$ が\hyperref[def:irr]{完全可約}
		\item $V$ の任意の\hyperref[def:gmod-directsum]{部分 $\mathfrak{g}$-加群} $W \subset V$ に対して,\hyperref[def:gmod-directsum]{部分 $\mathfrak{g}$-加群} $W^c \subset V$ \footnote{$W$ の\textbf{補表現} (complement representation) と言う.}が存在して $V \cong W \oplus W^c$ を充たす.
	\end{enumerate}
\end{myprop}

\begin{proof}
	\begin{description}
		\item[\textbf{(1) $\bm{\Longrightarrow}$ (2)}] 
		$V$ が既約な部分 $\mathfrak{g}$-加群の族 $\Familyset[\big]{V_i}{i \in I}$ によって
		\begin{align}
			\label{eq:ds1}
			V = \bigoplus_{i \in I} V_i 
		\end{align}
		と書けるとする.$V$ の任意の部分 $\mathfrak{g}$-加群 $W \subset V$ を1つ固定する.
		このとき標準的射影 $p \colon V \lto V/W$ は全射な\hyperref[def:g-module-hom]{$\mathfrak{g}$-加群の準同型}なので,補題\ref{lem:semisimple-split}から$\mathfrak{g}$-加群の短完全列
		\begin{align}
			0 \lto \Ker p \cong W \hookrightarrow V \xrightarrow{p} V/W \lto 0
		\end{align}
		が\hyperref[lem:splitting]{分裂}する.よって系\ref{col:split}から
		\begin{align}
			V \cong W \oplus (V/W)
		\end{align}
		が言えた.
		% i.e. $p \circ i = \mathrm{id}_W$ を充たす $\mathfrak{g}$-加群の準同型 $i \colon V/W \lto V$ が存在する.$p$ は全射なので $i$ は単射であり,$V/W \cong \Im i$ が言える.よって 

		\item[\textbf{(1) $\bm{\Longleftarrow}$ (2)}] 
		
		% $V$ の任意の\hyperref[def:gmod-directsum]{部分 $\mathfrak{g}$-加群}が補空間を持つとする.
		% このとき $\dim V$ に関する数学的帰納法から
		$V$ の\hyperref[def:irr]{既約な部分 $\mathfrak{g}$-加群}全体の集合を $\mathcal{V}$ と書く.$\mathcal{S}$ を以下の条件を充たす組 $(I,\, V_I)$ 全体の集合とする:
		\begin{itemize}
			\item $I \subset \mathcal{V}$
			\item 内部直和の意味で $V_I = \bigoplus_{V_i \in I} V_i \subset V$
		\end{itemize}
		$\mathcal{S}$ 上の2項関係を
		\begin{align}
			(I,\, V_I) \le (J,\, V_J) \DEF I \subset J
		\end{align}
		と定義すると組 $(\mathcal{S},\, \le)$ は順序集合になる.$V$ の $0$ でない部分 $\mathfrak{g}$-加群のうち極小のものを $V_1$ とすると,定義から $V_1 \in \mathcal{V}$ なので $(\{V_1\},\, V_{V_1}) \in \mathcal{S}$ となり $\mathcal{S}$ は空でない.
		また $\mathcal{S}' = \Familyset[\big]{(J_a,\, V_{J_a})}{a \in A}$ を $\mathcal{S}$ の任意の全順序部分集合とすると $\Bigl(\bigcup_{a\in A} J_a,\, V_{\bigcup_{a \in A} J_a}\Bigr) \in \mathcal{S}$ であり,これが $\mathcal{S}'$ の上界を与える.i.e. $\mathcal{S}$ は帰納的順序集合である.
		したがってZornの補題を使うことができ,$\mathcal{S}$ は極大元 $(I_0,\, V_{I_0}) \in \mathcal{S}$ を持つ.
		このとき $V = V_{I_0}$ であることを背理法により示そう.

		 $V \neq V_{I_0}$ を仮定する.このとき (2) より $V$ の $0$ でない部分 $\mathfrak{g}$-加群 $V_{I_0}^c$ が存在して $V \cong V_{I_0} \oplus V_{I_0}^c$ を充たす.このとき $V_{I_0}^c$ に含まれる $0$ でない極小の部分 $\mathfrak{g}$-加群 $W$ をとることができるが,定義からこの $W$ は既約である.よって
		\begin{align}
			W \oplus V_{I_0} \subset V
		\end{align}
		もまた既約部分 $\mathfrak{g}$-加群の直和となり,$V_{I_0}$ の極大性に矛盾する.
	\end{description}
	
\end{proof}

\begin{mylem}[label=lem:Schur]{Schurの補題}
	\underline{任意の体}\footnote{代数閉体でなくても良い} $\mathbb{K}$ 上の\hyperref[ax:g-module]{$\mathfrak{g}$-加群} $V,\, W$,および \underline{$0$ でない}\hyperref[def:g-module-hom]{$\mathfrak{g}$-加群の準同型} $f \colon V \lto W$ を与える.
	このとき以下が成り立つ:
	\begin{enumerate}
		\item $V$ が\hyperref[def:irr]{既約}ならば $f$ は単射
		\item $W$ が\hyperref[def:irr]{既約}ならば $f$ は全射
	\end{enumerate}
	
\end{mylem}

\begin{proof}
	\begin{enumerate}
		\item \exref{ex:gmod-Ker-Im}より $\Ker f$ は $V$ の\hyperref[def:sub-g-module]{部分 $\mathfrak{g}$-加群}だが,$V$ が既約なので $\Ker f = 0,\, V$ のどちらかである.仮定より $f$ は $0$ でないので $\Ker f = 0$,i.e. $f$ は単射である.
		\item \exref{ex:gmod-Ker-Im}より $\Im f$ は $W$ の\hyperref[def:sub-g-module]{部分 $\mathfrak{g}$-加群}だが,$W$ が既約なので $\Im f = 0,\, W$ のどちらかである.仮定より $f$ は $0$ でないので $\Im f = W$,i.e. $f$ は全射である.
	\end{enumerate}
	
\end{proof}

\begin{mycol}[label=col:Schur-closed]{代数閉体上のSchurの補題}
	\underline{代数閉体} $\mathbb{K}$ 上の有限次元\hyperref[ax:g-module]{$\mathfrak{g}$-加群} $V$ を与える.
	このとき $V$ が\hyperref[def:irr]{既約}ならば,任意の\hyperref[def:g-module-hom]{$\mathfrak{g}$-加群の自己準同型} $\phi \in \End V$ はある $\lambda \in \mathbb{K}$ を使って $\phi = \lambda\, \mathrm{id}_V$(i.e. スカラー倍)と書ける.
\end{mycol}

\begin{proof}
	仮定より $V$ が既約なので,補題\ref{lem:Schur}-(1), (2) より任意の\hyperref[def:g-module-hom]{$\mathfrak{g}$-加群の自己準同型} $\phi \colon V \lto V$ は\hyperref[def:g-module-hom]{$\mathfrak{g}$-加群の同型}か $0$ のどちらかである.
	ここで $\lambda \in \mathbb{K}$ を $\phi$ の固有値とする.$\mathbb{K}$ が代数閉体なので $\lambda$ は確かに存在する.
	このとき写像 $\phi - \lambda\, \mathrm{id}_V \colon V \lto V$ もまた\hyperref[def:g-module-hom]{$\mathfrak{g}$-加群の自己準同型}となるが,固有値の定義から $\det (\phi - \lambda\, \mathrm{id}_V) = 0$ なので同型写像ではあり得ない.よって $\phi - \lambda\, \mathrm{id}_V = 0 \IFF \phi = \lambda\, \mathrm{id}_V$ である.
\end{proof}

\begin{mycol}[label=col:solvable-irrep]{可換なLie代数の有限次元既約表現}
	\underline{代数閉体}上のLie代数 $\mathfrak{g}$ が可換ならば,$\mathfrak{g}$ の任意の\underline{有限次元}\hyperref[def:irr]{既約表現}は1次元である.
\end{mycol}

\begin{proof}
	$\phi \colon \mathfrak{g} \lto \Lgl (V)$ を $\mathfrak{g}$ の有限次元既約表現とする.
	このとき $\mathfrak{g}$ が可換であることから $\forall x,\, y \in \mathfrak{g},\, \forall v \in V$ に対して
	\begin{align}
		\phi(x)(y \btr v) 
		&= \phi(x) \circ \phi(y) (v) \\
		&= \comm{\phi(x)}{\phi(y)}(v) + \phi(y) \circ \phi(x)(v) \\
		&= \phi(\comm{x}{y})(v) + \phi(y) \circ \phi(x)(v) \\
		&= \phi(0)(v) + \phi(y) \circ \phi(x)(v) \\
		&= \phi(y) \circ \phi(x)(v) \\
		&= y \btr \bigl( \phi(x)(v) \bigr)
	\end{align}
	が言える.i.e. $\forall x \in \mathfrak{g}$ に対して $\phi(x) \colon V \lto V$ は\hyperref[def:g-module-hom]{$\mathfrak{g}$-加群の準同型}である.
	よって\hyperref[col:Schur-closed]{Schurの補題}から $\phi(x)$ がスカラー倍だとわかる.
	故に $V$ の任意の1次元部分ベクトル空間は自動的に\hyperref[def:sub-g-module]{部分 $\mathfrak{g}$-加群}になる.
	然るに $V$ は仮定より既約だから $V$ の\hyperref[def:sub-g-module]{部分 $\mathfrak{g}$-加群}は $0,\, V$ しかあり得ない.さらに $V \neq 0$ なので $\dim V = 1$ でなくてはならない.
\end{proof}

\subsection{$\mathfrak{g}$-加群のHomとテンソル積}

この小節でも $\mathbb{K}$ を\underline{任意の体}とする.

\begin{mydef}[label=def:gmod-tensor]{$\mathfrak{g}$-加群のテンソル積}
	$(V_1,\, +,\, \cdot\;,\, \btr_1) ,\, (V_2,\, +,\, \cdot\;,\, \btr_2)$ を\underline{有限次元}\hyperref[ax:g-module]{ $\mathfrak{g}$-加群}とする.このとき
	\begin{itemize}
		\item \hyperref[def:univ-vec-tensor]{$\mathbb{K}$-ベクトル空間のテンソル積} $V_1 \otimes V_2$
		\item $V_1 \otimes V_2$ への $\mathfrak{g}$ の左作用\footnote{正確には,これの右辺を線型に拡張したもの}
		\begin{align}
			\btr \colon \mathfrak{g} \times (V_1 \otimes V_2) \lto V_1 \otimes V_2,\; (x,\, v_1 \otimes v_2) \lmto (x \btr_1 v_1) \otimes v_2 + v_1 \otimes (x \btr_2 v_2)
		\end{align}
	\end{itemize}
	の組として得られる $\mathfrak{g}$-加群 $(V_1 \otimes V_2,\, +,\, \cdot\;,\, \btr)$ を\textbf{$\bm{\mathfrak{g}}$-加群のテンソル積} (tensor product) と呼び,$\bm{V_1 \otimes V_2}$ と略記する.
\end{mydef}

実際 $V_1 \otimes V_2$ が $\mathfrak{g}$-加群になっていることを確認しておこう:
\begin{align}
	\comm{x}{y} \btr (v_1 \otimes v_2)
	&= (\comm{x}{y} \btr_1 v_1) \otimes v_2 + v_1 \otimes (\comm{x}{y} \btr_2 v_2) \\
	&= (x \btr_1 y \btr_1 v_1) \otimes v_2 - (y \btr_1 x \btr_1 v_1) \otimes v_2 \\
	&\quad + v_1 \otimes (x \btr_2 y \btr_2 v_2) - v_1 \otimes (y \btr_2 x \btr_2 v_2) \\
	&= \bigl( (x \btr_1 y \btr_1 v_1) \otimes v_2 + v_1 \otimes (x \btr_2 y \btr_2 v_2) \bigr) \\
	&\quad - \bigl( (y \btr_1 x \btr_1 v_1) \otimes v_2 + v_1 \otimes (y \btr_2 x \btr_2 v_2) \bigr), \\
	x \btr y \btr (v_1 \otimes v_2) - y \btr x \btr (v_1 \otimes v_2)
	&= x \btr \bigl( (y \btr_1 v_1) \otimes v_2 + v_1 \otimes (y \btr_2 v_2) \bigr) \\
	&\quad - y \btr \bigl( (x \btr_1 v_1) \otimes v_2 + v_1 \otimes (x \btr_2 v_2) \bigr) \\
	&= \bigl( (x \btr_1 y \btr_1 v_1) \otimes v_2 + v_1 \otimes (x \btr_2 y \btr_2 v_2) \bigr) \\
	&\quad - \bigl( (y \btr_1 x \btr_1 v_1) \otimes v_2 + v_1 \otimes (y \btr_2 x \btr_2 v_2) \bigr)
\end{align}
なので
\begin{align}
	\comm{x}{y} \btr (v_1 \otimes v_2)  = x \btr y \btr (v_1 \otimes v_2) - y \btr x \btr (v_1 \otimes v_2)
\end{align}
がわかった.

\begin{mydef}[label=def:gmod-dual]{$\mathfrak{g}$-加群の双対}
	$(V,\, +,\, \cdot\;,\, \btr)$ を\underline{有限次元} \hyperref[ax:g-module]{ $\mathfrak{g}$-加群}とする.このとき
	\begin{itemize}
		\item \hyperref[def:hom-vec]{双対ベクトル空間} $V^* = \Hom{\mathbb{K}}(V,\, \mathbb{K})$
		\item $V^*$ への $\mathfrak{g}$ の左作用
		\begin{align}
			\btr \colon \mathfrak{g} \times V^* \lto V^*,\; (x,\, f) \lmto \bigl(v \mapsto - f(x \btr v)\bigr)
		\end{align}
	\end{itemize}
	の組として得られる $\mathfrak{g}$-加群 $(V^*,\, +,\, \cdot\;,\, \btr)$ を\textbf{$\bm{\mathfrak{g}}$-加群の双対} (dual)\footnote{\textbf{反傾} (contragradient) と呼ぶ場合もあるようだが,現在はあまり使われていないような気がする.} と呼び,$\bm{V^*}$ と略記する.
	
\end{mydef}

実際 $V^*$ が $\mathfrak{g}$-加群になっていることを確認しておこう:
\begin{align}
	(\comm{x}{y} \btr f)(v)
	&= - f(\comm{x}{y} \btr f)(v) \\
	&= - f(x \btr y \btr v - y \btr x \btr v) \\
	&= - f(x \btr y \btr v)  + f(y \btr x \btr v) \\
	&= (x \btr f)(y \btr v) - (y \btr f)(x \btr v) \\
	&= -\bigl( y \btr (x \btr f) \bigr) (v) + \bigl( x \btr (y \btr f) \bigr) (v) \\
	&= (x \btr y \btr f)(v) - (y \btr x \btr f)(v)
\end{align}
なので
\begin{align}
	\comm{x}{y} \btr f = x \btr y \btr f - y \btr x \btr f
\end{align}
がわかった.

ここで, $\mathbb{K}$-ベクトル空間の自然な同型(命題\ref{prop:tensor-hom})
\begin{align}
	V^* \otimes W \cong \Hom{\mathbb{K}}(V,\, W)
\end{align}
の具体形が
\begin{align}
	\label{eq:isom-tensorhom}
	\alpha \colon f \otimes w \lmto \bigl( v \mapsto f(v) \cdot w  \bigr) 
\end{align}
となっていたことを思い出そう.このことから,$\mathbb{K}$-ベクトル空間 $\Hom{\mathbb{K}}(V,\, W)$ の上の $\mathfrak{g}$ の左作用を
\begin{align}
	x \btr (f \otimes w) = - f (x \btr \mhyphen) \otimes w + f \otimes (x \btr w)
\end{align}
に着想を得て
\begin{align}
	(x \btr F)(v) = - F(x \btr v) + x \btr F(v)\quad (\forall F \in \Hom{\mathbb{K}}(V,\, W))
\end{align}
と定義しようと思うのが自然である.というのも,こう定義することで $\mathbb{K}$-ベクトル空間の同型写像\eqref{eq:isom-tensorhom}が
\begin{align}
	\alpha \bigl( x \btr (f \otimes w) \bigr) (v)
	&= - \alpha \bigl(f (x \btr \mhyphen) \otimes w\bigr)(v) + \alpha \bigl( f \otimes (x \btr w) \bigr)(v) \\
	&= - f (x \btr v)\cdot w + f(v) \cdot (x \btr w) \\
	&= - f (x \btr v)\cdot w + x \btr \bigl( f(v) \cdot w \bigr) \\
	&= \bigl(x \btr \alpha (f \otimes w) \bigr)(v)
\end{align}
となって\hyperref[def:g-module-hom]{$\mathfrak{g}$-加群の同型写像}になる!

\begin{mydef}[label=def:gmod-hom]{$\mathfrak{g}$-加群のHom}
	$(V,\, +,\, \cdot\;,\, \btr_1) ,\, (W,\, +,\, \cdot\;,\, \btr_2)$ を\underline{有限次元}\hyperref[ax:g-module]{ $\mathfrak{g}$-加群}とする.このとき
	\begin{itemize}
		\item \hyperref[def:vec-hom]{$\mathbb{K}$-ベクトル空間} $\Hom{\mathbb{K}}(V,\, W)$\footnote{$V$ から $W$ への\hyperref[def:g-module-hom]{$\mathfrak{g}$-加群の準同型}全体の集合\underline{ではない}.}
		\item $\Hom{\mathbb{K}}(V,\, W)$ への $\mathfrak{g}$ の左作用
		\begin{align}
			\btr \colon \mathfrak{g} \times \Hom{\mathbb{K}}(V,\, W) \lto \Hom{\mathbb{K}}(V,\, W),\; F \lmto \bigl( v \mapsto - F(x \btr_1 v) + x \btr_2 F(v) \bigr) 
		\end{align}
	\end{itemize}
	の組として得られる $\mathfrak{g}$-加群 $\bigl(\Hom{\mathbb{K}}(V,\, W),\, +,\, \cdot\;,\, \btr\bigr)$ を $\bm{\Hom{\mathbb{K}}(V,\, W)}$ と略記する.
\end{mydef}


\subsection{Casimir演算子}

この小節では $\mathbb{K}$ は\underline{標数 $0$ の体}とする.

\begin{mydef}[label=def:faithful]{忠実な表現}
	\hyperref[def:rep-LieAlg]{Lie代数 $\mathfrak{g}$ の表現} $\rho \colon \mathfrak{g} \lto \Lgl (V)$ が\textbf{忠実} (faithful)\footnote{群作用の文脈では\textbf{効果的な作用} (effective action) と呼ぶ.} であるとは,$\rho$ が単射であることを言う.
\end{mydef}

% % この小節の以降では有限次元表現を考える.
% Lie代数 $\mathfrak{g}$ の\underline{有限次元}表現 $\phi \colon \mathfrak{g} \lto \Lgl (V)$ は忠実であるとする.
% $\mathfrak{g}$ 上の対称な双線型形式 $\beta \in L(\mathfrak{g},\, \mathfrak{g};\, \mathbb{K})$ を
% \begin{align}
% 	\beta \colon \mathfrak{g} \times \mathfrak{g} \lto \mathbb{K},\; (x,\, y) \lmto \Tr \bigl( \phi(x) \circ \phi(y) \bigr) 
% \end{align}
% と定義する.このとき $\Tr$ の循環性から
% \begin{align}
% 	\beta (x,\, \comm{y}{z}) 
% 	&= \Tr \bigl( \phi(x) \circ \phi(\comm{y}{z}) \bigr) \\
% 	&= \Tr \bigl( \phi(x) \circ \phi(y) \circ \phi(z) \bigr) - \Tr \bigl( \phi(x) \circ \phi(z) \circ \phi(y) \bigr) \\
% 	&= \Tr \bigl( \phi(x) \circ \phi(y) \circ \phi(z) \bigr) - \Tr \bigl( \phi(y) \circ \phi(x) \circ \phi(z) \bigr) \\
% 	&= \Tr \bigl( \phi(\comm{x}{y}) \circ \phi(z) \bigr) \\
% 	&= \beta (\comm{x}{y},\, z)
% \end{align}
% が成り立つので,$\beta$ の\hyperref[def:radical-bilinear]{radical}
% \begin{align}
% 	S_\beta \coloneqq \bigl\{\, x \in \mathfrak{g} \bigm| \forall y \in \mathfrak{g},\; \beta (x,\, y) = 0 \,\bigr\} 
% \end{align}
% は $\mathfrak{g}$ の\hyperref[def:ideal-LieAlg]{イデアル}となる.
% さらに $\mathfrak{g}$ が\hyperref[def:semisimple-LieAlg]{半単純}ならば $S_\beta = 0$ であること,i.e. $\beta$ が\hyperref[def:radical-bilinear]{非退化}であることが言える:

\begin{mylem}[label=lem:nondegenerate]{}
	\begin{itemize}
		\item \hyperref[def:semisimple-LieAlg]{半単純Lie代数} $\mathfrak{g}$ の\hyperref[def:faithful]{忠実}な\underline{有限次元}表現 $\phi \colon \mathfrak{g} \lto \Lgl (V)$ 
		\item $\mathfrak{g}$ 上の対称な双線型形式
		\begin{align}
			\beta \colon \mathfrak{g} \times \mathfrak{g} \lto \mathbb{K},\; (x,\, y) \lmto \Tr \bigl( \phi(x) \circ \phi(y) \bigr) 
		\end{align}
	\end{itemize}
	を与える.$\beta$ の\hyperref[def:radical-bilinear]{radical}を
	\begin{align}
		S_\beta \coloneqq \bigl\{\, x \in \mathfrak{g} \bigm| \forall y \in \mathfrak{g},\; \beta (x,\, y) = 0 \,\bigr\} 
	\end{align}
	とおく.このとき以下が成り立つ:
	\begin{enumerate}
		\item $S_\beta$ は $\mathfrak{g}$ の\hyperref[def:ideal-LieAlg]{イデアル}である.
		\item $S_\beta = 0$,i.e. $\beta$ は\hyperref[def:radical-bilinear]{非退化}である.
	\end{enumerate}
\end{mylem}

\begin{proof}
	\begin{enumerate}
		\item $\Tr$ の循環性から
		\begin{align}
			\beta (x,\, \comm{y}{z}) 
			&= \Tr \bigl( \phi(x) \circ \phi(\comm{y}{z}) \bigr) \\
			&= \Tr \bigl( \phi(x) \circ \phi(y) \circ \phi(z) \bigr) - \Tr \bigl( \phi(x) \circ \phi(z) \circ \phi(y) \bigr) \\
			&= \Tr \bigl( \phi(x) \circ \phi(y) \circ \phi(z) \bigr) - \Tr \bigl( \phi(y) \circ \phi(x) \circ \phi(z) \bigr) \\
			&= \Tr \bigl( \phi(\comm{x}{y}) \circ \phi(z) \bigr) \\
			&= \beta (\comm{x}{y},\, z)
		\end{align}
		が成り立つので,$\forall x \in \mathfrak{g},\; \forall y \in S_\beta$ に対して
		\begin{align}
			\forall z \in \mathfrak{g},\; \beta(\comm{x}{y},\, z) = -\beta (y,\, \comm{x}{z}) = 0
		\end{align}
		が成り立つ.i.e. $\comm{x}{y} \in S_\beta$ が言えた.
		\item $S_\beta$ の定義から $\comm{\phi(x)}{\phi(y)}$ の形をした $\comm{\phi(S_\beta)}{\phi(S_\beta)}$ の任意の元および $\forall \phi(z) \in \phi(S_\beta)$ に対して
		\begin{align}
			\Tr\bigl(\comm{\phi(x)}{\phi(y)} \circ \phi(z)\bigr) 
			&= \Tr \bigl( \phi(\comm{x}{y}) \circ \phi(z) \bigr) \\
			&= \beta (\comm{x}{y},\, z) \\
			&= 0
		\end{align}
		が成り立つので,定理\ref{thm:Cartan-crit}より $\phi(S_\beta)$ は可解である.$\phi$ は忠実なので $\Ker \phi = 0$ であり,
		\hyperref[prop:homo]{準同型定理}から $\phi(S_\beta) \cong S_\beta / \Ker \phi = S_\beta$ が言える.従って (1) も併せると $S_\beta$ は $\mathfrak{g}$ の可解イデアルである.
		仮定より $\mathfrak{g}$ は半単純だったから,\hyperref[def:semisimple-LieAlg]{半単純Lie代数の定義}から $S_\beta = 0$ が言える.
	\end{enumerate}
	
\end{proof}


\begin{mylem}[label=lem:Casimir,breakable]{}
	\begin{itemize}
		\item \hyperref[de:semisimple-LieAlg]{半単純Lie代数} $\mathfrak{g}$ の基底 $\{e_\mu\}$
		\item 対称かつ\hyperref[def:radical-bilinear]{非退化}な双線型形式 $\beta \in L(\mathfrak{g},\, \mathfrak{g};\, \mathbb{K})$
		であって $\forall x,\, y,\, z \in \mathfrak{g}$ に対して
		\begin{align}
			\beta(x,\, \comm{y}{z}) = \beta (\comm{x}{y},\, z)
		\end{align}
		を充たすもの
	\end{itemize}
	を与える.このとき以下が成り立つ: 
	\begin{enumerate}
		\item 
		$\mathfrak{g}$ の基底 $\{e^\mu\}$ であって\footnote{$\mathfrak{g}^*$ の元ではないが,Einsteinの規約との便宜上添字を上付きにする.},$\forall (\mu,\, \nu) \in \{1,\, \dots,\, \dim \mathfrak{g}\}^2$ に対して
		\begin{align}
			\beta(e_\mu,\, e^\nu) = \delta_{\mu}^{\nu}
		\end{align}
		を充たすものが一意的に存在する.
		\item $\forall x \in \mathfrak{g}$ を一つ固定する.このとき $\ad(x) \colon \mathfrak{g} \lto \Lgl (\mathfrak{g})$ の基底 $\{e_\mu\}$ による表現行列 $\bigl[ a_\mu{}^\nu \bigr]_{1 \le \mu,\, \nu \le \dim \mathfrak{g}}$ と,(1) の基底 $\{e^\mu\}$ による表現行列 $\bigl[ b^\mu{}_\nu \bigr]_{1 \le \mu,\, \nu \le \dim \mathfrak{g}}$ について
		\begin{align}
			a_\mu{}^\nu = -b^\nu{}_\mu
		\end{align}
		が成り立つ.
	\end{enumerate}
\end{mylem}

\begin{proof}
	\begin{enumerate}
		\item 
		$\beta_{\mu\nu} \coloneqq \beta(e_\mu,\, e_\nu)$ とおく.このとき $x = x^\mu e_\mu \in \mathfrak{g}$ に対して
		\begin{align}
			\forall y = y^\nu e_\nu \in \mathfrak{g},\; \beta(x,\, y) = 0 
			&\IFF \forall \mqty[y^1 \\ \vdots \\ y^{\dim \mathfrak{g}}] \in \mathbb{K}^{\dim \mathfrak{g}},\; \beta_{\mu\nu} x^\mu y^\nu = 0 \\
			&\IFF 1 \le \forall \nu \le \dim \mathfrak{g},\; \beta_{\nu\mu} x^\mu = 0 \\
			&\IFF \mqty[x^1 \\ \vdots \\ x^{\dim \mathfrak{g}}] \in \Ker \bigl[ \beta_{\mu\nu} \bigr] \subset \mathbb{K}^{\dim \mathfrak{g}}
		\end{align}
		が言える.ただし2つ目の同値変形で $\beta$ が対称であることを使った.したがって $\beta$ が非退化であることは $\Ker \bigl[ \beta_{\mu\nu} \bigr] = 0$ と同値であり,このことはさらに補題\ref{lem:finvec-basic}-(3) より $\det \bigl[ \beta_{\mu\nu} \bigr] \neq 0$ と同値である\footnote{Cramerの公式は任意の体 $\mathbb{K}$ 上で成り立つ.}.
		よって $\bigl[ \beta_{\mu\nu} \bigr]$ の逆行列 $\bigl[ \alpha^{\mu\nu} \bigr]$ が一意的に存在するので,
		$e^\mu \coloneqq e_\nu \alpha^{\mu\nu}$ と定めると,
		\begin{align}
			\beta(e_\mu,\, e^\nu) 
			&= \alpha^{\nu\rho} \beta_{\mu\rho} = \delta_{\mu}^{\nu}
		\end{align}
		が成り立つ.
		\item $\ad(x)(e_\mu) \eqqcolon a_\mu{}^\nu e_\nu,\; \ad(e^\mu) \eqqcolon b^{\mu}{}_\nu e^\nu$ とおくと,
		\begin{align}
			a_\mu{}^\nu 
			&= a_\mu{}^\rho \delta_{\rho}^{\nu} \\
			&= a_\mu{}^\rho \beta(e_\rho,\, e^\nu) \\
			&= \beta \bigl(\ad(x)(e_\mu),\, e^\nu\bigr) \\
			&= \beta \bigl(-\comm{e_\mu}{x},\, e^\nu\bigr) \\
			&= \beta \bigl(e_\mu,\, - \ad(x){e^\nu}\bigr) \\
			&= -b^\nu{}_\rho \beta \bigl(e_\mu,\, e^\rho\bigr) \\
			&= -b^\nu{}_\mu
		\end{align}
	\end{enumerate}
	
\end{proof}

\begin{mydef}[label=def:Casimir,breakable]{忠実な表現のCasimir演算子}
	\begin{itemize}
		\item \hyperref[def:semisimple-LieAlg]{半単純Lie代数} $\mathfrak{g}$ の\underline{有限次元}\hyperref[ax:g-module]{表現} $\phi \colon \mathfrak{g} \lto \Lgl(V)$
		\item \hyperref[de:semisimple-LieAlg]{半単純Lie代数} $\mathfrak{g}$ の基底 $\{e_\mu\}$
		\item 対称かつ\hyperref[def:radical-bilinear]{非退化}な双線型形式 $\beta \in L(\mathfrak{g},\, \mathfrak{g};\, \mathbb{K})$
		であって $\forall x,\, y,\, z \in \mathfrak{g}$ に対して
		\begin{align}
			\beta(x,\, \comm{y}{z}) = \beta (\comm{x}{y},\, z)
		\end{align}
		を充たすもの
	\end{itemize}
	を与える.与えられた $\mathfrak{g}$ の基底 $\{e_\mu\}$ から補題\ref{lem:Casimir}により構成した $\mathfrak{g}$ の基底 $\{e^\mu\}$ をとる.このとき

	\begin{itemize}
		\item $\mathbb{K}$-線型変換
		\begin{align}
			\bm{c_\phi(\beta)} \colon V \lto V,\; v \lmto \sum_{\mu = 1}^{\dim \mathfrak{g}} \phi(e_\mu) \circ \phi(e^\mu) (v)
		\end{align}
		を $\beta,\, \phi$ の前Casimir演算子と呼ぶ.
		\item $\phi$ が\hyperref[def:faithful]{忠実な表現}で,かつ
		\begin{align}
			\beta(x,\, y) \coloneqq \Tr \bigl(\phi(x) \circ \phi(y) \bigr) 
		\end{align}
		であるとき\footnote{補題\ref{lem:nondegenerate}よりこの $\beta$ は非退化である},$\beta,\, \phi$ の前Casimir演算子のことを\textbf{$\bm{\phi}$ のCasimir演算子} (Casimir operator of $\phi$) と呼んで $\bm{c_\phi}$ と略記する.
	\end{itemize}
\end{mydef}

\begin{myprop}[label=prop:Casimir-basic,breakable]{Casimir演算子の性質}
	\begin{itemize}
		\item \hyperref[def:semisimple-LieAlg]{半単純Lie代数} $\mathfrak{g}$ の\underline{有限次元}\hyperref[ax:g-module]{表現} $\phi \colon \mathfrak{g} \lto \Lgl(V)$
		\item \hyperref[de:semisimple-LieAlg]{半単純Lie代数} $\mathfrak{g}$ の基底 $\{e_\mu\}$
		\item 対称かつ\hyperref[def:radical-bilinear]{非退化}な双線型形式 $\beta \in L(\mathfrak{g},\, \mathfrak{g};\, \mathbb{K})$
		であって $\forall x,\, y,\, z \in \mathfrak{g}$ に対して
		\begin{align}
			\beta(x,\, \comm{y}{z}) = \beta (\comm{x}{y},\, z)
		\end{align}
		を充たすもの
	\end{itemize}
	を与える.与えられた $\mathfrak{g}$ の基底 $\{e_\mu\}$ から補題\ref{lem:Casimir}により構成した $\mathfrak{g}$ の基底 $\{e^\mu\}$ をとる.

	\begin{enumerate}
		\item \hyperref[def:Casimir]{前Casimir演算子} $c_{\phi}(\beta) \in \End (V)$ は,$\forall x \in \mathfrak{g}$ に対して
		\begin{align}
			\comm{\phi(x)}{c_\phi(\beta)} = 0
		\end{align}
		を充たす.従って $c_\phi (\beta)$ は\hyperref[def:g-module-hom]{$\mathfrak{g}$-加群の準同型である}.	
		\item $\phi$ が\hyperref[def:faithful]{忠実な表現}ならば,\hyperref[def:Casimir]{Casimir演算子} $c_\phi \in \End V$ について
		\begin{align}
			\Tr c_\phi = \dim \mathfrak{g}
		\end{align}
		が成り立つ.
		\item $\mathbb{K}$ が\underline{代数閉体}でかつ $\phi$ が\hyperref[def:faithful]{忠実な表現}でかつ $\phi$ が\hyperref[def:irr]{既約表現}ならば,\hyperref[def:Casimir]{Casimir演算子} $c_\phi \in \End V$ は $\mathfrak{g}$ の基底の取り方によらずに
		\begin{align}
			c_\phi = \frac{\dim \mathfrak{g}}{\dim V}\, \mathrm{id}_{V}
		\end{align}
		と書ける.
	\end{enumerate}
\end{myprop}

\begin{proof}
	\begin{enumerate}
		\item $\forall x,\,y,\, z\in \End (V)$ に対して
		% \footnote{写像の合成を略記した.}
		\begin{align}
			\comm{x}{y \circ z} = \comm{x}{y} \circ z + y \circ \comm{x}{z}
		\end{align}
		が成り立つことと補題\ref{lem:Casimir}-(2) より,
		\begin{align}
			\comm{\phi(x)}{c_\phi(\beta)} 
			&= \sum_{\mu=1}^{\dim \mathfrak{g}} \comm{\phi(x)}{\phi(e_\mu)} \circ \phi(e^\mu) + \sum_{\mu=1}^{\dim \mathfrak{g}} \phi(e_\mu) \circ \comm{\phi(x)}{\phi(e^\mu)} \\
			&= \sum_{\mu=1}^{\dim \mathfrak{g}} \ad\bigl(\phi(x)\bigr)\bigl(\phi(e_\mu)\bigr) \circ \phi(e^\mu) + \sum_{\mu=1}^{\dim \mathfrak{g}} \phi(e_\mu) \circ  \ad\bigl(\phi(x)\bigr)\bigl(\phi(e^\mu)\bigr) \\
			&= \sum_{\mu=1}^{\dim \mathfrak{g}} \phi\bigl(\ad(x)(e_\mu)\bigr) \circ \phi(e^\mu) + \sum_{\mu=1}^{\dim \mathfrak{g}} \phi(e_\mu) \circ  \phi\bigl(\ad(x)(e^\mu)\bigr) \\
			&= \sum_{\mu=1}^{\dim \mathfrak{g}} a_\mu{}^\nu \phi(e_\nu) \circ \phi(e^\mu) + \sum_{\mu=1}^{\dim \mathfrak{g}} b^\mu{}_\nu\phi(e_\mu) \circ \phi(e^\nu) \\
			&= 0
		\end{align}
		が言えた.
		\item 補題\ref{lem:Casimir}-(1) より
		\begin{align}
			\Tr c_\phi 
			&= \sum_{\mu}^{\dim \mathfrak{g}} \Tr \bigl(\phi(e_\mu) \circ \phi(e^\mu) \bigr) \\
			&= \sum_{\mu}^{\dim \mathfrak{g}} \beta (e_\mu,\, e^\mu) \\
			&= \dim \mathfrak{g}
		\end{align}
		\item $\mathbb{K}$ が代数閉体でかつ $\phi$ が既約なので,(1), (2) と\hyperref[col:Schur-closed]{代数閉体上のSchurの補題}から $c_\phi \colon V \lto V$ は $(\dim \mathfrak{g} / \dim V) \, \mathrm{id}_V$ に等しい.
	\end{enumerate}
	
\end{proof}


$\phi \colon \mathfrak{g} \lto \Lgl (V)$ が忠実でない場合は次のように考える:
まず,$\mathfrak{g}$ が\hyperref[def:semisimple-LieAlg]{半単純}なので,$\Ker \phi$($\mathfrak{g}$ のイデアルである)は系\ref{col:semisimple-decomp}から $\mathfrak{g}$ の\hyperref[def:simple-LieAlg]{単純}イデアルの直和である.
定理\ref{thm:semisimple-decomp}を使って $\mathfrak{g}^\perp$ を $\mathfrak{g} \eqqcolon \Ker \phi \oplus \mathfrak{g}^\perp$ で定義すると,$\mathfrak{g}^\perp \cong \mathfrak{g}/\Ker \phi$ なので,制限
\begin{align}
	\phi|_{\mathfrak{g}^\perp} \colon \mathfrak{g}^\perp \lto \Lgl (V)
\end{align}
は忠実な表現になる.そして $\mathfrak{g}^\perp$ の基底に対して定義\ref{def:Casimir}を適用するのである.



\subsection{Weylの定理}

この小節では $\mathbb{K}$ を\underline{標数 $0$ の体}とする.~\cite[第7章, p.80-86]{Satake1987LieAlg}に倣ってWeylの定理を証明する.

\begin{mylem}[label=lem:Weyl]{}
	$\phi \colon \mathfrak{g} \lto \Lgl (V)$ を\hyperref[def:semisimple-LieAlg]{半単純Lie代数}の\underline{有限次元}\hyperref[ax:g-module]{表現}とする.
	このとき
	\begin{align}
		\phi(\mathfrak{g}) \subset \Lsl (V)
	\end{align}
	が成り立つ.特に,$\dim V = 1$ ならば $\phi$ は零写像である\footnote{これを\textbf{自明な表現} (trivial representation) と言う.}
\end{mylem}

\begin{proof}
	\exref{def:typeA}より,$\Lsl (V)$ の基底は行列単位 $e_{ij}$ を使って
	\begin{align}
		\bigl\{\, e_{ij} - e_{ji} \bigm| 1 \le i\neq j \le \dim V \,\bigr\} \cup \bigl\{\, e_{ii} - e_{i+1,i+1} \bigm| 1 \le i \le \dim V - 1 \,\bigr\} = \comm{\{e_i\}}{\{e_j\}}
	\end{align}
	と書けた.よって $\Lsl (V) = \comm{\Lgl(V)}{\Lgl(V)}$ である.
	一方で $\mathfrak{g}$ が半単純なので系\ref{col:semisimple-decomp}-(1) より $\mathfrak{g} = \comm{\mathfrak{g}}{\mathfrak{g}}$ であるから,
	\begin{align}
		\phi(\mathfrak{g}) = \phi(\comm{\mathfrak{g}}{\mathfrak{g}}) = \comm{\phi(\mathfrak{g})}{\phi(\mathfrak{g})} \subset \comm{\Lgl(V)}{\Lgl(V)} = \Lsl(V)
	\end{align}
	が言えた.特に $\dim V = 0$ ならば $\dim \Lsl(V) = 1^2 - 1 = 0$ なので,$\Im \phi = 0$ である. 
\end{proof}

\begin{mylem}[label=lem:Whitehead]{Whiteheadの補題}
	\hyperref[def:semisimple-LieAlg]{半単純Lie代数}の\underline{有限次元}\hyperref[ax:g-module]{表現} $\phi \colon \mathfrak{g} \lto \Lgl (V)$ を与える.
	
	このとき
	\begin{align}
		\label{eq:cocycle}
		\forall x,\, y \in \mathfrak{g},\; f(\comm{x}{y}) = \phi(x) \circ f(y) - \phi(y) \circ f(x)
	\end{align}
	を充たす任意の\underline{$\mathbb{K}$-線型写像} $f \in \Hom{\mathbb{K}}(\mathfrak{g},\, V)$ に対して,
	ある $v \in V$ が存在して
	\begin{align}
		\forall x \in \mathfrak{g},\; f(x) = \phi(x)(v)
	\end{align}
	が成り立つ.
\end{mylem}

\begin{proof}
	\begin{description}
		\item[\textbf{case1: $\bm{\phi}$ が既約かつ忠実な場合}] 
		
		\eqref{eq:cocycle}を充たす任意の $f \in \Hom{\mathbb{K}}(\mathfrak{g},\, V)$ を1つとる.
		$\mathfrak{g}$ の基底 $\{e_\mu\}$ を1つ固定し,
		\begin{align}
			\beta \colon \mathfrak{g} \times \mathfrak{g} \lto \mathbb{K},\; (x,\, y) \lmto \Tr \bigl( \phi(x) \circ \phi(y) \bigr) 
		\end{align}
		を用いて補題\ref{lem:Casimir}-(1) の方法で対応する $\mathfrak{g}$ の基底 $\{e^\mu\}$ を作る.このとき
		\begin{align}
			v \coloneqq \sum_{\mu = 1}^{\dim \mathfrak{g}} \phi(e_\mu) \circ f(e^\mu) \in V
		\end{align}
		とおくと,$\forall x \in \mathfrak{g}$ に対して補題\ref{lem:Casimir}と同じ記号の下で
		\begin{align}
			\phi(x)(v) 
			&= \sum_{\mu = 1}^{\dim \mathfrak{g}} \phi(x) \circ \phi(e_\mu) \circ f(e^\mu) \\
			&= \sum_{\mu = 1}^{\dim \mathfrak{g}} \comm{\phi(x)}{\phi(e_\mu)} \circ f(e^\mu)  + \sum_{\mu = 1}^{\dim \mathfrak{g}} \phi(e_\mu) \circ \phi(x) \circ f(e^\mu) \\
			&= \sum_{\mu = 1}^{\dim \mathfrak{g}} \phi\bigl(\ad(x)(e_\mu)\bigr) \circ f(e^\mu) + \sum_{\mu = 1}^{\dim \mathfrak{g}} \phi(e_\mu) \circ \phi(x) \circ f(e^\mu) \\
			&= \sum_{\mu = 1}^{\dim \mathfrak{g}} a_\mu{}^\nu \phi(e_\nu) \circ f(e^\mu) + \sum_{\mu = 1}^{\dim \mathfrak{g}} \phi(e_\mu) \circ \phi(x) \circ f(e^\mu) \\
			c_\phi \circ f (x)
			&= \sum_{\mu=1}^{\dim \mathfrak{g}} \phi(e_\mu) \circ \phi(e^\nu) \circ f(x) \\
			&= \sum_{\mu=1}^{\dim \mathfrak{g}} \phi(e_\mu) \circ f(\comm{e^\nu}{x}) 
			+ \sum_{\mu=1}^{\dim \mathfrak{g}} \phi(e_\mu) \circ \phi(x) \circ f(e^\nu) \\
			&= -\sum_{\mu=1}^{\dim \mathfrak{g}} \phi(e_\mu) \circ f \bigl( \ad(x)(e^\nu) \bigr)  
			+ \sum_{\mu=1}^{\dim \mathfrak{g}} \phi(e_\mu) \circ \phi(x) \circ f(e^\nu) \\
			&= -\sum_{\mu=1}^{\dim \mathfrak{g}} b^\nu{}_\mu \phi(e_\mu) \circ f (e^\mu)
			+ \sum_{\mu=1}^{\dim \mathfrak{g}} \phi(e_\mu) \circ \phi(x) \circ f(e^\nu) 
		\end{align}
		と計算できるので,補題\ref{lem:Casimir}-(2) から
		\begin{align}
			\phi(x) (v) = c_\phi \circ f(x)
		\end{align}
		が言えた.仮定より\hyperref[ax:g-module]{ $\mathfrak{g}$-加群} $V$ は\hyperref[def:irr]{既約}なので,\hyperref[lem:Schur]{Schurの補題}-(1), (2) から\hyperref[def:g-module-hom]{ $\mathfrak{g}$-加群の準同型} $c_\phi \colon V \lto V$ は $\mathfrak{g}$-加群の同型であり,$c_\phi^{-1} (v) \in V$ が所望のベクトルとなる.

		\item[\textbf{case2: $\bm{\phi}$ が忠実とは限らない既約表現の場合}] 
		
		\eqref{eq:cocycle}を充たす任意の $f \in \Hom{\mathbb{K}}(\mathfrak{g},\, V)$ を1つとる.
		このとき $\forall \comm{x}{y} \in \comm{\Ker \phi}{\Ker \phi}$ に対して 
		\begin{align}
			f(\comm{x}{y}) = \phi(x) \circ f(y) - \phi(y) \circ f(x) = 0 \IFF \comm{x}{y} \in \Ker f
		\end{align}
		が言えるが,仮定より $\mathfrak{g}$ は半単純なので,系\ref{col:semisimple-decomp}-(3) よりそのイデアルである $\Ker \phi \subset \mathfrak{g}$ もまた半単純.故に系\ref{col:semisimple-decomp}-(1) から $\comm{\Ker \phi}{\Ker \phi} = \Ker \phi$ であり,
		\begin{align}
			\Ker \phi \subset \Ker f
		\end{align}
		がわかった.従ってこのとき\hyperref[prop:homo]{商ベクトル空間の普遍性}を使うことができ,以下の図式を可換にする $\overline{f} \in \Hom{\mathbb{K}}(V/\Ker \phi,\, \mathfrak{g})$ が一意的に存在する:
		\begin{center}
			\begin{tikzcd}[row sep=large, column sep=large]
				\mathfrak{g} \ar[d, "p"']\ar[r, "f"] &V \\
				\mathfrak{g}/\Ker \phi \arrow[ur, red, dashed, "\exists!\bar{f}"']&
			\end{tikzcd}
		\end{center}
		さらに商代数の普遍性から,表現 $\phi \colon \mathfrak{g} \lto \Lgl (V)$ は以下の図式を可換にする表現 $\overline{\phi}\colon \mathfrak{g}/\Ker \phi \lto \Lgl (V)$ に一意的に持ち上がる:
		\begin{center}
			\begin{tikzcd}[row sep=large, column sep=large]
				\mathfrak{g} \ar[d, "p"']\ar[r, "\phi"] &\Lgl (V) \\
				\mathfrak{g}/\Ker \phi \arrow[ur, red, dashed, "\exists!\bar{\phi}"']&
			\end{tikzcd}
		\end{center}
		このとき $\overline{\phi}$ は $\mathfrak{g} / \Ker \phi$ の忠実な既約表現であり,$\overline{f} \in \Hom{\mathbb{K}}(\mathfrak{g}/\Ker \phi,\, V)$ は\eqref{eq:cocycle}を充たす.
		よって \textsf{\textbf{case1}} からある $v \in V$ があって
		\begin{align}
			\forall x \in \mathfrak{g},\; \overline{f}(x + \Ker \phi) = \overline{\phi}(x + \Ker \phi) (v) 
			&\IFF \forall x \in \mathfrak{g},\; f(x) = \phi(x) (v) 
		\end{align}
		が成り立つ.
		% 逆に\eqref{eq:cocycle}を充たす任意の $g \in \Hom{\mathbb{K}}(\mathfrak{g}/\Ker \phi,\, V)$ に対して,\eqref{eq:cocycle}を充たす $f \in \Hom{\mathbb{K}}(\mathfrak{g},\, V)$ が一意的に存在して $f = g \circ p$ を充たす.

		\item[\textbf{case3: $\bm{\phi}$ が一般の場合}] 
		
		\eqref{eq:cocycle}を充たす任意の $f \in \Hom{\mathbb{K}}(\mathfrak{g},\, V)$ を1つ固定する.
		このときある $v \in V$ が存在して $\forall x \in \mathfrak{g},\; f(x) = \phi(x)(v)$ が成り立つことを $\dim V$ に関する数学的帰納法により示す.$\dim V = 1$ のとき,補題\ref{lem:Weyl}より $\phi$ が零写像なので $\forall v \in V$ に対して $\forall x \in \mathfrak{g},\; f(x) = \phi(x)(v)$ が成り立つ.
		
		 $\dim V > 1$ とする.\hyperref[ax:g-module]{$\mathfrak{g}$-加群} $V$ が\hyperref[def:irr]{既約}でないならば,\hyperref[def:sub-g-module]{部分 $\mathfrak{g}$-加群} $0 \subsetneq W \subsetneq V$ が存在する.標準的射影\footnote{このとき $V$ の $\mathbb{K}$-ベクトル空間としての構造しか見ない.} $p \colon V \lto V/W$ および $\forall x \in \mathfrak{g}$ について $W \subset \Ker \bigl( p \circ \phi(x)\bigr)$ であるから,\hyperref[prop:homo]{商代数の普遍性}より $\forall \phi(x) \in \phi(\mathfrak{g})$ に対して以下の図式を可換にする $\overline{\phi}(x) \in \Lgl(V/W)$ が一意的に存在する:
		\begin{center}
			\begin{tikzcd}[row sep=large, column sep=large]
				V \ar[d, "p"']\ar[r, "p \circ \phi(x)"] &V/W \\
				V/W \arrow[ur, red, dashed, "\exists!\overline{\phi}(x)"']&
			\end{tikzcd}
		\end{center}
		このとき写像
		\begin{align}
			\phi_1 \colon \mathfrak{g} \lmto \Lgl(V/W),\; x \lmto \bigl(v + W \mapsto \overline{\phi}(x)(v+W) \bigr)
		\end{align}
		はwell-definedなLie代数の準同型なので $\mathfrak{g}$ の表現である.
		$f_1 \coloneqq p \circ f \in \Hom{\mathbb{K}}(\mathfrak{g},\, V/W)$ は $\phi_1$ に関して\eqref{eq:cocycle}を充たすので,
		帰納法の仮定より\footnote{$\dim V/W < \dim V$ なので帰納法の仮定が使える.}ある $v_1 + W \in V/W$ が存在して
		\begin{align}
			\forall x \in \mathfrak{g},\; f_1(x) = f(x) + W = \phi_1(x) (v_1 + W)
		\end{align}
		が成り立つ.ここで $f_2 \in \Hom{\mathbb{K}}(\mathfrak{g},\, W)$ を
		\begin{align}
			f_2 (x) \coloneqq f(x) - \phi(x)(v_1)
		\end{align}
		と定義すると,$f_2$ は\hyperref[def:sub-g-module]{部分表現} $\phi|_W \colon \mathfrak{g} \lto \Lgl(W)$ に関して\eqref{eq:cocycle}を充たす.よって帰納法の仮定から\footnote{$\dim W < \dim V$ なので帰納法の仮定が使える.}ある $v_2 \in W$ が存在して
		\begin{align}
			\forall x \in \mathfrak{g},\; f_2(x) = \phi|_W (x)(v_2)
		\end{align}
		が成り立つ.以上より,$v \coloneqq v_1 + v_2 \in V$ とおけば
		\begin{align}
			\forall x \in \mathfrak{g},\; f(x) = f_2(x) + \phi(x)(v_1) = \phi|_W(x)(v_2) + \phi(x)(v_1) = \phi(x)(v)
		\end{align}
		が言えた.
		
	\end{description}
	
\end{proof}


\begin{mytheo}[label=thm:Weyl]{完全可約性に関するWeylの定理}
	$\phi \colon \mathfrak{g} \lto \Lgl (V)$ が\hyperref[def:semisimple-LieAlg]{半単純Lie代数}の\underline{有限次元}\hyperref[ax:g-module]{表現}ならば,
	$\mathfrak{\phi}$ は\hyperref[def:irr]{完全可約}である.
\end{mytheo}

\begin{proof}
	$V$ の任意の\hyperref[def:sub-g-module]{部分 $\mathfrak{g}$-加群} $W \subset V$ を1つ固定する.
	このとき命題\ref{prop:reducible-1}より,部分 $\mathfrak{g}$-加群 $W^c \subset V$ が存在して $V \cong W \oplus W^c$ が成り立つことを示せば良い.

	$\End V$ の\underline{部分ベクトル空間} $L_W$ を
	\begin{align}
		L_W \coloneqq \bigl\{\, t \in \End V \bigm| t(V) \subset W,\; t(W) = 0  \,\bigr\} 
	\end{align}
	と定める.$L_W$ への $\mathfrak{g}$ の左作用を
	\begin{align}
		x \btr t \coloneqq \comm{\phi(x)}{t}
	\end{align}
	と定義すると,$W$ が部分 $\mathfrak{g}$-加群であることおよび定義\ref{def:gmod-hom}より $L_W$ は $\mathfrak{g}$-加群になる.
	i.e. 
	\begin{align}
		\tilde{\phi} \colon \mathfrak{g} \lto \Lgl (L_W),\; x \lmto \bigl( t \lmto x \btr t \bigr) 
	\end{align}
	は半単純Lie代数 $\mathfrak{g}$ の有限次元表現である.
	
	ここで \underline{$\mathbb{K}$-ベクトル空間} $W$ への射影演算子\footnote{$p^2 = p$ かつ $p|_W = \mathrm{id}_W$} $p \colon V \lto V$ を1つとり,$f \in \Hom{\mathbb{K}}(\mathfrak{g},\, L_W)$ を
	\begin{align}
		f(x) \coloneqq \comm{p}{\phi(x)}
	\end{align}
	と定義しよう.このとき $\forall x,\, y \in \mathfrak{g}$ に対して\hyperref[ax:LieAlg]{Jacobi恒等式}から
	\begin{align}
		f(\comm{x}{y})
		&= \comm{p}{\comm{\phi(x)}{\phi(y)}} \\
		&= \comm{\phi(x)}{\comm{p}{\phi(y)}} - \comm{\phi(y)}{\comm{p}{\phi(x)}} \\
		&= \comm{\phi(x)}{f(y)} - \comm{\phi(y)}{f(x)} \\
		&= \tilde{\phi}(x) \circ f(y) - \tilde{\phi}(y) \circ f(x)
	\end{align}
	が成り立つので,補題\ref{lem:Whitehead}からある $t \in L_W$ が存在して
	\begin{align}
		\forall x \in \mathfrak{g},\; f(x) = \tilde{\phi}(x)(t) = \comm{\phi(x)}{t}
	\end{align}
	が成り立つ.よってこのとき $\forall x \in \mathfrak{g}$ に対して
	\begin{align}
		\comm{\phi(x)}{p+t} = \tilde{\phi}\bigl(\phi(x)\bigr)\bigl( p+t \bigr) = - \comm{p}{\phi(x)} + \comm{\phi(x)}{t} = -f(x) + \comm{\phi(x)}{t} = 0
	\end{align}
	が言えた.i.e. $p+t \in \End V$ は\hyperref[def:g-module-hom]{$\mathfrak{g}$-加群の準同型}である.
	さらに $t \in L_W$ であることから,$(p+t)(V) = W$ かつ $(p+t) \circ (p+t)|_W = \mathrm{id}_W$ が言える.i.e. $\mathfrak{g}$-加群の短完全列
	\begin{align}
		0 \hookrightarrow \Ker (p+t) \hookrightarrow V \xrightarrow{p+t} W \lto 0
	\end{align}
	は\hyperref[prop:split]{分裂}し,\hyperref[def:gmod-directsum]{$\mathfrak{g}$-加群の直和}として
	\begin{align}
		V \cong W \oplus \Ker (p+t)
	\end{align}
	が言えた.
\end{proof}



\subsection{Jordan分解の保存}

この小節でも引き続き $\mathbb{K}$ を\underline{標数 $0$ の代数閉体}とする.

部分Lie代数 $\mathfrak{h} \subset \mathfrak{g}$ に関して
\begin{align}
	\ad_{\mathfrak{h}}(x) \colon \mathfrak{h} \lto \mathfrak{g},\; v \lmto \comm{x}{v}
\end{align}
と定義する\footnote{$x \in \mathfrak{h}$ ならば,\hyperref[def:sub-LieAlg]{部分Lie代数の定義}から $\Im \ad_\mathfrak{h}(x) \subset \mathfrak{h}$ となる.この意味で $\ad_{\mathfrak{h}} \in \Lgl(\mathfrak{h})$ と書いても良い.}.

\begin{mytheo}[label=thm:JC]{Jordan-Chevalley分解と抽象Jordan分解}
	$V$ を\underline{有限次元} $\mathbb{K}$-ベクトル空間とし,$\mathfrak{g} \subset \Lgl(V)$ を\hyperref[def:semisimple-LieAlg]{半単純線型Lie代数}とする.
	
	このとき $\forall x \in \mathfrak{g}$ の\hyperref[prop:Jordan-Chevalley]{Jordan-Chevalley分解} $x = x_s + x_n$ について $x_s,\, x_n \in \mathfrak{g}$ が成り立つ.
	特に,$x$ の\hyperref[def:abstruct-JC]{抽象Jordan分解} $s_x,\, n_x \in \mathfrak{g}$ に関して $x_s = s_x,\; x_n = n_x$ が成り立つ.
\end{mytheo}

\begin{proof}
	後半の主張はJordan-Chevalley分解および抽象Jordan分解の一意性より従うので,前半を示せば良い.

	$\forall x \in \mathfrak{g}$ を1つ固定し,$x$ のJordan-Chevalley分解 $x = x_s + x_n$ をとる.
	$\ad_{\Lgl(V)}(x) (\mathfrak{g}) = \comm{x}{\mathfrak{g}} \subset \mathfrak{g}$ なので命題\ref{prop:Jordan-Chevalley}-(3) より $\ad_{\Lgl(V)}(x_s)(\mathfrak{g}) \subset \mathfrak{g},\; \ad_{\Lgl(V)}(x_n)(\mathfrak{g}) \subset \mathfrak{g}$ が言える.
	i.e. \hyperref[def:normalizer-LieAlg]{正規化代数}の言葉を使うと
	$x_s,\, x_n \in N_{\Lgl(V)} (\mathfrak{g}) \eqqcolon N$ が言える\footnote{補題\ref{lem:Weyl}より $\mathfrak{g} \subset \Lsl(V)$ なので $\mathfrak{g} \subsetneq N$ であることに注意}.

	包含準同型 $\iota\colon \mathfrak{g} \lto \Lgl (V),\; x \lmto x$ によって $V$ を\hyperref[ax:g-module]{$\mathfrak{g}$-加群}と見做し,$V$ の任意の\hyperref[def:sub-g-module]{部分 $\mathfrak{g}$-加群} $W$ をとる.このとき $\Lgl(V)$ の部分Lie代数 $L_W \subset \Lgl(V)$ を
	\begin{align}
		L_W \coloneqq \bigl\{\, y \in \Lgl (V) \bigm| y(W) \subset W \AND \Tr (y|_W) = 0 \,\bigr\}
	\end{align}
	と定義する\footnote{例えば $L_V = \Lsl(V)$ である}.
	仮定より $\mathfrak{g}$ は半単純Lie代数なので系\ref{col:semisimple-decomp}-(1) より $\mathfrak{g} = \comm{\mathfrak{g}}{\mathfrak{g}}$ であり,$\mathfrak{g} \subset L_W$ が言える.さらにこのとき命題\ref{prop:Jordan-Chevalley}-(3) より $x_s,\, x_n \in L_W$ も言える\footnote{$\Tr (x_n) = 0$ なので,$\Tr (x) = \Tr(x_s) = 0$.}.

	ここで
	\begin{align}
		\mathfrak{g}' \coloneqq N \cap \left(\bigcap_{\substack{W \subset V, \\ \text{部分} \mathfrak{g}\text{-加群}}} L_W \right)
	\end{align}
	とおくと,$\mathfrak{g}'$ は $\mathfrak{g}$ を\hyperref[def:ideal-LieAlg]{イデアル}としてもつ $N$ の部分Lie代数であり,$\forall x \in \mathfrak{g}$ に対して $x_s,\, x_n \in \mathfrak{g}'$ だとわかる.
	
	最後に $\mathfrak{g} = \mathfrak{g}'$ が成り立つことを示す.
	随伴表現 $\ad_{\mathfrak{g}'} \colon \mathfrak{g}' \lto \Lgl (\mathfrak{g}')$ の $\mathfrak{g} \subset \mathfrak{g}'$ への制限 $\ad_{\mathfrak{g}'}|_{\mathfrak{g}} \colon \mathfrak{g} \lto \Lgl(\mathfrak{g}')$ によって $\mathfrak{g}'$ を有限次元 $\mathfrak{g}$-加群と見做すと $\mathfrak{g} \subset \mathfrak{g}'$ 自身は $\mathfrak{g}'$ の部分 $\mathfrak{g}$-加群なので\footnote{$\forall x \in \mathfrak{g}$ に対して $\ad_{\mathfrak{g}'}|_{\mathfrak{g}}(x)(\mathfrak{g}) = \comm{x}{\mathfrak{g}} \subset \mathfrak{g}$.},命題\ref{prop:reducible-1}および\hyperref[thm:Weyl]{Weylの定理}よりある $\mathfrak{g}'$ の部分 $\mathfrak{g}$-加群 $\mathfrak{h}$ が存在して $\mathfrak{g}' = \mathfrak{g} \oplus \mathfrak{h}$ と書ける.
	示すべきは $\mathfrak{h} = 0$ である.
	ここで $V$ の任意の\hyperref[def:irr]{既約}な部分 $\mathfrak{g}$-加群 $W$ をとる\footnote{ここでは包含準同型 $\iota \colon \mathfrak{g} \lto \Lgl(V)$ によって $V$ を $\mathfrak{g}$ 加群と見做している.}.
	$\forall y \in \mathfrak{h}$ を1つとると,$\comm{\mathfrak{g}}{\mathfrak{g}'} \subset \mathfrak{g}$\footnote{$\mathfrak{g}' \subset N$ だから}であることから $\comm{\iota(\mathfrak{g})}{y} = \comm{\mathfrak{g}}{y} = 0$,i.e. $y \colon V \lto V$ は\hyperref[def:g-module-hom]{$\mathfrak{g}$-加群の準同型}である.よって\hyperref[col:Schur-closed]{代数閉体上のSchurの補題}から $y|_W \colon W \lto W$ はスカラー倍である.
	一方で $y \in L_W$ でもあるので $\Tr(y|_W) = 0$ であり,$y|_W = 0$ だと分かった.
	Weylの定理より $V$ は既約な部分 $\mathfrak{g}$-加群の直和であり,$W$ は任意だったので $y = 0$ が言えた.i.e. $\mathfrak{h} = 0$ が示された.

\end{proof}

\begin{mycol}[label=col:JC]{}
	$\mathfrak{g}$ を\hyperref[def:semisimple-LieAlg]{半単純Lie代数},$\phi \colon \mathfrak{g} \lto \Lgl(V)$ をその\underline{有限次元}表現とする.

	このとき $\forall x \in \mathfrak{g}$ の\hyperref[def:abstruct-JC]{抽象Jordan分解} $x = s_x + n_x$ に対して,$\phi(x) = \phi(s_x) + \phi(n_x) \in \Lgl(V)$ は\hyperref[prop:Jordan-Chevalley]{Jordan-Chevalley分解}である.i.e. $\phi(x)_s = \phi(s_x),\; \phi(x)_n = \phi(n_x)$ が成り立つ.
\end{mycol}

\begin{proof}
	$s_x$ は半単純なので $\ad (s_x) \colon \mathfrak{g} \lto \mathfrak{g}$ も半単純.i.e. $\ad(s_x)$ は対角化可能なので $\ad(s_x)$ の固有ベクトル全体がなす集合は $\mathfrak{g}$ の基底となる.
	従って $\ad_{\phi(\mathfrak{g})} \bigl( \phi(s_x) \bigr) \colon \phi(\mathfrak{g}) \lto \phi(\mathfrak{g})$ の $\phi(\mathfrak{g})$ の固有ベクトル全体がなす集合は $\phi(\mathfrak{g})$ の基底となる.i.e. $\ad_{\phi(\mathfrak{g})} \bigl( \phi(s_x) \bigr)$ は半単純である.
	同様に $\ad_{\phi(\mathfrak{g})} \bigl( \phi(n_x) \bigr)$ が冪零であることもわかるので,$\phi(x) = \phi(s_x) + \phi(n_x)$ は半単純Lie代数 $\phi(\mathfrak{g})$ における $\phi(x)$ の\hyperref[def:abstruct-JC]{抽象Jordan分解}である.よって定理\ref{thm:JC}からこれはJordan-Chevalley分解と一致する.
\end{proof}

\begin{marker}
	定理\ref{thm:JC}および系\ref{col:JC}の意味で\hyperref[prop:Jordan-Chevalley]{Jordan-Chevalley分解}と\hyperref[def:abstruct-JC]{抽象Jordan分解}は同一視できるので,以後抽象Jordan分解のことも $x = s_x + n_x$ の代わりに $x = \bm{x_s} + \bm{x_n}$ と表記する.
\end{marker}


\section{$\lsl{2}{\mathbb{K}}$ の表現}

この節において常に $\mathfrak{g} = \Lsl(2,\, \mathbb{K})$ とし,$\mathfrak{g}$ の全ての表現は有限次元であるとする.
\exref{def:typeA}に倣って $\mathfrak{g}$ の基底は
\begin{align}
	x &\coloneqq \mqty[0 & 1 \\ 0 & 0], \\
	y &\coloneqq \mqty[0 & 0 \\ 1 & 0], \\
	h &\coloneqq \mqty[1 & 0 \\ 0 & -1]
\end{align}
を採用する.このとき
\begin{align}
	\label{eq:sl2}
	\comm{h}{x} &= 2x, \\
	\comm{h}{y} &= -2y, \\
	\comm{x}{y} &= h
\end{align}
が成り立つ.

\subsection{ウエイトと極大ベクトル}

任意の\hyperref[ax:g-module]{$\mathfrak{g}$ の表現} $\phi \colon \mathfrak{g} \lto \Lgl(V)$ を与える.
$h$ は\hyperref[def:semisimple-end]{半単純}なので系\ref{col:JC}より $\phi(h)$ を対角化する $V$ の基底がある.よって
\begin{align}
	\label{def:weight-space}
	V_\lambda \coloneqq \bigl\{\, v \in V \bigm| h \btr v = \lambda v \,\bigr\} 
\end{align}
とおくと $V$ は異なる $\lambda$ に関する $V_\lambda$ の\hyperref[prop:subvec-directsum]{内部直和}になる(\hyperref[thm:eigen-decomp1]{固有空間分解})\footnote{$\lambda$ が $h$ の固有値でなければ $V_\lambda = 0$ となる.}:
\begin{align}
	\label{def:weight-space-decomp}
	V = \bigoplus_{\lambda \in \mathbb{K}} V_\lambda
\end{align}
特に $V_\lambda \neq 0$ のとき $\lambda$ を $h$ の\textbf{ウエイト} (weight) と呼び,$V_\lambda$ を\textbf{ウェイト空間} (weight space) と呼ぶ.

\begin{mylem}[label=lem:sl2-1]{}
	$v \in V_\lambda \IMP x \btr v \in V_{\lambda +2},\; y \btr v \in V_{\lambda - 2}$
\end{mylem}

\begin{proof}
	\begin{align}
		h \btr (x \btr v) = \comm{h}{x} \btr v + x \btr h \btr v = (\lambda + 2) x \btr v
	\end{align}
\end{proof}

$\dim V < \infty$ なので,$V_\lambda \neq 0 \AND V_{\lambda + 2} = 0$ を充たす $\lambda \in \mathbb{K}$ が存在する.このとき $v \in V_\lambda \setminus \{0\}$ のことをウエイト $\lambda$ の\textbf{極大ベクトル} (maximal vector) と呼ぶ.

\subsection{既約表現の分類}

\underline{$\mathfrak{g}$-加群 $V$ は\hyperref[def:irr]{既約}であるとする}.
極大ベクトル $v_0 \in V_\lambda$ をとり,
\begin{align}
	v_i \coloneqq 
	\begin{cases}
		0, & i = -1 \\
		& \\
		\displaystyle \frac{1}{i!} y^i \btr v_0, & i \ge 0 \\
	\end{cases}
\end{align}
とおく.

\begin{mylem}[label=lem:sl2-2]{}
	\begin{enumerate}
		\item $h \btr v_i = (\lambda - 2i) v_i$
		\item $y \btr v_i = (i+1) v_{i+1}$
		\item $x \btr v_i = (\lambda - i+1) v_{i-1}\WHERE i \ge 0$
	\end{enumerate}
\end{mylem}

\begin{proof}
	\begin{enumerate}
		\item 補題\ref{lem:sl2-1}より明らか.
		\item $v_i$ の定義より
		\begin{align}
			y \btr v_i = \frac{1}{i!} y^{i+1} \btr v_0 = (i+1) v_{i+1}
		\end{align}
		\item $i$ に関する数学的帰納法により示す.$i=0$ のときは自明.$i > 0$ とする.
		帰納法の仮定より $x \btr v_{i-1} = (\lambda - i + 2) v_{i-2}$ であるから,
		\begin{align}
			x \btr v_{i} 
			% &= \frac{1}{i!} x \btr y \btr y^{i-1} \btr v_0 \\
			&= \frac{1}{i} x \btr y \btr v_{i-1} \\
			&= \frac{1}{i} \comm{x}{y} \btr v_{i-1} + \frac{1}{i} y \btr (x \btr v_{i-1}) \\
			&= \frac{\lambda-2i+2}{i} v_{i-1} + \frac{(\lambda-i+2)(i-1)}{i} v_{i-1} \\
			&= (\lambda - i + 1)v_{i-1}
		\end{align}
		が言えて帰納法が完成する.
	\end{enumerate}
	
\end{proof}

補題\ref{lem:sl2-2}-(1) より $0$ でない $v_i \in V$ たちは全て線型独立であるが,$\dim V < \infty$ である.
そこで
\begin{align}
	\label{eq:maximal-weight}
	m \coloneqq \min \bigl\{\, m \in \mathbb{Z}_{\ge 0} \bigm| v_m \neq 0 \AND v_{m+1} = 0 \,\bigr\} 
\end{align}
とおく.$\forall i > 0$ に対して帰納的に $v_{m+i} = 0$ がわかるので,補題\ref{lem:sl2-2}-(1), (2), (3) より $\Span \{v_0,\, \dots,\, v_m\} \subset V$ は非零な\hyperref[def:def:sub-g-module]{部分 $\mathfrak{g}$-加群}である.$\mathfrak{g}$ は既約だったので $V = \Span \{v_0,\, \dots,\, v_m\}$ が言えた.

補題\ref{lem:sl2-2}-(3) を $i=m+1$ の場合に適用すると $0 = (\lambda - m)v_m$ となり,$\lambda = m = \dim V - 1 \in \mathbb{Z}_{\ge 0}$ がわかる.この $m$ を $\mathfrak{g}$-加群 $V$ の\textbf{最高ウエイト} (highest weight) と呼ぶ.
さらに,補題\ref{lem:sl2-2}-(1) より $h$ のウエイトは全て異なる.i.e. $0 \le \forall \mu \le m$ に対して,対応するウエイト空間の次元 $\dim V_{m - 2\mu} = 1$ である.特に最高ウエイトは $\dim V$ によって一意に決まるので,$v_0 \in V_m$ は零でないスカラー倍を除いて一意に定まる.

以上の考察をまとめて次の定理を得る:

\begin{mytheo}[label=thm:irr-sl2,breakable]{}
	$V$ を $\mathfrak{g} = \Lsl(2,\, \mathbb{K})$ の\hyperref[def:irr]{既約表現}とする.
	\begin{enumerate}
		\item $m \coloneqq \dim V - 1$ とおくと \underline{$\mathbb{K}$-ベクトル空間として}
		\begin{align}
			V = \bigoplus_{\mu = 0}^m V_{m-2\mu}\quad \WHERE 0 \le \forall \mu \le m,\; \dim V_{m - 2\mu} = 1
		\end{align}
		が成り立つ.
		\item $V$ の極大ベクトルは零でないスカラー倍を除いて一意に決まり,そのウエイトは $m$ である.
		\item $\mathfrak{g}$ の $V$ への作用は補題\ref{lem:sl2-2}によって完全に決まる.
	\end{enumerate}
\end{mytheo}

\begin{mycol}[label=col:sl2]{}
	$V$ を任意の有限次元 $\mathfrak{g} = \Lsl(2,\, \mathbb{K})$-加群とする.このとき $\phi(h) \colon V \lto V$  の固有値は全て整数で,かつ自身にマイナス符号をつけたものとちょうど同じ回数だけ出現する.
	さらに,$V$ の任意の既約表現への内部直和分解において,直和因子の個数はちょうど $\dim V_0 + \dim V_1$ 個である.
\end{mycol}

\begin{proof}
	$V = 0$ なら明らか.
	$V \neq 0$ とする.\hyperref[thm:Weyl]{Weylの定理}により $V$ を既約な部分 $\mathfrak{g}$-加群の直和に分解すると,既約な部分 $\mathfrak{g}$-加群は全て定理\ref{thm:irr-sl2}-(1) の形をしているので前半が従う.

	後半を示す.$V$ の既約な部分 $\mathfrak{g}$-加群への直和分解を
	\begin{align}
		V = \bigoplus_i W_i
	\end{align}
	と書くと,定理\ref{thm:irr-sl2}より $\forall i$ に対して何かしらの $m_i \ge 0$ が存在して $W_i \cong \bigoplus_{\mu = 0}^{m_i} V_{m_i - 2\mu}$ と書ける.
	逆に $\forall m \ge 0$ に対して,$\bigoplus_{\mu = 0}^{m} V_{m - 2\mu}$ と\hyperref[def:gmod-hom]{$\mathfrak{g}$-加群として同型}な全ての $W_i$ の直和を $V^{(m)}$ と書くと,
	\begin{align}
		V = \bigoplus_{m=0}^\infty V^{(m)}
	\end{align}
	となる.よって $V^{(m)}$ の直和因子の個数を $k_m$ とかくと
	\begin{align}
		\dim V_\lambda = \sum_{\substack{m \ge \abs{\lambda}, \\ m \equiv \lambda \mod 2}} k_m
	\end{align}
	となる.特に
	\begin{align}
		\dim V_0 + \dim V_1 = \sum_{m=0}^\infty k_m
	\end{align}
	なので示された.
\end{proof}


\section{ルート空間分解}

この節では体 $\mathbb{K}$ を\underline{標数 $0$ の代数閉体}とし,
$\mathfrak{g} \neq 0$ を体 $\mathbb{K}$-上の\hyperref[def:semisimple-LieAlg]{半単純Lie代数}とする.

\subsection{極大トーラスとルート}

もし $\forall x \in \mathfrak{g}$ が冪零ならば $\ad(x) \in \Lgl (\mathfrak{g})$ もまた冪零であり\footnote{$\ad \colon \mathfrak{g} \lto \Lgl(\mathfrak{g})$ はLie代数の準同型なので.},従って\hyperref[thm:Engel]{Engelの定理}から $\mathfrak{g}$ は\hyperref[def:nilpotent-LieAlg]{冪零Lie代数}となる.
然るにこれは $\mathfrak{g}$ が\hyperref[def:semisimple-LieAlg]{半単純Lie代数}であることに矛盾する
\footnote{
	一般に\hyperref[def:nilpotent-LieAlg]{冪零Lie代数} $\IMP$ \hyperref[def:solvable-LieAlg]{可解Lie代数}だが,系\ref{col:semisimple-decomp}-(1) より $\mathfrak{g}$ が半単純Lie代数 $\IMP$ $\mathfrak{g}$ は可解Lie代数でない $\IMP$ $\mathfrak{g}$ は冪零Lie代数でない.
	
	もしくは,命題\ref{prop:nilpo-basic}-(4)から $\mathfrak{g} \neq 0$ かつ $\mathfrak{g}$ が冪零Lie代数 $\IMP$ $\mathfrak{g}$ は半単純Lie代数でない
}
.

従って冪零でない $x \in \mathfrak{g}$ が存在するが,このとき $x$ の\hyperref[def:abstruct-JC]{抽象Jordan分解}\footnote{系\ref{col:JC}の注意に従って抽象Jordan分解をこのように表記する.} $x = x_s + x_n\WHERE x_s,\, x_n \in \mathfrak{g}$ において $x_s \neq 0$ である.
このことは,\hyperref[def:semisimple-end]{半単純}成分のみで構成される $\mathfrak{g}$ の\underline{自明でない}部分Lie代数の存在を示唆する.

\begin{mydef}[label=def:toral-subLieAlg]{トーラス}
	$\mathfrak{g}$ の \underline{$0$ でない}部分Lie代数 $\mathfrak{t} \subset \mathfrak{g}$ が\textbf{トーラス} (toral subalgebra) であるとは,$\forall x \in \mathfrak{t}$ の\hyperref[def:abstruct-JC]{抽象Jordan分解} $x = x_s + x_n$ において $x_n = 0$ であることを言う.
\end{mydef}

\begin{mylem}[label=lem:torus]{トーラスは可換}
	$\mathfrak{g}$ の任意の\hyperref[def:toral-subLieAlg]{トーラス} $\mathfrak{t} \subset \mathfrak{g}$ はLieブラケットについて可換である.
\end{mylem}

\begin{proof}
	$\mathfrak{t}$ を $\mathfrak{g}$ のトーラスとする.
	$\forall x \in \mathfrak{t}$ を1つ固定する.示すべきは $\ad_{\mathfrak{t}} (x) = 0$ が成り立つことであるが,
	$\mathfrak{t}$ がトーラスであるという仮定より $\ad (x) \in \Lgl (\mathfrak{g})$ は\hyperref[def:semisimple-end]{半単純} $\IFF$ 対角化可能だから,$\ad_{\mathfrak{t}}(x)$ の固有値が全て $0$ であることを背理法により示す.

	$\exists y \in \mathfrak{t} \setminus \{0\},\; \exists \lambda \in \mathbb{K} \setminus \{0\},\; \comm{x}{y} = \lambda y$ を仮定する.このとき $\ad_{\mathfrak{t}}(y)(x) = -\lambda y$ は $\ad_{\mathfrak{t}}(y)$ の固有値 $0$ に属する固有ベクトルである.
	一方で $y \in \mathfrak{t}$ なので $\ad_{\mathfrak{t}}(y) \in \Lgl(\mathfrak{t})$ は対角化可能であり,$\ad_{\mathfrak{t}}(y)$ の固有ベクトル $v_1,\, \dots,\, v_{\dim \mathfrak{t}} \in \mathfrak{t} \setminus \{0\}$ は $\mathfrak{t}$ の基底をなす\footnote{固有ベクトル $v_\mu$ は固有値 $\lambda_\mu$ に属するとする.なお,以下ではEinsteinの規約を使わない.}. 
	従って $x = \sum_{\mu=1}^{\dim \mathfrak{t}} x^\mu v_\mu$ と書けるが,このとき
	\begin{align}
		0 = \ad_{\mathfrak{t}} (y) \bigl(\ad_{\mathfrak{t}}(y)(x)\bigr) &= \sum_{\mu=1}^{\dim \mathfrak{t}} (\lambda_\mu)^2 x^\mu v_\mu 
	\end{align}
	が成り立つ.$v_\mu$ の線型独立性から $(\lambda_\mu)^2 x^\mu = 0$ だが,$x$ は任意だったので $\lambda_1 = \cdots = \lambda_{\dim \mathfrak{t}} = 0$\footnote{体は\hyperref[def:domain-basic]{整域}なので.},i.e. $\ad_{\mathfrak{t}}(y) = 0$ が分かった.然るにこれは仮定に矛盾し,背理法が完成した.
\end{proof}

% \begin{mydef}[label=def:maximal-toral-subLieAlg]{極大トーラス}
	
% \end{mydef}

$\mathfrak{g}$ の\hyperref[def:toral-subLieAlg]{トーラス}のうち極大なもの\footnote{トーラス $\mathfrak{t} \subsetneq \mathfrak{g}$ が存在して $\mathfrak{h} \subset \mathfrak{t}$ が成り立つならば $\mathfrak{t} = \mathfrak{h}$.}を1つ固定し,それを $\mathfrak{h} \subset \mathfrak{g}$ と書こう.
$\mathfrak{h}$ は\textbf{極大トーラス} (maximal toral subalgebra) と呼ばれる.
\begin{marker}
	極大トーラスを $\mathfrak{t}$ ではなく $\mathfrak{h}$ と書くのは慣習である.
	後述する\textbf{Cartan部分代数} (Cartan subalgebra; CSA) を $\mathfrak{h}$ と書く慣習と対応しているように思われる?
\end{marker}

ここで,与えられた $h \in \mathfrak{h}$ に対して線型変換 $\ad_{\mathfrak{g}}(h) \in \Lgl(\mathfrak{g})$ の固有値のうちのどれか1つを返す写像
\begin{align}
	\alpha \colon \mathfrak{h} \lto \mathbb{K},\; h \lmto \alpha(h) \WHERE \exists x \in \mathfrak{g}\setminus \{0\},\; \alpha(h)x \coloneqq \ad_{\mathfrak{g}}(h)(x)
\end{align}
を考えよう.このとき $\mathfrak{g}_\alpha$ の\underline{部分ベクトル空間}\footnote{$\alpha \neq 0$ のとき $\mathfrak{g}_\alpha$ は部分Lie代数\underline{ではない}.実際,$\forall x,\, y \in \mathfrak{g}_\alpha$ および $\forall h \in \mathfrak{h}$ に対して $\ad_{\mathfrak{g}}(h)(\comm{x}{y}) = \comm{\ad_{\mathfrak{g}}(h)(x)}{y} + \comm{x}{\ad_{\mathfrak{g}}(h)(y)} = 2\alpha(h) \comm{x}{y}$ となり,$\alpha(h) \neq 0$ ならば $\comm{x}{y} \not\in \mathfrak{g}_\alpha$ である.} $\mathfrak{g}_\alpha$ を
\begin{align}
	\label{def:ga}
	\mathfrak{g}_\alpha \coloneqq \bigl\{\, x \in \mathfrak{g} \bigm| \forall h \in \mathfrak{h},\; \ad_{\mathfrak{g}}(h)(x) = \alpha(h) x \,\bigr\} 
\end{align}
と定義すると,$\mathfrak{g}_\alpha \neq 0$ ならば $\alpha \in \mathfrak{h}^*$ だとわかる.
\footnote{
	% まず\hyperref[def:eigen]{固有値の定義}から $x \in \mathfrak{g}_\alpha \setminus \{0\}$ が存在する.
	$\mathfrak{g}_\alpha \neq 0$ なので $x \in \mathfrak{g}_\alpha \setminus \{0\}$ が存在する.
	このとき $\forall h,\, h_1,\, h_2 \in \mathfrak{h},\; \forall \lambda \in \mathbb{K}$ に対して $\alpha(h)x = \ad_{\mathfrak{g}}(h)(x),\; \alpha(h_i)x = \ad_{\mathfrak{g}}(h_i)(x)$ が成り立つが,$\mathfrak{h}$ は $\mathfrak{g}$ の部分Lie代数なので $h_1 + h_2,\, \lambda h \in \mathfrak{h}$ であること,および $\ad_{\mathfrak{g}} \colon \mathfrak{g} \lto \Lgl(\mathfrak{g})$ の線型性から
	\begin{align}
		\alpha(h_1 + h_2)x &= \ad_{\mathfrak{g}}(h_1 + h_2)(x) = \ad_{\mathfrak{g}}(h_1)(x) + \ad_{\mathfrak{g}}(h_2)(x) = \bigl(\alpha(h_1) + \alpha(h_2)\bigr)x, \\
		\alpha(\lambda h)x &= \ad_{\mathfrak{g}}(\lambda h)(x) = \lambda \ad_{\mathfrak{g}}(h)(x) = \bigl(\lambda \alpha(h)\bigr)x
	\end{align}
	が言える.$x \neq 0$ なので $\alpha(h_1 + h_2) = \alpha(h_1) + \alpha(h_2),\; \alpha(\lambda h) = \lambda \alpha(h)$,i.e. $\alpha \colon \mathfrak{h} \lto \mathbb{K}$ が線型写像であることが分かった.
}
一方,補題\ref{lem:torus}より $\forall h,\, k \in \mathfrak{h}$ に対して $\comm{\ad_{\mathfrak{g}}(h)}{\ad_{\mathfrak{g}}(k)} = \ad_{\mathfrak{g}}(\comm{h}{k}) = \ad_{\mathfrak{g}}(0) = 0$ が成り立つ,i.e. $\ad_{\mathfrak{g}}(h),\; \ad_{\mathfrak{g}} (k) \in \Lgl(\mathfrak{g})$ は互いに可換な線型変換なので,$\forall \ad_{\mathfrak{g}}(h) \in \ad_{\mathfrak{g}}(\mathfrak{h})$ は\underline{同時対角化可能}である.
従って $\mathfrak{g}$ の\hyperref[thm:eigen-decomp-1]{固有空間分解}
\begin{align}
	\label{eq:root-decomp1}
	\mathfrak{g} = \bigoplus_{\alpha \in \mathfrak{h}^*} \mathfrak{g}_\alpha
\end{align}
が成り立つ.特に $\dim \mathfrak{g} < \infty$ であることから
% $\mathfrak{g}$ の\textbf{ルート} (roots) と呼ばれる
集合
\begin{align}
	\Phi \coloneqq \bigl\{\, \alpha \in \mathfrak{h}^* \setminus \{0\} \bigm| \mathfrak{g}_\alpha \neq 0 \,\bigr\} 
\end{align}
は有限集合になるので,\eqref{eq:root-decomp1}は有限個の直和として
\begin{align}
	\label{eq:root-decomp}
	\mathfrak{g} = C_{\mathfrak{g}}(\mathfrak{h}) \oplus \bigoplus_{\alpha \in \Phi} \mathfrak{g}_\alpha
\end{align}
と書くこともできる\footnote{\hyperref[def:normalizer-LieAlg]{中心化代数の定義}を思い出すと $\mathfrak{g}_0 = C_{\mathfrak{g}}(\mathfrak{h})$ である.}.
$\Phi$ の元のことを $\mathfrak{g}$ の\textbf{ルート} (root) と呼び,\eqref{eq:root-decomp}のことを $\mathfrak{g}$ の\textbf{ルート空間分解} (root space decomposition)\footnote{\textbf{Cartan分解} (Cartan decomposition) と呼ぶこともある.} と呼ぶ.

\begin{myprop}[label=prop:root-decomp-basic1]{}
	$\kappa \in L(\mathfrak{g},\, \mathfrak{g};\, \mathbb{K})$ を $\mathfrak{g}$ の\hyperref[def:Killing-form]{Killing形式}とする.
	$\forall \alpha,\, \beta \in \mathfrak{h}^*$ に対して以下が成り立つ:
	\begin{enumerate}
		\item $\comm{\mathfrak{g}_\alpha}{\mathfrak{g}_\beta} \subset \mathfrak{g}_{\alpha + \beta}$
		\item $\alpha \neq 0 \IMP \forall x \in \mathfrak{g}_\alpha$ に対して $\ad_{\mathfrak{g}}(x) \in \Lgl(\mathfrak{g})$ は冪零である.
		\item $\alpha + \beta \neq 0 \IMP \kappa$ について $\mathfrak{g}_\alpha$ と $\mathfrak{g}_\beta$ は直交する.i.e. $\kappa|_{\mathfrak{g}_\alpha \times \mathfrak{g}_\beta} = 0$
		\item $\kappa$ に関する $\mathfrak{g}_\alpha$ の\hyperref[def:radical-bilinear]{直交補空間}は
		\begin{align}
			\mathfrak{g}_\alpha{}^\perp = \bigoplus_{\gamma \in \mathfrak{h}^* \setminus \{-\alpha\}} \mathfrak{g}_{\gamma}
		\end{align}
		と書ける.
	\end{enumerate}
\end{myprop}

\begin{proof}
	\begin{enumerate}
		\item $\forall x \in \mathfrak{g}_\alpha,\; \forall y \in \mathfrak{g}_\beta$ をとる.このとき $\ad_{\mathfrak{g}} \colon \mathfrak{g} \lto \Lgl(\mathfrak{g})$ の像が\hyperref[def:derivation-Alg]{微分}であることから,$\forall h \in \mathfrak{h}$ に対して
		\begin{align}
			\ad_{\mathfrak{g}}(h)(\comm{x}{y}) 
			&= \comm{\ad_{\mathfrak{g}}(h)(x)}{y} + \comm{x}{\ad_{\mathfrak{g}}(h)(y)} \\
			&= \bigl(\alpha(h) + \beta(h)\bigr) \comm{x}{y} \\
			&= (\alpha + \beta)(h) \comm{x}{y}
		\end{align}
		が成り立つ.i.e. $\comm{x}{y} \in \mathfrak{g}_{\alpha + \beta}$ である.
		\item $\alpha \neq 0$ とし,$\forall x \in \mathfrak{g}_\alpha$ を1つとる.\eqref{eq:root-decomp1}が $\mathfrak{g}$ の直和分解なので,$\forall \beta \in \mathfrak{h}^*$ に対してある $n \in \mathbb{N}$ が存在して $\bigl(\ad_{\mathfrak{g}}(x)\bigr)^n (\mathfrak{g}_\beta) = 0$ を充たすことを示せばよい.
		
		 $\forall n \in \mathbb{N}$ に対して $\bigl(\ad_{\mathfrak{g}}(x)\bigr)^n (\mathfrak{g}_\beta) \neq 0$ が成り立つと仮定する.
		ところが (1) よりこれは
		$0 \subsetneq \bigl(\ad_{\mathfrak{g}}(x) \bigr)^n(\mathfrak{g}_\beta) \subset \mathfrak{g}_{n\alpha + \beta}$ を意味し,$\Phi \subset \mathfrak{h}^*$ が有限集合であることに矛盾する.従って背理法により示された.
		\item 仮定より $\alpha + \beta \neq 0 \in \mathfrak{h}^*$ なので,ある $h \in \mathfrak{h}$ が存在して $(\alpha + \beta)(h) \neq 0$ を充たす.
		このとき $\forall (x,\, y) \in \mathfrak{g}_\alpha \times \mathfrak{g}_{\beta}$ をとると
		\begin{align}
			\alpha(h) \kappa(x,\, y)
			&= \kappa\bigl(\ad_{\mathfrak{g}}(h)(x),\, y\bigr) \\
			&= -\kappa(\comm{x}{h},\, y) \\
			&= -\kappa(x,\, \comm{h}{y}) \\
			&= -\kappa\bigl(x,\, \ad_{\mathfrak{g}}(h)(y)\bigr) \\
			&= -\beta(h) \kappa(x,\, y)
		\end{align}
		が成り立つので $\kappa(x,\, y) = 0$ が言えた.
		\item (3) より
		\begin{align}
			\label{eq:perp-inclusion1}
			\mathfrak{g}_\alpha{}^\perp \supset \bigoplus_{\gamma \in \mathfrak{h}^* \setminus \{-\alpha\}} \mathfrak{g}_{\gamma}
		\end{align}
		がわかる.
		定理\ref{thm:semisimple-LieAlg-iff}より $\kappa$ は\hyperref[def:radical-bilinear]{非退化}なので,$\mathfrak{g}$ が有限次元であることから
		\begin{align}
			\dim \mathfrak{g}_\alpha{}^\perp = \dim \mathfrak{g} - \dim \mathfrak{g}_\alpha
		\end{align}
		が成り立つ.よって\eqref{eq:perp-inclusion1}から $\dim \mathfrak{g} - \dim \mathfrak{g}_{-\alpha} \le \dim \mathfrak{g} - \dim \mathfrak{g}_\alpha \IFF \dim \mathfrak{g}_\alpha \le \dim \mathfrak{g}_{-\alpha}$ が言える.$\alpha \in \mathfrak{h}^*$ は任意だったので $\dim \mathfrak{g}_{-\alpha} \le \dim \mathfrak{g}_{\alpha}$ も言えて $\dim \mathfrak{g}_\alpha = \dim \mathfrak{g}_{-\alpha}$ が従う.
		故に
		\begin{align}
			\dim \mathfrak{g}_\alpha{}^\perp = \dim \mathfrak{g} - \dim \mathfrak{g}_{\alpha} = \dim \mathfrak{g} - \dim \mathfrak{g}_{-\alpha} = \dim \bigoplus_{\gamma \in \mathfrak{h}^* \setminus \{-\alpha\}} \mathfrak{g}_{\gamma}
		\end{align}
		が成り立ち,\eqref{eq:perp-inclusion1}の包含が等号だと分かった.
	\end{enumerate}
	
\end{proof}

\begin{mycol}[label=col:c-Killing]{}
	$\kappa|_{\mathfrak{g}_0 \times \mathfrak{g}_0}$ は\hyperref[def:radical-bilinear]{非退化}である.
\end{mycol}

\begin{proof}
	命題\ref{prop:root-decomp-basic1}-(4) より $\mathfrak{g}_0 \cap \mathfrak{g}_0{}^\perp = 0$,i.e. $\kappa|_{\mathfrak{g}_0 \times \mathfrak{g}_0}$ は非退化である.
\end{proof}


\subsection{極大トーラスの中心化代数}

\begin{mylem}[label=lem:8-2]{}
	$V$ を有限次元 $\mathbb{K}$-ベクトル空間とし,$x,\, y \in \End V$ は $\comm{x}{y} = 0$ を充たす(i.e. 互いに可換)とする.

	このとき,$x,\, y$ のどちらか一方が冪零ならば $x \circ y$ も冪零である.特に $\Tr (x \circ y) = 0$ が成り立つ.
\end{mylem}

\begin{proof}
	$x$ が冪零であるとする.このときある $n > 0$ が存在して $x^n = 0$ となるので,$\comm{x}{y} = 0 \IFF x \circ y = y \circ x$ より $(x \circ y)^n = y^{n} \circ x^n = 0$ が言える.
	特に $x\circ y$ の表現行列を $\mathfrak{n}(\dim V,\, \mathbb{K})$ の元にするような $V$ の基底が存在するので $\Tr(x\circ y) = 0$ である.
\end{proof}


\begin{myprop}[label=prop:torus-centralize]{}
	$\mathfrak{g}$ の\hyperref[def:toral-subLieAlg]{極大トーラス} $\mathfrak{h}$ に対して,
	\begin{align}
		\mathfrak{h} = C_{\mathfrak{g}}(\mathfrak{h}) \; (= \mathfrak{g}_0)
	\end{align}
\end{myprop}

\begin{proof}
	$C \coloneqq  C_{\mathfrak{g}}(\mathfrak{h})$ とおく.命題\ref{lem:torus}より $\mathfrak{h}$ は可換なので $\mathfrak{h} \subset C$ が成り立つ.
	$x \in \mathfrak{g}$ の\hyperref[def:abstruct-JC]{抽象Jordan分解}を $x = x_s + x_n$ と書く.
	\begin{description}
		\item[\textbf{step1: $\bm{\forall x = x_s + x_n \in C}$ に対して $\bm{x_s},\, \bm{x_n \in C}$}] 
		
		$x \in C \IFF \ad_{\mathfrak{g}}(x)(\mathfrak{h}) = 0$ である.このとき命題\ref{prop:Jordan-Chevalley}-(3) より $\ad_{\mathfrak{g}}(x)_s(\mathfrak{h}) = \ad_{\mathfrak{g}}(x)_n(\mathfrak{h}) = 0$ が言える.系\ref{col:JC}より $\ad_{\mathfrak{g}}(x)_s = \ad_{\mathfrak{g}}(x_s),\; \ad_{\mathfrak{g}}(x)_n = \ad_{\mathfrak{g}}(x_n)$ なので示された.
		
		\item[\textbf{step2: $\bm{\forall x = x_s + x_n \in C}$ に対して $\bm{x_s \in \mathfrak{h}}$}] 
		
		$\forall x = x_s + x_n \in C$ をとる.このとき $\forall h \in \mathfrak{h}$ に対して $0 = \ad_{\mathfrak{g}}(0) = \ad_{\mathfrak{g}}(\comm{h}{x}) = \comm{\ad_{\mathfrak{g}}(h)}{\ad_{\mathfrak{g}}(x)} = 0$ が成り立つので $\ad_{\mathfrak{g}}(h),\, \ad_{\mathfrak{g}}(x),\,  \ad_{\mathfrak{g}}(x_s) \in \Lgl(\mathfrak{g})$ は同時対角化可能であり,従って $\ad_{\mathfrak{g}}(h+x_s) = \ad_{\mathfrak{g}}(h) + \ad_{\mathfrak{g}}(x_s) \in \Lgl(\mathfrak{g})$ もまた半単純だとわかる.
		従って $\mathfrak{h} + \mathbb{K}x_s$ は $\mathfrak{h}$ を含む $\mathfrak{g}$ のトーラスだが,$\mathfrak{h}$ の極大性より $\mathfrak{h} + \mathbb{K} x_s = \mathfrak{h}$,i.e. $x_s \in \mathfrak{h}$ が言えた.

		\item[\textbf{step3: $\bm{\kappa}|_{\bm{\mathfrak{h} \times \mathfrak{h}}}$ が非退化}] 
		
		$\forall h \in \mathfrak{h} \cap \mathfrak{h}^\perp$ をとる.このとき $\kappa(h,\, \mathfrak{h}) = 0$ が成り立つ.示すべきは $h = 0$ である.

		 ところで,$x \in C$ が冪零ならば,$\comm{x}{\mathfrak{h}} = 0$ かつ $\ad_{\mathfrak{g}}(x) \in \Lgl(\mathfrak{g})$ が冪零であることから
		補題\ref{lem:8-2}が使えて $\kappa(x,\, y) = \Tr \bigl( \ad_{\mathfrak{g}}(x) \circ \ad_{\mathfrak{g}}(y) \bigr) = 0$ が言える.i.e. $\kappa(x,\, \mathfrak{h}) = 0$ が成り立つ.
		ここで $\forall y = y_s + y_n \in C$ をとると,\textsf{\textbf{(step2)}}から $y_s \in \mathfrak{h}$ なので $h$ の取り方から $\kappa(y_s,\, h) = 0$ がわかり,\textsf{\textbf{(step1)}}から $y_n \in C$ がわかるが,$y_n$ は冪零なので結局 $\kappa(y,\, h) = \kappa(y_n,\, h) = 0$ だと分かった.
		$y \in C$ は任意だったのでこのことは $h \in C^\perp$ を意味するが,系\ref{col:c-Killing}より $\kappa|_{C \times C}$ は\hyperref[def:radical-bilinear]{非退化}なので $h = 0$ が言えた\footnote{$\mathfrak{h} \subset C$ より $h \in C \cap C^\perp$.}.
		
		\item[\textbf{step4: $\bm{C}$ は冪零Lie代数}] 
		
		$\forall x = x_s + x_n\in C$ をとる.
		\textsf{\textbf{(step2)}}より $x_s \in \mathfrak{h}$ であり,$\ad_C(x) = 0$ は冪零である.
		一方 $\ad_C(x_n)$ は定義から冪零なので,$\ad_C(x) = \ad_C(x_s) + \ad_C(x_n)$ も冪零である.よって\hyperref[thm:Engel]{Engelの定理}から $C$ は\hyperref[def:nilpotent-LieAlg]{冪零Lie代数}である.

		\item[\textbf{step5: $\bm{\mathfrak{h} \cap [C,\, C] = 0}$}] 
		
		$\kappa$ はLieブラケットについて結合的なので,$\comm{\mathfrak{h}}{C} = 0$ と併せて $\kappa(\mathfrak{h},\, \comm{C}{C}) = 0$ を得る.従って\textsf{\textbf{(step3)}}より $\mathfrak{h} \cap \comm{C}{C} = 0$ が言えた.

		\item[\textbf{step6: $\bm{C}$ は可換}] 
		
		背理法により示す.$\comm{C}{C} \neq 0$ とする.\textsf{\textbf{(step4)}}より $C$ は冪零Lie代数で,かつ $\comm{C}{C} \subset C$ はその\hyperref[def:ideal-LieAlg]{イデアル}であるから,このとき補題\ref{lem:nilpo-ideal}が使えて $Z(C)\cap \comm{C}{C} \neq 0$ が言える.i.e. $z = z_s + z_n \in (Z(C)\cap \comm{C}{C}) \setminus \{0\}$ が存在する.\textsf{\textbf{(step2)}}, \textsf{\textbf{(step5)}}より $z_n \neq 0$ でなくてはならない.
		よって \textsf{\textbf{(step1)}} から $z_n \in C \setminus \{0\}$ と言うことになる.
		一方,命題\ref{prop:Jordan-Chevalley}-(2) より $\forall c \in C$ に対して $0 = \comm{z}{c} = \comm{z_s}{c} + \comm{z_n}{c}$ だが,\textsf{\textbf{(step2)}}より $z_s \in \mathfrak{h}$ なので $\comm{z_s}{c} = 0$ であり,$\comm{z_n}{c} = 0$,i.e. $z_n \in Z(C)$ となる.よって補題\ref{lem:8-2}から $\kappa(z_n,\, C) = 0$ が言えて,系\ref{col:c-Killing}から $z_n = 0$ ということになり,$z_n \neq 0$ と矛盾する.

		\item[\textbf{step7: $\bm{C = \mathfrak{h}}$}] 
		
		背理法により示す.$C \neq \mathfrak{h}$ とすると,ある $x = x_s + x_n \in C$ が存在して $x_n \neq 0$ となる.
		このとき \textsf{\textbf{(step1)}} より $x_n \in C \setminus \{0\}$ であり,\textsf{\textbf{(step6)}} および補題\ref{lem:8-2}から $\forall y \in C$ に対して $\kappa(x_n,\, y) = \Tr \bigl( \ad(x_n) \circ  \ad(y)\bigr) = 0$ を充たすことになり,系\ref{col:c-Killing}に矛盾する.
	\end{description}
\end{proof}

\begin{mycol}[label=col:torus-centralizer]{}
	$\kappa|_{\mathfrak{h} \times \mathfrak{h}}$ は\hyperref[def:radical-bilinear]{非退化}である.
\end{mycol}

\begin{proof}
	系\ref{col:c-Killing}に命題\ref{prop:torus-centralize}を用いるだけである.
\end{proof}

$\mathbb{K}$-線型写像
\begin{align}
	\label{eq:isom-h}
	\tilde{\kappa} \colon \mathfrak{h} \lto \mathfrak{h}^*,\; h \lmto \bigl( x \lmto \kappa(h,\, x) \bigr) 
\end{align}
を考える.すると $\kappa$ に関する $\mathfrak{h}$ の\hyperref[def:radical-bilinear]{直交補空間}の定義を思い出して
\begin{align}
	\Ker \tilde{\kappa} = \bigl\{\, h \in \mathfrak{h} \bigm| \forall x \in \mathfrak{h},\; \kappa(h,\, x) = 0 \,\bigr\} = \mathfrak{h} \cap \mathfrak{h}^\perp
\end{align}
となることがわかるが,系\ref{col:torus-centralizer}より $\Ker \tilde{\kappa} = 0$ が言える.i.e. $\tilde{\kappa}$ は単射である.
$\dim \mathfrak{h} = \dim \mathfrak{h}^*$ なので,補題\ref{lem:finvec-basic}-(3) から $\bm{\tilde{\kappa}}$ \textbf{は} $\bm{\mathbb{K}}$\textbf{-ベクトル空間の同型写像である.}
従って $\forall \alpha \in \mathfrak{h}^*$ が与えられたとき,$\forall h \in \mathfrak{h}$  に対して $\alpha(h) = \kappa(\bm{t_\alpha},\, h) = \tilde{\kappa}(\bm{t_\alpha})(h)$ を充たすような $\bm{t_\alpha} \in \mathfrak{h}$ が存在する.
% \footnote{
% 	系\ref{col:torus-centralizer}より $\kappa|_{\mathfrak{h} \times \mathfrak{h}} \in L(\mathfrak{h},\, \mathfrak{h};\, \mathbb{K})$ は対称かつ非退化なので,
% 	$\mathfrak{h}$ の基底 $\{e_\mu\}$ を1つ固定すると補題\ref{lem:Casimir}-(1) の方法によって $\kappa(e_\mu,\, e^\nu) = \delta_\mu^\nu$ を充たす $\mathfrak{h}$ の基底 $\{e^\mu\}$ を一意的に構成できる.このとき $1 \le \forall \mu \le \dim \mathfrak{h}$ について $\alpha(e^\mu) = \alpha(e^\nu) \delta^\mu_\nu = \kappa \bigl( \alpha(e^\nu) e_\nu,\, e^\mu \bigr)$ が成り立つので,$t_\alpha \coloneqq \alpha(e^\nu) e_\nu \in \mathfrak{h}$ とおけば良い.
% }.i.e. $\bm{\tilde{\kappa}}$ \textbf{は} $\bm{\mathbb{K}}$\textbf{-ベクトル空間の同型写像である.}

\subsection{直交性}

\begin{myprop}[label=prop:root-decomp-ortho,breakable]{ルートの直交性}
	\begin{itemize}
		\item 体 $\mathbb{K}$ 上の有限次元\hyperref[def:semisimple-LieAlg]{半単純Lie代数} $\mathfrak{g}$
		\item $\mathfrak{g}$ の\hyperref[def:toral-subLieAlg]{極大トーラス} $\mathfrak{h} \subset \mathfrak{g}$
		\item $\mathfrak{h}$ のルート全体の集合 $\Phi \subset \mathfrak{h}^*$ 
	\end{itemize}
	を与える.$\mathfrak{g}$ の\hyperref[eq:root-decomp]{ルート空間分解}を命題\ref{prop:torus-centralize}によって
	\begin{align}
		\mathfrak{g} = \mathfrak{h} \oplus \bigoplus_{\alpha \in \Phi} \mathfrak{g}_\alpha
	\end{align}
	と書く.
	このとき,以下が成り立つ:
	\begin{enumerate}
		\item $\mathfrak{h}^* = \Span \Phi$
		\item $\alpha \in \Phi \IMP -\alpha \in \Phi$
		\item $\forall \alpha \in \Phi$ および $\forall x \in \mathfrak{g}_\alpha,\; \forall y \in \mathfrak{g}_{-\alpha}$ に対して,$\comm{x}{y} = \kappa(x,\, y) t_\alpha$
		\item $\alpha \in \Phi \IMP \comm{\mathfrak{g}_\alpha}{\mathfrak{g}_{-\alpha}} = \Span \{t_\alpha\}$
		\item $\alpha \in \Phi \IMP \alpha(t_\alpha) = \kappa(t_\alpha,\, t_\alpha) \neq 0$
		\item $\forall \alpha \in \Phi$ および $\forall x_\alpha \in \mathfrak{g}_\alpha \setminus \{0\}$ に対して $y_\alpha \in \mathfrak{g}_{-\alpha}$ が存在し,$h_\alpha \coloneqq \comm{x_\alpha}{y_\alpha} \in \mathfrak{h}$ とおくと
		\begin{align}
			x_\alpha &\lmto \mqty[0 & 1 \\ 0 & 0], & y_\alpha &\lmto \mqty[0 & 0 \\ 1 & 0], & h_\alpha &\lmto \mqty[1 & 0 \\ 0 & -1]
		\end{align}
		を線型に拡張することで定義される $\mathbb{K}$-線型写像
		\begin{align}
			\Span \bigl\{\, x_\alpha,\, y_\alpha,\, h_\alpha\,\bigr\} \lto \Lsl(2,\, \mathbb{K})
		\end{align}
		がLie代数の同型写像になる.
		\item 
		\begin{align}
			h_\alpha = \frac{2}{\kappa(t_\alpha,\, t_\alpha)} t_\alpha,\quad h_\alpha = - h_{-\alpha}
		\end{align}
	\end{enumerate}
	
\end{myprop}

\begin{proof}
	\begin{enumerate}
		\item 背理法により示す.
		$\mathfrak{h}^* \supsetneq \Span \Phi$ を仮定する.
		このとき $\Span \Phi$ の基底 $\bigl\{\,\varepsilon^\mu\,\bigr\}_{1 \le \mu \le r}\quad (r < \dim \mathfrak{h})$ を1つ固定すると,線型独立な $\dim \mathfrak{h} - r > 0$ 個の元 $\varepsilon^{r+1},\, \dots,\, \varepsilon^{\dim \mathfrak{h}} \in \mathfrak{h}^* \setminus \Phi$ が存在して $\bigl\{\,\varepsilon^\mu\,\bigr\}_{1 \le \mu \le r} \cup \bigl\{\, \varepsilon^\nu\,\bigr\}_{r+1 \le \nu \le \dim \mathfrak{h}}$ が $\mathfrak{h}^*$ の基底になる.
		さらに $\dim \mathfrak{h} < \infty$ なので $\mathfrak{h}^*$ の基底 $\bigl\{\, \varepsilon^\mu\,\bigr\}_{1 \le \mu \le \dim \mathfrak{h}}$ の\hyperref[prop:dual-basis]{双対基底} $\bigl\{\, e_\mu \in \mathfrak{h} \,\bigr\}_{1 \le \mu \le \dim \mathfrak{h}}$ をとることができるが,
		このとき $r + 1\le \nu \le \dim \mathfrak{h}$ を充たす $\nu$ が少なくとも1つ存在して $e_\nu \in \mathfrak{h} \setminus \{0\}$ が $\forall \alpha = \sum_{\mu = 1}^r\alpha_\mu \varepsilon^\mu \in \Phi$ に対して $\alpha(e_\nu) = \sum_{\mu = 1}^r\alpha_\mu \varepsilon^\mu(e_\nu) = 0$ を充たす.
		このような $e_\nu$ のうちのどれか1つを $h$ とおこう.

		% ならば非零な $\beta \in \mathfrak{h}^* \setminus \Span \Phi$ が存在する.
		% 従って\underline{非零な}\footnote{$\beta \colon \mathfrak{h} \lto \mathbb{K}$ は $\mathbb{K}$-線型写像なので,もし $h = 0$ ならば $h \in \Ker \beta$ となって矛盾.} $h \in \Bigl(\bigcap_{\alpha \in \Phi} \Ker \alpha \Bigr)\setminus \Ker \beta$ が存在する.
		 このとき\hyperref[def:ga]{$\mathfrak{g}_\alpha$ の定義}から,$\forall \alpha \in \Phi$ に対して $\comm{h}{\mathfrak{g}_\alpha} = 0$ が成り立つ.
		一方で,補題\ref{lem:torus}より $\comm{h}{\mathfrak{h}} = 0$ であるから,命題\ref{prop:torus-centralize}より $\comm{h}{\mathfrak{g}} = \comm{h}{\mathfrak{h} \oplus \bigoplus_{\alpha \in \Phi} \mathfrak{g}_\alpha} = \comm{h}{\mathfrak{h}} \oplus \bigoplus_{\alpha \in \Phi} \comm{h}{\mathfrak{g}_\alpha} = 0 \IFF h \in Z(\mathfrak{g})$ がわかる.然るに $\mathfrak{g}$ が\hyperref[def:semisimple-LieAlg]{半単純Lie代数}であることから $Z(\mathfrak{g}) = 0$ であり $h = 0$ と言うことになって矛盾.

		\item $\alpha \in \Phi$ とする.背理法により $-\alpha \in \Phi$ を示す.
		$-\alpha \notin \Phi$ を仮定する.このとき $\mathfrak{g}_{-\alpha} = 0$ であるから $\kappa|_{\mathfrak{g}_\alpha \times \mathfrak{g}_{-\alpha}} = 0$ であり,命題\ref{prop:root-decomp-basic1}-(3) と併せると $\forall \beta \in \mathfrak{h}^*$ に対して $\kappa|_{\mathfrak{g}_\alpha \times \mathfrak{g}_\beta} = 0$ がわかる.
		よって $\kappa$ の双線型性から $\kappa|_{\mathfrak{g}_\alpha \times \mathfrak{g}} = 0$,i.e. $\kappa$ の\hyperref[def:radical-bilinear]{radical} $S_\kappa$ について $0 \subsetneq \mathfrak{g}_\alpha \subset S_\kappa$ と言うことになって $\mathfrak{g}$ が半単純Lie代数であることに矛盾(定理\ref{thm:semisimple-LieAlg-iff}).

		\item $\forall \alpha \in \Phi$ および $\forall x \in \mathfrak{g}_\alpha,\; \forall y \in \mathfrak{g}_{-\alpha}$ を1つずつとる.
		$\forall h \in \mathfrak{h}$ を1つ固定しよう.$\kappa \in L(\mathfrak{g},\, \mathfrak{g};\, \mathbb{K})$ は対称かつLieブラケットについて結合的なので
		\begin{align}
			\kappa (h,\, \comm{x}{y}) 
			&= \kappa (\comm{h}{x},\, y) \\
			&= \kappa \bigl(\ad_{\mathfrak{g}}(h)(x),\, y\bigr) \\
			&= \alpha(h) \kappa(x,\, y) \\
			&= \kappa(t_\alpha,\, h) \kappa(x,\, y) \\
			&= \kappa \bigl( \kappa(x,\, y) t_\alpha,\, h \bigr) \\
			&= \kappa \bigl( h,\, \kappa(x,\, y) t_\alpha\bigr) \\
		\end{align}
		がわかる.$h \in \mathfrak{h}$ は任意だったので $\kappa \bigl( h,\, \comm{x}{y} - \kappa(x,\, y) t_\alpha \bigr) = 0$,i.e. \hyperref[def:radical-bilinear]{$\kappa$ に関する直交補空間}の意味で $\comm{x}{y} - \kappa(x,\, y) t_\alpha \in \mathfrak{h}^\perp$ だと分かった.
		一方で,命題\ref{prop:root-decomp-basic1}-(1),命題\ref{prop:torus-centralize}より $\comm{x}{y} \in \mathfrak{g}_0 = \mathfrak{h}$ が言えて,かつ定義より $t_\alpha \in \mathfrak{h}$ なので,$\comm{x}{y} - \kappa(x,\, y) t_\alpha \in \mathfrak{h}$ である.
		以上より $\comm{x}{y} - \kappa(x,\, y) t_\alpha \in \mathfrak{h} \cap \mathfrak{h}^\perp$ が分かったが,系\ref{col:torus-centralizer}より $\mathfrak{h} \cap \mathfrak{h}^\perp = 0$ なので $\comm{x}{y} - \kappa(x,\, y) t_\alpha = 0 \IFF \comm{x}{y} = \kappa(x,\, y) t_\alpha$ が示された.
		
		\item $\alpha \in \Phi$ とする.(3) より,$\comm{\mathfrak{g}_\alpha}{\mathfrak{g}_{-\alpha}} \neq 0$ を示せば十分である.
		$\alpha \in \Phi$ なので,$x \in \mathfrak{g}_\alpha \setminus \{0\}$ が存在する.
		もし $\kappa|_{\{x\} \times \mathfrak{g}_{-\alpha}} = 0$ ならば (2) の証明と同様の議論により $\kappa|_{\{x\} \times \mathfrak{g}} = 0$,i.e. $x \in S_\kappa$ と言う事になり定理\ref{thm:semisimple-LieAlg-iff}に矛盾.よって背理法からある $y \in \mathfrak{g}_{-\alpha}$ が存在して $\kappa(x,\, y) \neq 0$ を充たす.従って (3) から $\comm{x}{y} \in \comm{\mathfrak{g}_\alpha}{\mathfrak{g}_{-\alpha}} \setminus \{0\}$ が言えた.

		\item $\alpha \in \Phi$ とする.背理法により $\alpha(t_\alpha) \neq 0$ を示す.
		$\alpha(t_\alpha) = 0$ を仮定すると,$\forall x \in \mathfrak{g}_\alpha,\; \forall y \in \mathfrak{g}_{-\alpha}$ に対して $\comm{t_\alpha}{x} = 0 = \comm{t_\alpha}{y}$ となる.
		(4) と同様の議論により $\kappa(x,\, y) \neq 0$ であるような $x \in \mathfrak{g}_\alpha \setminus \{0\},\; y \in \mathfrak{g}_{-\alpha}$ が存在する.$x,\, y$ のどちらか一方を $\kappa(x,\, y)^{-1}$ 倍することで,(3) において $\comm{x}{y} = t_\alpha \in \mathfrak{h}$ となるようにできる.
		このとき $\mathfrak{g}$ の部分Lie代数 $\mathfrak{s} \coloneqq \Span \{x,\, y,\, t_\alpha\}$ は3次元の\hyperref[def:solvable-LieAlg]{可解Lie代数}であり,$\mathfrak{s} \cong \ad_{\mathfrak{g}}(\mathfrak{s}) \subset \Lgl(\mathfrak{g})$ である\footnote{\hyperref[def:abstruct-JC]{抽象Jordan分解の定義}の直前で述べたように,$\mathfrak{g}$ が\hyperref[def:semisimple-LieAlg]{半単純Lie代数}ならば $\ad_{\mathfrak{g}} \colon \mathfrak{g} \lto \Lgl(\mathfrak{g})$ は単射である.}.
		従って系\ref{col:Lie-2}が使えて,$\forall s \in \comm{\mathfrak{s}}{\mathfrak{s}}$ に対して $\ad_{\mathfrak{g}}(s) \in \Lgl(\mathfrak{g})$ は冪零である.$t_\alpha \in \comm{\mathfrak{s}}{\mathfrak{s}}$ なので $\ad_{\mathfrak{g}}(t_\alpha)$ は冪零である.
		一方で $t_\alpha \in \mathfrak{h}$ なので $\ad_{\mathfrak{g}}(t_\alpha)$ は\hyperref[def:semisimple-end]{半単純}でもあるから $\ad_{\mathfrak{g}}(t_\alpha) = 0 \IFF t_\alpha \in \Ker \ad_{\mathfrak{g}} = 0 \IMP t_\alpha = 0$ となるが,これは $\alpha \in \Phi$ (従って $\alpha \neq 0$)であることに矛盾する.

		\item $\forall \alpha \in \Phi$ および $\forall x_\alpha \in \mathfrak{g}_\alpha \setminus \{0\}$ を与える.
		このとき (4) の証明と同様の議論によりある $y \in \mathfrak{g}_{-\alpha}$ が存在して $\kappa(x_\alpha,\, y) \neq 0$ を充たす.さらに (5) から $\kappa(t_\alpha,\, t_\alpha) \neq 0$ なので,
		\begin{align}
			y_\alpha \coloneqq \frac{2}{\kappa(t_\alpha,\, t_\alpha)\kappa(x_\alpha,\, y)} y
		\end{align}
		とおけば
		\begin{align}
			\kappa(x_\alpha,\, y_\alpha) = \frac{2}{\kappa(t_\alpha,\, t_\alpha)}
		\end{align}
		を充たす.よってさらに
		\begin{align}
			h_\alpha \coloneqq \frac{2}{\kappa(t_\alpha,\, t_\alpha)} t_\alpha \in \mathfrak{h}
		\end{align}
		とおけば (3) より
		\begin{align}
			\comm{x_\alpha}{y_\alpha} &= \kappa(x_\alpha,\, y_\alpha) t_\alpha = h_\alpha, \\
			\comm{h_\alpha}{x_\alpha} &= \frac{2}{\alpha(t_\alpha)} \comm{t_\alpha}{x_\alpha} = \frac{2}{\alpha(t_\alpha)}\alpha(t_\alpha) x_\alpha = 2x_\alpha, \\
			\comm{h_\alpha}{y_\alpha} &= \frac{2}{\alpha(t_\alpha)} \comm{t_\alpha}{y_\alpha} = \frac{2}{\alpha(t_\alpha)}\bigl(-\alpha(t_\alpha)\bigr) y_\alpha = -2y_\alpha
		\end{align}
		が成り立ち,$\Span \bigl\{x_\alpha,\, y_\alpha,\, h_\alpha\bigr\} \subset \mathfrak{g}$ は $\Lsl (2,\, \mathbb{K})$ とLie代数として同型になる.

		\item (6) の証明から即座に
		\begin{align}
			h_\alpha = \frac{2}{\kappa(t_\alpha,\, t_\alpha)} t_\alpha
		\end{align}
		が従う.$\kappa(t_\alpha,\, h) \coloneqq \alpha(h)\quad (\forall h \in \mathfrak{h})$ が $t_\alpha$ の定義だったので,
		\begin{align}
			\forall h \in \mathfrak{h},\; \kappa(t_{-\alpha},\, h) = -\alpha(h) = \kappa(-t_{\alpha},\, h)
		\end{align}
		が言える.よって系\ref{col:torus-centralizer}から $t_{-\alpha} = -t_\alpha$ であり,
		\begin{align}
			h_\alpha = -h_{-\alpha}
		\end{align}
		が言えた.
	\end{enumerate}
	
\end{proof}

\subsection{整性}

命題\ref{prop:root-decomp-ortho}-(6) で構成した $(x_\alpha,\, y_\alpha,\, h_\alpha) \in \mathfrak{g}_\alpha \times \mathfrak{g}_{-\alpha} \times \mathfrak{h}$ について,
\begin{align}
	\mathfrak{s}_\alpha \coloneqq \Span \bigl\{\, x_\alpha,\, y_\alpha,\, h_\alpha \,\bigr\} \cong \Lsl(2,\, \mathbb{K})
\end{align}
とおく.

\begin{myprop}[label=prop:root-decomp-int,breakable]{}
	\begin{itemize}
		\item 体 $\mathbb{K}$ 上の有限次元\hyperref[def:semisimple-LieAlg]{半単純Lie代数} $\mathfrak{g}$
		\item $\mathfrak{g}$ の\hyperref[def:toral-subLieAlg]{極大トーラス} $\mathfrak{h} \subset \mathfrak{g}$
		\item $\mathfrak{h}$ のルート全体の集合 $\Phi \subset \mathfrak{h}^*$ 
	\end{itemize}
	を与える.$\mathfrak{g}$ の\hyperref[eq:root-decomp]{ルート空間分解}を命題\ref{prop:torus-centralize}によって
	\begin{align}
		\mathfrak{g} = \mathfrak{h} \oplus \bigoplus_{\alpha \in \Phi} \mathfrak{g}_\alpha
	\end{align}
	と書く.このとき,$\forall \alpha,\, \beta \in \Phi$ に対して以下が成り立つ:
	\begin{enumerate}
		\item $\dim \mathfrak{g}_\alpha = 1$.
		特に $\mathfrak{h}_\alpha \coloneqq \comm{\mathfrak{g}_\alpha}{\mathfrak{g}_{-\alpha}}$ とおくと
		\begin{align}
			\mathfrak{s}_\alpha = \mathfrak{g}_\alpha \oplus \mathfrak{g}_{-\alpha} \oplus \mathfrak{h}_{\alpha}
		\end{align}
		が成り立ち,$\forall x_\alpha \in \mathfrak{g}_{\alpha}$ に対して $\comm{x_\alpha}{y_\alpha} = h_\alpha \in \mathfrak{h}_\alpha$ を充たす $y_\alpha \in \mathfrak{g}_{-\alpha}$ が一意的に存在する.
		\item $\lambda \alpha \in \Phi \IMP \lambda = +1 \OR \lambda = -1$
		\item $\beta(h_\alpha) \in \mathbb{Z}$ で,かつ $\beta - \beta(h_\alpha)\alpha \in \Phi$
		\item $\alpha + \beta \in \Phi \IMP \comm{\mathfrak{g}_\alpha}{\mathfrak{g}_\beta} = \mathfrak{g}_{\alpha+\beta}$
		\item $\beta \neq \pm \alpha$ ならば,
		\begin{align}
			r &\coloneqq \max \bigl\{\, \lambda \in \mathbb{Z} \bigm| \beta - \lambda \alpha \in \Phi \,\bigr\}, \\
			q &\coloneqq \max \bigl\{\, \lambda \in \mathbb{Z} \bigm| \beta + \lambda \alpha \in \Phi \,\bigr\}
		\end{align}
		とおいたとき $-r \le \forall \lambda \le q$ に対して
		\begin{align}
			\beta + \lambda\alpha \in \Phi \AND \beta(h_\alpha) = r-q \in \mathbb{Z}
		\end{align}
		が成り立つ.
		% \item Lie代数としての同型
		% \begin{align}
		% 	\mathfrak{g} \cong \mathbb{K}^{\oplus \mathfrak{g}_\alpha}
		% \end{align}
		% が成り立つ.
	\end{enumerate}
	
\end{myprop}

\begin{marker}
	(3) の $\beta(h_\alpha) \in \mathbb{Z}$ は\textbf{Cartan数} (Cartan integer) と呼ばれる.(5) で得られた系列 $\{\beta + \lambda\alpha \in \Phi\}_{-r \le \lambda \le q}$ のことを\textbf{$\bm{\alpha}$-string through $\bm{\beta}$}と呼ぶ.
\end{marker}


\begin{proof}
	$\forall \alpha \in \Phi$ を1つ固定する.
	$\mathfrak{s}_\alpha$ の随伴表現
	\begin{align}
		\ad_{\mathfrak{g}}|_{\mathfrak{s}_\alpha} \colon \mathfrak{s}_\alpha \lto \Lgl (\mathfrak{g}),\; x \lmto \ad_{\mathfrak{g}}(x)
	\end{align}
	によって $\mathfrak{g}$ を\hyperref[ax:g-module]{$\mathfrak{s}_\alpha$-加群}と見做す.
	このとき $h_\alpha \in \mathfrak{s}_\alpha$ の,\underline{$\mathfrak{g}$ における}ウエイト $\mu$ の\hyperref[def:weight-space]{ウエイト空間}を $V_\mu \subset \mathfrak{g}$ と書く.
	\hyperref[thm:Weyl]{Weylの定理}から $\mathfrak{s}_\alpha$-加群 $\mathfrak{g}$ は\hyperref[def:irr]{既約}な部分 $\mathfrak{s}_\alpha$-加群の族 $\Familyset[\big]{\mathfrak{w}_i}{i \in I}$ の($\mathfrak{s}_\alpha$-加群としての)\hyperref[def:gmod-directsum]{直和}
	\begin{align}
		\label{eq:prop2.5.4-Weyl}
		\mathfrak{g} = \bigoplus_{i \in I} \mathfrak{w}_i
	\end{align}
	に分解し,それぞれの直和因子 $\mathfrak{w}_i$ の既約な部分 $\mathfrak{s}_\alpha \cong \Lsl(2,\, \mathbb{K})$-加群としての構造は定理\ref{thm:irr-sl2}より定まる.
	% \begin{itemize}
	% 	\item ある $m_i \in \mathbb{Z}_{\ge 0}$ が存在して,\underline{$\mathbb{K}$-ベクトル空間としての}直和
	% 	\begin{align}
	% 		\mathfrak{w}_i = \bigoplus_{\mu = 0}^{m_i} V_{m_i - 2\mu}
	% 	\end{align}
	% 	が成り立つ.
	% 	\item 
	% \end{itemize}
	特に
	% $\ad_{\mathfrak{g}}(h_\alpha) \in \Lgl(\mathfrak{g})$ の固有ベクトル全体は $\mathfrak{g}$ の基底を成し,
	系\ref{col:sl2}より 
	\begin{itemize}
		\item $\ad_{\mathfrak{g}}(h_\alpha)$ の固有値(i.e. 随伴表現 $\ad_{\mathfrak{g}}|_{\mathfrak{s}_\alpha}$ における $h_\alpha \in \mathfrak{s}_\alpha$ のウエイト)は全て整数
		\item 偶数ウエイトのウエイト空間の直和として書ける $\mathfrak{w}_i$ の個数は $\dim V_0$ 個
		\item 奇数ウエイトのウエイト空間の直和として書ける $\mathfrak{w}_i$ の個数は $\dim V_1$ 個
	\end{itemize}
	が成り立つ.
	% 命題\ref{prop:root-decomp-ortho}-(6) で構成した同型写像 $f \colon \Lsl(2,\, \mathbb{K}) \xrightarrow{\cong} \mathfrak{s}_\alpha$ を用いて $\ad_{\mathfrak{g}}|_{\mathfrak{s}_\alpha} \circ f$ を考えることでこれは $\Lsl(2,\, \mathbb{K})$-加群と見做せるので,

	\begin{description}
		\item[(1), (2)] 
		
		$\mathfrak{g}$ の\underline{部分ベクトル空間}
		\begin{align}
			\label{eq:prop2.5.4-1}
			\mathfrak{m} \coloneqq \mathfrak{h} \oplus \bigoplus_{\lambda \in \mathbb{K}^*} \mathfrak{g}_{\lambda\alpha} \subset \mathfrak{g}
		\end{align}
		を考える\footnote{$\mathbb{K}^*$ の元は $\mathbb{K} \lto \mathbb{K}$ なる $\mathbb{K}$-線型写像ということなので,要するにスカラー倍である.}.
		命題\ref{prop:root-decomp-basic1}-(1) より $\forall \lambda \in \mathbb{K}^*$ に対して
		\begin{align}
			x_\alpha \btr \mathfrak{g}_{\lambda\alpha} &= \ad_{\mathfrak{g}}(x_\alpha) (\mathfrak{g}_{\lambda\alpha}) = \comm{x_\alpha}{\mathfrak{g}_{\lambda\alpha}} \subset \mathfrak{g}_{(\lambda+1)\alpha}, \\
			y_\alpha \btr \mathfrak{g}_{\lambda\alpha} &= \ad_{\mathfrak{g}}(y_\alpha) (\mathfrak{g}_{\lambda\alpha}) = \comm{y_\alpha}{\mathfrak{g}_{\lambda\alpha}} \subset \mathfrak{g}_{(\lambda-1)\alpha}, \\
			h_\alpha \btr \mathfrak{g}_{\lambda\alpha} &= \ad_{\mathfrak{g}}(h_\alpha) (\mathfrak{g}_{\lambda\alpha}) = \comm{h_\alpha}{\mathfrak{g}_{\lambda\alpha}} \subset \mathfrak{g}_{\lambda\alpha}
		\end{align}
		が成り立つので,$\mathfrak{m}$ は $\mathfrak{g}$ の\hyperref[def:sub-g-module]{部分 $\mathfrak{s}_\alpha$-加群}である.
		また,\hyperref[def:ga]{$\mathfrak{g}_\alpha$ の定義}および命題\ref{prop:root-decomp-ortho}-(5), (7) から,$\lambda\alpha \in \Phi$ を充たす任意の $\lambda \in \mathbb{K}^*$ に対して
		\begin{align}
			\forall x \in \mathfrak{g}_{\lambda \alpha},\; h_\alpha \btr x = \ad_{\mathfrak{g}}(h_\alpha)(x) = \lambda \alpha(h_\alpha) x = 2\lambda x
		\end{align}
		が成り立つ.i.e. $\mathfrak{g}_{\lambda\alpha} = V_{2\lambda} \cap \mathfrak{m}$ である.
		さらに補題\ref{lem:torus}より $h_\alpha \btr \mathfrak{h} = \ad_{\mathfrak{g}}(h_\alpha)(\mathfrak{h}) = \comm{h_\alpha}{\mathfrak{h}} = 0$ 
		% \hyperref[def:ga]{$\mathfrak{g}_\alpha$ の定義}および命題\ref{prop:root-decomp-ortho}-(5), (7) から $\mathfrak{g}_{\lambda\alpha} \neq 0$ を充たす任意の $\lambda \in \mathbb{K}^* \setminus \{0\}$ および $\forall x \in \mathfrak{g}_{\lambda\alpha}$ に対して $h_\alpha \btr x = \ad_{\mathfrak{g}}(h_\alpha)(x) = \lambda \alpha(h_\alpha) x = 2\lambda x$ が
		が言えるので,$\mathfrak{h} = \mathfrak{g}_0 = V_0 \cap \mathfrak{m}$ である.
		% 言えるので,$\mathfrak{g}_{\lambda\alpha} \neq 0 \IMP \lambda \in \frac{1}{2} \mathbb{Z}$ が分かった.

		 ここで $\dim (\Ker \alpha) = \dim \mathfrak{h} - 1$ であり\footnote{$\alpha \neq 0$ なので $\dim (\Im \alpha) = 1$ である.よって\hyperref[col:rank-nullity]{階数・退化次元の定理}から $\dim (\Ker \alpha) = \dim \mathfrak{h} - 1$.},$\forall h \in \Ker \alpha$ に対して
		\begin{align}
			x_\alpha \btr h &= -\ad_{\mathfrak{g}}(h)(x_\alpha) = \alpha(h)x_\alpha = 0, \\
			y_\alpha \btr h &= -\ad_{\mathfrak{g}}(h)(y_\alpha) = -\alpha(h)x_\alpha = 0, \\
			h_\alpha \btr h &= -\ad_{\mathfrak{g}}(h)(h_\alpha) = 0
		\end{align}
		が成り立つ.i.e. $\mathfrak{s}_\alpha$ は $\Ker \alpha$ に自明に作用するので,$\Ker \alpha$ は $\dim \mathfrak{h} - 1$ 個の自明な(従ってウエイト $0$ の)既約 $\mathfrak{s}_\alpha$-部分加群の直和である.
		一方,$\mathfrak{s}_\alpha \subset \mathfrak{g}$ 自身もまた $\mathfrak{g}$ の既約な部分 $\mathfrak{s}_\alpha$-加群であり,命題\ref{prop:root-decomp-ortho}-(6) より
		\begin{align}
			h_\alpha \btr x_\alpha &= \ad_{\mathfrak{g}}(h_\alpha)(x_\alpha) = \comm{h_\alpha}{x_\alpha} = 2x_\alpha, \\
			h_\alpha \btr y_\alpha &= \ad_{\mathfrak{g}}(h_\alpha)(x_\alpha) = \comm{h_\alpha}{x_\alpha} = -2y_\alpha, \\
			h_\alpha \btr h_\alpha &= \ad_{\mathfrak{g}}(h_\alpha)(h_\alpha) = 0
		\end{align}
		が成り立つのでウエイト $0,\, \pm 2$ を持つ.i.e. $\mathfrak{g}_{\pm \alpha} \subset \mathfrak{s}_\alpha$ である.従って $\Ker \alpha \oplus \mathfrak{s}_\alpha$ は\footnote{$\mathfrak{h} = \Ker \alpha \oplus \mathbb{K}h_\alpha$ なので $\Ker \alpha \cap \mathfrak{s}_\alpha = 0$ である.} 
		$(\dim \mathfrak{h} - 1) + 1 = \dim \mathfrak{h}$ 個の,偶数ウエイトを持つ既約な部分 $\mathfrak{s}_\alpha$-加群の直和に分解するが,系\ref{col:sl2}よりこのような既約部分 $\mathfrak{s}_\alpha$-加群の $\mathfrak{m}$ 内における総数は $\dim (V_0 \cap \mathfrak{m}) = \dim \mathfrak{h}$ 個なので,
		結局 $\mathfrak{m}$ 内の $h_\alpha$ の偶数ウエイトは $0,\, \pm 2$ 以外に出現しないことが分かった.
		従って $2\alpha,\, \frac{1}{2}\alpha \notin \Phi$ である\footnote{議論の冒頭で $\alpha \in \Phi$ は任意にとっていたので,もし $2\alpha \in \Phi$ ならば,$\beta \coloneqq 2\alpha \in \Phi$ とおいたときに $2\beta = 4\alpha \in \Phi$ という事になり矛盾.$\alpha = 2 (\frac{1}{2}\alpha) \in \Phi$ なので $\frac{1}{2}\alpha \notin \Phi$ も言えた.}.
		このことから $V_1 \cap \mathfrak{m} = 0$ がわかり,系\ref{col:sl2}から $\mathfrak{m}$ 内に奇数ウエイトが出現しないこともわかった.
		以上の考察より
		\begin{align}
			\mathfrak{m} = \Ker \alpha \oplus \mathfrak{s}_\alpha
		\end{align}
		なので $\mathbb{K}h_\alpha \subset \mathfrak{s}_\alpha$ であり,$\dim \mathfrak{s}_\alpha = 3$ であることから $\dim \mathfrak{g}_{\alpha} = 1$ が言えた.特に命題\ref{prop:root-decomp-ortho}-(4) より $\mathbb{K}h_\alpha = \comm{\mathfrak{g}_{\alpha}}{\mathfrak{g}_{-\alpha}} \eqqcolon \mathfrak{h}_\alpha$ であり,
		\begin{align}
			\mathfrak{s}_\alpha = \mathfrak{g}_{\alpha} \oplus \mathfrak{g}_{-\alpha} \oplus \mathfrak{h}_{\alpha}
		\end{align}
		も言えた.
		\item[(3), (4), (5)] 
		
		$\beta = \pm \alpha$ のとき,$\alpha(h_\alpha) = 2$ なので (3) が成り立つ.
		よって以下では $\forall \beta \in \Phi \setminus \{\pm \alpha\}$ を1つとる.
		$\mathfrak{g}$ の\underline{部分ベクトル空間}
		\begin{align}
			\mathfrak{k} \coloneqq \bigoplus_{\lambda \in \mathbb{Z}} \mathfrak{g}_{\beta + \lambda \alpha}
		\end{align}
		を考える.命題\ref{prop:root-decomp-basic1}-(1) より $\forall \lambda \in \mathbb{Z}$ に対して
		\begin{align}
			x_\alpha \btr \mathfrak{g}_{\beta + \lambda\alpha} &= \comm{x_\alpha}{\mathfrak{g}_{\beta + \lambda\alpha}} \subset \mathfrak{g}_{\beta + (\lambda+1)\alpha}, \\
			y_\alpha \btr \mathfrak{g}_{\beta + \lambda\alpha} &= \comm{y_\alpha}{\mathfrak{g}_{\beta + \lambda\alpha}} \subset \mathfrak{g}_{\beta + (\lambda-1)\alpha}, \\
			h_\alpha \btr \mathfrak{g}_{\beta + \lambda\alpha} &= \comm{h_\alpha}{\mathfrak{g}_{\beta + \lambda\alpha}} \subset \mathfrak{g}_{\beta + \lambda\alpha}
		\end{align}
		が成り立つので,$\mathfrak{k}$ は $\mathfrak{g}$ の\hyperref[def:sub-g-module]{部分 $\mathfrak{s}_\alpha$-加群}である.
		(1) において $\alpha$ は任意だったので $\forall \lambda \in \mathbb{Z}$ に対して $\dim \mathfrak{g}_{\beta + \lambda \alpha} = 1$ である.
		\hyperref[def:ga]{$\mathfrak{g}_\alpha$ の定義}および命題\ref{prop:root-decomp-ortho}-(5), (7) から,$\beta + \lambda\alpha \in \Phi$ を充たす任意の $\lambda \in \mathbb{Z}$ に対して
		% さらに $\beta + \lambda\alpha \neq 0$ なので,
		\begin{align}
			\forall x \in \mathfrak{g}_{\beta + \lambda\alpha},\; h_\alpha \btr x = \bigl(\beta(h_\alpha) + 2\lambda\bigr)x
		\end{align}
		が成り立つ.i.e. $\mathfrak{g}_{\beta + \lambda\alpha} = V_{\beta(h_\alpha) + 2\lambda} \cap \mathfrak{k}$ である.特に $\beta(h_\alpha) + 2\lambda \in \mathbb{Z}$ なので $\beta(h_\alpha) \in \mathbb{Z}$ でなくてはならない(\textbf{(3) の前半}).
		% (2) より $-\beta = \beta - \beta(h_\alpha)\alpha \in \Phi$
		
		 $\alpha,\, \beta$ を固定したとき,$\beta(h_\alpha) \in \mathbb{Z}$ だと分かったので $\beta(h_\alpha) \equiv 0,\, 1 \mod 2$ のどちらかである.
		$\beta(h_\alpha) \equiv 0 \mod 2$ ならば
		\begin{align}
			V_0 \cap \mathfrak{k} = \mathfrak{g}_{\beta - \frac{\beta(h_\alpha)}{2}\alpha},\quad V_1 \cap \mathfrak{k}= 0
		\end{align}
		$\beta(h_\alpha) \equiv 1 \mod 2$ ならば
		\begin{align}
			V_0 \cap \mathfrak{k} = 0,\quad  V_1 \cap \mathfrak{k}= \mathfrak{g}_{\beta - \frac{\beta(h_\alpha) - 1}{2}\alpha}
		\end{align}
		なので,いずれの場合も $\dim (V_0 \cap \mathfrak{k}) + \dim (V_1 \cap \mathfrak{k}) = 1$ であり,系\ref{col:sl2}より $\mathfrak{k}$ は\hyperref[def:irr]{既約}な $\mathfrak{s}_\alpha \cong \Lsl(2,\, \mathbb{K})$-加群であることが分かった.
		従って $\mathfrak{k}$ の\hyperref[eq:maximal-weight]{最高(resp. 最低)ウエイト} $m$(resp. $l$)が存在し,
		\begin{align}
			\beta(h_\alpha) + 2q &\coloneqq m, \\
			\beta(h_\alpha) - 2r &\coloneqq l
		\end{align}
		とおけば
		\begin{align}
			q &= \max \bigl\{\, \lambda \in \mathbb{Z} \bigm| \beta + \lambda\alpha \in \Phi \,\bigr\}, \\
			r &= \max \bigl\{\, \lambda \in \mathbb{Z} \bigm| \beta - \lambda\alpha \in \Phi \,\bigr\} 
		\end{align}
		が成り立ち,さらに定理\ref{thm:irr-sl2}-(1) より $-r \le \forall \lambda \le q$ に対して $\beta(h_\alpha) + 2\lambda = (\beta + \lambda\alpha)(h_\alpha)$ は $\mathfrak{k}$ のウエイトなので
		\begin{align}
			\beta + \lambda\alpha \in \Phi
		\end{align}
		である.特に $l = -m$ なので
		\begin{align}
			\beta(h_\alpha) = r-q \in \mathbb{Z}
		\end{align}
		が言えた(\textbf{(5)}).さらに $q,\, r \ge 0$ なので $-r \le q-r \le q$ が成り立ち,
		\begin{align}
			\beta - \beta(h_\alpha)\alpha = \beta + (q-r)\alpha \in \Phi
		\end{align}
		が分かった(\textbf{(3)の後半}).

		 (2) および $0 \notin \Phi$ であることから $\alpha + \beta \in \Phi$ ならば $\beta \neq \pm \alpha$ である.よって補題\ref{lem:sl2-2}より $x_\alpha \btr \mathfrak{g}_\beta = \comm{x_\alpha}{\mathfrak{g}_\beta} \neq 0$ である.(1) より $\dim \mathfrak{g}_{\alpha+\beta} = 1$ なので,命題\ref{prop:root-decomp-basic1}-(1) から $\comm{\mathfrak{g}_\alpha}{\mathfrak{g}_\beta} = \mathfrak{g}_{\alpha+\beta}$ が言えた(\textbf{(4)}).

		% \item[(6)] 
		

		\end{description}
\end{proof}

\subsection{有理性}

\begin{itemize}
	\item 体 $\mathbb{K}$ 上の有限次元\hyperref[def:semisimple-LieAlg]{半単純Lie代数} $\mathfrak{g}$
	\item $\mathfrak{g}$ の\hyperref[def:toral-subLieAlg]{極大トーラス} $\mathfrak{h} \subset \mathfrak{g}$
	\item $\mathfrak{h}$ のルート全体の集合 $\Phi \subset \mathfrak{h}^*$ 
\end{itemize}
を与える.$\mathfrak{g}$ の\hyperref[eq:root-decomp]{ルート空間分解}を命題\ref{prop:torus-centralize}によって
\begin{align}
	\mathfrak{g} = \mathfrak{h} \oplus \bigoplus_{\alpha \in \Phi} \mathfrak{g}_\alpha
\end{align}
と書く.系\ref{col:torus-centralizer}から得られる同型\eqref{eq:isom-h}によって,$\mathfrak{h}^*$ 上に\hyperref[def:radical-bilinear]{非退化}かつ対称な双線型形式
\begin{align}
	(\; ,\, ) \colon \mathfrak{h}^* \times \mathfrak{h}^* \lto \mathbb{K},\; (\alpha,\, \beta) \lmto \kappa\bigl(\tilde{\kappa}^{-1}(\alpha),\, \tilde{\kappa}^{-1}(\beta)\bigr) = \kappa(t_\alpha,\, t_\beta)
\end{align}
を定義する.

命題\ref{prop:root-decomp-ortho}-(1) より,$\dim \mathfrak{h}$ 個のルート $\alpha^1,\, \dots,\, \alpha^{\dim \mathfrak{h}} \in \Phi$ であって $\mathfrak{h}^*$ の \underline{$\mathbb{K}$-ベクトル空間として}の基底となるようなものが存在する.

\begin{mylem}[label=lem:root-Q]{}
	$\forall \beta \in \mathfrak{h}^*$ を基底 $\alpha^1,\, \dots,\, \alpha^{\dim \mathfrak{h}} \in \Phi$ で
	\begin{align}
		\beta = c_\mu \alpha^\mu
	\end{align}
	と展開すると,$1 \le \forall \mu \le \dim \mathfrak{h}$ に対して $c_\mu \in \mathbb{Q}$ が成り立つ.
\end{mylem}

\begin{proof}
	この証明ではEinsteinの規約を使わない.$1 \le \forall \mu \le \dim \mathfrak{h}$ に対して
	\begin{align}
		\label{eq:LE}
		\frac{2(\alpha^\mu,\, \beta) }{(\alpha^\mu,\, \alpha^\mu)}
		&= \sum_{\nu = 1}^{\dim \mathfrak{h}} c_\nu \frac{2(\alpha^\mu,\, \alpha^\nu)}{(\alpha^\mu,\, \alpha^\mu)}
	\end{align}
	命題\ref{prop:root-decomp-ortho}-(7), 命題\ref{prop:root-decomp-int}-(3) より $\frac{2(\alpha^\mu,\, \beta) }{(\alpha^\mu,\, \alpha^\mu)} = \frac{2}{\kappa(t_{\alpha^\mu},\, t_{\alpha^\mu})}\beta(t_{\alpha^\mu}) = \beta(h_{\alpha^\mu}) \in \mathbb{Z}$ などが成り立つので,
	\eqref{eq:LE}は $(c_\nu) \in \mathbb{K}^{\dim \mathfrak{h}}$ に関する $\mathbb{Z}$-係数線型連立方程式である.特に $\alpha^\mu$ は線型独立なので $\det \bigl[ (\alpha^\mu,\, \alpha^\nu) \bigr]_{1 \le \mu,\, \nu \le \dim \mathfrak{h}} \neq 0$ であり,
	\begin{align}
		\det \left[ \frac{2(\alpha^\mu,\, \alpha^\nu)}{(\alpha^\mu,\, \alpha^\mu)} \right]_{1 \le \mu,\, \nu \le \dim \mathfrak{h}} = \left( \prod_{\nu = 1}^{\dim \mathfrak{h}} \frac{2}{(\alpha^\mu,\, \alpha^\mu)}\right)  \det \bigl[ (\alpha^\mu,\, \alpha^\nu) \bigr]_{1 \le \mu,\, \nu \le \dim \mathfrak{h}} \neq 0
	\end{align}
	i.e. \eqref{eq:LE} は $\mathbb{Q}^{\dim \mathfrak{h}}$ に一意的な解を持つ.
\end{proof}

補題\ref{lem:root-Q}により,$\Span_{\mathbb{Q}} \Phi \subset \mathfrak{h}^*$ の次元が $\dim_{\mathbb{K}} \mathfrak{h}^*$ だと分かった.

$\forall \alpha,\, \beta \in \mathfrak{h}^*$ をとる.
このとき\hyperref[def:ga]{$\mathfrak{g}_\alpha$ の定義}を思い出すと,$\forall x \in \mathfrak{g}$ をルート空間分解することで $x = x_0 + \sum_{\gamma \in \Phi} x_\gamma \WHERE x_0 \in \mathfrak{h},\; x_\gamma \in \mathfrak{g}_\gamma$ と書けるので
\begin{align}
	\ad_{\mathfrak{g}}(t_\alpha) \circ \ad_{\mathfrak{g}}(t_\beta)(x)
	&= \sum_{\gamma \in \Phi} \ad_{\mathfrak{g}}(t_\alpha)\circ \ad_{\mathfrak{g}}(t_\beta)(x_\gamma) \\
	&= \sum_{\gamma \in \Phi} \gamma(t_\alpha)\gamma(t_\beta)\, \mathrm{id}_{\mathfrak{g}}|_{\mathfrak{g}_{\gamma}}(x_\gamma)
\end{align}
となる.さらに命題\ref{prop:root-decomp-int}-(1) より $\dim \mathfrak{g}_\gamma = 1$ なので $\Tr( \mathrm{id}_{\mathfrak{g}}|_{\mathfrak{g}_{\gamma}}) = 1$ であり,
\begin{align}
	\kappa(t_\alpha,\, t_\beta)
	&= \Tr \bigl( \ad_{\mathfrak{g}}(t_\alpha) \circ \ad_{\mathfrak{g}}(t_\beta)(x) \bigr) \\
	&= \sum_{\gamma \in \Phi} \gamma(t_\alpha)\gamma(t_\beta)
\end{align}
が分かった.従って
\begin{align}
	(\alpha,\, \beta) 
	&= \kappa(t_\alpha,\, t_\beta) \\
	&= \sum_{\gamma \in \Phi} \gamma(t_\alpha) \gamma(t_\beta) \\
	&= \sum_{\gamma \in \Phi} (\gamma,\, \alpha)(\gamma,\, \beta)
\end{align}
が成り立つ.特に
\begin{align}
	(\beta,\, \beta) 
	&= \sum_{\gamma \in \Phi} (\gamma,\, \beta)^2 \\
	&= (\beta,\, \beta)^2 \left(\sum_{\gamma \in \Phi} \frac{1}{4} \left( \frac{2(\gamma,\, \beta)}{(\beta,\, \beta)} \right)^2\right) \ge 0
\end{align}
で $\frac{2(\gamma,\, \beta)}{(\beta,\, \beta)} \in \mathbb{Z}$ であるから $(\beta,\, \beta) \in \mathbb{Q}$ であり,従って $(\alpha,\, \beta) \in \mathbb{Q}$ であることもわかった.
このことから,
\begin{align}
	\mathbb{E}_{\mathbb{Q}} \coloneqq \Span_{\mathbb{Q}} \Phi
\end{align}
とおけば,制限
\begin{align}
	(\;,\, )|_{\mathbb{E}_{\mathbb{Q}} \times \mathbb{E}_{\mathbb{Q}}} \colon \mathbb{E}_{\mathbb{Q}} \times \mathbb{E}_{\mathbb{Q}} \lto \mathbb{Q}
\end{align}
は \underline{$\mathbb{Q}$-ベクトル空間} $\mathbb{E}_{\mathbb{Q}}$ 上の\underline{正定値内積}を定める.

\hyperref[def:field-extention]{体の拡大} $\mathbb{R}/\mathbb{Q}$ によってベクトル空間の係数拡大 $\mathbb{E} \coloneqq \mathbb{R} \otimes_{\mathbb{Q}} \mathbb{E}_{\mathbb{Q}}$ を行うことで,$(\;,\, )|_{\mathbb{E}_{\mathbb{Q}}}$ は $\mathbb{E} \cong \mathbb{R}^{\dim \mathfrak{h}}$ 上の正定値内積へと自然に拡張される.i.e. 組 $\bigl(\mathbb{E},\, (\;,\, )\bigr)$ はEuclid空間である!

本章の考察をまとめて次の定理を得る:

\begin{mytheo}[label=thm:root-decomp-Q,breakable]{}
	\begin{itemize}
		\item 体 $\mathbb{K}$ 上の有限次元\hyperref[def:semisimple-LieAlg]{半単純Lie代数} $\mathfrak{g}$
		\item $\mathfrak{g}$ の\hyperref[def:toral-subLieAlg]{極大トーラス} $\mathfrak{h} \subset \mathfrak{g}$
		\item $\mathfrak{h}$ のルート全体の集合 $\Phi \subset \mathfrak{h}^*$ 
		\item \hyperref[def:field-extention]{体の拡大} $\mathbb{R}/\mathbb{Q}$ による $\Span_{\mathbb{Q}}\Phi$ の係数拡大 $\mathbb{E}$
		\item $\mathbb{E}$-上の正定値内積
		\begin{align}
			(\; ,\, ) \colon \mathbb{E} \times \mathbb{E} \lto \mathbb{R},\; (\alpha,\, \beta) \lmto \kappa(t_\alpha,\, t_\beta)
		\end{align}
	\end{itemize}
	を与える.このとき以下が成り立つ:
	\begin{enumerate}
		\item $\mathbb{E} = \Span_{\mathbb{R}}\Phi,\quad 0 \notin \Phi$
		\item $\lambda \alpha \in \Phi \IMP \lambda = \pm 1$
		\item $\alpha,\, \beta \in \Phi \IMP \beta - \frac{2(\beta,\, \alpha)}{(\alpha,\, \alpha)} \alpha \in \Phi$
		\item $\alpha,\, \beta \in \Phi \IMP \frac{2(\beta,\, \alpha)}{(\alpha,\, \alpha)} \in \mathbb{Z}$
	\end{enumerate}
\end{mytheo}

定理\ref{thm:root-decomp-Q}によって,\hyperref[def:semisimple-LieAlg]{半単純Lie代数} $\mathfrak{g}$ とその\hyperref[def:toral-subLieAlg]{極大トーラス} $\mathfrak{h}$ の組 $(\mathfrak{g},\, \mathfrak{h})$ 
から,Euclid空間とその上の\hyperref[eq:root-decomp1]{ルート} $\Phi$ の組 $(\Phi,\, \mathbb{E})$ の間の対応
\begin{align}
	(\mathfrak{g},\, \mathfrak{h}) \lto (\Phi,\, \mathbb{E})
\end{align}
が得られた事になる.
定理\ref{thm:root-decomp-Q}の条件を充たす組 $(\Phi,\, \mathbb{E})$ は\textbf{ルート系} (root system) と呼ばれる:

\begin{myaxiom}[label=ax:root-system,breakable]{ルート系}
	\begin{itemize}
		\item 有限次元Euclid空間\footnote{有限次元 $\mathbb{R}$-ベクトル空間であって,対称かつ正定値な双線型形式 $(\;,\, ) \colon \mathbb{E} \times \mathbb{E} \lto \mathbb{R}$ を持つもののこと.} $\bigl(\mathbb{E},\, (\;,\, )\bigr)$ 
		\item $\mathbb{E}$ の部分集合 $\Phi \subset \mathbb{E}$
	\end{itemize}
	の組 $(\Phi,\, \mathbb{E})$ が\textbf{ルート系} (root system) であるとは,以下の条件を充たすことを言う:
	\begin{description}
		\item[\textbf{(Root-1)}] $\Phi$ は $0$ を含まない有限集合で,かつ $\mathbb{E} = \Span_{\mathbb{R}} \Phi$ を充たす.
		\item[\textbf{(Root-2)}] $\lambda \alpha \in \Phi \IMP \lambda = \pm 1$
		\item[\textbf{(Root-3)}] $\alpha,\, \beta \in \Phi \IMP \beta - \frac{2(\beta,\, \alpha)}{(\alpha,\, \alpha)} \alpha \in \Phi$
		\item[\textbf{(Root-4)}] $\alpha,\, \beta \in \Phi \IMP \frac{2(\beta,\, \alpha)}{(\alpha,\, \alpha)} \in \mathbb{Z}$
	\end{description}
\end{myaxiom}

実は,勝手なルート系が与えられると,それに対応する $(\mathfrak{g},\, \mathfrak{h})$ がLie代数の同型を除いて一意に決まることが後に明らかになる.
差し当たって,まず次の章ではルート系 $(\Phi,\, \mathbb{E})$ の完全な分類を行う.

\end{document}