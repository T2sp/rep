\documentclass[rep_main]{subfiles}

\begin{document}

\setcounter{chapter}{1}

\chapter{半単純Lie代数}

この章において,特に断らない限り体 $\mathbb{K}$ は代数閉体\footnote{つまり,定数でない任意の1変数多項式 $f(x) \in \mathbb{K}[x]$ に対してある $\alpha \in \mathbb{K}$ が存在して $f(\alpha) = 0$ を充たす.}で,かつ $\character \mathbb{K} = 0$ であるとする.

\section{Lieの定理・Cartanの判定条件}

\subsection{Lieの定理}

\subsection{Jordan-Chevalley分解}
\subsection{Cartanの判定条件}

\section{Killing形式}

\section{半単純性の判定条件}
\section{単純イデアル}
\section{内部微分}
\section{抽象Jordan分解}

\section{表現の完全可約性}

ベクトル空間のテンソル積について復習する:

\begin{mydef}[label=def:tensor-vec]{ベクトル空間のテンソル積}
    $\mathbb{K}$ を任意の体とし,$\mathbb{K}$-ベクトル空間 $V,\, W$ を与える.
    \begin{itemize}
        \item $\mathbb{K}$-ベクトル空間 $\bm{V \otimes W}$ 
        \item 双線型写像 $\bm{\Phi} \colon V \times W \lto \bm{V \otimes W}$ 
    \end{itemize}
    の組 $(V \otimes W,\, \Phi)$ が $V,\, W$ の\textbf{テンソル積} (tensor product) であるとは,以下の性質を充たすことをいう:
    \begin{description}
        \item[\textbf{テンソル積の普遍性}] 
        
        任意の $\mathbb{K}$-ベクトル空間 $\textcolor{blue}{Z}$ および任意の双線型写像 $\textcolor{blue}{f} \colon V \times W \lto \textcolor{blue}{Z}$ に対して,以下の図式を可換にする線型写像 $\textcolor{red}{\overline{f}} \colon \bm{V \otimes W} \lto \textcolor{blue}{Z}$ が一意的に存在する:
        \begin{center}
            \begin{tikzcd}[row sep=large, column sep=large]
                V \times W \ar[d, "\bm{\Phi}"']\ar[r, blue, "f"] &\forall \textcolor{blue}{Z} \\
                \bm{V \otimes W} \arrow[ur, red, dashed, "\exists!\bar{f}"']&
            \end{tikzcd}
        \end{center}
    \end{description}
\end{mydef}

テンソル積をとる操作は結合的である.i.e. $V_1 \otimes (V_2 \otimes V_3) \cong (V_1 \otimes V_2) \otimes V_3$ が成り立つ.
従って以降では3つ以上のベクトル空間のテンソル積を括弧を省略して書く.

\begin{myprop}[label=prop:unique-tensor-vec]{テンソル積の一意性}
	\hyperref[def:tensor-vec]{テンソル積}は,存在すればベクトル空間の同型を除いて一意である.
\end{myprop}

\begin{proof}
	$\mathbb{K}$ 上のベクトル空間 $V,\, W \in \Obj{\VEC{\mathbb{K}}}$ を与える.
	体 $\mathbb{K}$ 上のベクトル空間と双線型写像の組 $(\bm{T},\, \bm{\Phi} \colon V \times W \lto \bm{T})$ および $(\bm{T},\, \bm{\Phi'} \colon V \times W \lto \bm{T'})$ がどちらも $V,\, W$ の\hyperref[def:tensor-vec]{テンソル積}であるとする.
	このとき\hyperref[def:tensor-vec]{テンソル積の普遍性}から $\VEC{\mathbb{K}}$ の\hyperref[def:commutative]{可換図式}
	\begin{figure}[H]
		\centering
		\begin{subfigure}{0.4\columnwidth}
			\centering
			\begin{tikzcd}[row sep=large, column sep=large]
				&V \times W \ar[d, "\bm{\Phi}"] \ar[r, blue, "\Phi'"] &\textcolor{blue}{T'} \\
				&\bm{T} \ar[ur, red, dashed, "\exists! u"] &
			\end{tikzcd}
		\end{subfigure}
		\hspace{5mm}
		\begin{subfigure}{0.4\columnwidth}
			\centering
			\begin{tikzcd}[row sep=large, column sep=large]
				&V \times W \ar[d, "\bm{\Phi'}"] \ar[r, blue, "\Phi"] &\textcolor{blue}{T} \\
				&\bm{T'} \ar[ur, red, dashed, "\exists! u'"] &
			\end{tikzcd}
		\end{subfigure}
	\end{figure}%
	が成り立つので,これらの図式を併せた $\VEC{\mathbb{K}}$ の可換図式
	\begin{center}
		\begin{tikzcd}[row sep=large, column sep=large]
			&\bm{T} & \\
			&V \times W \ar[u, "\bm{\Phi}"]\ar[d, "\bm{\Phi}"] \ar[r, blue, "\Phi'"] &\textcolor{blue}{T'}\ar[ul, red, dashed, "\exists! u'"] \\
			&\bm{T} \ar[ur, red, dashed, "\exists! u"] &
		\end{tikzcd}
	\end{center}
	が存在する.然るに $\VEC{\mathbb{K}}$ の可換図式
	\begin{center}
		\begin{tikzcd}[row sep=large, column sep=large]
			&V \times W \ar[d, "\bm{\Phi}"] \ar[r, "\bm{\Phi}"] &\bm{T'} \\
				&\bm{T} \ar[ur, red, dashed, "\mathrm{id}_T"] &
		\end{tikzcd}
	\end{center}
	も成り立ち,\hyperref[def:tensor-vec]{テンソル積の普遍性}より赤点線で書いた線型写像は一意でなくてはならないので,
	\begin{align}
		\textcolor{red}{u'} \circ \textcolor{red}{u} = \mathrm{id}_T
	\end{align}
	がわかる.同様の議論から
	\begin{align}
		\textcolor{red}{u} \circ \textcolor{red}{u'} = \mathrm{id}_{T'}
	\end{align}
	も従うので,線型写像 $\textcolor{red}{u} \colon \bm{T} \lto \bm{T'},\; \textcolor{red}{u'} \colon \bm{T'} \lto \bm{T}$ は互いに逆写像,i.e. 同型写像である.
\end{proof}


命題\ref{prop:unique-tensor-vec}からテンソル積の一意性が言えたが,そもそもテンソル積が存在しなければ意味がない.
そこで,\hyperref[ax.ring]{体} $\mathbb{K}$ 上の任意のベクトル空間 $V,\, W \in \Obj{\VEC{\mathbb{K}}}$ を素材にして\hyperref[def:tensor-vec]{テンソル積} $(V \otimes W,\, \Phi \colon V \times W \lto V \otimes W)$ を具体的に構成してみよう.

$\mathbb{K} \in \Obj{\VEC{\mathbb{K}}}$ なので,任意の集合 $S$ に対して\hyperref[prop:sum-vect]{ベクトル空間の直和}
\begin{align}
	\mathbb{K}^{\oplus S} \in \Obj{\VEC{\mathbb{K}}}
\end{align}
を考えることができる.$\mathbb{K}^{\oplus S}$ の元 $f$ とは,有限個の元 $x_1,\, \dots,\; x_n \in S$ を除いた全ての $x \in S$ に対して値 $0 \in \mathbb{K}$ を返すような $\mathbb{K}$ 値関数
$f \colon S \lto \mathbb{K}$ のことである
\footnote{これは集合論の記法である:ある集合 $\Lambda$ から集合 $A$ への写像 $a \colon \Lambda \lto A$ のことを $\Lambda$ によって\textbf{添字づけられた} $A$ の元の\textbf{族}と呼び,$\forall \lambda \in \Lambda$ に対して $a(\lambda) \in A$ のことを $\bm{a_\lambda}$ と書き,$\bm{a \colon \Lambda \lto A}$ \textbf{自身のこと}を $\bm{(a_\lambda)_{\lambda \in \Lambda}}$ と書くのである.
なお,$\Familyset[\big]{a_\lambda}{\lambda \in \Lambda}$ と書いたときは $A$ の部分集合 $\bigl\{\, a(\lambda) \bigm| \lambda \in \Lambda \,\bigr\} \subset A$ のことを意味する.}.

ところで,$\forall x \in S$ に対して次のような関数 $\delta_{x} \in \mathbb{K}^{\oplus S}$ が存在する:
\begin{align}
	\delta_{x} \bigl(y\bigr) =
	\begin{cases}
		1, &y = x \\
		0, &y \neq x
	\end{cases}
\end{align}
この $\delta_{x}$ を $x$ そのものと同一視してしまうことで,先述の $f \in \mathbb{K}^{\oplus (V \times W)}$ は
\begin{align}
	f = \sum_{i=1}^n \lambda_i x_i \quad \WHERE \lambda_i \coloneqq f (x_i) \in \mathbb{K}
\end{align}
の形に一意的に書ける.
\footnote{
	というのも,このように書けば $\forall y \in S$ に対して
	\begin{align}
		f (y)  &= \sum_{i=1}^n \lambda_i \delta_{x_i}(y) =
		\begin{cases}
			f (x_i), &y = x_i \\
			0, &\text{otherwise}
		\end{cases}
	\end{align}
	が言えるので.特に,この式の中辺は $\mathbb{K}$ の元の有限和なので意味を持つ.
}
この意味で,$\mathbb{K}^{\oplus (V \times W)}$ は $V \times W$ の元の\textbf{形式的な} $\bm{\mathbb{K}}$ \textbf{係数線形結合}全体がなす $\mathbb{K}$ ベクトル空間と言うことができ,集合 $V \times W$ 上の\textbf{自由ベクトル空間}と呼ばれる.
\hyperref[def:free-mod]{自由加群}の特別な場合と言っても良い.自由ベクトル空間は次の普遍性によって特徴づけられる:

\begin{mylem}[label=lem:univ-free-vec, breakable]{自由ベクトル空間の普遍性}
	任意の集合 $S$ および任意の $\mathbb{K}$ ベクトル空間 $\textcolor{blue}{Z} \in \Obj{\VEC{\mathbb{K}}}$ を与える.
	包含写像
	\begin{align}
		\iota \colon S \lto \mathbb{K}^{\oplus S},\; x \lmto \delta_x
	\end{align}
	を考える.
	このとき,任意の写像 $\textcolor{blue}{f} \colon S \lto \textcolor{blue}{Z}$ に対して線型写像
	$\textcolor{red}{u} \colon \mathbb{K}^{\oplus S} \lto \textcolor{blue}{Z}$ が一意的に存在して,図式\ref{cmtd:free-vec}を可換にする:
	\begin{figure}[H]
		\centering
		\begin{tikzcd}[row sep=large, column sep=large]
			&S \ar[r, blue, "f"]\ar[d, "\iota"] &\forall \textcolor{blue}{Z} \\
			&\mathbb{K}^{\oplus S} \ar[ur, red, dashed, "\exists! u"] &
		\end{tikzcd}
		\caption{自由ベクトル空間の普遍性}
		\label{cmtd:free-vec}
	\end{figure}%
\end{mylem}

\begin{proof}
	写像
	\begin{align}
		\textcolor{red}{u} \colon \mathbb{K}^{\oplus S} \lto \textcolor{blue}{Z},\; \sum_{i=1}^n \lambda_i \delta_{x_i} \lmto \sum_{i=1}^n \lambda_i \textcolor{blue}{f}(x_i)
	\end{align}
	は右辺が有限和なのでwell-definedであり,$\forall x \in S$ に対して $\textcolor{red}{u} \bigl( \iota(x) \bigr) = \textcolor{blue}{f}(x)$ を充たす.
	
	別の線型写像 $g \colon \mathbb{K}^{\oplus S} \lto \textcolor{blue}{Z}$ が $g \circ \iota = \textcolor{blue}{f}$ を充たすとする.このとき $\forall x \in S$ に対して $g(\delta_x) = \textcolor{blue}{f}(x)$ であるから,
	$\forall v = \sum_{i=1}^n \lambda_i \delta_{x_i} \in \mathbb{K}^{\oplus S}$ に対して
	\begin{align}
		g(v) = g \left( \sum_{i=1}^n \lambda_i \delta_{x_i} \right) = \sum_{i=1}^n \lambda_i g(\delta_{x_i}) = \sum_{i=1}^n \lambda_i \textcolor{blue}{f}(x_i) = \textcolor{red}{u} (v)
	\end{align}
	が言える.よって $g = \textcolor{red}{u}$ である.
\end{proof}

さて,\hyperref[cmtd:free-vec]{自由加群の普遍性の図式}と\hyperref[def:tensor-vec]{テンソル積の普遍性の図式}はとても似ているので,$\bm{V \otimes W} \in \Obj{\VEC{\mathbb{K}}}$ の候補として $\mathbb{K}^{\oplus (V \times W)}$ を考えてみる.
しかしそのままでは $\iota \colon V \times W \lto \mathbb{K}^{\oplus (V \times W)}$ が双線型写像になってくれる保証はない.
そこで,
\begin{align}
	\iota(\lambda v,\, w) &\sim \lambda \iota(v,\, w), \\
	\iota(v,\, \lambda w) &\sim \lambda \iota(v,\, w), \\
	\iota(v_1 + v_2,\, w) &\sim \iota(v_1,\, w) + \iota(v_2,\, w), \\
	\iota(v,\, w_1 + w_2) &\sim \iota(v,\, w_1) + \iota(v,\, w_2)
\end{align}
を充たすような上手い\hyperref[ax.eq]{同値関係}による\hyperref[prop:quotient-vec]{商ベクトル空間}を構成する.
% そこで,$\iota (V \times W) \in \mathbb{K}^{\oplus (V \times W)}$ の上に適切な関係式を課してできる $\mathbb{K}^{\oplus (V \times W)}$ の部分集合が\hyperref[prop:gen-submodule]{生成する部分ベクトル空間}で $\mathbb{K}^{\oplus (V \times W)}$ の\hyperref[prop:quotient-vec]{商ベクトル空間}を作ることを試みる.
% \hyperref[prop:universality-dp]{直和の普遍性}から $\forall \textcolor{blue}{V} \in \Obj{\VEC{\mathbb{K}}}$ に対して可換図式
% \begin{center}
% 	\begin{tikzcd}[row sep=large, column sep=large]
% 		&V \times W \ar[d, "\bm{\Phi}"] \ar[r, blue, "f"] &\forall \textcolor{blue}{Z} \\
% 		&\bm{V \otimes W} \ar[ur, red, dashed, "\exists! u"] &
% 	\end{tikzcd}
% \end{center}


\begin{myprop}[label=prop:tensor-vec, breakable]{テンソル積の構成}
	$\mathbb{K}^{\oplus (V \times W)}$ の部分集合
	\begin{align}
		S_1 &\coloneqq \bigl\{\iota(\lambda v,\, w) - \lambda \iota(v,\, w) \bigm| \forall v \in V,\, \forall w \in W,\, \forall \lambda \in \mathbb{K} \bigr\} , \\
		S_2 &\coloneqq \bigl\{\iota(v,\, \lambda w) - \lambda \iota(v,\, w) \bigm| \forall v \in V,\, \forall w \in W,\, \forall \lambda \in \mathbb{K} \bigr\}, \\
		S_3 &\coloneqq \bigl\{\iota(v_1 + v_2,\, w) - \iota(v_1,\, w) - \iota(v_2,\, w) \bigm| \forall v_1,\, \forall v_2 \in V,\, \forall w \in W,\, \forall \lambda \in \mathbb{K} \bigr\}, \\
		S_4 &\coloneqq \bigl\{\iota(v,\, w_1 + w_2) - \iota(v,\, w_1) - \iota(v,\, w_2) \bigm| \forall v \in V,\, \forall w_1,\, w_2 \in W,\, \forall \lambda \in \mathbb{K} \bigr\}
	\end{align}
	の和集合 $S_1 \cup S_2 \cup S_3 \cup S_4$ が\hyperref[prop:gen-submodule]{生成する} $\mathbb{K}$ ベクトル空間\footnote{これらの元の形式的な$\mathbb{K}$ 係数線型結合全体が成すベクトル空間のこと.}を $\mathcal{R}$ と書き,
	\hyperref[prop:quotient-vec]{商ベクトル空間}
	$\mathbb{K}^{\oplus (V \times W)} / \mathcal{R}$ 
	の商写像を
	\begin{align}
		\pi \colon \mathbb{K}^{\oplus (V \times W)} \lto \mathbb{K}^{\oplus (V \times W)} / \mathcal{R},\; \sum_{i=1}^n \lambda_i \iota(v_i,\, w_i) \lmto \left( \sum_{i=1}^n \lambda_i \iota(v_i,\, w_i) \right) + \mathcal{R}
	\end{align}
	と書き,$\bm{v \otimes w} \coloneqq \pi \bigl( \iota (v,\, w) \bigr)$ とおく.
	このとき,
	\begin{itemize}
		\item $\mathbb{K}$ ベクトル空間 $\bm{\mathbb{K}^{\oplus (V \times W)} / \mathcal{R}}$
		\item 写像 $\bm{\Phi} = \pi \circ \iota \colon V \times W \lto \bm{\mathbb{K}^{\oplus (V \times W)} / \mathcal{R}},\; (v,\, w) \lmto v \otimes w$
	\end{itemize}
	の組は $V,\, W$ の\hyperref[def:tensor-vec]{テンソル積}である.
\end{myprop}

\begin{proof}
	まず,$\bm{\Phi}$ が双線型写像であることを示す.\hyperref[prop:quotient-vec]{商ベクトル空間}の和とスカラー乗法の定義から
	\begin{align}
		\Phi (\lambda v,\, w) &= \iota(v,\, w) + \mathcal{R} = \bigl( \lambda \iota(v,\, w) + \iota (\lambda v,\, w) - \lambda \iota (v,\, w) \bigr) + \mathcal{R} \\
		&= \lambda \iota (v,\, w) + \mathcal{R} = \lambda \bigl( \iota(v,\, w) + \mathcal{R} \bigr) = \lambda \Phi(v,\, w) \\
		\Phi (v_1 + v_2,\, w) &= \iota(v_1 + v_2,\, w) + \mathcal{R} = \bigl( \iota(v_1,\, w) + \iota (v_2,\, w) + \iota(v_1+v_2,\, w) - \iota (v_1,\, w) - \iota(v_2,\, w) \bigr) + \mathcal{R} \\
		&= \bigl(\iota (v_1,\, w) + \iota (v_2,\, w) \bigr) + \mathcal{R} = \bigl( \iota(v_1,\, w) + \mathcal{R} \bigr) + \bigl( \iota(v_2,\, w) + \mathcal{R} \bigr) \\
		&= \Phi(v_1,\, w) + \Phi(v_2,\, w)
	\end{align}
	が言える.第2引数に関しても同様であり,$\bm{\Phi}$ は双線型写像である.

	次に,上述の構成が\hyperref[def:tensor-vec]{テンソル積の普遍性}を充たすことを示す.$\forall  \textcolor{blue}{Z} \in \Obj{\VEC{\mathbb{K}}}$ と任意の双線型写像 $\textcolor{blue}{f} \colon V \times W \lto \textcolor{blue}{Z}$ を与える.
	\hyperref[lem:univ-free-vec]{自由ベクトル空間の普遍性}から $\VEC{\mathbb{K}}$ の可換図式
	\begin{center}
		\begin{tikzcd}[row sep=large, column sep=large]
			&V \times W \ar[r, blue, "f"]\ar[d, "\iota"'] &\forall \textcolor{blue}{Z} \\
			&\mathbb{K}^{\oplus (V \times W)} \ar[ur, red, dashed, "\exists! \overline{f}"] &
		\end{tikzcd}
	\end{center}
	が存在する.$\textcolor{blue}{f}$ が双線型なので,
	\begin{align}
		\textcolor{red}{\overline{f}} \bigl( \iota (\lambda v,\, w) \bigr) &= f (\lambda v,\, w) = \lambda f(v,\, w) \\ 
		&= \lambda\textcolor{red}{\overline{f}} \bigl( \iota (v,\, w) \bigr) = \textcolor{red}{\overline{f}}\bigl( \lambda \iota(v,\, w) \bigr) , \\
		\textcolor{red}{\overline{f}} \bigl( \iota (v_1 + v_2,\, w) \bigr) &= f (v_1 + v_2,\, w) = f(v_1,\, w) + f(v_2,\, w)\\
		&= \textcolor{red}{\overline{f}} \bigl( \iota(v_1,\, w) \bigr) + \textcolor{red}{\overline{f}} \bigl(\iota(v_2,\, w) \bigr) = \textcolor{red}{\overline{f}} \bigl( \iota(v_1,\, w) + \iota(v_2,\, w)\bigr) 
	\end{align}
	が成り立つ.第2引数についても同様なので,$\mathcal{R} \subset \Ker \textcolor{red}{\overline{f}}$ である.よって\hyperref[thm.homo1]{準同型定理}から
	$\VEC{\mathbb{K}}$ の可換図式
	\begin{center}
		\begin{tikzcd}[row sep=large, column sep=large]
			&V \times W \ar[r, blue, "f"]\ar[d, "\iota"'] &\forall \textcolor{blue}{Z} \\
			&\mathbb{K}^{\oplus (V \times W)} \ar[d, "\pi"']\ar[ur, red, dashed, "\exists! \overline{f}"] & \\
			&\mathbb{K}^{\oplus (V \times W)} /\mathcal{R}\ar[uur, red, dashed, "\exists! u"] &
		\end{tikzcd}
	\end{center}
	が存在する.この図式の外周部は\hyperref[def:tensor-vec]{テンソル積の普遍性の図式}である.
\end{proof}

\begin{myprop}[label=prop:basis-tensor]{有限次元テンソル積の基底}
	\underline{有限次元} $\mathbb{K}$-ベクトル空間 $V,\, W$($\dim V \eqqcolon n,\; \dim W \eqqcolon m$)を与える.
    $V,\, W$ の基底をそれぞれ $\{e_1,\, \dots,\, e_n\},\; \{f_1,\, \dots,\, f_m\}$ と書く.
	このとき,集合
	\begin{align}
		\mathcal{E} \coloneqq \bigl\{\, e_\mu \otimes f_\nu \bigm| 1 \le \mu \le n,\, 1 \le \nu \le m \,\bigr\} 
	\end{align}
	は $V \otimes W$ の基底である.従って $\dim V \otimes W = nm$ である.
\end{myprop}

\begin{proof}
	\hyperref[prop:tensor-vec]{テンソル積の構成}から,$\forall t \in V \otimes W$ は有限個の $(v_i,\, w_i) \in V \times W\; (i=1,\, \dots l)$ を使って
	\begin{align}
		t = \left(\sum_{i=1}^l t_{i} \iota(v_i,\, w_i)\right) = \sum_{i=1}^l t_i v_i \otimes w_i
	\end{align}
	と書ける.$v_i = v_i{}^\mu e_\mu,\; w_i = w_i{}^{\mu} f_\mu$ のように展開することで,
	\begin{align}
		t &= \sum_{i=1}^l t_i (v_i{}^\mu e_\mu) \otimes (w_i{}^\nu f_\nu) \\
		&= \sum_{i=1}^l t_i v_i{}^\mu w_i{}^\nu e_{\mu} \otimes f_\nu
	\end{align}
	と書ける.ただし添字 $\mu,\, \nu$ に関してはEinsteinの規約を適用した.従って $\mathcal{E}$ は $V \otimes W$ を生成する.

	$\mathcal{E}$ の元が線型独立であることを示す.
	\begin{align}
		t^{\mu\nu} e_\mu \otimes e_{\nu} = 0
	\end{align}
	を仮定する.$\{e_\mu\},\, \{f_\mu\}$ の\hyperref[def.basisforDVS]{双対基底}をそれぞれ $\{\varepsilon^\mu\},\, \{\eta^\nu\}$ と書き,
	全ての添字の組み合わせ $(\mu,\, \nu) \in \{	1,\, \dots,\, n\} \times \{1,\, \dots,\, m\}$ に対して双線型写像
	\begin{align}
		\tau^{\mu\nu} \colon V \times W \lto \mathbb{R},\; (v,\, w) \lmto \varepsilon^\mu (v) \eta^\nu (w)
	\end{align}
	を定める.$\tau^{\mu\nu}$ は双線型なので\hyperref[cmtd:tensor-vec]{テンソル積の普遍性}から $\VEC{\mathbb{K}}$ の可換図式
	\begin{center}
		\begin{tikzcd}[row sep=large, column sep=large]
			&V \times W \ar[d, "\pi \circ \iota"'] \ar[r,"\tau^{\mu\nu}"] &\mathbb{R} \\
			&\bm{V \otimes W} \ar[ur, red, dashed, "\exists! \overline{\tau}^{\mu\nu}"'] &
		\end{tikzcd}
	\end{center}
	が存在する.このことは,
	\begin{align}
		0 &= \textcolor{red}{\overline{\tau}^{\mu\nu}} (t^{\rho\sigma} e_{\rho} \otimes f_{\sigma}) \\
		&= t^{\rho\sigma} (\textcolor{red}{\overline{\tau}^{\mu\nu}} \circ \pi \circ \iota) (e_\rho ,\,  f_\sigma) \\
		&= t^{\rho\sigma} \tau^{\mu\nu}(e_\rho ,\,  f_\sigma) = t^{\mu\nu}
	\end{align}
	を意味する.従って$\mathcal{E}$ の元は線型独立である.
\end{proof}

これでもまだ直接の計算には向かない.より具体的な構成を探そう.

任意の $\mathbb{K}$-ベクトル空間\footnote{有限次元でなくても良い.} $V_1,\, \dots,V_n,\, W \in \Obj{\VEC{\mathbb{K}}}$ に対して,集合
\begin{align}
	\bm{L(V_1},\,\dots,\, \bm{V_n;\, W)} \coloneqq \bigl\{\, F \colon V_1 \times \cdots \times V_n \lto W \bigm| F\;\text{は多重線型写像} \,\bigr\} 
\end{align}
を考える.$L(V_1,\, \dots,\, V_n;\, W)$ の上の加法とスカラー乗法を $\forall v_i \in V_i,\, \forall \lambda \in \mathbb{K}$ に対して
\begin{align}
	(F + G)(v_1,\, \dots,\, v_n) &\coloneqq F(v_1,\, \dots,\, v_n) + G(v_1,\, \dots,\, v_n), \\
	(\lambda F)(v_1,\, \dots,\, v_n) &\coloneqq \lambda \bigl(F(v_1,\, \dots,\, v_n)\bigr)
\end{align}
と定義すると $L(V_1,\, \dots,\, V_n;\, W)$ は $\mathbb{K}$ ベクトル空間になる.
そして $\forall \omega_i \in V_i^*$ に対して,$\bm{\omega_1 \otimes \cdots \otimes \omega_n}$ と書かれる $L(V_1,\, \dots,\, V_n;\, \mathbb{K})$ の元を
\begin{align}
	\bm{\omega_1 \otimes \cdots \otimes \omega_n} \colon V_1 \times \cdots \times V_n \lto \mathbb{K},\; (v_1,\, \dots,\, v_n) \lmto \prod_{i=1}^n \omega_i(v_i)
\end{align}
によって定義する.ただし右辺の総積記号は $\mathbb{K}$ の積についてとる.

\begin{myprop}[label=prop:basis-L]{}
	有限次元 $\mathbb{K}$-ベクトル空間 $V,\, W$($\dim V \eqqcolon n,\, \dim W \eqqcolon m$)の基底をそれぞれ $\{e_\mu\},\, \{f_\nu\}$ と書き,その\hyperref[def.basisforDVS]{双対基底}をそれぞれ $\{\varepsilon^\mu\},\, \{\eta^\mu\}$ と書く.このとき,集合
	\begin{align}
		\mathcal{B} \coloneqq \bigl\{\, \varepsilon^\mu \otimes \eta^\nu \bigm| 1 \le \mu \le n,\, 1 \le \nu \le m \,\bigr\} 
	\end{align}
	は $L(V,\, W;\, \mathbb{K})$ の基底である.従って $\dim L(V,\, W;\, \mathbb{K}) = nm$ である.
\end{myprop}

\begin{proof}
	$\forall F  \in L(V,\, W;\, \mathbb{K})$ を1つとり,全ての添字の組み合わせ $(\mu,\, \nu)$ に対して
	\begin{align}
		F_{\mu\nu} \coloneqq F(e_\mu,\, e_\nu)
	\end{align}
	とおく.$\forall (v,\, w) \in V \times W$ を $v = v^\mu e_\mu,\; w = w^\nu f_\nu$ と展開すると,
	\begin{align}
		F_{\mu\nu} \varepsilon^\mu \otimes \eta^\nu (v,\, w) &= F_{\mu\nu} \varepsilon^{\mu}(v) \eta^\nu(w) \\
		&= F_{\mu\nu} v^\mu w^\nu
	\end{align}
	が成り立つ.一方,双線型性から
	\begin{align}
		F(v,\, w) = v^\mu w^\nu F(e_\mu,\, e_\nu) = F_{\mu\nu} v^\mu w^\nu
	\end{align}
	も成り立つので $F = F_{\mu\nu} \varepsilon^\mu \otimes \eta^\nu$ が言えた.i.e. 集合 $\mathcal{B}$ は $L(V,\, W;\, \mathbb{K})$ を生成する.

	次に,$\mathcal{B}$ の元が線型独立であることを示す.
	\begin{align}
		F_{\mu\nu} \varepsilon^\mu \otimes \eta^\nu = 0
	\end{align}
	を仮定する.全ての添字の組み合わせについて,$(e_\mu,\, f_\nu)$ に左辺を作用させることで,$F_{\mu\nu} = 0$ が従う.i.e. $\mathcal{B}$ の元は互いに線型独立である.
\end{proof}


\begin{myprop}[label=prop:tensor-multillinear]{テンソル積の構成その2}
	任意の\underline{有限次元} $\mathbb{K}$-ベクトル空間 $V,\, W$ に対して
    \begin{align}
		L(V,\, W;\, \mathbb{K}) \cong V^* \otimes W^*
	\end{align}
\end{myprop}

\begin{proof}
	写像
	\begin{align}
		\Phi \colon V^* \times W^* &\lto L(V,\, W;\, \mathbb{K}), \\
		(\omega,\, \eta) &\lmto \bigl( (v,\, w) \lmto \omega(v) \eta(w) \bigr) 
	\end{align}
	は双線型写像なので
	\hyperref[cmtd:tensor-vec]{テンソル積の普遍性}から $\VEC{\mathbb{K}}$ の可換図式
	\begin{center}
		\begin{tikzcd}[row sep=large, column sep=large]
			&V^* \times W^* \ar[d, "\pi \circ \iota"'] \ar[r,"\Phi"] &L(V,\, W;,\, \mathbb{K}) \\
			&\bm{V^* \otimes W^*} \ar[ur, red, dashed, "\exists! \overline{\Phi}"'] &
		\end{tikzcd}
	\end{center}
	が存在する.
	$V,\, W$($\dim V = n,\, \dim W = m$)の基底をそれぞれ $\{e_\mu\},\, \{f_\nu\}$ と書き,その\hyperref[def.basisforDVS]{双対基底}をそれぞれ $\{\varepsilon^\mu\},\, \{\eta^\mu\}$ と書く.
	命題\ref{prop:basis-tensor}より $V^* \otimes W^*$ の基底として
	\begin{align}
		\mathcal{E} \coloneqq \bigl\{\, \varepsilon^\mu \otimes \eta^\nu \bigm| 1 \le \mu \le n,\, 1 \le \nu \le m \,\bigr\} 
	\end{align}
	がとれ,命題\ref{prop:basis-L}より $L(V,\, W;\, \mathbb{K})$ の基底として
	\begin{align}
		\mathcal{B} \coloneqq \bigl\{\, \varepsilon^\mu \otimes \eta^\nu \bigm| 1 \le \mu \le n,\, 1 \le \nu \le m \,\bigr\} 
	\end{align}
	がとれる(記号が同じだが,違う定義である).
	このとき,$\forall (v,\, w) \in V \times W$ に対して
	\begin{align}
		\textcolor{red}{\overline{\Phi}}(\varepsilon^\mu \otimes \eta^\nu )(v,\, w) = \textcolor{red}{\overline{\Phi}}\circ \pi \circ \iota(\varepsilon^\mu,\,\eta^\nu )(v,\, w) = \Phi(\varepsilon^\mu,\, \eta^\nu)(v,\, w) = \varepsilon^\mu (v) \eta^\nu (w) = \varepsilon^\mu \otimes \eta^\nu(v,\, w)
	\end{align}
	が成り立つ(ただし,左辺の $\otimes$ は命題\ref{prop:tensor-vec},右辺は命題\ref{prop:basis-L}で定義したものである)ので,
	$\textcolor{red}{\overline{\Phi}}$ は $\mathcal{E}$ の元と $\mathcal{B}$ の元の1対1対応を与える.i.e. 同型写像である.
\end{proof}

\begin{mycol}[label=col:tensor-hom]{}
    任意の\underline{有限次元} $\mathbb{K}$-ベクトル空間 $V,\, W$ に対して
    \begin{align}
        L(V;\, W) = \Hom{\mathbb{K}} (V,\, W) \cong V^* \otimes W
    \end{align}
\end{mycol}

\begin{proof}
    写像
    \begin{align}
        \Phi \colon V^* \times W &\lto \Hom{\mathbb{K}}(V,\, W), \\
        (\omega,\, w) &\lmto \bigl( v \lmto \omega(v) w \bigr) 
    \end{align}
    は双線型なので,\hyperref[def:tensor-vec]{テンソル積の普遍性}から $\VEC{\mathbb{K}}$ の可換図式
    \begin{center}
		\begin{tikzcd}[row sep=large, column sep=large]
			&V^* \times W \ar[d, "\pi \circ \iota"'] \ar[r,"\Phi"] &L(V,\, W;,\, \mathbb{K}) \\
			&\bm{V^* \otimes W} \ar[ur, red, dashed, "\exists! \overline{\Phi}"'] &
		\end{tikzcd}
	\end{center}
	が存在する.
	$V,\, W$($\dim V = n,\, \dim W = m$)の基底をそれぞれ $\{e_\mu\},\, \{f_\nu\}$ と書き,その\hyperref[def.basisforDVS]{双対基底}をそれぞれ $\{\varepsilon^\mu\},\, \{\eta^\mu\}$ と書く.
	命題\ref{prop:basis-tensor}より $V^* \otimes W*$ の基底として
	\begin{align}
		\mathcal{E} \coloneqq \bigl\{\, \varepsilon^\mu \otimes f_\nu \bigm| 1 \le \mu \le n,\, 1 \le \nu \le m \,\bigr\} 
	\end{align}
	がとれる.一方,$\forall \omega \in V^*,\; \forall w \in W$ に対して
    \begin{align}
        \label{eq:tensorbasis}
        \omega \otimes w \coloneqq \Phi(\omega,\, w) \colon V \lto W,\; v \lmto \omega(v) w
    \end{align}
    とおくと $L(V;\, W)$ の基底として
	\begin{align}
		\mathcal{B} \coloneqq \bigl\{\, \varepsilon^\mu \otimes f_\nu \bigm| 1 \le \mu \le n,\, 1 \le \nu \le m \,\bigr\} 
	\end{align}
	がとれる
    \footnote{
        $\forall F \in L(V;\, W)$ をとる.$F_{\mu}{}^\nu \coloneqq \eta^\nu \bigl( F(e_\mu) \bigr)$ とおく.このとき $\forall v = v^\mu e_\mu \in V$ に対して
        \begin{align}
            F_{\mu}{}^{\nu} \varepsilon^\mu \otimes f_\nu (v) = F_\mu{}^\nu \varepsilon^\mu(v) f_\nu = F_\mu{}^\nu v^\mu f_\nu
        \end{align}
        一方で,線形性および双対基底の定義から
        \begin{align}
            F(v) = v^\mu F(e_\mu) = v^\mu \eta^\nu \bigl( F(e_\mu) \bigr) f_\nu = v^\mu F_\mu{}^\nu f_\nu
        \end{align}
        が成り立つので $F = F_\mu{}^\nu \varepsilon^\mu \otimes f_\nu$ が言えた.i.e. $\mathcal{B}$ は $L(V;\, W)$ を生成する.

        次に,$\mathcal{B}$ の元が線型独立であることを示す.
        \begin{align}
            F_{\mu}{}^\nu \varepsilon^\mu \otimes f_\nu = 0
        \end{align}
        を仮定する.$1 \le \forall \mu \le \dim V$ について右辺を $e_\mu$ に作用させることで $F_\mu{}^\nu f_\nu = 0$ が従うが,$f_\nu$ の線型独立性から $F_{\mu}{}^\nu = 0$ である.
    }
    (記号が同じだが,$\mathcal{E}$ とは違う定義である).
	このとき,$\forall v \in V$ に対して
	\begin{align}
		\textcolor{red}{\overline{\Phi}}(\varepsilon^\mu \otimes f_\nu )(v) = \textcolor{red}{\overline{\Phi}} \circ \pi \circ \iota(\varepsilon^\mu,\, f_\nu )(v) = \Phi(\varepsilon^\mu,\, f_\nu)(v) = \varepsilon^\mu (v) f_\nu = \varepsilon^\mu \otimes f_\nu(v)
	\end{align}
	が成り立つ(ただし,左辺の $\otimes$ は命題\ref{prop:tensor-vec},右辺は\eqref{eq:tensorbasis}で定義したものである)ので,
	$\textcolor{red}{\overline{\Phi}}$ は $\mathcal{E}$ の元と $\mathcal{B}$ の元の1対1対応を与える.i.e. 同型写像である.
\end{proof}

\begin{mycol}[label=col:tensor-multillinear]{}
    任意の\underline{有限次元} $\mathbb{K}$-ベクトル空間 $V_1,\, \dots,\, V_n,\, W$ に対して
    \begin{align}
        L(V_1,\, \dots,\, V_n;\, W) \cong V_1^* \otimes \cdots \otimes V_n^* \otimes W
    \end{align}
\end{mycol}

\begin{proof}
    命題\ref{prop:tensor-multillinear}, 系\ref{col:tensor-hom}をあわせて
    \begin{align}
        L(V_1,\, \dots,\, V_n;\, W) \cong L(V_1,\, \dots,\, V_n;\, \mathbb{K}) \otimes W \cong V_1^* \otimes \cdots \otimes V_n^* \otimes W
    \end{align}
    を得る.
\end{proof}

\subsection{$\mathfrak{g}$-加群と表現}

まず,環上の加群の定義を復習する:

\begin{myaxiom}[label=ax:R-module,breakable]{環上の加群の公理}
	\begin{itemize}
		\item $R$ を環とする.\textbf{左 $\bm{R}$ 加群} (left $R$-module) とは,可換群 $(M,\, +,\, 0)$ と写像\footnote{この写像 $\cdot$ は\textbf{スカラー乗法} (scalar multiplication) と呼ばれる.}
		\begin{align}
			\cdot \; \colon R \times M \to M,\; (a,\, x) \mapsto a \cdot x
		\end{align}
		の組 $(M,\, +,\,\cdot\mathrel{})$ であって, $\forall x,\, x_1,\, x_2 \in M,\; \forall a,\, b \in R$ に対して以下を充たすもののことを言う:
		\begin{description}
			\item[\textbf{(LM1)}] $a \cdot (b \cdot x) = (\textcolor{red}{ab}) \cdot x$
			\item[\textbf{(LM2)}] $(a+b) \cdot x = a \cdot x + b \cdot x$
			\item[\textbf{(LM3)}] $a \cdot (x_1 + x_2) = a \cdot x_1 + a\cdot x_2$
			\item[\textbf{(LM4)}] $1 \cdot x = x$
		\end{description}
		ただし,$1 \in R$ は\underline{環 $R$ の}乗法単位元である.
		\item $R$ を環とする.\textbf{右 $\bm{R}$ 加群} (left $R$-module) とは,可換群 $(M,\, +,\, 0)$ と写像
		\begin{align}
			\cdot \; \colon M \times R \to M,\; (x,\, a) \mapsto x \cdot a
		\end{align}
		の組 $(M,\, +,\,\cdot\mathrel{})$ であって, $\forall x,\, x_1,\, x_2 \in M,\; \forall a,\, b \in R$ に対して以下を充たすもののことを言う:
		\begin{description}
			\item[\textbf{(RM1)}] $(x \cdot b) \cdot a = x \cdot (\textcolor{red}{ba})$
			\item[\textbf{(RM2)}] $x \cdot (a+b) = x \cdot a + x \cdot b$
			\item[\textbf{(RM3)}] $(x_1 + x_2) \cdot a = x_1 \cdot a + x_2 \cdot a$
			\item[\textbf{(RM4)}] $x \cdot 1 = x$
		\end{description}
		\item $R,\, S$ を環とする.\textbf{$\bm{(R,\, S)}$ 両側加群} ($(R,\, S)$-bimodule) とは,可換群 $(M,\, +,\, 0)$ と写像
		\begin{align}
			\irm{\cdot}{L} \mathrel{} &\colon R \times M \to M,\; (a,\, x) \mapsto a \irm{\cdot}{L} x \\
			\irm{\cdot}{R} \mathrel{} &\colon M \times R \to M,\; (x,\, a) \mapsto x \irm{\cdot}{R} a
		\end{align}
		の組  $(M,\, +,\, \irm{\cdot}{L}\mathrel{},\, \irm{\cdot}{R}\mathrel{})$ であって, 
		$\forall x\in M,\; \forall a\in R,\; \forall b \in S$ に対して以下を充たすもののことを言う:
		\begin{description}
			\item[\textbf{(BM1)}] 左スカラー乗法 $\irm{\cdot}{L}$ に関して $M$ は左 $R$ 加群になる
			\item[\textbf{(BM2)}] 右スカラー乗法 $\irm{\cdot}{R}$ に関して $M$ は右 $S$ 加群になる
			\item[\textbf{(BM3)}] $(a \irm{\cdot}{L} x) \irm{\cdot}{R} b = a \irm{\cdot}{L} (x \irm{\cdot}{R} b)$
		\end{description}
	\end{itemize}
\end{myaxiom}

$R$ が\textbf{可換環}の場合,\textsf{\textbf{(LM1)}} と \textsf{\textbf{(RM1)}} が同値になるので,左 $R$ 加群と右 $R$ 加群の概念は同値になる.これを単に\textbf{$\bm{R}$ 加群} ($R$-module) と呼ぶ.

$R$ が\textbf{体}の場合,$R$ 加群のことを \textbf{$\bm{R}$-ベクトル空間}と呼ぶ.

\begin{marker}
	以下では,なんの断りもなければ $R$ 加群と言って左 $R$ 加群を意味する.
\end{marker}

$\mathfrak{g}$ を体 $\mathbb{K}$ 上の\hyperref[ax:LieAlg]{Lie代数}とする.このとき,\hyperref[ax:R-module]{環上の加群の公理}を少し修正することでLie代数 $\mathfrak{g}$ 上の加群の概念を得る:

\begin{myaxiom}[label=ax:g-module]{Lie代数上の加群}
    $\mathfrak{g}$ を体 $\mathbb{K}$ 上の\hyperref[ax:LieAlg]{Lie代数}とする.$\bm{\mathfrak{g}}$\textbf{-加群}とは,$\mathbb{K}$-ベクトル空間 $(V,\, +,\, \cdot\;)$ と写像
    \begin{align}
        \btr \colon \mathfrak{g} \times V \lto V,\; (x,\, v) \lmto x \btr v
    \end{align}
    の4つ組 $(V,\, +,\, \cdot\;,\, \btr)$ であって,$\forall x,\, x_1,\, x_2 \in \mathfrak{g},\; \forall v,\, v_1,\, v_2 \in V,\; \forall \lambda,\, \mu \in \mathbb{K}$ に対して以下を充たすもののことを言う:
    \begin{description}
        \item[\textbf{(M1)}] $(\lambda \cdot x_1 + \mu \cdot x_2) \btr v = \lambda \cdot (x_1 \btr v) + \mu \cdot (x_2 \btr v)$
        \item[\textbf{(M2)}] $x \btr (\lambda \cdot v_1 + \mu \cdot v_2) = \lambda \cdot (x \btr v_1) + \mu \cdot (x \btr v_2)$
        \item[\textbf{(M3)}] $\comm{x}{y} \btr v = x \btr (y \btr v) - y (\btr x \btr v)$
    \end{description}
    \tcblower
    同値な定義として,\hyperref[def:rep-LieAlg]{Lie代数の表現}
    \begin{align}
        \phi \colon \mathfrak{g} \lto \Lgl (V),\; x \lmto \bigl( v \mapsto \phi(x)(v) \bigr) 
    \end{align}
    において
    \begin{align}
        \btr \colon \mathfrak{g} \times V \lto V,\; (x,\, v) \lmto \phi(x)(v)
    \end{align}
    とおいて得られる4つ組 $(V,\, +,\, \cdot\;,\, \btr)$ のことである\footnote{\exref{def:gl-alg}より $\Lgl (V)$ のLieブラケットは交換子だったので $\comm{x}{y} \btr \mhyphen = \phi(\comm{x}{y}) = \comm{\phi(x)}{\phi(y)} = \phi(x) \circ \phi(y) - \phi(y) \circ \phi(x) = x \btr (y\btr \mhyphen) - y \btr (x \btr \mhyphen)$ となる.}.
\end{myaxiom}

\begin{marker}
    $\mathfrak{g}$-加群に備わっている3つの演算(加法,スカラー乗法,左作用)をいちいち全て明記するのは面倒なので $(V,\, +,\, \cdot\;,\, \btr)$ のことを「$\bm{\mathfrak{g}}$\textbf{-加群} $\bm{V}$」と略記する.この略記において,今まで通りスラカー乗法 $\cdot$ は省略して $\lambda v$ の様に書き,左作用はなんの断りもなく $x \btr v$ の様に書くことにする.
\end{marker}

全く同様に\hyperref[def:Alg]{代数}上の加群,\hyperref[def:Alg]{結合代数}上の加群を定義することもできるが,本章では以降 $\mathfrak{g}$-加群と言ったら\hyperref[ax:g-module]{Lie代数上の加群}を指すことにする.
\hyperref[def:rep-LieAlg]{Lie代数の表現}を考えることは $\mathfrak{g}$-加群を考えることと同値なのである.

\begin{mydef}[label=def:g-module-hom]{$\mathfrak{g}$-加群の準同型}
    $\mathfrak{g}$ を\hyperref[ax:LieAlg]{Lie代数},$V,\, W$ を $\mathfrak{g}$-\hyperref[ax:g-module]{加群}とする.
    線型写像 $f \colon V \lto W$ が $\bm{\mathfrak{g}}$\textbf{-加群の準同型} (homomorphism of $\mathfrak{g}$-module)
    \footnote{
        \textbf{同変写像} (equivalent map) と言うこともある.
        \textbf{絡作用素} (intertwining operator),\textbf{インタートウィナー} (intertwiner) と言う場合もあるが,そこまで普及していない気がする.
    } であるとは,$\forall x \in \mathfrak{g},\; \forall v \in V$ に対して
    \begin{align}
        f(x \btr v) = x \btr f(v)
    \end{align}
    が成り立つこと\footnote{スカラー乗法についての線型性の定義を $\btr$ について拡張しただけ.}.
    \tcblower
    \begin{itemize}
        \item $\mathfrak{g}$-加群の準同型 $f \colon V \lto W$ が\textbf{同型} (isomorphism) であるとは,
        $f$ が\underline{ベクトル空間の同型写像}であることを言う.
        \item 同型な $\mathfrak{g}$-加群のことを,\textbf{同値な} $\bm{\mathfrak{g}}$ \textbf{の表現} (equivalent representation of $\mathfrak{g}$)とも言う.
    \end{itemize}
\end{mydef}


\begin{mydef}[label=def:irr]{Lie代数の表現の既約性}
    
\end{mydef}


\subsection{表現のCasimir元}
\subsection{Weylの定理}
\subsection{Jordan分解の保存}

\section{$\lsl{2}{\mathbb{K}}$ の表現}

\subsection{ウエイトと極大ベクトル}
\subsection{既約加群の分類}

\section{ルート空間分解}

\subsection{極大トーラスとルート}
\subsection{極大トーラスの中心化代数}
\subsection{直交性}
\subsection{整性}
\subsection{有理性}

\end{document}