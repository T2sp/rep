\documentclass[rep_main]{subfiles}

\begin{document}

\setcounter{chapter}{1}

この章において,特に断らない限り体 $\mathbb{K}$ は代数閉体\footnote{つまり,定数でない任意の1変数多項式 $f(x) \in \mathbb{K}[x]$ に対してある $\alpha \in \mathbb{K}$ が存在して $f(\alpha) = 0$ を充たす.}で,かつ $\character \mathbb{K} = 0$ であるとする.

\chapter{半単純Lie代数}

\section{Lieの定理・Cartanの判定条件}

\subsection{Lieの定理}

\subsection{Jordan-Chevalley分解}
\subsection{Cartanの判定条件}

\section{Killing形式}

\section{半単純性の判定条件}
\section{単純イデアル}
\section{内部微分}
\section{抽象Jordan分解}

\section{表現の完全可約性}

\subsection{$\mathfrak{g}$-加群と表現}
\subsection{表現のCasimir元}
\subsection{Weylの定理}
\subsection{Jordan分解の保存}

\section{$\lsl{2}{\mathbb{K}}$ の表現}

\subsection{ウエイトと極大ベクトル}
\subsection{既約加群の分類}

\section{ルート空間分解}

\subsection{極大トーラスとルート}
\subsection{極大トーラスの中心化代数}
\subsection{直交性}
\subsection{整性}
\subsection{有理性}

\end{document}