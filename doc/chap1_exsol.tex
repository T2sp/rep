\documentclass{ltjsarticle}

%%%%%%%%%%packages%%%%%%%%%%
%% colors and links
\usepackage[svgnames]{xcolor}
\usepackage[colorlinks,citecolor=DarkGreen,linkcolor=DeepPink,linktocpage,unicode]{hyperref} 

%% equations
%%%% math
\usepackage{amsmath,amsfonts,amssymb,amsthm}
\usepackage{mathtools}
\usepackage{mathrsfs}
\usepackage{bm}
\usepackage{cancel}
\usepackage{dsfont}
%%%% physics
\usepackage{siunitx}
\usepackage{physics}
%%%% chemistry
\usepackage[version=3]{mhchem}

%% positioning
\usepackage{array}
\usepackage{arydshln}
\usepackage{float}

%% table
\usepackage{booktabs}
\usepackage{multirow}
\usepackage{hhline}
\usepackage{caption}
\captionsetup{format=hang}
\usepackage{subcaption}

%% figure
\usepackage{graphicx}
\usepackage{tikz}
\usetikzlibrary{intersections,calc,patterns,through,positioning,arrows,backgrounds,cd,tqft,braids}
\usepackage{tikz-feynman}
\usepackage{pgfplots}
\usepgfplotslibrary{patchplots}
\pgfplotsset{compat=1.15}

%% decorations
\usepackage{titlesec}
\usepackage{picture}

%% framing
\usepackage{fancybox}
\usepackage{boites}
\usepackage{tcolorbox}
\tcbuselibrary{skins,theorems,breakable}

%% citation
\usepackage{cite}

%% miscellaneous
%%%% comment
\usepackage{comment}
%%%% Japanese
\usepackage{ascmac}
\usepackage{listings} %日本語のコメントアウトをする場合jlisting(もしくはjvlisting)が必要
\lstset{
    basicstyle={\ttfamily},
    identifierstyle={\small},
    commentstyle={\smallitshape},
    keywordstyle={\small\bfseries},
    ndkeywordstyle={\small},
    stringstyle={\small	tfamily},
    frame={tb},
    breaklines=true,
    columns=[l]{fullflexible},
    numbers=left,
    xrightmargin=0pt,
    xleftmargin=27pt,
    numberstyle={\scriptsize},
    stepnumber=1,
    lineskip=-0.5ex
}

%%macros
\usepackage{mylogics}
\usepackage{myalgebra}
\usepackage{mygeometry}
\usepackage{myQM}

%%%%%%%%%%optional settings%%%%%%%%%%

%%%%%図表並列%%%%%
\makeatletter
\newcommand{\figcaption}[1]{\def\@captype{figure}\caption{#1}}
\newcommand{\tblcaption}[1]{\def\@captype{table}\caption{#1}}
\makeatother

%%%%%itemization%%%%%
\renewcommand{\labelenumi}{\theenumi}
\renewcommand{\theenumi}{(\arabic{enumi})} % 箇条書きをローマ数字に

%%%%%%theorem environments%%%%%
\newtheoremstyle{mystyle}%   % スタイル名
    {}%                      % 上部スペース
    {}%                      % 下部スペース
    {\normalfont}%          % 本文フォント
    {}%                      % インデント量
    {\bf}%                  % 見出しフォント
    {.}%                      % 見出し後の句読点
    {\newline}%                     % 見出し後のスペース
    {\underline{\thmname{#1}\thmnumber{#2}\thmnote{(#3)}}}%
    % 見出しの書式 (can be left empty, meaning `normal')
\theoremstyle{mystyle} % スタイルの適用

\newtheorem{theorem}{定理}[section]
\newtheorem{definition}{定義}[section]
\newtheorem{proposition}[definition]{命題}
\newtheorem{corollary}[theorem]{系}
\renewcommand{\proofname}{証明}

%proof
\makeatletter % use at mark
\renewcommand{\qedsymbol}{$\blacksquare$}
\renewenvironment{proof}[1][\proofname]{\par
    \pushQED{\qed}%
    \normalfont \topsep6\p@\@plus6\p@\relax
    \trivlist
    \item[\hskip\labelsep
        \itshape
    \textbf{\underline{#1}}]\ignorespaces
    % {\bf\underline{#1}\@addpunct{.}}]\ignorespaces % ピリオドあり
}{%
    \popQED\endtrivlist\@endpefalse
}
\makeatother % end at mark

%%%%%%mathtools%%%%%
\mathtoolsset{showonlyrefs=true} % 被参照数式のみ数式番号割り振り
\numberwithin{equation}{section}

\makeatletter
\@addtoreset{equation}{section}
\makeatother

%%%%%%framing%%%%%

\newtcolorbox[auto counter,number within=section]{myaxiom}[2][]{%
colback=red!5!white,colframe=red!75!black,fonttitle=\bfseries,
title=公理~\thetcbcounter: #2,#1}

\newtcolorbox[auto counter,number within=section]{mytheo}[2][]{%
colback=blue!5!white,colframe=blue!75!black,fonttitle=\bfseries,
title=定理~\thetcbcounter: #2,#1}

\newtcolorbox[auto counter,number within=section]{myprop}[2][]{%
colback=blue!5!white,colframe=blue!75!black,fonttitle=\bfseries,
title=命題~\thetcbcounter: #2,#1}

\newtcolorbox[auto counter,number within=section]{mylem}[2][]{%
colback=blue!5!white,colframe=blue!75!black,fonttitle=\bfseries,
title=補題~\thetcbcounter: #2,#1}

\newtcolorbox[auto counter,number within=section]{mydef}[2][]{%
colback=green!5!white,colframe=green!75!black,fonttitle=\bfseries,
title=定義~\thetcbcounter: #2,#1}

\newtcolorbox[use counter from=mytheo]{mycol}[2][]{%
fonttitle=\bfseries,enhanced,
attach boxed title to top left=
{xshift=-2mm,yshift=-2mm},
title=系~\thetcbcounter: #2,#1}

\newtcolorbox{marker}[1][breakable]{enhanced,
  	before skip=2mm,after skip=3mm,
  	boxrule=0.4pt,left=5mm,right=2mm,top=1mm,bottom=1mm,
  	colback=yellow!50,
  	colframe=yellow!20!black,
  	sharp corners,rounded corners=southeast,arc is angular,arc=3mm,
  	underlay={%
  	  	\path[fill=tcbcolback!80!black] ([yshift=3mm]interior.south east)--++(-0.4,-0.1)--++(0.1,-0.2);
  	  	\path[draw=tcbcolframe,shorten <=-0.05mm,shorten >=-0.05mm] ([yshift=3mm]interior.south east)--++(-0.4,-0.1)--++(0.1,-0.2);
  	  	\path[fill=yellow!50!black,draw=none] (interior.south west) rectangle node[white]{\Huge\bfseries !} ([xshift=4mm]interior.north west);
  	  	},
  	drop fuzzy shadow,#1}


\newtcolorbox[auto counter, number within=section]{myproblem}[2][]{
    breakable,
    % empty,
    outer arc=0pt,
    arc=0pt,
    attach boxed title to top left={
    %     % empty,
        xshift=-2mm,
        yshift=-2mm
    },
    % coltitle=black,
    % colframe=black,
    colback=white,
    fonttitle=\bfseries,
    title=問題~\thetcbcounter #2,#1,
}
\newcommand{\probref}[1]{\textbf{【問題\ref{#1}】}}


%%%%%title decorations%%%%%

%% section
% \titleformat{\section}[block]
% {}{}{0pt}
% {
%     \colorbox{teal}{\begin{picture}(0,20)\end{picture}}
%     \hspace{0pt}
%     \normalfont \Huge\bfseries \thesection
%     \hspace{0.5em}
% }
% [
% \begin{picture}(100,0)
%     \put(3,30){\color{teal}\line(1,0){400}}
% \end{picture}
% \
% \vspace{-50pt}
% ]

% % subsection
% \titleformat{\subsection}[block]
% {}{}{0pt}
% {
%     \colorbox{blue}{\begin{picture}(0,10)\end{picture}}
%     \hspace{0pt}
%     \normalfont \Large\bfseries \thesubsection
%     \hspace{0.5em}
% }
% [
% \begin{picture}(100,0)
%     \put(3,18){\color{blue}\line(1,0){300}}
% \end{picture}
% \
% \vspace{-30pt}
% ]

% % subsubsection
% \titleformat{\subsubsection}[block]
% {}{}{0pt}
% {
%     \normalfont \large\bfseries \thesubsubsection
%     \hspace{0.5em}
% }
% [
% \begin{picture}(100,0)
%     \put(3,15){\color{black}\line(1,0){200}}
% \end{picture}
% \
% \vspace{-25pt}
% ]
\newcommand{\lto}{\longrightarrow}
\newcommand{\lmto}{\longmapsto}
\usepackage{blkarray}

\begin{document}

\title{Humphreys Chapter I Exercises \\ 解答例(2023/10/16 実施分)}
\author{高間俊至}
\date{\today}
\maketitle

\setcounter{section}{1}

\cite[p.5, Exercise 8]{Humphreys1972introduction}と\cite[p.14, Exercise 2]{Humphreys1972introduction}の解答例です.

\begin{myproblem}[label=ex:1-1-8]{p.3 および p.5 の Exercise 8}
    % \setcounter{\tcbcounter}{7}
    $\character \mathbb{K} \neq 2$ とする,$A_l,\, B_l,\, C_l,\, D_l$ 型の線型Lie代数の各々について基底を構成し,その次元を答えよ.ただし $\lgl{l}{\mathbb{K}}$ の標準的な基底を行列単位とし,$e_{ij}$ と書く.
\end{myproblem}

\begin{description}
    \item[\textbf{$\bm{A_l}$ 型}] $A_l$ 型の定義は
    \begin{align}
        \mathfrak{sl} (l + 1,\, \mathbb{K}) \coloneqq \bigl\{\, x \in \mathfrak{gl}(l+1,\, \mathbb{K}) \bigm| \Tr x = 0 \,\bigr\} 
    \end{align}
    であった.従って $\mathfrak{sl} (l + 1,\, \mathbb{K})$ の基底として,例えば
    \begin{align}
        &\bigl\{\, e_{ij} \bigm| 1 \le i \neq j \le l+1 \,\bigr\} \\
        \cup\, &\bigl\{\, e_{ii} - e_{i+1,\, i+1} \bigm| 1 \le i \le l \,\bigr\} 
    \end{align}
    を採ることができる.故に
    \begin{align}
        \dim \mathfrak{sl} (l + 1,\, \mathbb{K}) = (l+1)^2 - (l+1) + l = (l+1)^2 - 1
    \end{align}
    である.
    \item[\textbf{$\bm{B_l}$ 型}] $B_l$ 型の定義は,対称行列
    \begin{align}
        s \coloneqq 
        \left[
            \begin{array}{c:cc}
                1 &\bm{0}^{\mathsf{T}} &\bm{0}^{\mathsf{T}} \\ 
                \hdashline 
                \bm{0} &O_l &\unity_l \\ 
                \bm{0} &\unity_l &O_l
            \end{array}
        \right]
    \end{align}
    を用いて
    \begin{align}
        \label{eq:ex1-1-8B}
        \mathfrak{o} (2l + 1,\, \mathbb{K}) \coloneqq \bigl\{\, x \in \mathfrak{gl}(2l+1,\, \mathbb{K}) \bigm| x^{\mathsf{T}}s = -sx  \,\bigr\} 
    \end{align}
    であった.ただし $O_l$ は全ての要素が $0 \in \mathbb{K}$ であるような $l \times l$ 行列である.
    $\forall x \in \mathfrak{o} (2l + 1,\, \mathbb{K})$ を1つとり,行列の区分けを
    \begin{align}
        x = \left[
            \begin{array}{c:cc}
                a & \bm{b}_1{}^{\mathsf{T}} & \bm{b}_2{}^{\mathsf{T}} \\ 
                \hdashline 
                \bm{c}_1 & M & N \\  
                \bm{c}_2 & P & Q 
            \end{array}
        \right]
    \end{align}
    のように行うと
    \begin{align}
        x^{\mathsf{T}} s &= \left[
            \begin{array}{ccc}
                a & \bm{c}_1{}^{\mathsf{T}} & \bm{c}_2{}^{\mathsf{T}} \\ 
                \bm{b}_1 & M^{\mathsf{T}} & P^{\mathsf{T}} \\  
                \bm{b}_2 & N^{\mathsf{T}} & Q^{\mathsf{T}} 
            \end{array}
        \right] \left[
            \begin{array}{ccc}
                1 &\bm{0}^{\mathsf{T}} &\bm{0}^{\mathsf{T}} \\ 
                \bm{0} &O_l &\unity_l \\ 
                \bm{0} &\unity_l &O_l
            \end{array}
        \right]
        = \left[
            \begin{array}{ccc}
                a & \bm{c}_2{}^{\mathsf{T}} & \bm{c}_1{}^{\mathsf{T}} \\ 
                \bm{b}_1 & P^{\mathsf{T}} & M^{\mathsf{T}} \\  
                \bm{b}_2 & Q^{\mathsf{T}} & N^{\mathsf{T}} 
            \end{array}
        \right] \\
        sx &= \left[
            \begin{array}{ccc}
                1 &\bm{0}^{\mathsf{T}} &\bm{0}^{\mathsf{T}} \\ 
                \bm{0} &O_l &\unity_l \\ 
                \bm{0} &\unity_l &O_l
            \end{array}
        \right] \left[
            \begin{array}{ccc}
                a & \bm{b}_1{}^{\mathsf{T}} & \bm{b}_2{}^{\mathsf{T}} \\ 
                \bm{c}_1 & M & N \\  
                \bm{c}_2 & P & Q 
            \end{array}
        \right]
        = \left[
            \begin{array}{ccc}
                a & \bm{b}_1{}^{\mathsf{T}} & \bm{b}_2{}^{\mathsf{T}} \\ 
                \bm{c}_2 & P & Q \\  
                \bm{c}_1 & M & N 
            \end{array}
        \right]
    \end{align}
    と計算できるので,\eqref{eq:ex1-1-8B}の条件は
    \begin{align}
        a &= -a \IFF a = 0\label{eq:ex-1-1-8-a} \\
        \bm{b}_1 &= -\bm{c}_2 \label{eq:ex-1-1-8-b1} \\
        \bm{b}_2 &= -\bm{c}_1 \label{eq:ex-1-1-8-b2} \\
        P^{\mathsf{T}} &= -P \label{eq:ex-1-1-8-P} \\
        M^{\mathsf{T}} &= -Q \label{eq:ex-1-1-8-MQ} \\
        N^{\mathsf{T}} &= -N \label{eq:ex-1-1-8-N} 
    \end{align}
    と同値である.ただし\eqref{eq:ex-1-1-8-a}の同値変形に $\character \mathbb{K} \neq 2$ であることを使った.
    \eqref{eq:ex-1-1-8-b1}から
    \begin{align}
        \bigl\{\, e_{1,\, 1 + \textcolor{red}{i}} - e_{1 + l + \textcolor{red}{i},\, 1} \bigm| 1 \le \textcolor{red}{i} \le l \,\bigr\} 
    \end{align}
    が,\eqref{eq:ex-1-1-8-b2}から
    \begin{align}
        \bigl\{\, e_{1,\, 1 + l + \textcolor{red}{i}} - e_{1 + \textcolor{red}{i},\, 1} \bigm| 1 \le \textcolor{red}{i} \le l \,\bigr\} 
    \end{align}
    が,\eqref{eq:ex-1-1-8-P}から
    \begin{align}
        \bigl\{\, e_{1 + l + \textcolor{red}{i},\, 1 + \textcolor{blue}{j}} - e_{1 + l + \textcolor{blue}{j},\, 1 + \textcolor{red}{i}} \bigm| 1 \le \textcolor{red}{i} < \textcolor{blue}{j} \le l \,\bigr\} 
    \end{align}
    が,\eqref{eq:ex-1-1-8-MQ}から
    \begin{align}
        \bigl\{\, e_{1 + \textcolor{blue}{j},\, 1 + \textcolor{red}{i}} - e_{1 + l + \textcolor{red}{i},\, 1 + l + \textcolor{blue}{j}} \bigm| 1 \le \textcolor{red}{i},\, \textcolor{blue}{j} \le l \,\bigr\} 
    \end{align}
    が,\eqref{eq:ex-1-1-8-N}から
    \begin{align}
        \bigl\{\, e_{1 + \textcolor{red}{i},\, 1 + l + \textcolor{blue}{j}} - e_{1 + \textcolor{blue}{j},\, 1 + l + \textcolor{red}{i}} \bigm| 1 \le \textcolor{red}{i} < \textcolor{blue}{j} \le l \,\bigr\} 
    \end{align}
    が基底を構成することが分かった.以上をまとめると $\Lo (2l+1,\, \mathbb{K})$ の基底として
    \begin{align}
        &\bigl\{\, e_{1,\, 1 + \textcolor{red}{i}} - e_{1 + l + \textcolor{red}{i},\, 1} \bigm| 1 \le \textcolor{red}{i} \le l \,\bigr\} \\
        \cup\, &\bigl\{\, e_{1,\, 1 + l + \textcolor{red}{i}} - e_{1 + \textcolor{red}{i},\, 1} \bigm| 1 \le \textcolor{red}{i} \le l \,\bigr\} \\
        \cup\, &\bigl\{\, e_{1 + l + \textcolor{red}{i},\, 1 + \textcolor{blue}{j}} - e_{1 + l + \textcolor{blue}{j},\, 1 + \textcolor{red}{i}} \bigm| 1 \le \textcolor{red}{i} < \textcolor{blue}{j} \le l \,\bigr\} \\
        \cup\, &\bigl\{\, e_{1 + \textcolor{blue}{j},\, 1 + \textcolor{red}{i}} - e_{1 + l + \textcolor{red}{i},\, 1 + l + \textcolor{blue}{j}} \bigm| 1 \le \textcolor{red}{i},\, \textcolor{blue}{j} \le l \,\bigr\}  \\
        \cup\, &\bigl\{\, e_{1 + \textcolor{red}{i},\, 1 + l + \textcolor{blue}{j}} - e_{1 + \textcolor{blue}{j},\, 1 + l + \textcolor{red}{i}} \bigm| 1 \le \textcolor{red}{i} < \textcolor{blue}{j} \le l \,\bigr\} 
    \end{align}
    を採ることができる.故に
    \begin{align}
        \dim \Lo (2l+1,\, \mathbb{K}) = 2l + 2 \cdot \frac{l (l-1)}{2} + l^2 = 2l^2 + l
    \end{align}
    である.
    \item[\textbf{$\bm{C_l}$ 型}] $C_l$ 型の定義は,歪対称行列
    \begin{align}
        s \coloneqq 
        \left[
            \begin{array}{cc}
                O_l &\unity_l \\ 
                -\unity_l &O_l
            \end{array}
        \right]
    \end{align}
    を用いて
    \begin{align}
        \label{eq:ex1-1-8C}
        \mathfrak{sp} (2l,\, \mathbb{K}) \coloneqq \bigl\{\, x \in \mathfrak{gl}(2l,\, \mathbb{K}) \bigm| x^{\mathsf{T}}s = -sx  \,\bigr\} 
    \end{align}
    であった.$\forall x \in \mathfrak{sp} (2l,\, \mathbb{K})$ を1つとり,行列の区分けを
    \begin{align}
        x = \left[
            \begin{array}{cc}
                M & N \\  
                P & Q 
            \end{array}
        \right]
    \end{align}
    のように行うと\eqref{eq:ex1-1-8C}の条件は
    \begin{align}
        P^{\mathsf{T}} &= P  \\
        M^{\mathsf{T}} &= -Q \\
        N^{\mathsf{T}} &= N
    \end{align}
    と同値である.$B_l$ 型のときと同様の考察から $\Lsp (2l,\, \mathbb{K})$ の基底として
    \begin{align}
        &\bigl\{\, e_{l + \textcolor{red}{i},\, \textcolor{red}{i}} \bigm| 1 \le \textcolor{red}{i} \le l \,\bigr\} \cup \bigl\{\, e_{l + \textcolor{red}{i},\, \textcolor{blue}{j}} + e_{l + \textcolor{blue}{j},\, \textcolor{red}{i}} \bigm| 1 \le \textcolor{red}{i} < \textcolor{blue}{j} \le l \,\bigr\} \\
        \cup\, &\bigl\{\, e_{\textcolor{blue}{j},\, \textcolor{red}{i}} - e_{l + \textcolor{red}{i},\, l + \textcolor{blue}{j}} \bigm| 1 \le \textcolor{red}{i},\, \textcolor{blue}{j} \le l \,\bigr\}  \\
        \cup\, &\bigl\{\, e_{\textcolor{red}{i},\, l + \textcolor{red}{i}} \bigm| 1 \le \textcolor{red}{i} \le l \,\bigr\} \cup \bigl\{\, e_{\textcolor{red}{i},\, l + \textcolor{blue}{j}} + e_{1 + \textcolor{blue}{j},\, l + \textcolor{red}{i}} \bigm| 1 \le \textcolor{red}{i} < \textcolor{blue}{j} \le l \,\bigr\} 
    \end{align}
    を採ることができる\footnote{$\textcolor{red}{i} \le \textcolor{blue}{j}$ のように等号を含むことに注意.}.故に
    \begin{align}
        \dim \Lsp (2l,\, \mathbb{K}) = 2l + 2 \cdot \frac{l (l-1)}{2} + l^2 = 2l^2 + l
    \end{align}
    である.
    \item[\textbf{$\bm{D_l}$ 型}] $D_l$ 型の定義は,対称行列
    \begin{align}
        s \coloneqq 
        \left[
            \begin{array}{cc}
                O_l &\unity_l \\ 
                \unity_l &O_l
            \end{array}
        \right]
    \end{align}
    を用いて
    \begin{align}
        \label{eq:ex1-1-8D}
        \mathfrak{o} (2l,\, \mathbb{K}) \coloneqq \bigl\{\, x \in \mathfrak{gl}(2l,\, \mathbb{K}) \bigm| x^{\mathsf{T}}s = -sx  \,\bigr\} 
    \end{align}
    であった.
    $\forall x \in \mathfrak{o} (2l,\, \mathbb{K})$ を1つとり,行列の区分けを
    \begin{align}
        x = \left[
            \begin{array}{cc}
                M & N \\  
                P & Q 
            \end{array}
        \right]
    \end{align}
    のように行うと\eqref{eq:ex1-1-8D}の条件は $B_l$ 型のときと同様に
    \begin{align}
        P^{\mathsf{T}} &= -P  \\
        M^{\mathsf{T}} &= -Q \\
        N^{\mathsf{T}} &= -N
    \end{align}
    と同値である.$B_l$ 型のときと同様の考察から $\Lo (2l,\, \mathbb{K})$ の基底として
    \begin{align}
        &\bigl\{\, e_{l + \textcolor{red}{i},\, \textcolor{blue}{j}} - e_{l + \textcolor{blue}{j},\, \textcolor{red}{i}} \bigm| 1 \le \textcolor{red}{i} < \textcolor{blue}{j} \le l \,\bigr\} \\
        \cup\, &\bigl\{\, e_{\textcolor{blue}{j},\, \textcolor{red}{i}} - e_{l + \textcolor{red}{i},\, l + \textcolor{blue}{j}} \bigm| 1 \le \textcolor{red}{i},\, \textcolor{blue}{j} \le l \,\bigr\}  \\
        \cup\, &\bigl\{\, e_{\textcolor{red}{i},\, l + \textcolor{blue}{j}} - e_{\textcolor{blue}{j},\, l + \textcolor{red}{i}} \bigm| 1 \le \textcolor{red}{i} < \textcolor{blue}{j} \le l \,\bigr\} 
    \end{align}
    を採ることができる.故に
    \begin{align}
        \dim \Lo (2l,\, \mathbb{K}) = 2 \cdot \frac{l (l-1)}{2} + l^2 = 2l^2 - l
    \end{align}
    である.
\end{description}


\begin{myproblem}[label=ex:1-3-8]{p.14 の Exercise 2}
    % \setcounter{\tcbcounter}{7}
    標数が $2$ でない体 $\mathbb{K}$ 上の\underline{有限次元} Lie代数 $\mathfrak{g}$ を与える.このとき以下の (1), (2) が同値であることを示せ:
    \begin{enumerate}
        \item Lie代数 $\mathfrak{g}$ が可解 
        \item $\mathfrak{g}$ の部分Lie代数の減少列
        \begin{align}
            \mathfrak{g} = \mathfrak{g}_0 \supset \mathfrak{g}_1 \supset \cdots \supset \mathfrak{g}_k = \{o\}
        \end{align}
        であって,$\mathfrak{g}_{i+1}$ が $\mathfrak{g}_i$ のイデアルであり,かつ $\mathfrak{g}_i / \mathfrak{g}_{i+1}$ がLieブラケットについて可換であるようなものが存在する.
    \end{enumerate}
\end{myproblem}

\begin{marker}
    $\character \mathbb{K} \neq 2$ なので,$\mathfrak{g}_i / \mathfrak{g}_{i+1}$ がLieブラケットについて可換であるとは $\mathfrak{g}_i / \mathfrak{g}_{i+1}$ の任意の2つの元のLieブラケットが零ベクトルになることと同値である.
\end{marker}

\begin{proof}
    $D \mathfrak{g} \coloneqq \comm{\mathfrak{g}}{\mathfrak{g}}$ とおき,帰納的に $D^{i+1} \mathfrak{g} \coloneqq D^{i} \mathfrak{g}$ と定義する.
    \begin{description}
        \item[\textbf{(1) $\bm{\Longrightarrow}$ (2)}] $\mathfrak{g}$ の導来列そのものが (2) の条件を充たしていることを示す.
        仮定より $\mathfrak{g}$ が可解なので,ある $k > 0$ が存在して $D^k \mathfrak{g} = \{o\}$ となる.また,第1回の資料の\textbf{補題1.1.1}(p.13)より $D^{i+1}\mathfrak{g} = D (D^i \mathfrak{g}) \subset D^i \mathfrak{g}$ は $D^i \mathfrak{g}$ のイデアルであり,商代数 $D^i \mathfrak{g} / D^{i+1} \mathfrak{g}$ が定義できる.
        あとは $D^i \mathfrak{g} / D^{i+1} \mathfrak{g}$ がLieブラケットについて可換であることを示せばよい.
        
         $i = 0$ のとき,$\forall x + D \mathfrak{g},\, y + D \mathfrak{g} \in \mathfrak{g} /  D \mathfrak{g} = D^0\mathfrak{g}/D^1 \mathfrak{g}$ をとる\footnote{$D^0\mathfrak{g}/D^1 \mathfrak{g}$ の任意の元は $x + D \mathfrak{g}$ の形で書ける.標準的射影 $p \colon \mathfrak{g} \lto \mathfrak{g} / D \mathfrak{g},\, x \lmto x + D\mathfrak{g}$ が全射であるとも言える.}.
        商代数のLieブラケットの定義と $\comm{x}{y} \in \comm{\mathfrak{g}}{\mathfrak{g}} = D \mathfrak{g}$ であることから
        \begin{align}
            \comm{x + D \mathfrak{g}}{y + D \mathfrak{g}} &= \comm{x}{y} + D \mathfrak{g}
            = D \mathfrak{g}
        \end{align}
        なので\footnote{商代数 $D^0\mathfrak{g}/D^1 \mathfrak{g}$ の加法単位元(零ベクトル)は $D \mathfrak{g}$ である.},$D^0\mathfrak{g}/D^1 \mathfrak{g}$ はLieブラケットについて可換である.
        $i > 0$  のときも全く同様に示される.
        \item[\textbf{(1) $\bm{\Longleftarrow}$ (2)}] (2) の条件を充たす部分Lie代数の減少列 $\mathfrak{g} = \mathfrak{g}_0 \supset \mathfrak{g}_1 \supset \cdots \supset \mathfrak{g}_k = \{o\}$ が与えられたとき,$0 \le \forall i \le k$ に対して $D^i \mathfrak{g} \subset \mathfrak{g}_i$ が成り立つ\footnote{つまり,\textbf{導来列は (2) の条件を充たす部分Lie代数の減少列のうち最小のものである}.なお明示していなかったが,$\subset$ は $=$ も含むとする.一方,真 (\textbf{proper}) の包含は $\subsetneq$ と書く.}ことを $i$ についての数学的帰納法により示す.
        
         $i = 0$ のときは $D^0 \mathfrak{g} = \mathfrak{g} \subset \mathfrak{g} = \mathfrak{g}_0$ なので明らか.
        $i > 0$ のとき,$D^{i-1} \mathfrak{g} \subset \mathfrak{g}_{i-1}$ が示されているとする.このとき $D^i \mathfrak{g} \subset \mathfrak{g}_i$ であることを2段階に分けて示す:
        \begin{description}
            \item[$\bm{D \mathfrak{g}_{i-1} \subset \mathfrak{g}_i}$ ]  
            
            $\forall x,\, y \in \mathfrak{g}_{i-1}$ に対して,$\mathfrak{g}_{i-1} / \mathfrak{g}_i$ が可換であると言う仮定から
            \begin{align}
                &\mathfrak{g}_i = \comm{x + \mathfrak{g}_i}{y + \mathfrak{g}_i} = \comm{x}{y} + \mathfrak{g_{i}} \\
                \IFF &\comm{x}{y} \in \mathfrak{g}_i
            \end{align}
            が言える\footnote{商代数 $\mathfrak{g}_{i-1}/ \mathfrak{g}_i$ の加法単位元(零ベクトル)は $\mathfrak{g}_{i}$ である.}.従って $D \mathfrak{g}_{i-1} \subset \mathfrak{g}_i$ である\footnote{$D \mathfrak{g_{i-1}}$ の勝手な元は,ある $n < \infty$ を用いて $\sum_{i=1}^n \comm{x_i}{y_i}$ と書ける.$\mathfrak{g}_i$ は部分Lie代数であり,加法について閉じているので $\sum_{i=1}^n \comm{x_i}{y_i} \in \mathfrak{g}_i$ が言える.}.
            \item[$\bm{D^{i} \mathfrak{g} \subset \mathfrak{g}_{i}}$] 
            
            帰納法の仮定より $D (D^{i-1} \mathfrak{g}) \subset D \mathfrak{g}_{i-1}$ が言える\footnote{任意のイデアル $\mathfrak{i},\, \mathfrak{j} \subset \mathfrak{g}$ に対して,$\mathfrak{i} \subset \mathfrak{j}$ ならば $D$ の定義から $D \mathfrak{i} \subset D \mathfrak{j}$ が成り立つ.}ので,
            \begin{align}
                D^i \mathfrak{g} = D (D^{i-1} \mathfrak{g}) \subset D \mathfrak{g}_{i-1} \subset \mathfrak{g}_i
            \end{align}
            が示された.
        \end{description}
    \end{description}
    
\end{proof}


%参考文献のスタイル(style.bstを参照)
\bibliographystyle{myh-physrev}
%参考文献(reference.bibを参照)
\bibliography{repref}

\end{document}