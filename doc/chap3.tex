\documentclass[rep_main]{subfiles}

\newcommand{\Euc}[1]{\bigl(\, #1,\, (\;,\,)_{#1}\,\bigr)}
\newcommand{\Weyl}[2]{\mathscr{W}_{#1}(#2)}

\begin{document}

\setcounter{chapter}{2}

\chapter{ルート系}

この章において,特に断らない限り体 $\mathbb{K}$ は代数閉体\footnote{つまり,定数でない任意の1変数多項式 $f(x) \in \mathbb{K}[x]$ に対してある $\alpha \in \mathbb{K}$ が存在して $f(\alpha) = 0$ を充たす.}で,かつ $\character \mathbb{K} = 0$ であるとする.
また,\hyperref[ax:LieAlg]{Lie代数} $\mathfrak{g}$ は常に\underline{有限次元}であるとする.

\section{公理的方法}

\label{def:Euclid-space}\textbf{Euclid空間} (Euclid space) とは,
\begin{itemize}
	\item 体 $\mathbb{R}$ 上の\underline{有限次元}ベクトル空間 $\mathbb{E}$
	\item 対称かつ正定値な双線型形式 $\rpair{\;}{\,}_{\mathbb{E}}\colon \mathbb{E} \times \mathbb{E} \lto \mathbb{R}$
\end{itemize}
の組 $\Euc{\mathbb{E}}$ のことを言う
\footnote{Euclid空間と言って位相空間のことを指す場合があるが,そのときは双線型形式 $\rpair{\;}{\,}_{\mathbb{E}}$ を使って $\mathbb{E}$ 上の距離関数を $d_{\mathbb{E}} \colon \mathbb{E} \times \mathbb{E} \lto \mathbb{R}_{\ge 0},\; (x,\, y) \lmto \rpair{x-y}{x-y}_{\mathbb{E}}$ と定義し(これは通常\textbf{Euclid距離}と呼ばれる),$\mathbb{E}$ に $d_{\mathbb{E}}$ による距離位相を入れる.}.
Euclid空間 $\Euc{\mathbb{E}}$ の任意の元 $\alpha \in \mathbb{E}$ に対して,
\begin{itemize}
    \item \textbf{鏡映面} (reflecting hyperplane)\footnote{余次元 $1$ の部分 $\mathbb{R}$-ベクトル空間.最右辺は\hyperref[def:radical-bilinear]{対称かつ非退化な双線型形式 $(\;,\,)_{\mathbb{E}}$ による直交補空間}の意味である.}
    \begin{align}
        \label{def:hyperplane}
        \bm{P_\alpha} \coloneqq \bigl\{\, \beta \in \mathbb{E} \bigm| \rpair{\beta}{\alpha}_{\mathbb{E}} = 0 \,\bigr\} = (\mathbb{R}\alpha)^\perp
    \end{align}
    \item 鏡映面 $P_\alpha$ に関する\textbf{鏡映} (reflecting)
    \begin{align}
        \bm{\sigma_\alpha} \colon \mathbb{E} \lto \mathbb{E},\; \beta \lmto \beta - 2 \frac{\rpair{\beta}{\alpha}_{\mathbb{E}}}{\rpair{\alpha}{\alpha}_{\mathbb{E}}} \alpha
    \end{align}
\end{itemize}
を考える.


\begin{marker}
    $2 \frac{(\beta,\, \alpha)}{(\alpha,\, \alpha)} \in \mathbb{R}$ が頻繁に登場するので,
    \begin{align}
		\sspair{\beta}{\alpha} \coloneqq 2\frac{(\beta,\, \alpha)_{\mathbb{E}}}{(\alpha,\, \alpha)_{\mathbb{E}}}
	\end{align}
	と略記することにする.写像 $\sspair{\;}{\,} \colon \mathbb{E} \times \mathbb{E} \lto \mathbb{R}$ は記号的には内積のように見えるかもしれないが,あくまで第一引数についてのみ線型なのであって,\underline{対称でも双線型でもない}ことに注意.
\end{marker}

$\sigma_\alpha$ は $\mathbb{R}$-線型でかつ $\forall \beta \in \mathbb{E}$ に対して
\begin{align}
	\sigma_\alpha \circ \sigma_\alpha (\beta) 
	&= \bigl(\beta - \sspair{\beta}{\alpha} \alpha\bigr) - \sspair[\Big]{\bigl(\beta - \sspair{\beta}{\alpha} \alpha\bigr)}{\alpha}\alpha \\
	&= \beta - \sspair{\beta}{\alpha} \alpha - \sspair{\beta}{\alpha}\alpha + \sspair{\beta}{\alpha}\sspair{\alpha}{\alpha} \alpha \\
	&= \beta - 2\sspair{\beta}{\alpha}\alpha + 2\sspair{\beta}{\alpha}\alpha \\
	&= \beta
\end{align}
を充たす,i.e. $\sigma_\alpha^{-1} = \sigma_\alpha$ なので,$\sigma_\alpha \in \LGL(\mathbb{E})$ である.

\begin{mylem}[label=lem:refrect]{鏡映の特徴付け}
	Euclid空間 $\mathbb{E}$ と,
	\begin{itemize}
		\item $\mathbb{E} = \Span_{\mathbb{R}}\Phi$ 
		\item $\forall \alpha \in \Phi,\; \sigma_\alpha(\Phi) = \Phi$ 
	\end{itemize}
	を充たす $\mathbb{E}$ の有限部分集合 $\Phi \subset \mathbb{E}$ を与える.

	このとき,$\sigma \in \LGL(\mathbb{E})$ が
	\begin{description}
		\item[\textbf{(RF-1)}]  $\sigma(\Phi) = \Phi$
		\item[\textbf{(RF-2)}]  余次元 $1$ の部分ベクトル空間 $P \subset \mathbb{R}$ が存在して,$\forall \beta \in P,\; \sigma(\beta) = \beta$ が成り立つ
		\item[\textbf{(RF-3)}]  $\exists \alpha \in \Phi \setminus \{0\},\; \sigma(\alpha) = -\alpha$
	\end{description}
	の3条件を満たすならば $\sigma = \sigma_\alpha$ (かつ $P = P_\alpha$)である.
\end{mylem}

\begin{proof}
	$\tau \coloneqq \sigma \circ \sigma_\alpha \, (= \sigma \circ \sigma_\alpha^{-1})$ とおき,$\tau = \mathrm{id}_{\mathbb{E}}$ であることを示す.

	\textsf{\textbf{(RF-1)}} より $\tau(\Phi) = \Phi,\; \tau(\alpha) = \alpha$ が成り立つので $\tau|_{\mathbb{R}\alpha} = \mathrm{id}_{\mathbb{R}\alpha}$ である.
	さらに $\mathbb{R}$-線型写像
	\begin{align}
		\overline{\tau} \colon \mathbb{E}/\mathbb{R}\alpha \lto \mathbb{E}/\mathbb{R}\alpha,\; \beta + \mathbb{R}\alpha \lmto \tau(\beta) + \mathbb{R}\alpha
	\end{align}
	はwell-definedだが,\textsf{\textbf{(RF-2)}}より $\overline{\tau} = \mathrm{id}_{ \mathbb{E}/\mathbb{R}\alpha}$ である.よって $\tau$ の固有値は全て $1$ であり,$\tau$ の\hyperref[def:minimal-poly]{最小多項式} $f(t)$ は $(t-1)^{\dim \mathbb{E}}$ の\hyperref[def:domain-basic]{約元}である.
	一方,$\Phi$ は有限集合なので,$\forall \beta \in \Phi$ に対してある $k_\beta \in \mathbb{N}$ が存在して $\tau^{k_\beta}(\beta) = \beta$ を充たす.ここで $k \coloneqq \max \{k_\beta\mid \beta \in \Phi\}$ とおくと,\textsf{\textbf{(RF-1)}} より $\tau^k = \mathrm{id}_{\mathbb{E}}$ が言える.よって $f(t)$ は $t^k - 1$ の約元でもある.
	従って,$f(t) = \GCD \bigl( (t-1)^{\dim \mathbb{E}},\, t^k - 1 \bigr) = t-1$ だと分かった.故に $\tau = \mathrm{id}_{\mathbb{E}}$ である.
\end{proof}


\subsection{ルート系}

前章で与えたルート系の公理を再掲するところから始めよう:

\begin{myaxiom}[label=ax:root-system,breakable]{ルート系}
	\begin{itemize}
		\item 有限次元Euclid空間 $\Euc{\mathbb{E}}$ 
		\item $\mathbb{E}$ の部分集合 $\Phi \subset \mathbb{E}$
	\end{itemize}
	の組 $(\mathbb{E},\, \Phi)$ が\textbf{ルート系} (root system) であるとは,以下の条件を充たすことを言う:
	\begin{description}
		\item[\textbf{(Root-1)}] $\Phi$ は $0$ を含まない有限集合で,かつ $\mathbb{E} = \Span_{\mathbb{R}} \Phi$ を充たす.
		\item[\textbf{(Root-2)}] $\lambda \alpha \in \Phi \IMP \lambda = \pm 1$
		\item[\textbf{(Root-3)}] $\alpha,\, \beta \in \Phi \IMP \sigma_\alpha(\beta) \in \Phi$
		\item[\textbf{(Root-4)}] $\alpha,\, \beta \in \Phi \IMP \sspair{\beta}{\alpha} \in \mathbb{Z}$
	\end{description}
	\tcblower
	$\Phi$ の元のことを\textbf{ルート} (root) と呼ぶ.
\end{myaxiom}

\begin{marker}
	本資料の以降では,文脈上直積集合の要素との混同が起きる恐れがないときは\hyperref[def:Euclid-space]{Euclid空間} $\Euc{\mathbb{E}}$ に備わっている双線型形式を $\rpair{\;}{\,}_{\mathbb{E}}$ と書く代わりに $\rpair{\;}{\,}$ と略記する.
\end{marker}


ルート系と言ったときに,\textsf{\textbf{(Root-2)}} を除外する場合がある.その場合我々が採用した定義に該当するものは\textbf{簡約ルート系} (reduced root system) と呼ばれる.

\begin{mydef}[label=def:rank-root]{}
	\hyperref[ax:root-system]{ルート系} $(\mathbb{E},\, \Phi)$ の\textbf{ランク} (rank) とは,$\dim \mathbb{E} \in \mathbb{N}$ のことを言う.
\end{mydef}

公理 \textsf{\textbf{(Root-4)}} は,任意のルートの2つ組の配位に非常に強い制約を与える.
というのも,2つのベクトルのなす角の定義を思い出すと,$\forall \alpha,\, \beta \in \Phi$ に対してある $\theta \in [0,\, \pi]$ が存在して
\begin{align}
	\sspair{\beta}{\alpha} 
	&= 2 \frac{\norm*{\beta}}{\norm*{\alpha}} \cos \theta \in \mathbb{Z} \label{eq:root4-1} \\
	\sspair{\alpha}{\beta}\sspair{\beta}{\alpha} 
	&= 4 \cos^2 \theta \in \mathbb{Z} \label{eq:root4-2}
\end{align}
が成り立たねばならないのである.$\sspair{\alpha}{\beta},\,\sspair{\beta}{\alpha} \in \mathbb{Z}$ かつ $0 \le \cos^2 \theta \le 1$ なので,\eqref{eq:root4-2}から
\begin{align}
	\cos^2 \theta = 0,\, \frac{(\pm 1) \cdot (\pm 1)}{4},\, \frac{(\pm 1) \cdot (\pm 2)}{4},\, \frac{(\pm 1) \cdot (\pm 3)}{4},\, \frac{ (\pm 1) \cdot (\pm 4)}{4},\, \frac{(\pm 2) \cdot (\pm 2)}{4} \quad (\text{複号同順})
\end{align}
しかあり得ないとわかる.\textsf{\textbf{(Root-2)}}も考慮すると,$\norm*{\alpha} \le \norm*{\beta}$ ならば\footnote{このとき\eqref{eq:root4-1}より $\sspair{\alpha}{\beta} \le \sspair{\beta}{\alpha}$}あり得る可能性は以下の通り\footnote{表\ref{tab:rootsystem}の最後の2段は $\beta = \pm \alpha$ の場合に相当する.}:

\begin{table}[H]
	\centering
	\caption[]{可能なルートの2つ組 $\alpha,\, \beta$}
	\label{tab:rootsystem}
	\begin{tabular}{cc|cc}
		\multicolumn{1}{c}{$\sspair{\alpha}{\beta}$} 
		&\multicolumn{1}{c}{$\sspair{\beta}{\alpha}$} 
		&\multicolumn{1}{|c}{$\theta$} 
		&\multicolumn{1}{c}{$\norm*{\beta}^2/\norm*{\alpha}^2$} \\
		\hhline{--|--}
		$0 $ &$0 $ &$\frac{\pi}{2}$ 	&- \\
		$1 $ &$1 $ &$\frac{\pi}{3}$ 	&$1$ \\
		$-1$ &$-1$ &$\frac{2\pi}{3}$ 	&$1$ \\
		$1 $ &$2 $ &$\frac{\pi}{4}$ 	&$2$ \\
		$-1$ &$-2$ &$\frac{3\pi}{4}$ 	&$2$ \\
		$1 $ &$3 $ &$\frac{\pi}{6}$ 	&$3$ \\
		$-1$ &$-3$ &$\frac{5\pi}{6}$ 	&$3$ \\
		$2 $ &$2 $ &$0$ 				&$1$ \\
		$-2$ &$-2$ &$\pi$ 				&$1$ \\
	\end{tabular}
\end{table}%

\begin{mylem}[label=lem:root-basic]{}
	$(\mathbb{E},\, \Phi)$ を\hyperref[ax:root-system]{ルート系}とする.
	$\alpha \neq \pm \beta$ を充たす任意の $\alpha,\, \beta \in \Phi$ に対して以下が成り立つ:
	\begin{enumerate}
		\item $\rpair{\alpha}{\beta} > 0 \IMP \alpha - \beta \in \Phi$
		\item $\rpair{\alpha}{\beta} < 0 \IMP \alpha + \beta \in \Phi$
	\end{enumerate}
\end{mylem}

\begin{proof}
	$\rpair{\alpha}{\beta} > 0$ とする.このとき $\sspair{\alpha}{\beta} > 0$ であるから,表\ref{tab:rootsystem}より $\sspair{\alpha}{\beta},\, \sspair{\beta}{\alpha}$ の少なくとも一方は $1$ に等しい.$\sspair{\alpha}{\beta} = 1$ だとすると,\textsf{\textbf{(Root-3)}}から $\sigma_\beta(\alpha) = \alpha - \beta \in \Phi$ がいえる.$\sspair{\beta}{\alpha} = 1$ ならば $\sigma_\alpha(\beta) = \beta - \alpha \in \Phi$ であり,$\sigma_{\beta-\alpha}(\beta-\alpha) = \alpha - \beta \in \Phi$ が従う.
	(2) は (1) において $\beta$ の代わりに $-\beta = \sigma_\beta(\beta) \in \Phi$ を用いて同じ議論をすれば良い.
\end{proof}

\begin{mydef}[label=def:a-sting]{$\alpha$-string through $\beta$}
	$(\mathbb{E},\, \Phi)$ を\hyperref[ax:root-system]{ルート系}とする.
	$\alpha \neq \pm \beta$ を充たす任意の $\alpha,\, \beta \in \Phi$ に対して,$\Phi$ の部分集合
	\begin{align}
		\bigl\{\, \beta + \lambda \alpha \in \mathbb{E} \bigm| \lambda \in \mathbb{Z} \,\bigr\} \cap \Phi
	\end{align}
	のことを\textbf{$\bm{\alpha}$-string through $\bm{\beta}$} と呼ぶ.
\end{mydef}

\begin{myprop}[label=prop:a-string-basic]{$\alpha$-string through $\beta$ の性質}
	$(\mathbb{E},\, \Phi)$ を\hyperref[ax:root-system]{ルート系}とする.
	$\alpha \neq \pm \beta$ を充たす任意の $\alpha,\, \beta \in \Phi$ に対して
	\begin{align}
		r &\coloneqq \max \bigl\{\, \lambda \in \mathbb{Z}_{\ge 0} \bigm| \beta - \lambda \alpha \in \Phi \,\bigr\}, \\
		q &\coloneqq \max \bigl\{\, \lambda \in \mathbb{Z}_{\ge 0} \bigm| \beta + \lambda \alpha \in \Phi \,\bigr\} 
	\end{align}
	とおく.このとき以下が成り立つ:
	\begin{enumerate}
		\item 
		\hyperref[def:a-sting]{$\alpha$-string through $\beta$}は $\mathbb{E}$ の部分集合
		\begin{align}
			\label{eq:a-string}
			\bigl\{\, \beta + \lambda\alpha \in \mathbb{E} \bigm| -r \le \lambda \le q \,\bigr\} 
		\end{align}
		に等しい.i.e. $-r \le \forall \lambda \le q$ に対して $\beta + \lambda\alpha \in \Phi$ である.
		\item \hyperref[def:a-sting]{$\alpha$-string through $\beta$}は鏡映 $\sigma_\alpha$ の作用の下で不変である.
		\item $r-q = \sspair{\beta}{\alpha}$.特に\hyperref[def:a-sting]{$\alpha$-string through $\beta$}の長さは $4$ 以下である.
	\end{enumerate}
\end{myprop}

\begin{proof}
	\begin{enumerate}
		\item 背理法により示す.ある $-r < \lambda < q$ に対して $\beta + \lambda\alpha \notin \Phi$ であるとする.
		このときある $-r \le s < p \le q$ が存在して $\beta + s\alpha \in \Phi,\, \beta + (s-1)\alpha \notin \Phi,\, \beta + (p-1)\alpha \notin \Phi,\, \beta + p\alpha \in \Phi$ を充たすが,補題\ref{lem:root-basic}の対偶よりこのとき $\rpair{\alpha}{\beta + s\alpha} \le 0 \le \rpair{\alpha}{\beta + p\alpha}$ が成り立つ.
		よって $(p-s) \rpair{\alpha}{\alpha} \ge 0$ である.
		然るに $p < s$ かつ $\rpair{\alpha}{\alpha} > 0$ なのでこれは矛盾である.
		\item $-r \le \forall \lambda \le q$ に対して
		\begin{align}
			\sigma_\alpha(\beta + \lambda\alpha) 
			&= \beta + \lambda\alpha - \sspair{\beta}{\alpha}\alpha - \lambda \sspair{\alpha}{\alpha}\alpha \\
			&= \beta - \bigl( \sspair{\beta}{\alpha} + \lambda \bigr) \alpha
		\end{align}
		が成り立つ.\textsf{\textbf{(Root-3)}} より最右辺は $\Phi$ の元であり,かつ \textsf{\textbf{(Root-4)}}より $\sspair{\beta}{\alpha} + \lambda \in \mathbb{Z}$ なので,(1) より $-r \le \sspair{\beta}{\alpha} + \lambda \le q$ だと分かった.
		\item $r,\, q$ の定義より,(2) の証明において
		\begin{align}
			\sigma_\alpha(\beta + q\alpha) = \beta - \bigl( \sspair{\beta}{\alpha} + q \bigr) \alpha = \beta - r \alpha
		\end{align}
		が成り立たねばならない.よって
		\begin{align}
			r-q = \sspair{\beta}{\alpha}
		\end{align}
		である.表\ref{tab:rootsystem}より $\abs{r-q} \le 3$ であるから, 集合\eqref{eq:a-string}の要素数は $4$ 以下である.
	\end{enumerate}
	
\end{proof}

\begin{mydef}[label=def:dual-root]{双対ルート系}
	\hyperref[ax:root-system]{ルート系} $(\mathbb{E},\, \Phi)$ に対して
	\begin{align}
		\bm{\Phi^{\vee}} \coloneqq \biggl\{\, \frac{2}{(\alpha,\, \alpha)} \alpha \in \mathbb{E} \biggm| \alpha \in \Phi \,\biggr\} 
	\end{align}
	とおき,組 $\bm{(\mathbb{E},\, \Phi^\vee)}$ のことを $(\mathbb{E},\, \Phi)$ の\textbf{双対ルート系} (dual root system) と呼ぶ.
	
	\tcblower
	
	$\alpha \in \Phi$ に対して
	\begin{align}
		\bm{\alpha^\vee} \coloneqq \frac{2}{(\alpha,\, \alpha)} \alpha \in \Phi^\vee
	\end{align}
	と書く.
\end{mydef}

双対ルート系が\hyperref[ax:root-system]{ルート系の公理}を充たすことを確認しよう:
\begin{description}
	\item[\textbf{(Root-1,2)}] $(\mathbb{E},\, \Phi)$ がルート系なので明らか.
	\item[\textbf{(Root-3)}] $\forall \alpha^\vee,\, \beta^\vee \in \Phi^\vee$ をとる.このとき
	\begin{align}
		\label{eq:dual-root3}
		\sigma_{\alpha^\vee} \left( \beta^\vee  \right) 
		&= \frac{2}{(\beta,\, \beta)} \beta - \sspair*{\frac{2}{(\beta,\, \beta)} \beta}{\frac{2}{(\alpha,\, \alpha)} \alpha } \frac{2}{(\alpha,\, \alpha)} \alpha \\
		&= \frac{2}{(\beta,\, \beta)} \beta - \frac{\frac{2}{(\beta,\, \beta)}}{\frac{2}{(\alpha,\, \alpha)}} \sspair*{\beta}{\alpha} \frac{2}{(\alpha,\, \alpha)} \alpha \\
		&= \frac{2}{(\beta,\, \beta)} \sigma_{\alpha}(\beta)
	\end{align}
	だが,
	\begin{align}
		\rpair[\big]{\sigma_\alpha(\beta)}{\sigma_\alpha(\beta)}
		&= \rpair{\beta}{\beta} - 2 \sspair{\beta}{\alpha} \rpair{\beta}{\alpha} + \sspair{\beta}{\alpha}^2 \rpair{\alpha}{\alpha} \\
		&= \rpair{\beta}{\beta} - 2 \sspair{\beta}{\alpha} \rpair{\beta}{\alpha} + 2\sspair{\beta}{\alpha}\rpair{\beta}{\alpha} \\
		&= \rpair{\beta}{\beta}
	\end{align}
	なので
	\begin{align}
		\sigma_{\alpha^\vee} \left( \beta^\vee  \right) = \sigma_\alpha(\beta)^\vee \in \Phi^\vee
	\end{align}
	だと分かった.
	\item[\textbf{(Root-4)}] $\forall \frac{2}{(\alpha,\, \alpha)} \alpha,\, \frac{2}{(\beta,\, \beta)} \beta \in \Phi^\vee$ をとる.このとき
	\begin{align}
		\sspair{\beta^\vee}{\alpha^\vee}
		&= \sspair*{\frac{2}{(\beta,\, \beta)} \beta}{\frac{2}{(\alpha,\, \alpha)} \alpha} \\
		&= \frac{\frac{2}{(\beta,\, \beta)}}{\frac{2}{(\alpha,\, \alpha)}} \sspair{\beta}{\alpha} \\
		&= \frac{\frac{2(\alpha,\, \beta)}{(\beta,\, \beta)}}{\frac{2(\beta,\, \alpha)}{(\alpha,\, \alpha)}} \sspair{\beta}{\alpha} \\
		&= \frac{\sspair{\alpha}{\beta}}{\sspair{\beta}{\alpha}} \sspair{\beta}{\alpha} \\
		&= \sspair{\alpha}{\beta} \in \mathbb{Z}
	\end{align}
	が言えた.
\end{description}

\subsection{Weyl群}

\begin{mydef}[label=def:Weylgroup]{Weyl群}
	$(\mathbb{E},\, \Phi)$ を\hyperref[ax:root-system]{ルート系}とする.
	$\LGL (\mathbb{E})$ の部分集合 $\bigl\{\, \sigma_\alpha \in \LGL(\mathbb{E}) \bigm| \alpha \in \Phi \,\bigr\}$ が生成する $\LGL(\mathbb{E})$ の部分群のことをルート系 $(\mathbb{E},\, \Phi)$ の\textbf{Weyl群} (Weyl group) と呼び,$\bm{\Weyl{\mathbb{E}}{\Phi}}$ と書く.
	
	\tcblower
	
	$\forall \sigma \in \mathscr{W}$ に対して,
	\begin{align}
		\bm{\len (\sigma)} \coloneqq \min \bigl\{\, t \in \mathbb{Z}_{\ge 0} \bigm| \sigma = \sigma_{\alpha_1} \circ \cdots\circ \sigma_{\alpha_t} \WHERE \alpha_i \in \Phi \,\bigr\} 
	\end{align}
	を $\sigma$ の\textbf{長さ} (length) と呼ぶ.ただし $\len (\mathrm{id}_{\mathbb{E}}) \coloneqq 0$ と定義する.
\end{mydef}

\begin{marker}
	本資料の以降では,文脈上考えているルート系が明らかな場合 $\Weyl{\mathbb{E}}{\Phi}$ を $\bm{\mathscr{W}}$ と略記する.
\end{marker}


\textsf{\textbf{(Root3)}}より,$\forall \tau \in \Weyl{\mathbb{E}}{\Phi}$ の $\Phi$ への制限は全単射である.その上 \textsf{\textbf{(Root-1)}} から $\Phi$ は有限集合でかつ $\mathbb{E} = \Span_{\mathbb{R}} \Phi$ が成り立つので,$\Weyl{\mathbb{E}}{\Phi}$ を $\Phi$ に作用する対称群 $\mathfrak{S}_{\abs{\Phi}}$ の部分群と同一視できる.特に $\Weyl{\mathbb{E}}{\Phi}$ は\underline{有限群}である.

\begin{mylem}[label=lem:Weylgroup]{}
	$(\mathbb{E},\, \Phi)$ を\hyperref[ax:root-system]{ルート系}とする.
	$\tau \in \LGL(\mathbb{E})$ が $\tau(\Phi) = \Phi$ を充たすならば,$\forall \alpha,\, \beta \in \Phi$ に対して以下が成り立つ:
	\begin{enumerate}
		\item $\tau \circ \sigma_\alpha \circ \tau^{-1} = \sigma_{\tau(\alpha)}$
		\item $\sspair{\beta}{\alpha} = \sspair{\tau(\beta)}{\tau(\alpha)}$
	\end{enumerate}
\end{mylem}

\begin{proof}
	$\forall \alpha,\, \beta \in \Phi$ をとる.\textsf{\textbf{(Root-3)}} より $\sigma_\alpha(\beta) \in \Phi$ なので,$\tau \circ \sigma_\alpha \circ \tau^{-1}\bigl(\tau(\beta)\bigr) = \sigma \circ \sigma_\alpha(\beta) \in \Phi$ が成り立つ.
	一方で
	\begin{align}
		\label{eq:lemWeylgroup1}
		\tau \circ \sigma_\alpha \circ \tau^{-1}\bigl(\tau(\beta)\bigr) = \tau \bigl( \beta - \sspair{\beta}{\alpha}\alpha \bigr) = \tau(\beta) - \sspair{\beta}{\alpha}\tau(\alpha)
	\end{align}
	である.$\beta \in \Phi$ は任意で $\tau \in \LGL(\mathbb{E})$ は全単射なので 
	\begin{description}
		\item[\textbf{(RF-1)}] $\tau \circ \sigma_\alpha \circ \tau^{-1} (\Phi) = \Phi$
		\item[\textbf{(RF-2)}] $\forall \beta \in P_\alpha,\; \tau \circ \sigma_\alpha \circ \tau^{-1}(\beta) = \beta$
		\item[\textbf{(RF-3)}] $\tau(\alpha) \in \Phi \setminus \{0\},\; \tau \circ \sigma_\alpha \circ \tau^{-1} \bigl( \tau(\alpha) \bigr) = -\tau(\alpha)$
	\end{description}
	が成り立つことが分かった.よって補題\ref{lem:refrect}より $\tau \circ \sigma_\alpha \circ \tau^{-1} = \sigma_{\tau(\alpha)}$ である.
	
	さらに\eqref{eq:lemWeylgroup1}から
	\begin{align}
		\tau(\beta) - \sspair{\beta}{\alpha}\tau(\alpha) = \sigma_{\tau(\alpha)} \bigl(\tau(\beta)\bigr) = \tau(\beta) - \sspair{\tau(\beta)}{\tau(\alpha)} \tau(\alpha)
	\end{align}
	が分かるので (2) が従う.
\end{proof}

\begin{mydef}[label=def:isom-root]{ルート系の同型}
	2つの\hyperref[ax:root-system]{ルート系} $(\mathbb{E},\, \Phi),\; (\mathbb{E}',\, \Phi')$ を与える.
	写像
	\begin{align}
		\phi \colon \mathbb{E} \lto \mathbb{E}'
	\end{align}
	が\textbf{ルート系の同型写像} (isomorphism) であるとは,以下の3条件を満たすことを言う:
	\begin{enumerate}
		\item $\phi$ は $\mathbb{R}$-ベクトル空間の同型写像
		\item $\phi(\Phi) = \Phi'$
		\item $\forall \alpha,\, \beta \in \Phi$ に対して $\sspair{\phi(\beta)}{\phi(\alpha)} = \sspair{\beta}{\alpha}$
	\end{enumerate}
	\tcblower
	\hyperref[ax:root-system]{ルート系} $(\mathbb{E},\, \Phi)$ の\textbf{自己同型} (automorphism) とは,$\phi \in \LGL(\mathbb{E})$ であって $\phi(\Phi) = \Phi$ を充たすもののことを言う.これは補題\ref{lem:Weylgroup}-(2) により自動的にルート系の同型となる.
	ルート系の自己同型全体が写像の合成に関してなす群のことを\textbf{ルート系の自己同型群}と呼び,$\bm{\Aut \Phi}$ と書く.
\end{mydef}

\hyperref[def:isom-root]{ルート系の同型}
\begin{align}
	\phi \colon (\mathbb{E},\, \Phi) \lto (\mathbb{E}',\, \Phi')
\end{align}
について,$\forall \alpha,\, \beta \in \Phi$ に対して
\begin{align}
	\sigma_{\phi(\alpha)} \circ \phi(\beta) = \phi(\beta) - \sspair{\phi(\beta)}{\phi(\alpha)} \phi(\alpha) = \phi\bigl(\beta - \sspair{\beta}{\alpha} \alpha\bigr) = \phi \circ \sigma_\alpha(\beta)
\end{align}
が成り立つ.i.e. ルート系の図式
\begin{center}
	\begin{tikzcd}[row sep=large, column sep=large]
		&\Phi \ar[d, "\sigma_\alpha"']\ar[r, "\phi"] &\Phi' \ar[d, "\sigma_{\phi(\alpha)}"]\\
		&\Phi \ar[r, "\phi"'] &\Phi'
	\end{tikzcd}
\end{center}
は可換である.よってルート系の同型は\hyperref[def:Weylgroup]{Weyl群}の自然な(群の)同型
\begin{align}
	\label{eq:induced-Weyl-isom}
	\overline{\phi} \colon \Weyl{\mathbb{E}}{\Phi} &\lto \Weyl{\mathbb{E}'}{\Phi'}, \\
	\sigma &\lmto \phi \circ \sigma \circ \phi^{-1}
\end{align}
を引き起こす.

\begin{mylem}[label=lem:Weylgroup-basic]{}
	\hyperref[ax:root-system]{ルート系} $(\mathbb{E},\, \Phi)$ の\hyperref[def:Weylgroup]{Weyl群} $\Weyl{\mathbb{E}}{\Phi}$ は,\hyperref[def:isom-root]{ルート系の自己同型群} $\Aut \Phi$ の正規部分群である.
\end{mylem}

\begin{proof}
	$\forall \sigma \in \Weyl{\mathbb{E}}{\Phi}$ を1つとる.このときある $\alpha_1,\, \dots,\, \alpha_k \in \Phi$ が存在して $\sigma = \sigma_{\alpha_1} \circ \cdots \circ \sigma_{\alpha_k}$ と書ける\footnote{$\sigma_{\alpha_i}^{-1} = \sigma_{\alpha_i}$ なのでこれで良い.}. 
	従って $\forall \tau \in \Aut \Phi$ に対して,補題\ref{lem:Weylgroup}-(1) より
	\begin{align}
		\tau \circ \sigma \circ \tau^{-1} = (\tau \circ \sigma_{\alpha_1} \circ \tau^{-1}) \circ \cdots \circ (\tau \circ \sigma_{\alpha_k} \circ \tau^{-1}) = \sigma_{\tau(\alpha_1)} \circ \cdots \circ \sigma_{\tau(\alpha_k)} \in \Weyl{\mathbb{E}}{\Phi}
	\end{align}
	が成り立つ.i.e. $\Weyl{\mathbb{E}}{\Phi} \triangleleft \Aut \Phi$ である.
\end{proof}

\begin{mylem}[label=lem:Weylgroup-dual]{双対ルートのWeyl群}
	\begin{align}
		\Weyl{\mathbb{E}}{\Phi} \cong \Weyl{\mathbb{E}}{\Phi^\vee}
	\end{align}
\end{mylem}

\begin{proof}
	\eqref{eq:dual-root3}より,写像 $\sigma_\alpha \lmto \sigma_{\alpha^\vee}$ は同型写像である.
\end{proof}


\section{単純ルートとWeyl群}

この節では $(\mathbb{E},\, \Phi)$ を任意の\hyperref[def:rank-root]{ランク $l$} の\hyperref[ax:root-system]{ルート系}とし,その\hyperref[def:Weylgroup]{Weyl群}を $\mathscr{W}$ と略記する. 

\subsection{ルート系の底とWeylの区画}

\begin{mydef}[label=def:base-root,breakable]{ルート系の底}
	$\Phi$ の部分集合 $\Delta \subset \Phi$ が\textbf{底} (base) であるとは,以下を充たすことをいう:
	\begin{description}
		\item[\textbf{(B-1)}] $\Delta$ は $\mathbb{R}$-ベクトル空間 $\mathbb{E}$ の基底である.
		\item[\textbf{(B-2)}] $\forall \beta \in \Phi$ に対して\underline{整数}の族 $\Familyset[\big]{\beta_\alpha}{\alpha \in \Delta} \in \prod_{\alpha \in \Delta} \textcolor{red}{\mathbb{Z}}$ が一意的に存在して
		\begin{align}
			\beta = \sum_{\alpha \in \Delta} \beta_\alpha \alpha
		\end{align}
		を充たし,$\forall \alpha \in \Delta,\; \beta_\alpha \ge 0$ であるか $\forall \alpha \in \Delta,\; \beta_\alpha \le 0$ であるかのどちらかである.
	\end{description}
	\tcblower
	\begin{itemize}
		\item $\Delta$ の元のことを\textbf{単純ルート} (simple root) と呼ぶ.
		\item $\beta = \sum_{\alpha \in \Delta} \beta_\alpha \alpha \in \Phi$ に対して
		\begin{align}
			\bm{\hight \beta} \coloneqq \sum_{\alpha \in \Delta} \beta_\alpha \in \mathbb{Z}
		\end{align}
		と定義し,底 $\Delta$ に関するルート $\beta$ の\textbf{高さ} (height) と呼ぶ. 
		\item $\beta = \sum_{\alpha \in \Delta} \beta_\alpha \alpha \in \Phi$ が\textbf{正}(resp. \textbf{負})であるとは,$\forall \alpha \in \Delta,\; \beta_\alpha \ge 0$ (resp. $\forall \alpha \in \Delta,\; \beta_\alpha \le 0$)が成り立つことを言い,$\bm{\beta \succ 0}$(resp. $\bm{\beta \prec 0}$)と書く\footnote{\LaTeX コマンドは $\succ$ が \texttt{\textbackslash succ} , $\prec$ が \texttt{\textbackslash prec} である.}.
		\item 正(resp. 負)のルート全体の集合のことを $\bm{\Phi^+}$ (resp. $\bm{\Phi^-}$)と書く\footnote{明らかに $\Phi^- = - \Phi^+$ である.}.
		\item \underline{$\mathbb{E}$ 上の}半順序 $\bm{\prec}\; \subset \mathbb{E} \times \mathbb{E}$を
		\begin{align}
			\mu \prec \lambda \DEF  \exists! \Familyset[\big]{k_\alpha}{\alpha \in \Delta} \in \prod_{\alpha \in \Delta} \mathbb{R}_{\textcolor{red}{\ge 0}},\; \lambda - \mu = \sum_{\alpha \in \Delta} k_\alpha \alpha
		\end{align}
		と定義する.
	\end{itemize}
\end{mydef}

$\prec$ が半順序になっていることを確認しておこう:
\begin{description}
	\item[\textbf{(反射律)}] $\forall \mu \in \mathbb{E}$ に対して $\mu - \mu = 0$ が成り立つので $\mu \prec \mu$ である.
	\item[\textbf{(反対称律)}] $\mu \prec \lambda \AND \lambda \prec \mu$ だとする.
	このとき $\Familyset[\big]{k_\alpha}{\alpha \in \Delta},\; \Familyset[\big]{l_\alpha}{\alpha \in \Delta} \in \prod_{\alpha \in \Delta} \mathbb{R}_{\ge 0}$ が一意的に存在して
	\begin{align}
		\lambda - \mu &= \sum_{\alpha \in \Delta} k_\alpha \alpha, & \mu - \lambda &= \sum_{\alpha \in \Delta} l_\alpha \alpha
	\end{align}
	と書ける.辺々足すと\hyperref[def:base-root]{\textsf{\textbf{(B-1)}}}より $\forall \alpha \in \Delta$ に対して $k_\alpha + l_\alpha = 0$ だと分かるが,$k_\alpha \ge 0 \AND l_\alpha \ge 0$ なので $k_\alpha = l_\alpha = 0$,i.e. $\mu - \lambda = 0 \IFF \mu  = \lambda$ が言えた.
	\item[\textbf{(推移律)}] $\mu \prec \lambda \AND \lambda \prec \nu$ だとする.このとき $\nu - \mu = (\nu - \lambda) + (\lambda - \mu)$ なので明らかに $\mu \prec \nu$ である.
\end{description}

ルート系 $(\mathbb{E},\, \Phi)$ の\hyperref[def:base-root]{底}を定義したのは良いが,存在しなくては意味がない.

\begin{mylem}[label=lem:base]{}
	$\Delta$ が $\Phi$ の\hyperref[def:base-root]{底}ならば,
	相異なる任意の2つの単純ルート $\alpha,\, \beta \in \Delta$ に対して $\rpair{\alpha}{\beta} \le 0$ であり,$\alpha - \beta \notin \Delta$ である.
\end{mylem}

\begin{proof}
	背理法により示す.$\rpair{\alpha}{\beta} > 0$ だとする.仮定より $\alpha \neq \beta$ であり,かつ $\Delta$ の元の線型独立性から $\beta \neq -\alpha$ なので,補題\ref{lem:root-basic}から $\alpha - \beta \in \Delta$ と言うことになる.
	然るにこのとき $\alpha - \beta \in \Phi$ が $\alpha,\, \beta \in \Delta$ の係数 $1,\, -1$ の線型結合で書けていることになり \textsf{\textbf{(B-2)}} に矛盾する.
\end{proof}

\begin{mydef}[label=def:decomposable]{}
	$\forall \gamma \in \mathbb{E}$ に対して以下を定義する:
	\begin{itemize}
		\item $\Phi$ の部分集合
		\begin{align}
			\bm{\Phi^+(\gamma)} \coloneqq \bigl\{\, \alpha \in \Phi \bigm| \rpair{\gamma}{\alpha} > 0 \,\bigr\} 
		\end{align}
		\item \begin{align}
			\gamma \in \mathbb{E} \setminus \bigcup_{\alpha \in \Phi} P_\alpha
		\end{align}
		のとき,$\gamma$ は\textbf{正則} (regular) であるという.$\gamma$ が正則でないとき\textbf{特異} (sigular) であると言う.
		\item $\alpha \in \Phi^+ (\gamma)$ が
		\begin{align}
			\exists \beta_1,\, \beta_2 \in \Phi^+(\gamma),\; \alpha = \beta_1 + \beta_2
		\end{align}
		を充たすとき,$\alpha$ は\textbf{分割可能} (decomposable) であると言う.分割可能でないとき\textbf{分割不可能} (indecomposable) であると言う.
	\end{itemize}
\end{mydef}

$\gamma$ が正則ならば $\forall \alpha \in \Phi$ に対して $\rpair{\gamma}{\alpha} \neq 0$ なので,\hyperref[ax:root-system]{\textsf{\textbf{(Root-2)}}}から $\Phi = \Phi^+(\gamma) \amalg \bigl( -\Phi^+(\gamma) \bigr)$ (disjoint union)が成り立つ.

\begin{mytheo}[label=thm:base-exist]{底の存在}
	\hyperref[def:decomposable]{正則}な任意の $\gamma \in \mathbb{E}$ を与える.このとき集合
	\begin{align}
		\Delta(\gamma) \coloneqq \bigl\{\, \alpha \in \Phi^+(\gamma) \bigm| \text{\hyperref[def:decomposable]{分割不可能}} \,\bigr\} 
	\end{align}
	は $\Phi$ の\hyperref[def:base-root]{底}である.
	逆に $\Phi$ の任意の底 $\Delta$ に対してある正則な $\gamma \in \mathbb{E}$ が存在して $\Delta = \Delta(\gamma)$ となる.
\end{mytheo}

\begin{proof}
	\begin{description}
		\item[\textbf{step1: $\bm{\Phi^+(\gamma)}$ の任意の元は $\bm{\Delta (\gamma)}$ の $\bm{\mathbb{Z}_{\ge 0}}$-係数線型結合で書ける}] 
		
		背理法により示す.$\Delta (\gamma)$ の $\mathbb{Z}_{\ge 0}$-係数線型結合で書けない $\alpha \in \Phi^+(\gamma)$ が存在するとする.
		このとき,そのような $\alpha$ のうち $\rpair{\gamma}{\alpha}$ が最小であるようなものが存在するのでそれを $\alpha_0$ とおく.$\alpha_0 \notin \Delta(\gamma)$ なので\footnote{$\alpha_0 \in \Delta(\gamma)$ だとすると,$\alpha_0 \in \Delta(\gamma)$ の係数 $1 \in \mathbb{Z}_{\ge 0}$ の線型結合として書けていることになり矛盾.}$\alpha_0$ は分割可能であり,ある $\beta_1,\, \beta_2 \in \Phi^+(\gamma)$ が存在して $\alpha = \beta_1 + \beta_2$ と書ける.このとき
		\begin{align}
			\rpair{\gamma}{\alpha_0} = \rpair{\gamma}{\beta_1} + \rpair{\gamma}{\beta_2} > \rpair{\gamma}{\beta_i}
		\end{align}
		が成り立つので,$\alpha_0$ の最小性から $\beta_1,\, \beta_2$ はどちらも $\Delta(\gamma)$ の元の $\mathbb{Z}_{\ge 0}$-係数線型結合で書ける.然るにこのとき $\alpha$ も $\Delta(\gamma)$ の元の $\mathbb{Z}_{\ge 0}$-係数線型結合で書けることになって矛盾.

		\item[\textbf{step2: $\bm{\alpha,\, \beta \in \Delta(\gamma)}$ かつ $\bm{\alpha \neq \beta}$ ならば,$\bm{(\alpha,\, \beta) \le 0}$}] 
		
		背理法により示す.$\rpair{\alpha}{\beta} > 0$ を仮定する.このとき補題\ref{lem:root-basic}-(1) より $\alpha - \beta \in \Phi$ であり,$\beta - \alpha = \sigma_{\alpha-\beta}(\alpha - \beta) \in \Phi$ もわかる.よって $\alpha - \beta \in \Phi^+(\gamma)$ または $\beta - \alpha \in \Phi^+(\gamma)$ である.
		然るに前者の場合 $\alpha = \beta + (\alpha - \beta)$ なので $\alpha \in \Delta(\gamma)$ が分割可能ということになって矛盾し,後者の場合は $\beta = \alpha + (\beta - \alpha)$ なので $\beta \in \Delta(\gamma)$ が分割可能ということになって矛盾である.

		\item[\textbf{step3: $\bm{\Delta(\gamma)}$ の元は互いに線型独立}] 
		
		$\Familyset[\big]{k_\alpha}{\alpha \in \Delta (\gamma)} \in \prod_{\alpha \in \Delta} \mathbb{R}$ に対して $\sum_{\alpha \in \Delta (\gamma)} k_\alpha \alpha = 0$ を仮定する.$\forall \alpha \in \Delta(\gamma)$ に対して
		\begin{align}
			% s_\alpha &\coloneqq
			% \begin{cases}
			% 	k_\alpha, & k_\alpha > 0 \\
			% 	0, & k_\alpha \le 0
			% \end{cases}
			% & t_\alpha &\coloneqq
			% \begin{cases}
			% 	-k_\alpha, & k_\alpha < 0 \\
			% 	0, & k_\alpha \ge 0
			% \end{cases}
			\Delta^+(\gamma) &\coloneqq \bigl\{\, \alpha \in \Delta(\gamma) \bigm| k_\alpha > 0 \,\bigr\}, &
			\Delta^-(\gamma) &\coloneqq \bigl\{\, \alpha \in \Delta(\gamma) \bigm| k_\alpha < 0 \,\bigr\}
		\end{align}
		とおくと $\Delta^+(\gamma) \cap \Delta^-(\gamma) = \emptyset$ で,仮定は
		\begin{align}
			\sum_{\alpha \in \Delta^+(\gamma)} k_\alpha \alpha = \sum_{\alpha \in \Delta^-(\gamma)} (-k_\alpha) \alpha
		\end{align}
		と同値である.
		$\varepsilon \coloneqq \sum_{\alpha \in \Delta^+ (\gamma)} k_\alpha \alpha$ とおくと,
		\textsf{\textbf{step2}}から
		\begin{align}
			0 \le \rpair{\varepsilon}{\varepsilon} = \sum_{\alpha \in \Delta^+(\gamma),\, \beta \in \Delta^-(\gamma)} k_\alpha (-k_\beta) \rpair{\alpha}{\beta} \le 0
		\end{align}
		が成り立つので $\varepsilon = 0$ だとわかる.よって
		\begin{align}
			0 &= \rpair{\gamma}{\varepsilon}
			= \sum_{\alpha \in \Delta^+(\gamma)} k_\alpha \rpair{\gamma}{\alpha}
			= \sum_{\alpha \in \Delta^-(\gamma)} (-k_\alpha) \rpair{\gamma}{\alpha}
		\end{align}
		であり,$\forall \alpha \in \Delta(\gamma),\; k_\alpha = 0$ が言えた.
		\item[\textbf{step4: $\bm{\Delta(\gamma)}$ は $\bm{\Phi}$ の底}] 
		
		$\Phi = \Phi^+ (\gamma) \amalg \bigl( -\Phi^+ (\gamma) \bigr)$ なので,
		\textsf{\textbf{step1}}と併せて\hyperref[def:base-root]{\textsf{\textbf{(B-2)}}}が,
		\textsf{\textbf{step3}}と併せて\hyperref[def:base-root]{\textsf{\textbf{(B-1)}}}が従う.

		\item[\textbf{step5: 任意の底 $\bm{\Delta}$ に対してある正則な $\bm{\gamma \in \mathbb{E}}$ が存在して $\bm{\Delta = \Delta(\gamma)}$ となる}] 
		
		$\Phi$ の\hyperref[def:base-root]{底} $\Delta$ が与えられたとき,$\forall \alpha \in \Delta$ に対して $\rpair{\gamma}{\alpha} > 0$ を充たす $\gamma \in \mathbb{E}$ をとる
		\footnote{
			このような $\gamma$ が存在することを示そう.\textsf{\textbf{(B-1)}} より $\Delta$ は $\mathbb{E}$ の基底だから,$\forall \alpha$ に対して $\gamma_\alpha \in \mathbb{E}$ を,$\mathbb{E}$ の部分ベクトル空間 $\Span_{\mathbb{K}} \bigl(\Delta \setminus \{\alpha\}\bigr)$ の $\rpair{\;}{\,}$ に関する\hyperref[def:radical-bilinear]{直交補空間} $\Bigl(\Span_{\mathbb{K}} \bigl(\Delta \setminus \{\alpha\}\bigr)\Bigr)^\perp$ への $\alpha$ の射影とする.$\Delta$ の元は全て互いに線型独立なので $\gamma_\alpha \neq 0$ である.このとき,$\Familyset[\big]{k_\alpha}{\alpha \in \Delta} \in \prod_{\alpha \in \Delta} \mathbb{R}_{\textcolor{red}{> 0}} $ に対して $\gamma \coloneqq \sum_{\alpha \in \Delta} k_\alpha \gamma_\alpha$ とおけば,$\forall \alpha \in \Delta$ に対して $\rpair{\gamma}{\alpha} = k_\alpha \rpair{\gamma_\alpha}{\alpha} > 0$ が成り立つ.}.
		\textsf{\textbf{(B-2)}} より $\gamma$ は\hyperref[def:decomposable]{正則}であり,かつ $\forall \beta = \sum_{\alpha \in \Delta} \beta_\alpha \alpha \in \Phi^+$ に対して
		\begin{align}
			\rpair{\gamma}{\beta} = \sum_{\alpha \in \Delta} \beta_\alpha \rpair{\gamma}{\alpha} > 0
		\end{align}
		が成り立つので $\beta \in \Phi^+(\gamma)$,i.e. $\Phi^+ \subset \Phi^+(\gamma),\; \Phi^-  = - \Phi^- \subset -\Phi^+(\gamma)$ も分かる.
		ところが \textsf{\textbf{step4}} より $\Phi = \Phi^+ \amalg \Phi^- = \Phi^+(\gamma) \amalg \bigl( -\Phi^+(\gamma) \bigr)$ なので,$\Phi^+ = \Phi^+(\gamma)$ でなくてはいけない.
		従って $\forall \alpha \in \Delta$ は\hyperref[def:decomposable]{分割不可能}であり
		\footnote{$\beta_1 = \sum_{\alpha \in \Delta} \beta_1{}_\alpha \alpha,\, \beta_2 = \sum_{\alpha \in \Delta} \beta_2{}_\alpha \alpha \in \Phi^+ = \Phi^+(\gamma)$ を用いて $\alpha = \beta_1 + \beta_2$ と書けたとする.このとき $\Delta$ の元の線型独立性から $\beta_1{}_\alpha + \beta_2{}_\alpha = 1 \AND \forall \gamma \in \Delta \setminus \{\alpha\},\; \beta_1{}_\gamma + \beta_2{}_\gamma = 0$ が成り立つが,\textsf{\textbf{(B-2)}}より $(\beta_1{}_\alpha,\, \beta_2{}_\alpha) = (1,\, 0) \OR (0,\, 1) \AND \forall \gamma \in \Delta \setminus \{\alpha\},\; \beta_1{}_\gamma = \beta_2{}_\gamma = 0$,i.e. $(\beta_1,\, \beta_2) = (\alpha,\, 0)\OR (0,\, \alpha)$ でなくてはならず,$0 \notin \Phi^+$ に矛盾.}
		,$\Delta \subset \Delta(\gamma)$ だと分かった.
		\textsf{\textbf{(B-1)}}および\textsf{\textbf{step4}}より $\abs{\Delta} = \abs{\Delta(\gamma)} = l$ なので\footnote{~\cite{Humphreys1972introduction}では集合の\textbf{濃度} (cardinality) の意味で $\Card \Delta$ と書かれていた.} $\Delta = \Delta(\gamma)$ が言えた.
	\end{description}
\end{proof}

\begin{mydef}[label=def:Weylchamber]{Weylの区画}
	\begin{itemize}
		\item 位相空間\footnote{\hyperref[def:Euclid-space]{Euclid空間の定義}の脚注を参照.} $\mathbb{E}$ の部分空間 $\mathbb{E} \setminus \bigcup_{\alpha \in \Phi} P_\alpha$ の連結成分の1つのことを(開な)\textbf{Weylの区画} (Weyl chamber) \footnote{この訳語は筆者が勝手につけたものである.~\cite{Satake1987LieAlg}では\textbf{Weylの部屋}と呼ばれていた.}と呼ぶ.
		\item \hyperref[def:decomposable]{正則}な $\gamma \in \mathbb{E}$ が属するWeylの区画のことを $\bm{\mathfrak{C}(\gamma)}$ と書く\footnote{\LaTeX コマンドは \texttt{\textbackslash mathfrak\{C\}}}.
		\item $\Phi$ の\hyperref[def:base-root]{底} $\Delta$ に対して定理\ref{thm:base-exist}の意味で $\Delta = \Delta(\gamma)$ ならば $\bm{\mathfrak{C}(\Delta)} \coloneqq \mathfrak{C}(\gamma)$ とおき,\textbf{$\bm{\Delta}$ に関するWeylの基本区画} (fundamental Weyl chamber relative to $\Delta$)\footnote{これの訳語は全く普及していない気がする.\textbf{$\bm{\Delta}$ に関する基本的Weylの部屋}だと語感が悪いと思ったのでこのような訳語を充てた.} と呼ぶ.
	\end{itemize}
\end{mydef}

\begin{mylem}[label=lem:Weylchamber-basic]{Weylの区画の基本性質}
	\hyperref[def:decomposable]{正則}な任意の $\gamma,\, \gamma' \in \mathbb{E}$ および任意の $\Phi$ の\hyperref[def:base-root]{底} $\Delta$ を与える.定理\ref{thm:base-exist}によって得られる $\Phi$ の\hyperref[def:base-root]{底}を $\Delta(\gamma)$ と書く.
	\begin{enumerate}
		\item $\mathfrak{C}(\gamma) = \mathfrak{C}(\gamma') \IFF \Delta(\gamma) = \Delta(\gamma')$
		\item 写像
		\begin{align}
			\{\,\Phi\; \text{の\hyperref[def:base-root]{底}全体の集合}\, \} &\lto \{\, \text{\hyperref[def:Weylchamber]{Weylの区画}全体の集合}\, \}, \\
			\Delta &\lmto \mathfrak{C}(\Delta)
		\end{align}
		は全単射である.
		\item $\mathfrak{C}(\Delta) = \bigl\{\, \beta \in \mathbb{E} \bigm| \forall \alpha \in \Delta,\; \rpair{\beta}{\alpha} > 0 \,\bigr\}$
	\end{enumerate}
\end{mylem}

\begin{proof}
	\begin{enumerate}
		\item $\alpha \in \Phi$ に関する鏡映面 $P_\alpha$ に関して
		\begin{align}
			P_\alpha^+ &\coloneqq \bigl\{\, \beta \in \mathbb{E} \bigm| \rpair{\beta}{\alpha} > 0 \,\bigr\}, &P_\alpha^- &\coloneqq \bigl\{\, \beta \in \mathbb{E} \bigm| \rpair{\beta}{\alpha} < 0 \,\bigr\} 
		\end{align}
		とおく.すると $\mathbb{E} \setminus P_\alpha = P_\alpha^+ \cup P_\alpha^-$ (disjoint union)となるから,
		\begin{align}
			\mathbb{E} \setminus \bigcup_{\alpha \in \Phi} P_\alpha
			&= \bigcap_{\alpha \in \Phi} (\mathbb{E} \setminus P_\alpha) \\
			&= \bigcap_{\alpha \in \Phi} (P_\alpha^+ \cup P_\alpha^-) \\
			&= \bigcup_{\Familyset{\mu_\alpha}{\alpha \in \Phi} \in \prod_{\alpha \in \Phi}\{\pm\}} \bigcap_{\alpha \in \Phi} P_\alpha^{\mu_\alpha}
		\end{align}
		である.最右辺の $\bigcap_{\alpha \in \Phi} P_\alpha^{\mu_\alpha}$ は凸集合の共通部分なので凸集合であり,従って連結である.さらに $\Familyset{\mu_\alpha}{\alpha \in \Phi} \in \prod_{\alpha \in \Phi}\{\pm\}$ に関する重複を除いて位相空間の意味でdisjointであるからWeylの区画である
		\footnote{より厳密には,$\mathbb{E} \setminus \bigcup_{\alpha \in \Phi} P_\alpha$ の部分空間として開かつ閉 (clopen) であり,かつそれ自身連結なので,連結成分の1つだと分かる.}
		.
		従って $\mathfrak{C}(\gamma)$ は,$\forall \alpha \in \Phi$ に対して
		\begin{align}
			\mu_\alpha(\gamma) \coloneqq 
			\begin{cases}
				+, & \rpair{\gamma}{\alpha} > 0, \\
				-, & \rpair{\gamma}{\alpha} < 0
			\end{cases}
		\end{align}
		とおけば
		\begin{align}
			\label{eq:Weylchamber}
			\mathfrak{C}(\gamma) = \bigcap_{\alpha \in \Phi} P_\alpha^{\mu_\alpha(\gamma)} = \bigl\{\, \beta \in \mathbb{E} \bigm| \forall \alpha \in \Phi,\; \rpair{\gamma}{\alpha}\rpair{\beta}{\alpha} > 0 \,\bigr\} 
		\end{align}
		と書ける.よって
		\begin{align}
			\mathfrak{C}(\gamma) = \mathfrak{C}(\gamma') 
			&\IFF \forall \alpha \in \Phi,\; \mu_\alpha(\gamma) = \mu_\alpha(\gamma') \\
			&\IFF \Phi^+(\gamma) = \Phi^+(\gamma') \\
			&\IFF \Delta(\gamma) = \Delta(\gamma')
		\end{align}
		が言える.
		\item (1),定理\ref{thm:base-exist},および\hyperref[def:Weylchamber]{Weylの基本区画の定義}から従う.
		\item $\Delta = \Delta(\gamma)$ を充たす正則な $\gamma \in \mathbb{E}$ を1つとる.すると\eqref{eq:Weylchamber}より
		\begin{align}
			\mathfrak{C}(\Delta) 
			&= \bigl\{\, \beta \in \mathbb{E} \bigm| \forall \alpha \in \Phi,\; \rpair{\gamma}{\alpha}\rpair{\beta}{\alpha} > 0 \,\bigr\}
		\end{align}
		だが,$\Phi = \Phi^+(\gamma) \amalg \bigl(-\Phi^+(\gamma)\bigr)$ でかつ\hyperref[def:decomposable]{$\Phi^+(\gamma)$ の定義}より $\forall \alpha \in \Phi^+(\gamma)$ に対して $\rpair{\gamma}{\alpha} > 0$ が成り立つので
		\begin{align}
			\mathfrak{C}(\Delta) 
			&= \bigl\{\, \beta \in \mathbb{E} \bigm| \forall \alpha \in \Phi^+(\gamma),\; \rpair{\beta}{\alpha} > 0 \,\bigr\} \\
			&= \bigl\{\, \beta \in \mathbb{E} \bigm| \forall \alpha \in \Delta(\gamma),\; \rpair{\beta}{\alpha} > 0 \,\bigr\}
		\end{align}
		だと分かった.
	\end{enumerate}
	
\end{proof}


\begin{mylem}[label=lem:Weylchamber-Weylgroup]{Weylの区画とWeyl群の関係}
	\hyperref[def:decomposable]{正則}な任意の $\gamma \in \mathbb{E}$ および任意の $\Phi$ の\hyperref[def:base-root]{底} $\Delta$ を与える.このとき$\forall \sigma \in \mathscr{W}$ に対して以下が成り立つ:
	\begin{enumerate}
		\item $\sigma\bigl(\Delta(\gamma)\bigr) = \Delta \bigl( \sigma(\gamma) \bigr)$
		\item $\sigma(\Delta)$ もまた $\Phi$ の\hyperref[def:base-root]{底}である.
		\item $\sigma \bigl( \mathfrak{C}(\gamma) \bigr) = \mathfrak{C} \bigl( \sigma(\gamma) \bigr)$
		\item $\sigma \bigl( \mathfrak{C}(\Delta) \bigr) = \mathfrak{C} \bigl( \sigma(\Delta) \bigr)$
	\end{enumerate}
\end{mylem}

\begin{proof}
	鏡映は等長変換(isometry)なので $\rpair[\big]{\sigma(\alpha)}{\sigma(\beta)} = \rpair{\alpha}{\beta}$ が成り立つ.
	\begin{enumerate}
		\item 
		\begin{align}
			\sigma\bigl(\Phi^+ (\gamma)\bigr) 
			&= \bigl\{\, \sigma(\alpha) \in \Phi \bigm| \rpair{\gamma}{\gamma} = \rpair[\big]{\sigma(\gamma)}{\sigma(\alpha)} > 0 \,\bigr\} \\
			&= \bigl\{\, \beta \in \Phi \bigm| \rpair[\big]{\sigma(\gamma)}{\beta} > 0 \,\bigr\} \\
			&=\Phi^+ \bigl( \sigma(\gamma) \bigr) 
		\end{align}
		が分かる.従って $\sigma\bigl(\Delta (\gamma)\bigr) = \sigma\bigl(\Delta (\gamma)\bigr) $ である.
		\item 定理\ref{thm:base-exist}より,ある正則な $\gamma \in \mathbb{E}$ が存在して $\Delta = \Delta(\gamma)$ と書ける.よって (1) から $\sigma(\Delta) = \Delta \bigl( \sigma(\gamma) \bigr)$ であるが,再度定理\ref{thm:base-exist}より右辺は $\Phi$ の底である.
		\item \eqref{eq:Weylchamber},および $\sigma (\Phi) = \Phi$ より
		\begin{align}
			\sigma \bigl( \mathfrak{C}(\gamma) \bigr) 
			&= \bigl\{\, \sigma(\beta) \in \mathbb{E} \bigm| \forall \alpha \in \Phi,\; \rpair{\gamma}{\alpha}\rpair{\beta}{\alpha} = \rpair[\big]{\sigma(\gamma)}{\sigma(\alpha)} \rpair[\big]{\sigma(\beta)}{\sigma(\alpha)} > 0 \,\bigr\} \\
			&= \bigl\{\, \beta \in \mathbb{E} \bigm| \forall \alpha \in \Phi,\; \rpair[\big]{\sigma(\gamma)}{\alpha} \rpair[\big]{\beta}{\alpha} > 0 \,\bigr\} \\
			&= \mathfrak{C} \bigl( \sigma(\gamma) \bigr) 
		\end{align}
		\item (4) および定理\ref{thm:base-exist}より従う.
	\end{enumerate}
	
\end{proof}


\subsection{単純ルートに関する補題}



\subsection{Weyl群の性質}

\subsection{既約なルート系}

\section{ルート系の分類}

\end{document}