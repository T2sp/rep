\documentclass[rep_main]{subfiles}

\begin{document}

\setcounter{chapter}{2}

\chapter{ルート系}

この章において,特に断らない限り体 $\mathbb{K}$ は代数閉体\footnote{つまり,定数でない任意の1変数多項式 $f(x) \in \mathbb{K}[x]$ に対してある $\alpha \in \mathbb{K}$ が存在して $f(\alpha) = 0$ を充たす.}で,かつ $\character \mathbb{K} = 0$ であるとする.
また,\hyperref[ax:LieAlg]{Lie代数} $\mathfrak{g}$ は常に\underline{有限次元}であるとする.

\section{公理的方法}

\label{def:Euclid-space}\textbf{Euclid空間} (Euclid space) とは,
\begin{itemize}
	\item 体 $\mathbb{R}$ 上の\underline{有限次元}ベクトル空間 $\mathbb{E}$
	\item 対称かつ正定値な双線型形式 $\rpair{\;}{\,}_{\mathbb{E}}\colon \mathbb{E} \times \mathbb{E} \lto \mathbb{R}$
\end{itemize}
の組 $\Euc{\mathbb{E}}$ のことを言う
\footnote{Euclid空間と言って位相空間のことを指す場合があるが,そのときは双線型形式 $\rpair{\;}{\,}_{\mathbb{E}}$ を使って $\mathbb{E}$ 上の距離関数を $d_{\mathbb{E}} \colon \mathbb{E} \times \mathbb{E} \lto \mathbb{R}_{\ge 0},\; (x,\, y) \lmto \rpair{x-y}{x-y}_{\mathbb{E}}$ と定義し(これは通常\textbf{Euclid距離}と呼ばれる),$\mathbb{E}$ に $d_{\mathbb{E}}$ による距離位相を入れる.}.
Euclid空間 $\Euc{\mathbb{E}}$ の任意の元 $\alpha \in \mathbb{E}$ に対して,
\begin{itemize}
    \item \textbf{鏡映面} (reflecting hyperplane)\footnote{余次元 $1$ の部分 $\mathbb{R}$-ベクトル空間.最右辺は\hyperref[def:radical-bilinear]{対称かつ非退化な双線型形式 $(\;,\,)_{\mathbb{E}}$ による直交補空間}の意味である.}
    \begin{align}
        \label{def:hyperplane}
        \bm{P_\alpha} \coloneqq \bigl\{\, \beta \in \mathbb{E} \bigm| \rpair{\beta}{\alpha}_{\mathbb{E}} = 0 \,\bigr\} = (\mathbb{R}\alpha)^\perp
    \end{align}
    \item 鏡映面 $P_\alpha$ に関する\textbf{鏡映} (reflecting)
    \begin{align}
        \bm{\sigma_\alpha} \colon \mathbb{E} \lto \mathbb{E},\; \beta \lmto \beta - 2 \frac{\rpair{\beta}{\alpha}_{\mathbb{E}}}{\rpair{\alpha}{\alpha}_{\mathbb{E}}} \alpha
    \end{align}
\end{itemize}
を考える.


\begin{marker}
    $2 \frac{(\beta,\, \alpha)}{(\alpha,\, \alpha)} \in \mathbb{R}$ が頻繁に登場するので,
    \begin{align}
		\sspair{\beta}{\alpha} \coloneqq 2\frac{(\beta,\, \alpha)_{\mathbb{E}}}{(\alpha,\, \alpha)_{\mathbb{E}}}
	\end{align}
	と略記することにする.写像 $\sspair{\;}{\,} \colon \mathbb{E} \times \mathbb{E} \lto \mathbb{R}$ は記号的には内積のように見えるかもしれないが,あくまで第一引数についてのみ線型なのであって,\underline{対称でも双線型でもない}ことに注意.
\end{marker}

$\sigma_\alpha$ は $\mathbb{R}$-線型でかつ $\forall \beta \in \mathbb{E}$ に対して
\begin{align}
	\sigma_\alpha \circ \sigma_\alpha (\beta) 
	&= \bigl(\beta - \sspair{\beta}{\alpha} \alpha\bigr) - \sspair[\Big]{\bigl(\beta - \sspair{\beta}{\alpha} \alpha\bigr)}{\alpha}\alpha \\
	&= \beta - \sspair{\beta}{\alpha} \alpha - \sspair{\beta}{\alpha}\alpha + \sspair{\beta}{\alpha}\sspair{\alpha}{\alpha} \alpha \\
	&= \beta - 2\sspair{\beta}{\alpha}\alpha + 2\sspair{\beta}{\alpha}\alpha \\
	&= \beta
\end{align}
を充たす,i.e. $\sigma_\alpha^{-1} = \sigma_\alpha$ なので,$\sigma_\alpha \in \LGL(\mathbb{E})$ である.

\begin{mylem}[label=lem:refrect]{鏡映の特徴付け}
	Euclid空間 $\mathbb{E}$ と,
	\begin{itemize}
		\item $\mathbb{E} = \Span_{\mathbb{R}}\Phi$ 
		\item $\forall \alpha \in \Phi,\; \sigma_\alpha(\Phi) = \Phi$ 
	\end{itemize}
	を充たす $\mathbb{E}$ の有限部分集合 $\Phi \subset \mathbb{E}$ を与える.

	このとき,$\sigma \in \LGL(\mathbb{E})$ が
	\begin{description}
		\item[\textbf{(RF-1)}]  $\sigma(\Phi) = \Phi$
		\item[\textbf{(RF-2)}]  余次元 $1$ の部分ベクトル空間 $P \subset \mathbb{R}$ が存在して,$\forall \beta \in P,\; \sigma(\beta) = \beta$ が成り立つ
		\item[\textbf{(RF-3)}]  $\exists \alpha \in \Phi \setminus \{0\},\; \sigma(\alpha) = -\alpha$
	\end{description}
	の3条件を満たすならば $\sigma = \sigma_\alpha$ (かつ $P = P_\alpha$)である.
\end{mylem}

\begin{proof}
	$\tau \coloneqq \sigma \circ \sigma_\alpha \, (= \sigma \circ \sigma_\alpha^{-1})$ とおき,$\tau = \mathrm{id}_{\mathbb{E}}$ であることを示す.

	\textsf{\textbf{(RF-1)}} より $\tau(\Phi) = \Phi,\; \tau(\alpha) = \alpha$ が成り立つので $\tau|_{\mathbb{R}\alpha} = \mathrm{id}_{\mathbb{R}\alpha}$ である.
	さらに $\mathbb{R}$-線型写像
	\begin{align}
		\overline{\tau} \colon \mathbb{E}/\mathbb{R}\alpha \lto \mathbb{E}/\mathbb{R}\alpha,\; \beta + \mathbb{R}\alpha \lmto \tau(\beta) + \mathbb{R}\alpha
	\end{align}
	はwell-definedだが,\textsf{\textbf{(RF-2)}}より $\overline{\tau} = \mathrm{id}_{ \mathbb{E}/\mathbb{R}\alpha}$ である.よって $\tau$ の固有値は全て $1$ であり,$\tau$ の\hyperref[def:minimal-poly]{最小多項式} $f(t)$ は $(t-1)^{\dim \mathbb{E}}$ の\hyperref[def:domain-basic]{約元}である.
	一方,$\Phi$ は有限集合なので,$\forall \beta \in \Phi$ に対してある $k_\beta \in \mathbb{N}$ が存在して $\tau^{k_\beta}(\beta) = \beta$ を充たす.ここで $k \coloneqq \max \{k_\beta\mid \beta \in \Phi\}$ とおくと,\textsf{\textbf{(RF-1)}} より $\tau^k = \mathrm{id}_{\mathbb{E}}$ が言える.よって $f(t)$ は $t^k - 1$ の約元でもある.
	従って,$f(t) = \GCD \bigl( (t-1)^{\dim \mathbb{E}},\, t^k - 1 \bigr) = t-1$ だと分かった.故に $\tau = \mathrm{id}_{\mathbb{E}}$ である.
\end{proof}


\subsection{ルート系}

前章で与えたルート系の公理を再掲するところから始めよう:

\begin{myaxiom}[label=ax:root-system,breakable]{ルート系}
	\begin{itemize}
		\item 有限次元Euclid空間 $\Euc{\mathbb{E}}$ 
		\item $\mathbb{E}$ の部分集合 $\Phi \subset \mathbb{E}$
	\end{itemize}
	の組 $(\mathbb{E},\, \Phi)$ が\textbf{ルート系} (root system) であるとは,以下の条件を充たすことを言う:
	\begin{description}
		\item[\textbf{(Root-1)}] $\Phi$ は $0$ を含まない有限集合で,かつ $\mathbb{E} = \Span_{\mathbb{R}} \Phi$ を充たす.
		\item[\textbf{(Root-2)}] $\lambda \alpha \in \Phi \IMP \lambda = \pm 1$
		\item[\textbf{(Root-3)}] $\alpha,\, \beta \in \Phi \IMP \sigma_\alpha(\beta) \in \Phi$
		\item[\textbf{(Root-4)}] $\alpha,\, \beta \in \Phi \IMP \sspair{\beta}{\alpha} \in \mathbb{Z}$
	\end{description}
	\tcblower
	$\Phi$ の元のことを\textbf{ルート} (root) と呼ぶ.
\end{myaxiom}

\begin{marker}
	本資料の以降では,文脈上直積集合の要素との混同が起きる恐れがないときは\hyperref[def:Euclid-space]{Euclid空間} $\Euc{\mathbb{E}}$ に備わっている双線型形式を $\rpair{\;}{\,}_{\mathbb{E}}$ と書く代わりに $\rpair{\;}{\,}$ と略記する.
\end{marker}


ルート系と言ったときに,\textsf{\textbf{(Root-2)}} を除外する場合がある.その場合我々が採用した定義に該当するものは\textbf{簡約ルート系} (reduced root system) と呼ばれる.

\begin{mydef}[label=def:rank-root]{}
	\hyperref[ax:root-system]{ルート系} $(\mathbb{E},\, \Phi)$ の\textbf{ランク} (rank) とは,$\dim \mathbb{E} \in \mathbb{N}$ のことを言う.
\end{mydef}

公理 \textsf{\textbf{(Root-4)}} は,任意のルートの2つ組の配位に非常に強い制約を与える.
というのも,2つのベクトルのなす角の定義を思い出すと,$\forall \alpha,\, \beta \in \Phi$ に対してある $\theta \in [0,\, \pi]$ が存在して
\begin{align}
	\sspair{\beta}{\alpha} 
	&= 2 \frac{\norm*{\beta}}{\norm*{\alpha}} \cos \theta \in \mathbb{Z} \label{eq:root4-1} \\
	\sspair{\alpha}{\beta}\sspair{\beta}{\alpha} 
	&= 4 \cos^2 \theta \in \mathbb{Z} \label{eq:root4-2}
\end{align}
が成り立たねばならないのである.$\sspair{\alpha}{\beta},\,\sspair{\beta}{\alpha} \in \mathbb{Z}$ かつ $0 \le \cos^2 \theta \le 1$ なので,\eqref{eq:root4-2}から
\begin{align}
	\cos^2 \theta = 0,\, \frac{(\pm 1) \cdot (\pm 1)}{4},\, \frac{(\pm 1) \cdot (\pm 2)}{4},\, \frac{(\pm 1) \cdot (\pm 3)}{4},\, \frac{ (\pm 1) \cdot (\pm 4)}{4},\, \frac{(\pm 2) \cdot (\pm 2)}{4} \quad (\text{複号同順})
\end{align}
しかあり得ないとわかる.\textsf{\textbf{(Root-2)}}も考慮すると,$\norm*{\alpha} \le \norm*{\beta}$ ならば\footnote{このとき\eqref{eq:root4-1}より $\sspair{\alpha}{\beta} \le \sspair{\beta}{\alpha}$}あり得る可能性は以下の通り\footnote{表\ref{tab:rootsystem}の最後の2段は $\beta = \pm \alpha$ の場合に相当する.}:

\begin{table}[H]
	\centering
	\caption[]{可能なルートの2つ組 $\alpha,\, \beta$}
	\label{tab:rootsystem}
	\begin{tabular}{cc|cc}
		\multicolumn{1}{c}{$\sspair{\alpha}{\beta}$} 
		&\multicolumn{1}{c}{$\sspair{\beta}{\alpha}$} 
		&\multicolumn{1}{|c}{$\theta$} 
		&\multicolumn{1}{c}{$\norm*{\beta}^2/\norm*{\alpha}^2$} \\
		\hhline{--|--}
		$0 $ &$0 $ &$\frac{\pi}{2}$ 	&- \\
		$1 $ &$1 $ &$\frac{\pi}{3}$ 	&$1$ \\
		$-1$ &$-1$ &$\frac{2\pi}{3}$ 	&$1$ \\
		$1 $ &$2 $ &$\frac{\pi}{4}$ 	&$2$ \\
		$-1$ &$-2$ &$\frac{3\pi}{4}$ 	&$2$ \\
		$1 $ &$3 $ &$\frac{\pi}{6}$ 	&$3$ \\
		$-1$ &$-3$ &$\frac{5\pi}{6}$ 	&$3$ \\
		$2 $ &$2 $ &$0$ 				&$1$ \\
		$-2$ &$-2$ &$\pi$ 				&$1$ \\
	\end{tabular}
\end{table}%

\begin{mylem}[label=lem:root-basic]{}
	$(\mathbb{E},\, \Phi)$ を\hyperref[ax:root-system]{ルート系}とする.
	$\alpha \neq \pm \beta$ を充たす任意の $\alpha,\, \beta \in \Phi$ に対して以下が成り立つ:
	\begin{enumerate}
		\item $\rpair{\alpha}{\beta} > 0 \IMP \alpha - \beta \in \Phi$
		\item $\rpair{\alpha}{\beta} < 0 \IMP \alpha + \beta \in \Phi$
	\end{enumerate}
\end{mylem}

\begin{proof}
	$\rpair{\alpha}{\beta} > 0$ とする.このとき $\sspair{\alpha}{\beta} > 0$ であるから,表\ref{tab:rootsystem}より $\sspair{\alpha}{\beta},\, \sspair{\beta}{\alpha}$ の少なくとも一方は $1$ に等しい.$\sspair{\alpha}{\beta} = 1$ だとすると,\textsf{\textbf{(Root-3)}}から $\sigma_\beta(\alpha) = \alpha - \beta \in \Phi$ がいえる.$\sspair{\beta}{\alpha} = 1$ ならば $\sigma_\alpha(\beta) = \beta - \alpha \in \Phi$ であり,$\sigma_{\beta-\alpha}(\beta-\alpha) = \alpha - \beta \in \Phi$ が従う.
	(2) は (1) において $\beta$ の代わりに $-\beta = \sigma_\beta(\beta) \in \Phi$ を用いて同じ議論をすれば良い.
\end{proof}

\begin{mydef}[label=def:a-sting]{$\alpha$-string through $\beta$}
	$(\mathbb{E},\, \Phi)$ を\hyperref[ax:root-system]{ルート系}とする.
	$\alpha \neq \pm \beta$ を充たす任意の $\alpha,\, \beta \in \Phi$ に対して,$\Phi$ の部分集合
	\begin{align}
		\bigl\{\, \beta + \lambda \alpha \in \mathbb{E} \bigm| \lambda \in \mathbb{Z} \,\bigr\} \cap \Phi
	\end{align}
	のことを\textbf{$\bm{\alpha}$-string through $\bm{\beta}$} と呼ぶ.
\end{mydef}

\begin{myprop}[label=prop:a-string-basic]{$\alpha$-string through $\beta$ の性質}
	$(\mathbb{E},\, \Phi)$ を\hyperref[ax:root-system]{ルート系}とする.
	$\alpha \neq \pm \beta$ を充たす任意の $\alpha,\, \beta \in \Phi$ に対して
	\begin{align}
		r &\coloneqq \max \bigl\{\, \lambda \in \mathbb{Z}_{\ge 0} \bigm| \beta - \lambda \alpha \in \Phi \,\bigr\}, \\
		q &\coloneqq \max \bigl\{\, \lambda \in \mathbb{Z}_{\ge 0} \bigm| \beta + \lambda \alpha \in \Phi \,\bigr\} 
	\end{align}
	とおく.このとき以下が成り立つ:
	\begin{enumerate}
		\item 
		\hyperref[def:a-sting]{$\alpha$-string through $\beta$}は $\mathbb{E}$ の部分集合
		\begin{align}
			\label{eq:a-string}
			\bigl\{\, \beta + \lambda\alpha \in \mathbb{E} \bigm| -r \le \lambda \le q \,\bigr\} 
		\end{align}
		に等しい.i.e. $-r \le \forall \lambda \le q$ に対して $\beta + \lambda\alpha \in \Phi$ である.
		\item \hyperref[def:a-sting]{$\alpha$-string through $\beta$}は鏡映 $\sigma_\alpha$ の作用の下で不変である.
		\item $r-q = \sspair{\beta}{\alpha}$.特に\hyperref[def:a-sting]{$\alpha$-string through $\beta$}の長さは $4$ 以下である.
	\end{enumerate}
\end{myprop}

\begin{proof}
	\begin{enumerate}
		\item 背理法により示す.ある $-r < \lambda < q$ に対して $\beta + \lambda\alpha \notin \Phi$ であるとする.
		このときある $-r \le s < p \le q$ が存在して $\beta + s\alpha \in \Phi,\, \beta + (s-1)\alpha \notin \Phi,\, \beta + (p-1)\alpha \notin \Phi,\, \beta + p\alpha \in \Phi$ を充たすが,補題\ref{lem:root-basic}の対偶よりこのとき $\rpair{\alpha}{\beta + s\alpha} \le 0 \le \rpair{\alpha}{\beta + p\alpha}$ が成り立つ.
		よって $(p-s) \rpair{\alpha}{\alpha} \ge 0$ である.
		然るに $p < s$ かつ $\rpair{\alpha}{\alpha} > 0$ なのでこれは矛盾である.
		\item $-r \le \forall \lambda \le q$ に対して
		\begin{align}
			\sigma_\alpha(\beta + \lambda\alpha) 
			&= \beta + \lambda\alpha - \sspair{\beta}{\alpha}\alpha - \lambda \sspair{\alpha}{\alpha}\alpha \\
			&= \beta - \bigl( \sspair{\beta}{\alpha} + \lambda \bigr) \alpha
		\end{align}
		が成り立つ.\textsf{\textbf{(Root-3)}} より最右辺は $\Phi$ の元であり,かつ \textsf{\textbf{(Root-4)}}より $\sspair{\beta}{\alpha} + \lambda \in \mathbb{Z}$ なので,(1) より $-r \le \sspair{\beta}{\alpha} + \lambda \le q$ だと分かった.
		\item $r,\, q$ の定義より,(2) の証明において
		\begin{align}
			\sigma_\alpha(\beta + q\alpha) = \beta - \bigl( \sspair{\beta}{\alpha} + q \bigr) \alpha = \beta - r \alpha
		\end{align}
		が成り立たねばならない.よって
		\begin{align}
			r-q = \sspair{\beta}{\alpha}
		\end{align}
		である.表\ref{tab:rootsystem}より $\abs{r-q} \le 3$ であるから, 集合\eqref{eq:a-string}の要素数は $4$ 以下である.
	\end{enumerate}
	
\end{proof}

\begin{mydef}[label=def:dual-root]{双対ルート系}
	\hyperref[ax:root-system]{ルート系} $(\mathbb{E},\, \Phi)$ に対して
	\begin{align}
		\bm{\Phi^{\vee}} \coloneqq \biggl\{\, \frac{2}{(\alpha,\, \alpha)} \alpha \in \mathbb{E} \biggm| \alpha \in \Phi \,\biggr\} 
	\end{align}
	とおき,組 $\bm{(\mathbb{E},\, \Phi^\vee)}$ のことを $(\mathbb{E},\, \Phi)$ の\textbf{双対ルート系} (dual root system) と呼ぶ.
	
	\tcblower
	
	$\alpha \in \Phi$ に対して
	\begin{align}
		\bm{\alpha^\vee} \coloneqq \frac{2}{(\alpha,\, \alpha)} \alpha \in \Phi^\vee
	\end{align}
	と書く.
\end{mydef}

双対ルート系が\hyperref[ax:root-system]{ルート系の公理}を充たすことを確認しよう:
\begin{description}
	\item[\textbf{(Root-1,2)}] $(\mathbb{E},\, \Phi)$ がルート系なので明らか.
	\item[\textbf{(Root-3)}] $\forall \alpha^\vee,\, \beta^\vee \in \Phi^\vee$ をとる.このとき
	\begin{align}
		\label{eq:dual-root3}
		\sigma_{\alpha^\vee} \left( \beta^\vee  \right) 
		&= \frac{2}{(\beta,\, \beta)} \beta - \sspair*{\frac{2}{(\beta,\, \beta)} \beta}{\frac{2}{(\alpha,\, \alpha)} \alpha } \frac{2}{(\alpha,\, \alpha)} \alpha \\
		&= \frac{2}{(\beta,\, \beta)} \beta - \frac{\frac{2}{(\beta,\, \beta)}}{\frac{2}{(\alpha,\, \alpha)}} \sspair*{\beta}{\alpha} \frac{2}{(\alpha,\, \alpha)} \alpha \\
		&= \frac{2}{(\beta,\, \beta)} \sigma_{\alpha}(\beta)
	\end{align}
	だが,
	\begin{align}
		\rpair[\big]{\sigma_\alpha(\beta)}{\sigma_\alpha(\beta)}
		&= \rpair{\beta}{\beta} - 2 \sspair{\beta}{\alpha} \rpair{\beta}{\alpha} + \sspair{\beta}{\alpha}^2 \rpair{\alpha}{\alpha} \\
		&= \rpair{\beta}{\beta} - 2 \sspair{\beta}{\alpha} \rpair{\beta}{\alpha} + 2\sspair{\beta}{\alpha}\rpair{\beta}{\alpha} \\
		&= \rpair{\beta}{\beta}
	\end{align}
	なので
	\begin{align}
		\sigma_{\alpha^\vee} \left( \beta^\vee  \right) = \sigma_\alpha(\beta)^\vee \in \Phi^\vee
	\end{align}
	だと分かった.
	\item[\textbf{(Root-4)}] $\forall \frac{2}{(\alpha,\, \alpha)} \alpha,\, \frac{2}{(\beta,\, \beta)} \beta \in \Phi^\vee$ をとる.このとき
	\begin{align}
		\sspair{\beta^\vee}{\alpha^\vee}
		&= \sspair*{\frac{2}{(\beta,\, \beta)} \beta}{\frac{2}{(\alpha,\, \alpha)} \alpha} \\
		&= \frac{\frac{2}{(\beta,\, \beta)}}{\frac{2}{(\alpha,\, \alpha)}} \sspair{\beta}{\alpha} \\
		&= \frac{\frac{2(\alpha,\, \beta)}{(\beta,\, \beta)}}{\frac{2(\beta,\, \alpha)}{(\alpha,\, \alpha)}} \sspair{\beta}{\alpha} \\
		&= \frac{\sspair{\alpha}{\beta}}{\sspair{\beta}{\alpha}} \sspair{\beta}{\alpha} \\
		&= \sspair{\alpha}{\beta} \in \mathbb{Z}
	\end{align}
	が言えた.
\end{description}

\subsection{Weyl群}

\begin{mydef}[label=def:Weylgroup]{Weyl群}
	$(\mathbb{E},\, \Phi)$ を\hyperref[ax:root-system]{ルート系}とする.
	$\LGL (\mathbb{E})$ の部分集合 $\bigl\{\, \sigma_\alpha \in \LGL(\mathbb{E}) \bigm| \alpha \in \Phi \,\bigr\}$ が生成する $\LGL(\mathbb{E})$ の部分群のことをルート系 $(\mathbb{E},\, \Phi)$ の\textbf{Weyl群} (Weyl group) と呼び,$\bm{\Weyl{\mathbb{E}}{\Phi}}$ と書く.
	
	% \tcblower
	
	% $\forall \sigma \in \mathscr{W}$ に対して,
	% \begin{align}
	% 	\bm{\len (\sigma)} \coloneqq \min \bigl\{\, t \in \mathbb{Z}_{\ge 0} \bigm| \sigma = \sigma_{\alpha_1} \circ \cdots\circ \sigma_{\alpha_t} \WHERE \alpha_i \in \Phi \,\bigr\} 
	% \end{align}
	% を $\sigma$ の\textbf{長さ} (length) と呼ぶ.ただし $\len (\mathrm{id}_{\mathbb{E}}) \coloneqq 0$ と定義する.
\end{mydef}

\begin{marker}
	本資料の以降では,文脈上考えているルート系が明らかな場合 $\Weyl{\mathbb{E}}{\Phi}$ を $\bm{\mathscr{W}}$ と略記する.
\end{marker}


\textsf{\textbf{(Root3)}}より,$\forall \tau \in \Weyl{\mathbb{E}}{\Phi}$ の $\Phi$ への制限は全単射である.その上 \textsf{\textbf{(Root-1)}} から $\Phi$ は有限集合でかつ $\mathbb{E} = \Span_{\mathbb{R}} \Phi$ が成り立つので,$\Weyl{\mathbb{E}}{\Phi}$ を $\Phi$ に作用する対称群 $\mathfrak{S}_{\abs{\Phi}}$ の部分群と同一視できる.特に $\Weyl{\mathbb{E}}{\Phi}$ は\underline{有限群}である.

\begin{mylem}[label=lem:Weylgroup]{}
	$(\mathbb{E},\, \Phi)$ を\hyperref[ax:root-system]{ルート系}とする.
	$\tau \in \LGL(\mathbb{E})$ が $\tau(\Phi) = \Phi$ を充たすならば,$\forall \alpha,\, \beta \in \Phi$ に対して以下が成り立つ:
	\begin{enumerate}
		\item $\tau \circ \sigma_\alpha \circ \tau^{-1} = \sigma_{\tau(\alpha)}$
		\item $\sspair{\beta}{\alpha} = \sspair{\tau(\beta)}{\tau(\alpha)}$
	\end{enumerate}
\end{mylem}

\begin{proof}
	$\forall \alpha,\, \beta \in \Phi$ をとる.\textsf{\textbf{(Root-3)}} より $\sigma_\alpha(\beta) \in \Phi$ なので,$\tau \circ \sigma_\alpha \circ \tau^{-1}\bigl(\tau(\beta)\bigr) = \sigma \circ \sigma_\alpha(\beta) \in \Phi$ が成り立つ.
	一方で
	\begin{align}
		\label{eq:lemWeylgroup1}
		\tau \circ \sigma_\alpha \circ \tau^{-1}\bigl(\tau(\beta)\bigr) = \tau \bigl( \beta - \sspair{\beta}{\alpha}\alpha \bigr) = \tau(\beta) - \sspair{\beta}{\alpha}\tau(\alpha)
	\end{align}
	である.$\beta \in \Phi$ は任意で $\tau \in \LGL(\mathbb{E})$ は全単射なので 
	\begin{description}
		\item[\textbf{(RF-1)}] $\tau \circ \sigma_\alpha \circ \tau^{-1} (\Phi) = \Phi$
		\item[\textbf{(RF-2)}] $\forall \beta \in P_\alpha,\; \tau \circ \sigma_\alpha \circ \tau^{-1}(\beta) = \beta$
		\item[\textbf{(RF-3)}] $\tau(\alpha) \in \Phi \setminus \{0\},\; \tau \circ \sigma_\alpha \circ \tau^{-1} \bigl( \tau(\alpha) \bigr) = -\tau(\alpha)$
	\end{description}
	が成り立つことが分かった.よって補題\ref{lem:refrect}より $\tau \circ \sigma_\alpha \circ \tau^{-1} = \sigma_{\tau(\alpha)}$ である.
	
	さらに\eqref{eq:lemWeylgroup1}から
	\begin{align}
		\tau(\beta) - \sspair{\beta}{\alpha}\tau(\alpha) = \sigma_{\tau(\alpha)} \bigl(\tau(\beta)\bigr) = \tau(\beta) - \sspair{\tau(\beta)}{\tau(\alpha)} \tau(\alpha)
	\end{align}
	が分かるので (2) が従う.
\end{proof}

\begin{mydef}[label=def:isom-root]{ルート系の同型}
	2つの\hyperref[ax:root-system]{ルート系} $(\mathbb{E},\, \Phi),\; (\mathbb{E}',\, \Phi')$ を与える.
	写像
	\begin{align}
		\phi \colon \mathbb{E} \lto \mathbb{E}'
	\end{align}
	が\textbf{ルート系の同型写像} (isomorphism) であるとは,以下の3条件を満たすことを言う:
	\begin{enumerate}
		\item $\phi$ は $\mathbb{R}$-ベクトル空間の同型写像
		\item $\phi(\Phi) = \Phi'$
		\item $\forall \alpha,\, \beta \in \Phi$ に対して $\sspair{\phi(\beta)}{\phi(\alpha)} = \sspair{\beta}{\alpha}$
	\end{enumerate}
	\tcblower
	\hyperref[ax:root-system]{ルート系} $(\mathbb{E},\, \Phi)$ の\textbf{自己同型} (automorphism) とは,$\phi \in \LGL(\mathbb{E})$ であって $\phi(\Phi) = \Phi$ を充たすもののことを言う.これは補題\ref{lem:Weylgroup}-(2) により自動的にルート系の同型となる.
	ルート系の自己同型全体が写像の合成に関してなす群のことを\textbf{ルート系の自己同型群}と呼び,$\bm{\Aut \Phi}$ と書く.
\end{mydef}

\hyperref[def:isom-root]{ルート系の同型}
\begin{align}
	\phi \colon (\mathbb{E},\, \Phi) \lto (\mathbb{E}',\, \Phi')
\end{align}
について,$\forall \alpha,\, \beta \in \Phi$ に対して
\begin{align}
	\sigma_{\phi(\alpha)} \circ \phi(\beta) = \phi(\beta) - \sspair{\phi(\beta)}{\phi(\alpha)} \phi(\alpha) = \phi\bigl(\beta - \sspair{\beta}{\alpha} \alpha\bigr) = \phi \circ \sigma_\alpha(\beta)
\end{align}
が成り立つ.i.e. ルート系の図式
\begin{center}
	\begin{tikzcd}[row sep=large, column sep=large]
		&\Phi \ar[d, "\sigma_\alpha"']\ar[r, "\phi"] &\Phi' \ar[d, "\sigma_{\phi(\alpha)}"]\\
		&\Phi \ar[r, "\phi"'] &\Phi'
	\end{tikzcd}
\end{center}
は可換である.よってルート系の同型は\hyperref[def:Weylgroup]{Weyl群}の自然な(群の)同型
\begin{align}
	\label{eq:induced-Weyl-isom}
	\overline{\phi} \colon \Weyl{\mathbb{E}}{\Phi} &\lto \Weyl{\mathbb{E}'}{\Phi'}, \\
	\sigma &\lmto \phi \circ \sigma \circ \phi^{-1}
\end{align}
を引き起こす.

\begin{mylem}[label=lem:Weylgroup-basic]{}
	\hyperref[ax:root-system]{ルート系} $(\mathbb{E},\, \Phi)$ の\hyperref[def:Weylgroup]{Weyl群} $\Weyl{\mathbb{E}}{\Phi}$ は,\hyperref[def:isom-root]{ルート系の自己同型群} $\Aut \Phi$ の正規部分群である.
\end{mylem}

\begin{proof}
	$\forall \sigma \in \Weyl{\mathbb{E}}{\Phi}$ を1つとる.このときある $\alpha_1,\, \dots,\, \alpha_k \in \Phi$ が存在して $\sigma = \sigma_{\alpha_1} \circ \cdots \circ \sigma_{\alpha_k}$ と書ける\footnote{$\sigma_{\alpha_i}^{-1} = \sigma_{\alpha_i}$ なのでこれで良い.}. 
	従って $\forall \tau \in \Aut \Phi$ に対して,補題\ref{lem:Weylgroup}-(1) より
	\begin{align}
		\tau \circ \sigma \circ \tau^{-1} = (\tau \circ \sigma_{\alpha_1} \circ \tau^{-1}) \circ \cdots \circ (\tau \circ \sigma_{\alpha_k} \circ \tau^{-1}) = \sigma_{\tau(\alpha_1)} \circ \cdots \circ \sigma_{\tau(\alpha_k)} \in \Weyl{\mathbb{E}}{\Phi}
	\end{align}
	が成り立つ.i.e. $\Weyl{\mathbb{E}}{\Phi} \triangleleft \Aut \Phi$ である.
\end{proof}

\begin{mylem}[label=lem:Weylgroup-dual]{双対ルートのWeyl群}
	\begin{align}
		\Weyl{\mathbb{E}}{\Phi} \cong \Weyl{\mathbb{E}}{\Phi^\vee}
	\end{align}
\end{mylem}

\begin{proof}
	\eqref{eq:dual-root3}より,写像 $\sigma_\alpha \lmto \sigma_{\alpha^\vee}$ は同型写像である.
\end{proof}


\section{単純ルートとWeyl群}

この節では $(\mathbb{E},\, \Phi)$ を任意の\hyperref[def:rank-root]{ランク $l$} の\hyperref[ax:root-system]{ルート系}とし,その\hyperref[def:Weylgroup]{Weyl群}を $\mathscr{W}$ と略記する. 

\subsection{ルート系の底とWeylの区画}

\begin{mydef}[label=def:base-root,breakable]{ルート系の底}
	$\Phi$ の部分集合 $\Delta \subset \Phi$ が\textbf{底} (base) であるとは,以下を充たすことをいう:
	\begin{description}
		\item[\textbf{(B-1)}] $\Delta$ は $\mathbb{R}$-ベクトル空間 $\mathbb{E}$ の基底である.
		\item[\textbf{(B-2)}] $\forall \beta \in \Phi$ に対して\underline{整数}の族 $\Familyset[\big]{\beta_\alpha}{\alpha \in \Delta} \in \prod_{\alpha \in \Delta} \textcolor{red}{\mathbb{Z}}$ が一意的に存在して
		\begin{align}
			\beta = \sum_{\alpha \in \Delta} \beta_\alpha \alpha
		\end{align}
		を充たし,$\forall \alpha \in \Delta,\; \beta_\alpha \ge 0$ であるか $\forall \alpha \in \Delta,\; \beta_\alpha \le 0$ であるかのどちらかである.
	\end{description}
	\tcblower
	\begin{itemize}
		\item $\Delta$ の元のことを\textbf{単純ルート} (simple root) と呼ぶ.
		\item $\beta = \sum_{\alpha \in \Delta} \beta_\alpha \alpha \in \Phi$ に対して
		\begin{align}
			\bm{\hight \beta} \coloneqq \sum_{\alpha \in \Delta} \beta_\alpha \in \mathbb{Z}
		\end{align}
		と定義し,底 $\Delta$ に関するルート $\beta$ の\textbf{高さ} (height) と呼ぶ. 
		\item $\beta = \sum_{\alpha \in \Delta} \beta_\alpha \alpha \in \Phi$ が\textbf{正}(resp. \textbf{負})であるとは,$\forall \alpha \in \Delta,\; \beta_\alpha \ge 0$ (resp. $\forall \alpha \in \Delta,\; \beta_\alpha \le 0$)が成り立つことを言い,$\bm{\beta \succ 0}$(resp. $\bm{\beta \prec 0}$)と書く\footnote{\LaTeX コマンドは $\succ$ が \texttt{\textbackslash succ} , $\prec$ が \texttt{\textbackslash prec} である.}.
		\item 正(resp. 負)のルート全体の集合のことを $\bm{\Phi^+}$ (resp. $\bm{\Phi^-}$)と書く\footnote{明らかに $\Phi^- = - \Phi^+$ である.}.
		\item \underline{$\mathbb{E}$ 上の}半順序 $\bm{\prec}\; \subset \mathbb{E} \times \mathbb{E}$を
		\begin{align}
			\mu \prec \lambda \DEF  \exists! \Familyset[\big]{k_\alpha}{\alpha \in \Delta} \in \prod_{\alpha \in \Delta} \mathbb{R}_{\textcolor{red}{\ge 0}},\; \lambda - \mu = \sum_{\alpha \in \Delta} k_\alpha \alpha
		\end{align}
		と定義する.
	\end{itemize}
\end{mydef}

$\prec$ が半順序になっていることを確認しておこう:
\begin{description}
	\item[\textbf{(反射律)}] $\forall \mu \in \mathbb{E}$ に対して $\mu - \mu = 0$ が成り立つので $\mu \prec \mu$ である.
	\item[\textbf{(反対称律)}] $\mu \prec \lambda \AND \lambda \prec \mu$ だとする.
	このとき $\Familyset[\big]{k_\alpha}{\alpha \in \Delta},\; \Familyset[\big]{l_\alpha}{\alpha \in \Delta} \in \prod_{\alpha \in \Delta} \mathbb{R}_{\ge 0}$ が一意的に存在して
	\begin{align}
		\lambda - \mu &= \sum_{\alpha \in \Delta} k_\alpha \alpha, & \mu - \lambda &= \sum_{\alpha \in \Delta} l_\alpha \alpha
	\end{align}
	と書ける.辺々足すと\hyperref[def:base-root]{\textsf{\textbf{(B-1)}}}より $\forall \alpha \in \Delta$ に対して $k_\alpha + l_\alpha = 0$ だと分かるが,$k_\alpha \ge 0 \AND l_\alpha \ge 0$ なので $k_\alpha = l_\alpha = 0$,i.e. $\mu - \lambda = 0 \IFF \mu  = \lambda$ が言えた.
	\item[\textbf{(推移律)}] $\mu \prec \lambda \AND \lambda \prec \nu$ だとする.このとき $\nu - \mu = (\nu - \lambda) + (\lambda - \mu)$ なので明らかに $\mu \prec \nu$ である.
\end{description}

ルート系 $(\mathbb{E},\, \Phi)$ の\hyperref[def:base-root]{底}を定義したのは良いが,存在しなくては意味がない.

\begin{mylem}[label=lem:base]{}
	$\Delta$ が $\Phi$ の\hyperref[def:base-root]{底}ならば,
	相異なる任意の2つの単純ルート $\alpha,\, \beta \in \Delta$ に対して $\rpair{\alpha}{\beta} \le 0$ であり,$\alpha - \beta \notin \Delta$ である.
\end{mylem}

\begin{proof}
	背理法により示す.$\rpair{\alpha}{\beta} > 0$ だとする.仮定より $\alpha \neq \beta$ であり,かつ $\Delta$ の元の線型独立性から $\beta \neq -\alpha$ なので,補題\ref{lem:root-basic}から $\alpha - \beta \in \Delta$ と言うことになる.
	然るにこのとき $\alpha - \beta \in \Phi$ が $\alpha,\, \beta \in \Delta$ の係数 $1,\, -1$ の線型結合で書けていることになり \textsf{\textbf{(B-2)}} に矛盾する.
\end{proof}

\begin{mydef}[label=def:decomposable]{}
	$\forall \gamma \in \mathbb{E}$ に対して以下を定義する:
	\begin{itemize}
		\item $\Phi$ の部分集合
		\begin{align}
			\bm{\Phi^+(\gamma)} \coloneqq \bigl\{\, \alpha \in \Phi \bigm| \rpair{\gamma}{\alpha} > 0 \,\bigr\} 
		\end{align}
		\item \begin{align}
			\gamma \in \mathbb{E} \setminus \bigcup_{\alpha \in \Phi} P_\alpha
		\end{align}
		のとき,$\gamma$ は\textbf{正則} (regular) であるという.$\gamma$ が正則でないとき\textbf{特異} (sigular) であると言う.
		\item $\alpha \in \Phi^+ (\gamma)$ が
		\begin{align}
			\exists \beta_1,\, \beta_2 \in \Phi^+(\gamma),\; \alpha = \beta_1 + \beta_2
		\end{align}
		を充たすとき,$\alpha$ は\textbf{分割可能} (decomposable) であると言う.分割可能でないとき\textbf{分割不可能} (indecomposable) であると言う.
	\end{itemize}
\end{mydef}

$\gamma$ が正則ならば $\forall \alpha \in \Phi$ に対して $\rpair{\gamma}{\alpha} \neq 0$ なので,\hyperref[ax:root-system]{\textsf{\textbf{(Root-2)}}}から $\Phi = \Phi^+(\gamma) \amalg \bigl( -\Phi^+(\gamma) \bigr)$ (disjoint union)が成り立つ.

\begin{mytheo}[label=thm:base-exist]{底の存在}
	\hyperref[def:decomposable]{正則}な任意の $\gamma \in \mathbb{E}$ を与える.このとき集合
	\begin{align}
		\Delta(\gamma) \coloneqq \bigl\{\, \alpha \in \Phi^+(\gamma) \bigm| \text{\hyperref[def:decomposable]{分割不可能}} \,\bigr\} 
	\end{align}
	は $\Phi$ の\hyperref[def:base-root]{底}である.
	逆に $\Phi$ の任意の底 $\Delta$ に対してある正則な $\gamma \in \mathbb{E}$ が存在して $\Delta = \Delta(\gamma)$ となる.
\end{mytheo}

\begin{proof}
	\begin{description}
		\item[\textbf{step1: $\bm{\Phi^+(\gamma)}$ の任意の元は $\bm{\Delta (\gamma)}$ の $\bm{\mathbb{Z}_{\ge 0}}$-係数線型結合で書ける}] 
		
		背理法により示す.$\Delta (\gamma)$ の $\mathbb{Z}_{\ge 0}$-係数線型結合で書けない $\alpha \in \Phi^+(\gamma)$ が存在するとする.
		このとき,そのような $\alpha$ のうち $\rpair{\gamma}{\alpha}$ が最小であるようなものが存在するのでそれを $\alpha_0$ とおく.$\alpha_0 \notin \Delta(\gamma)$ なので\footnote{$\alpha_0 \in \Delta(\gamma)$ だとすると,$\alpha_0 \in \Delta(\gamma)$ の係数 $1 \in \mathbb{Z}_{\ge 0}$ の線型結合として書けていることになり矛盾.}$\alpha_0$ は分割可能であり,ある $\beta_1,\, \beta_2 \in \Phi^+(\gamma)$ が存在して $\alpha = \beta_1 + \beta_2$ と書ける.このとき
		\begin{align}
			\rpair{\gamma}{\alpha_0} = \rpair{\gamma}{\beta_1} + \rpair{\gamma}{\beta_2} > \rpair{\gamma}{\beta_i}
		\end{align}
		が成り立つので,$\alpha_0$ の最小性から $\beta_1,\, \beta_2$ はどちらも $\Delta(\gamma)$ の元の $\mathbb{Z}_{\ge 0}$-係数線型結合で書ける.然るにこのとき $\alpha$ も $\Delta(\gamma)$ の元の $\mathbb{Z}_{\ge 0}$-係数線型結合で書けることになって矛盾.

		\item[\textbf{step2: $\bm{\alpha,\, \beta \in \Delta(\gamma)}$ かつ $\bm{\alpha \neq \beta}$ ならば,$\bm{(\alpha,\, \beta) \le 0}$}] 
		
		背理法により示す.$\rpair{\alpha}{\beta} > 0$ を仮定する.このとき補題\ref{lem:root-basic}-(1) より $\alpha - \beta \in \Phi$ であり,$\beta - \alpha = \sigma_{\alpha-\beta}(\alpha - \beta) \in \Phi$ もわかる.よって $\alpha - \beta \in \Phi^+(\gamma)$ または $\beta - \alpha \in \Phi^+(\gamma)$ である.
		然るに前者の場合 $\alpha = \beta + (\alpha - \beta)$ なので $\alpha \in \Delta(\gamma)$ が分割可能ということになって矛盾し,後者の場合は $\beta = \alpha + (\beta - \alpha)$ なので $\beta \in \Delta(\gamma)$ が分割可能ということになって矛盾である.

		\item[\textbf{step3: $\bm{\Delta(\gamma)}$ の元は互いに線型独立}] 
		
		$\Familyset[\big]{k_\alpha}{\alpha \in \Delta (\gamma)} \in \prod_{\alpha \in \Delta} \mathbb{R}$ に対して $\sum_{\alpha \in \Delta (\gamma)} k_\alpha \alpha = 0$ を仮定する.$\forall \alpha \in \Delta(\gamma)$ に対して
		\begin{align}
			% s_\alpha &\coloneqq
			% \begin{cases}
			% 	k_\alpha, & k_\alpha > 0 \\
			% 	0, & k_\alpha \le 0
			% \end{cases}
			% & t_\alpha &\coloneqq
			% \begin{cases}
			% 	-k_\alpha, & k_\alpha < 0 \\
			% 	0, & k_\alpha \ge 0
			% \end{cases}
			\Delta^+(\gamma) &\coloneqq \bigl\{\, \alpha \in \Delta(\gamma) \bigm| k_\alpha > 0 \,\bigr\}, &
			\Delta^-(\gamma) &\coloneqq \bigl\{\, \alpha \in \Delta(\gamma) \bigm| k_\alpha < 0 \,\bigr\}
		\end{align}
		とおくと $\Delta^+(\gamma) \cap \Delta^-(\gamma) = \emptyset$ で,仮定は
		\begin{align}
			\sum_{\alpha \in \Delta^+(\gamma)} k_\alpha \alpha = \sum_{\alpha \in \Delta^-(\gamma)} (-k_\alpha) \alpha
		\end{align}
		と同値である.
		$\varepsilon \coloneqq \sum_{\alpha \in \Delta^+ (\gamma)} k_\alpha \alpha$ とおくと,
		\textsf{\textbf{step2}}から
		\begin{align}
			0 \le \rpair{\varepsilon}{\varepsilon} = \sum_{\alpha \in \Delta^+(\gamma),\, \beta \in \Delta^-(\gamma)} k_\alpha (-k_\beta) \rpair{\alpha}{\beta} \le 0
		\end{align}
		が成り立つので $\varepsilon = 0$ だとわかる.よって
		\begin{align}
			0 &= \rpair{\gamma}{\varepsilon}
			= \sum_{\alpha \in \Delta^+(\gamma)} k_\alpha \rpair{\gamma}{\alpha}
			= \sum_{\alpha \in \Delta^-(\gamma)} (-k_\alpha) \rpair{\gamma}{\alpha}
		\end{align}
		であり,$\forall \alpha \in \Delta(\gamma),\; k_\alpha = 0$ が言えた.
		\item[\textbf{step4: $\bm{\Delta(\gamma)}$ は $\bm{\Phi}$ の底}] 
		
		$\Phi = \Phi^+ (\gamma) \amalg \bigl( -\Phi^+ (\gamma) \bigr)$ なので,
		\textsf{\textbf{step1}}と併せて\hyperref[def:base-root]{\textsf{\textbf{(B-2)}}}が,
		\textsf{\textbf{step3}}と併せて\hyperref[def:base-root]{\textsf{\textbf{(B-1)}}}が従う.

		\item[\textbf{step5: 任意の底 $\bm{\Delta}$ に対してある正則な $\bm{\gamma \in \mathbb{E}}$ が存在して $\bm{\Delta = \Delta(\gamma)}$ となる}] 
		
		$\Phi$ の\hyperref[def:base-root]{底} $\Delta$ が与えられたとき,$\forall \alpha \in \Delta$ に対して $\rpair{\gamma}{\alpha} > 0$ を充たす $\gamma \in \mathbb{E}$ をとる
		\footnote{
			このような $\gamma$ が存在することを示そう.\textsf{\textbf{(B-1)}} より $\Delta$ は $\mathbb{E}$ の基底だから,$\forall \alpha$ に対して $\gamma_\alpha \in \mathbb{E}$ を,$\mathbb{E}$ の部分ベクトル空間 $\Span_{\mathbb{K}} \bigl(\Delta \setminus \{\alpha\}\bigr)$ の $\rpair{\;}{\,}$ に関する\hyperref[def:radical-bilinear]{直交補空間} $\Bigl(\Span_{\mathbb{K}} \bigl(\Delta \setminus \{\alpha\}\bigr)\Bigr)^\perp$ への $\alpha$ の射影とする.$\Delta$ の元は全て互いに線型独立なので $\gamma_\alpha \neq 0$ である.このとき,$\Familyset[\big]{k_\alpha}{\alpha \in \Delta} \in \prod_{\alpha \in \Delta} \mathbb{R}_{\textcolor{red}{> 0}} $ に対して $\gamma \coloneqq \sum_{\alpha \in \Delta} k_\alpha \gamma_\alpha$ とおけば,$\forall \alpha \in \Delta$ に対して $\rpair{\gamma}{\alpha} = k_\alpha \rpair{\gamma_\alpha}{\alpha} > 0$ が成り立つ.}.
		\textsf{\textbf{(B-2)}} より $\gamma$ は\hyperref[def:decomposable]{正則}であり,かつ $\forall \beta = \sum_{\alpha \in \Delta} \beta_\alpha \alpha \in \Phi^+$ に対して
		\begin{align}
			\rpair{\gamma}{\beta} = \sum_{\alpha \in \Delta} \beta_\alpha \rpair{\gamma}{\alpha} > 0
		\end{align}
		が成り立つので $\beta \in \Phi^+(\gamma)$,i.e. $\Phi^+ \subset \Phi^+(\gamma),\; \Phi^-  = - \Phi^- \subset -\Phi^+(\gamma)$ も分かる.
		ところが \textsf{\textbf{step4}} より $\Phi = \Phi^+ \amalg \Phi^- = \Phi^+(\gamma) \amalg \bigl( -\Phi^+(\gamma) \bigr)$ なので,$\Phi^+ = \Phi^+(\gamma)$ でなくてはいけない.
		従って $\forall \alpha \in \Delta$ は\hyperref[def:decomposable]{分割不可能}であり
		\footnote{$\beta_1 = \sum_{\alpha \in \Delta} \beta_1{}_\alpha \alpha,\, \beta_2 = \sum_{\alpha \in \Delta} \beta_2{}_\alpha \alpha \in \Phi^+ = \Phi^+(\gamma)$ を用いて $\alpha = \beta_1 + \beta_2$ と書けたとする.このとき $\Delta$ の元の線型独立性から $\beta_1{}_\alpha + \beta_2{}_\alpha = 1 \AND \forall \gamma \in \Delta \setminus \{\alpha\},\; \beta_1{}_\gamma + \beta_2{}_\gamma = 0$ が成り立つが,\textsf{\textbf{(B-2)}}より $(\beta_1{}_\alpha,\, \beta_2{}_\alpha) = (1,\, 0) \OR (0,\, 1) \AND \forall \gamma \in \Delta \setminus \{\alpha\},\; \beta_1{}_\gamma = \beta_2{}_\gamma = 0$,i.e. $(\beta_1,\, \beta_2) = (\alpha,\, 0)\OR (0,\, \alpha)$ でなくてはならず,$0 \notin \Phi^+$ に矛盾.}
		,$\Delta \subset \Delta(\gamma)$ だと分かった.
		\textsf{\textbf{(B-1)}}および\textsf{\textbf{step4}}より $\abs{\Delta} = \abs{\Delta(\gamma)} = l$ なので\footnote{~\cite{Humphreys1972introduction}では集合の\textbf{濃度} (cardinality) の意味で $\Card \Delta$ と書かれていた.} $\Delta = \Delta(\gamma)$ が言えた.
	\end{description}
\end{proof}

\begin{mydef}[label=def:Weylchamber]{Weylの区画}
	\begin{itemize}
		\item 位相空間\footnote{\hyperref[def:Euclid-space]{Euclid空間の定義}の脚注を参照.} $\mathbb{E}$ の部分空間 $\mathbb{E} \setminus \bigcup_{\alpha \in \Phi} P_\alpha$ の連結成分の1つのことを(開な)\textbf{Weylの区画} (Weyl chamber) \footnote{この訳語は筆者が勝手につけたものである.~\cite{Satake1987LieAlg}では\textbf{Weylの部屋}と呼ばれていた.}と呼ぶ.
		\item \hyperref[def:decomposable]{正則}な $\gamma \in \mathbb{E}$ が属するWeylの区画のことを $\bm{\mathfrak{C}(\gamma)}$ と書く\footnote{\LaTeX コマンドは \texttt{\textbackslash mathfrak\{C\}}}.
		\item $\Phi$ の\hyperref[def:base-root]{底} $\Delta$ に対して定理\ref{thm:base-exist}の意味で $\Delta = \Delta(\gamma)$ ならば $\bm{\mathfrak{C}(\Delta)} \coloneqq \mathfrak{C}(\gamma)$ とおき,\textbf{$\bm{\Delta}$ に関するWeylの基本区画} (fundamental Weyl chamber relative to $\Delta$)\footnote{これの訳語は全く普及していない気がする.\textbf{$\bm{\Delta}$ に関する基本的Weylの部屋}だと語感が悪いと思ったのでこのような訳語を充てた.} と呼ぶ.
	\end{itemize}
\end{mydef}

\begin{mylem}[label=lem:Weylchamber-basic]{Weylの区画の基本性質}
	\hyperref[def:decomposable]{正則}な任意の $\gamma,\, \gamma' \in \mathbb{E}$ および任意の $\Phi$ の\hyperref[def:base-root]{底} $\Delta$ を与える.定理\ref{thm:base-exist}によって得られる $\Phi$ の\hyperref[def:base-root]{底}を $\Delta(\gamma)$ と書く.
	\begin{enumerate}
		\item $\mathfrak{C}(\gamma) = \mathfrak{C}(\gamma') \IFF \Delta(\gamma) = \Delta(\gamma')$
		\item 写像
		\begin{align}
			\{\,\Phi\; \text{の\hyperref[def:base-root]{底}全体の集合}\, \} &\lto \{\, \text{\hyperref[def:Weylchamber]{Weylの区画}全体の集合}\, \}, \\
			\Delta &\lmto \mathfrak{C}(\Delta)
		\end{align}
		は全単射である.
		\item $\mathfrak{C}(\Delta) = \bigl\{\, \beta \in \mathbb{E} \bigm| \forall \alpha \in \Delta,\; \rpair{\beta}{\alpha} > 0 \,\bigr\}$
	\end{enumerate}
\end{mylem}

\begin{proof}
	\begin{enumerate}
		\item $\alpha \in \Phi$ に関する鏡映面 $P_\alpha$ に関して
		\begin{align}
			P_\alpha^+ &\coloneqq \bigl\{\, \beta \in \mathbb{E} \bigm| \rpair{\beta}{\alpha} > 0 \,\bigr\}, &P_\alpha^- &\coloneqq \bigl\{\, \beta \in \mathbb{E} \bigm| \rpair{\beta}{\alpha} < 0 \,\bigr\} 
		\end{align}
		とおく.すると $\mathbb{E} \setminus P_\alpha = P_\alpha^+ \cup P_\alpha^-$ (disjoint union)となるから,
		\begin{align}
			\mathbb{E} \setminus \bigcup_{\alpha \in \Phi} P_\alpha
			&= \bigcap_{\alpha \in \Phi} (\mathbb{E} \setminus P_\alpha) \\
			&= \bigcap_{\alpha \in \Phi} (P_\alpha^+ \cup P_\alpha^-) \\
			&= \bigcup_{\Familyset{\mu_\alpha}{\alpha \in \Phi} \in \prod_{\alpha \in \Phi}\{\pm\}} \bigcap_{\alpha \in \Phi} P_\alpha^{\mu_\alpha}
		\end{align}
		である.最右辺の $\bigcap_{\alpha \in \Phi} P_\alpha^{\mu_\alpha}$ は凸集合の共通部分なので凸集合であり,従って連結である.さらに $\Familyset{\mu_\alpha}{\alpha \in \Phi} \in \prod_{\alpha \in \Phi}\{\pm\}$ に関する重複を除いて位相空間の意味でdisjointであるからWeylの区画である
		\footnote{より厳密には,$\mathbb{E} \setminus \bigcup_{\alpha \in \Phi} P_\alpha$ の部分空間として開かつ閉 (clopen) であり,かつそれ自身連結なので,連結成分の1つだと分かる.}
		.
		従って $\mathfrak{C}(\gamma)$ は,$\forall \alpha \in \Phi$ に対して
		\begin{align}
			\mu_\alpha(\gamma) \coloneqq 
			\begin{cases}
				+, & \rpair{\gamma}{\alpha} > 0, \\
				-, & \rpair{\gamma}{\alpha} < 0
			\end{cases}
		\end{align}
		とおけば
		\begin{align}
			\label{eq:Weylchamber}
			\mathfrak{C}(\gamma) = \bigcap_{\alpha \in \Phi} P_\alpha^{\mu_\alpha(\gamma)} = \bigl\{\, \beta \in \mathbb{E} \bigm| \forall \alpha \in \Phi,\; \rpair{\gamma}{\alpha}\rpair{\beta}{\alpha} > 0 \,\bigr\} 
		\end{align}
		と書ける.よって
		\begin{align}
			\mathfrak{C}(\gamma) = \mathfrak{C}(\gamma') 
			&\IFF \forall \alpha \in \Phi,\; \mu_\alpha(\gamma) = \mu_\alpha(\gamma') \\
			&\IFF \Phi^+(\gamma) = \Phi^+(\gamma') \\
			&\IFF \Delta(\gamma) = \Delta(\gamma')
		\end{align}
		が言える.
		\item (1),定理\ref{thm:base-exist},および\hyperref[def:Weylchamber]{Weylの基本区画の定義}から従う.
		\item $\Delta = \Delta(\gamma)$ を充たす正則な $\gamma \in \mathbb{E}$ を1つとる.すると\eqref{eq:Weylchamber}より
		\begin{align}
			\mathfrak{C}(\Delta) 
			&= \bigl\{\, \beta \in \mathbb{E} \bigm| \forall \alpha \in \Phi,\; \rpair{\gamma}{\alpha}\rpair{\beta}{\alpha} > 0 \,\bigr\}
		\end{align}
		だが,$\Phi = \Phi^+(\gamma) \amalg \bigl(-\Phi^+(\gamma)\bigr)$ でかつ\hyperref[def:decomposable]{$\Phi^+(\gamma)$ の定義}より $\forall \alpha \in \Phi^+(\gamma)$ に対して $\rpair{\gamma}{\alpha} > 0$ が成り立つので
		\begin{align}
			\mathfrak{C}(\Delta) 
			&= \bigl\{\, \beta \in \mathbb{E} \bigm| \forall \alpha \in \Phi^+(\gamma),\; \rpair{\beta}{\alpha} > 0 \,\bigr\} \\
			&= \bigl\{\, \beta \in \mathbb{E} \bigm| \forall \alpha \in \Delta(\gamma),\; \rpair{\beta}{\alpha} > 0 \,\bigr\}
		\end{align}
		だと分かった.
	\end{enumerate}
	
\end{proof}


\begin{mylem}[label=lem:Weylchamber-Weylgroup]{Weylの区画とWeyl群の関係}
	\hyperref[def:decomposable]{正則}な任意の $\gamma \in \mathbb{E}$ および任意の $\Phi$ の\hyperref[def:base-root]{底} $\Delta$ を与える.このとき$\forall \sigma \in \mathscr{W}$ に対して以下が成り立つ:
	\begin{enumerate}
		\item $\sigma\bigl(\Delta(\gamma)\bigr) = \Delta \bigl( \sigma(\gamma) \bigr)$
		\item $\sigma(\Delta)$ もまた $\Phi$ の\hyperref[def:base-root]{底}である.
		\item $\sigma \bigl( \mathfrak{C}(\gamma) \bigr) = \mathfrak{C} \bigl( \sigma(\gamma) \bigr)$
		\item $\sigma \bigl( \mathfrak{C}(\Delta) \bigr) = \mathfrak{C} \bigl( \sigma(\Delta) \bigr)$
	\end{enumerate}
\end{mylem}

\begin{proof}
	鏡映は等長変換(isometry)なので $\rpair[\big]{\sigma(\alpha)}{\sigma(\beta)} = \rpair{\alpha}{\beta}$ が成り立つ.
	\begin{enumerate}
		\item 
		\begin{align}
			\sigma\bigl(\Phi^+ (\gamma)\bigr) 
			&= \bigl\{\, \sigma(\alpha) \in \Phi \bigm| \rpair{\gamma}{\gamma} = \rpair[\big]{\sigma(\gamma)}{\sigma(\alpha)} > 0 \,\bigr\} \\
			&= \bigl\{\, \beta \in \Phi \bigm| \rpair[\big]{\sigma(\gamma)}{\beta} > 0 \,\bigr\} \\
			&=\Phi^+ \bigl( \sigma(\gamma) \bigr) 
		\end{align}
		が分かる.従って $\sigma\bigl(\Delta (\gamma)\bigr) = \sigma\bigl(\Delta (\gamma)\bigr) $ である.
		\item 定理\ref{thm:base-exist}より,ある正則な $\gamma \in \mathbb{E}$ が存在して $\Delta = \Delta(\gamma)$ と書ける.よって (1) から $\sigma(\Delta) = \Delta \bigl( \sigma(\gamma) \bigr)$ であるが,再度定理\ref{thm:base-exist}より右辺は $\Phi$ の底である.
		\item \eqref{eq:Weylchamber},および $\sigma (\Phi) = \Phi$ より
		\begin{align}
			\sigma \bigl( \mathfrak{C}(\gamma) \bigr) 
			&= \bigl\{\, \sigma(\beta) \in \mathbb{E} \bigm| \forall \alpha \in \Phi,\; \rpair{\gamma}{\alpha}\rpair{\beta}{\alpha} = \rpair[\big]{\sigma(\gamma)}{\sigma(\alpha)} \rpair[\big]{\sigma(\beta)}{\sigma(\alpha)} > 0 \,\bigr\} \\
			&= \bigl\{\, \beta \in \mathbb{E} \bigm| \forall \alpha \in \Phi,\; \rpair[\big]{\sigma(\gamma)}{\alpha} \rpair[\big]{\beta}{\alpha} > 0 \,\bigr\} \\
			&= \mathfrak{C} \bigl( \sigma(\gamma) \bigr) 
		\end{align}
		\item (4) および定理\ref{thm:base-exist}より従う.
	\end{enumerate}
	
\end{proof}

補題\ref{lem:Weylchamber-Weylgroup}より,写像\footnote{$\Isom (X)$ は距離空間 $X$ 上の全単射な等長変換全体が写像の合成に関して成す群(\textbf{等長変換群}; isometry group)を意味する.}
\begin{align}
	\irm{\pi}{W} \colon \mathscr{W} &\lto \Isom (\{\text{Weylの区画}\}),\; \sigma \lmto \Bigl(\, \mathfrak{C}(\gamma) \mapsto \mathfrak{C}\bigl(\sigma(\gamma)\bigr)\, \Bigr) \label{def:Weylgroup-action-Weylchamber} \\
	\irm{\pi}{b} \colon \mathscr{W} &\lto \Isom (\{\Phi\;\text{の底}\}),\; \sigma \lmto \Bigl(\, \Delta \mapsto \sigma(\Delta)\, \Bigr) \label{def:Weylgroup-action-base}
\end{align}
が群準同型であることが分かった.i.e. \hyperref[def:Weylgroup]{Weyl群} $\mathscr{W}$ は\hyperref[def:Weylchamber]{Weylの区画}全体の集合,および $\Phi$ の\hyperref[def:base-root]{底}全体の集合に群として左作用する.

\subsection{単純ルートに関する補題}

この小節では $\Phi$ の任意の\hyperref[def:base-root]{底} $\Delta$ を1つ固定する.

\begin{mylem}[label=lem:simpleroot-A]{}
	$\alpha \in \Phi \setminus \Delta$ が\hyperref[def:base-root]{正ルート}ならば,ある $\beta \in \Delta$ が存在して $\alpha - \beta \in \Phi$ となる.
	特に,$\alpha - \beta \succ 0$ である.
\end{mylem}

\begin{proof}
	$\forall \beta \in \Delta$ に対して $\rpair{\alpha}{\beta} \le 0$ が成り立つと仮定すると,定理\ref{thm:base-exist}の \textsf{\textbf{step3}} の証明から $\Delta \cup \{\alpha\}$ が線型独立となり,$\Delta$ が $\mathbb{E}$ の基底であることに矛盾.
	よってある $\beta \in \Delta$ が存在して $\rpair{\alpha}{\beta} > 0$ となる.$\alpha \notin \Delta$ なので $\alpha \neq \pm \beta$ であることと併せると,補題\ref{lem:root-basic}が使えて $\alpha - \beta \in \Phi$ だと分かる.

	仮定より $\alpha \succ 0$ なので $\Familyset[\big]{\alpha_\gamma}{\gamma \in \Delta} \in \prod_{\gamma \in \Delta} \mathbb{Z}_{\ge 0}$ が一意的に存在して $\alpha = \sum_{\gamma \in \Delta} \alpha_\gamma \gamma$ と書ける.
	$\alpha \notin \Delta$ なのである $\gamma \in \Delta \setminus \{\beta\}$ が存在して $\alpha_\gamma > 0$ を充たす.よって $\alpha - \beta \in \Phi$ を単純ルートで展開した $\mathbb{Z}$-係数のうち少なくとも1つは正である.よって\hyperref[def:base-root]{(B-2)}から $\alpha - \beta \succ 0$ が言える.
\end{proof}


\begin{mycol}[label=col:simpleroot-A]{}
	$\forall \beta \in \Phi^+$ に対して $\alpha_1,\, \dots,\, \alpha_k \in \Delta$ が存在して(重複も許す),
	\begin{align}
		\beta = \sum_{j=1}^k \alpha_j
	\end{align}
	かつ $1 \le \forall i \le k$ に対して 
	\begin{align}
		\sum_{j=1}^i \alpha_i \in \Phi
	\end{align}
	を充たす.
\end{mycol}

\begin{proof}
	仮定より $\Familyset[\big]{\beta_\gamma}{\gamma \in \Delta} \in \prod_{\gamma \in \Delta} \mathbb{Z}_{\ge 0}$ が一意的に存在して $\beta = \sum_{\gamma \in \Delta} \beta_\gamma \gamma$ と書ける.
	$\hight \beta$ に関する数学的帰納法により示す.$\hight \beta = 1$ のときは $\beta \in \Delta \subset \Phi$ なので明らか.

	$\hight \beta > 1$ とする.このとき $\beta \notin \Delta$ である.
	% もしこのような $\gamma \in \Delta$ がただ1つならば $\beta = \hight \beta \gamma \in \Phi$ と言うことになり\hyperref[ax:root-system]{\textsf{\textbf{(Root-2)}}}に矛盾.よって $\beta \notin \Delta$ である.
	従って補題\ref{lem:simpleroot-A}が使えて $\beta - \gamma \in \Phi^+$ となる.よって帰納法の仮定から示された.
\end{proof}

\begin{mylem}[label=lem:simpleroot-B]{}
	任意の\hyperref[def:base-root]{単純ルート} $\alpha \in \Delta$ に対して,
	$\sigma_\alpha \in \LGL(\mathbb{E})$ は $\sigma_\alpha\bigl(\Phi^+ \setminus \{\alpha\}\bigr) = \Phi^+ \setminus \{\alpha\}$ を充たす.
\end{mylem}

\begin{proof}
	$\forall \beta \in \Phi^+ \setminus \{\alpha\}$ を1つとる.このとき  $\Familyset[\big]{\beta_\gamma}{\gamma \in \Delta} \in \prod_{\gamma \in \Delta} \mathbb{Z}_{\ge 0}$ が一意的に存在して $\beta = \sum_{\gamma \in \Delta} \beta_\gamma \gamma$ と書ける.$\beta \neq \pm \alpha$ だから,ある $\gamma \in \Delta \setminus \{\alpha\}$ が存在して $\beta_\gamma > 0$ となる.このとき $\sigma_\alpha(\beta) = \beta - \sspair{\beta}{\alpha} \alpha$ の $\gamma$ の係数は $k_\gamma > 0$ のままである.
	よって\hyperref[ax:root-system]{\textsf{\textbf{(B-2)}}}から $\sigma_\alpha(\beta) \in \Phi^+$ が言える.$\alpha = \sigma_\alpha (-\alpha) \neq \sigma_\alpha(\beta)$ なので $\sigma_\alpha(\beta) \in \Phi^+ \setminus \{\alpha\}$ が言えた.
\end{proof}

\begin{mycol}[label=col:simpleroot-B]{}
	\begin{align}
		\delta \coloneqq \frac{1}{2} \sum_{\beta \in \Phi^+} \beta
	\end{align}
	とおくと,$\forall \alpha \in \Delta$ に対して
	\begin{align}
		\sigma_\alpha(\delta) = \delta -\alpha
	\end{align}
	が成り立つ.
\end{mycol}

\begin{proof}
	補題\ref{lem:simpleroot-B}より
	\begin{align}
		\sigma_\alpha(\delta) = \sigma_\alpha \left( \sum_{\beta \in \Phi^+ \setminus \{\alpha\}} \beta\right) + \sigma_\alpha(\alpha) = \delta - \alpha
	\end{align}
	
\end{proof}

\begin{mylem}[label=lem:simpleroot-C]{}
	重複を許して $\alpha_1,\, \dots,\, \alpha_t \in \Delta$ を任意にとる.

	このとき,もし $\sigma_{\alpha_1} \circ \cdots \circ \sigma_{\alpha_{t-1}}(\alpha_t) \prec 0$ ならば,ある $1 \le s < t$ が存在して
	\begin{align}
		\sigma_{\alpha_1} \circ \cdots \circ \sigma_{\alpha_{t}} = \sigma_{\alpha_1} \circ \cdots \circ \sigma_{\alpha_{s-1}} \circ \sigma_{\alpha_{s+1}} \circ \cdots \circ \sigma_{\alpha_{t-1}}
	\end{align}
	が成り立つ.
\end{mylem}

\begin{proof}
	$\circ$ を省略し,$\sigma_i \coloneqq \sigma_{\alpha_i}$ と略記する.
	\begin{align}
		\beta_i \coloneqq 
		\begin{cases}
			\sigma_{i+1} \cdots \sigma_{t-1}(\alpha_t), &0 \le i \le t-2 \\
			\alpha_t, &i = t-1 
		\end{cases}
	\end{align}
	とおく.$\beta_0 \prec 0$ かつ $\beta_{t-1} \succ 0$ なので,
	\begin{align}
		s \coloneqq \min \bigl\{\, i \in \mathbb{N} \bigm| \beta_i \succ 0  \,\bigr\} 
	\end{align}
	が存在する.このとき $\beta_s \in \Phi^+ \AND \sigma_s(\beta_s) = \beta_{s-1} \prec 0$ なので補題\ref{lem:simpleroot-B}より $\beta_s = \alpha_s$ でなくてはならない.
	$\sigma_{s+1} \cdots \sigma_{t-1} \in \mathscr{W}$ なので,このとき補題\ref{lem:Weylgroup-basic}より
	\begin{align}
		\sigma_s = \sigma_{\beta_s} = (\sigma_{s+1} \cdots \sigma_{t-1}) \sigma_t (\sigma_{t-1} \cdots \sigma_{s+1})
	\end{align}
	だと分かる.よって
	\begin{align}
		\sigma_{1} \cdots \sigma_{s-1} \sigma_s \sigma_{s+1} \cdots \sigma_{t}
		&= \sigma_{1} \cdots \sigma_{s-1} (\sigma_{s+1} \cdots \sigma_{t-1}) \sigma_t (\sigma_{t-1} \cdots \sigma_{s+1}) \sigma_{s+1} \cdots \sigma_{t} \\
		&= \sigma_{1} \cdots \sigma_{s-1} \sigma_{s+1} \cdots \sigma_t
	\end{align}
\end{proof}

\begin{mycol}[label=col:simpleroot-C]{}
	$\sigma \in \mathscr{W}$ が,$\alpha_1,\, \dots,\, \alpha_{t} \in \Delta$ を用いて $\sigma = \sigma_{\alpha_1} \circ \cdots \circ \sigma_{\alpha_t}$ と表示されているとする.
	このような $\sigma$ の表示のうち $t$ が最小のものに対しては $\sigma(\alpha_{t}) \prec 0$ である.
\end{mycol}

\begin{proof}
	\begin{align}
		\sigma(\alpha_{t}) = \sigma_{\alpha_1} \circ \cdots \circ \sigma_{\alpha_{t - 1}} (-\alpha_{t})
	\end{align}
	なので,もし $\sigma(\alpha_{t}) \succ 0$ ならば $\sigma_{\alpha_1} \circ \cdots \circ \sigma_{\alpha_{t - 1}}(\alpha_{t}) \prec 0$ となる.
	然るにこのとき補題\ref{lem:simpleroot-C}からある $1 \le s < \len (\sigma)$ が存在して
	\begin{align}
		\sigma_{\alpha_1} \circ \cdots \circ \sigma_{\alpha_{t}} = \sigma_{\alpha_1} \circ \cdots \circ \sigma_{\alpha_{s-1}} \circ \sigma_{\alpha_{s+1}} \circ \cdots \circ \sigma_{\alpha_{t -1}}
	\end{align}
	が成り立ち,$t$ の最小性に矛盾する.よって背理法から $\sigma(\alpha_{t}) \prec 0$ である.
\end{proof}

\subsection{Weyl群の性質}

準備が整ったので,\hyperref[def:Weylgroup]{Weyl群}の極めて重要な性質を示す.

\begin{mytheo}[label=thm:Weylgroup-basic]{Weyl群の性質}
	$\Phi$ の\hyperref[def:base-root]{底} $\Delta$ を任意に与える.
	\begin{enumerate}
		\item $\gamma \in \mathbb{E}$ が\hyperref[def:decomposable]{正則}ならば,
		ある $\sigma \in \mathscr{W}$ が存在して,$\forall \alpha \in \Delta,\; \rpair[\big]{\sigma(\gamma)}{\alpha} > 0$ を充たす.
		\item $\Delta'$ が $\Phi$ のもう1つの底ならば,ある $\sigma \in \mathscr{W}$ が存在して $\sigma (\Delta') = \Delta$ を充たす.
		\item $\forall \alpha \in \Phi$ に対してある $\sigma \in \mathscr{W}$ が存在して $\sigma(\alpha) \in \Delta$ を充たす.
		\item $\mathscr{W}$ は集合 $\bigl\{\, \sigma_\alpha \bigm| \alpha \in \Delta \,\bigr\} \subset \bigl\{\, \sigma_\alpha \bigm| \alpha \in \Phi \,\bigr\} $ によって生成される.
		\item $\sigma \in \mathscr{W}$ が $\sigma (\Delta) = \Delta$ を充たすならば $\sigma = \mathrm{id}_{\mathbb{E}}$ である.
	\end{enumerate}
	
\end{mytheo}

\begin{proof}
	集合 $\bigl\{\, \sigma_\alpha \bigm| \alpha \in \Delta \,\bigr\}$ によって生成される $\mathscr{W}$ の部分群を $\mathscr{W}'$ とする.まず (1)-(3) を $\mathscr{W}'$ において示してから $\mathscr{W}' = \mathscr{W}$ を示す.
	\begin{enumerate}
		\item 正則な $\gamma \in \mathbb{E}$ を任意にとる.
		$\delta \coloneqq \frac{1}{2} \sum_{\beta \in \Phi^+} \beta$ とおき,
		$\sigma \in \argmax_{\sigma \in \mathscr{W}'} \rpair[\big]{\sigma(\gamma)}{\delta}$
		を1つとる\footnote{$\mathscr{W}' \subset \mathscr{W}$ は有限群なのでこのような $\sigma$ は少なくとも1つ存在する.}.
		ここで $\forall \alpha \in \Delta$ を1つとると, 
		$\sigma_\alpha \circ \sigma \in \mathscr{W}$ であるから系\ref{col:simpleroot-B}より
		\begin{align}
			\rpair[\big]{\sigma(\gamma)}{\delta} \ge \rpair[\big]{\sigma_\alpha\circ \sigma(\gamma)}{\delta} = \rpair[\big]{\sigma(\gamma)}{\sigma_\alpha(\delta)} = \rpair[\big]{\sigma(\gamma)}{\delta - \alpha} = \rpair[\big]{\sigma(\gamma)}{\delta} - \rpair[\big]{\sigma(\gamma)}{\alpha}
		\end{align}
		が分かる.よって $\rpair[\big]{\sigma(\gamma)}{\alpha} \ge 0$ が分かった.$\gamma$ は\hyperref[def:decomposable]{正則}なので $\rpair[\big]{\sigma(\gamma)}{\alpha} = \rpair[\big]{\gamma}{\sigma^{-1}(\alpha)} > 0$ が言えた.
		\item 定理\ref{thm:base-exist}より,$\Delta' = \Delta(\gamma')$ を充たす正則な $\gamma' \in \mathbb{E}$ が存在する.
		このとき (1) および補題\ref{lem:Weylchamber-basic}-(3) よりある $\sigma \in \mathscr{W}'$ が存在して $\sigma (\gamma') \in \mathfrak{C}(\Delta)$ が成り立つから,再度補題\ref{lem:Weylchamber-Weylgroup}より $\sigma \bigl( \mathfrak{C}(\Delta') \bigr) = \sigma \bigl(\mathfrak{C}(\gamma')\bigr) = \mathfrak{C} \bigl( \sigma(\gamma') \bigr) = \mathfrak{C}(\Delta)$ が分かる.
		従って補題\ref{lem:Weylchamber-basic}-(2) より $\sigma(\Delta') = \Delta$ である.
		\item $\forall \alpha \in \Phi$ を1つとる.(2) より,$\alpha$ が少なくとも1つの($\Delta$ とは限らない) $\Phi$ の\hyperref[def:base-root]{底}に属することを言えば良い.
		
		 \hyperref[ax:root-system]{\textsf{\textbf{(Root-2)}}}より $\forall \beta \in \Phi \setminus \{\pm \alpha\}$ に対して $\beta \notin \mathbb{K}\alpha$ であるから $\mathbb{K}\alpha \cap \mathbb{K}\beta = 0$ である.
		よって $P_\alpha \cap P_\beta = (\mathbb{K}\alpha)^\perp \cap (\mathbb{K}\beta)^\perp = (\mathbb{K}\alpha \oplus \mathbb{K}\beta)^\perp$ となる\footnote{${}^\perp$ は\hyperref[def:Euclid-space]{Euclid空間}に備わっている双線型形式 $\rpair{\;}{\,}$ に関する\hyperref[def:radical-bilinear]{直交補空間}の意味である.}が,$\mathbb{K}\alpha \subsetneq \mathbb{K} \alpha \oplus \mathbb{K} \beta$ なので $P_\alpha \cap P_\beta = (\mathbb{K}\alpha \oplus \mathbb{K}\beta)^\perp \subsetneq (\mathbb{K}\alpha)^\perp = P_\alpha$ が成り立ち, 
		$\exists \gamma \in P_\alpha \setminus (P_\alpha \cap P_\beta)$ が言える.
		ここで $\gamma$ にEuclid距離の意味で十分近い\hyperref[def:decomposable]{正則}な $\gamma' \in \mathbb{E}$ を取れば,ある $\varepsilon > 0$ に対して $\rpair{\gamma'}{\alpha} = \varepsilon \AND \abs{\rpair{\gamma'}{\beta}} > \varepsilon$ を充たすようにできる.
		すると $\alpha \in \Phi^+(\gamma)$ で,かつ $\beta \in \Phi \setminus \{\pm \alpha\}$ は任意だったので $\alpha$ は\hyperref[def:decomposable]{分割不可能}であり,
		\footnote{$\alpha$ が分割可能とする.このときある $\beta_1,\, \beta_2 \in \Phi^+(\gamma')$ が存在して $\alpha = \beta_1 + \beta_2$ と書ける.ところが今 $\rpair{\gamma'}{\beta_i} > \varepsilon$ であるから,$\varepsilon = \rpair{\gamma'}{\alpha} = \rpair{\gamma'}{\beta_1} + \rpair{\gamma'}{\beta_2} > 2\varepsilon$ と言うことになり,$\varepsilon > 0$ に矛盾.}
		,$\alpha \in \Delta(\gamma')$ が言えた.
		\item $\mathscr{W} = \mathscr{W}'$ を言うには,$\forall \alpha \in \Phi$ に対して $\sigma_\alpha \in \mathscr{W}'$ が成り立つことを示せば十分である.
		ここで (3) よりある $\sigma \in \mathscr{W}'$ が存在して $\sigma(\alpha) \in \Delta$ を充たすようにできる.
		このとき補題\ref{lem:Weylgroup-basic}-(1) より $\sigma_{\sigma(\alpha)} = \sigma \circ \sigma_\alpha \circ \sigma^{-1}$ であるので,$\sigma_{\alpha(\alpha)} \in \mathscr{W}'$ より $\sigma_\alpha = \sigma^{-1} \circ \sigma_{\sigma(\alpha)} \circ \sigma \in \mathscr{W}'$ が言えた.
		\item 背理法により示す.$\sigma(\Delta) = \Delta$ でかつ $\sigma \neq \mathrm{id}_{\mathbb{E}}$ を仮定する.
		このとき (4) よりある $\alpha_1,\, \dots,\, \alpha_{t} \in \Delta$ が存在して $\sigma = \sigma_{\alpha_1} \circ \cdots \circ \sigma_{\alpha_{t}}$ と表示できる.このような $\sigma$ の表示のうち,$t$ が最小のものをとる.
		然るにこのとき仮定より $\sigma (\alpha_{t}) \succ 0$ と言うことになって系\ref{col:simpleroot-C}に矛盾.
	\end{enumerate}
	
\end{proof}

\begin{mydef}[label=def:Weylgroup-presentation]{Weyl群の簡約表示}
	$\Phi$ の\hyperref[def:base-root]{底} $\Delta$ を任意に与える.
	\begin{itemize}
		\item $\forall \sigma \in \mathscr{W}$ に対して一意に定まる非負整数
		\begin{align}
			\bm{\len (\sigma)} \coloneqq \min \bigl\{\, t \in \mathbb{Z}_{\ge 0} \bigm| \exists \alpha_1,\, \dots,\, \alpha_t \in \Delta,\; \sigma = \sigma_{\alpha_1} \circ \cdots \circ \sigma_{\alpha_t} \,\bigr\} 
		\end{align}
		を $\Delta$ に関する $\sigma$ の\textbf{長さ} (length) と呼ぶ.ただし $\len (\mathrm{id}_{\mathbb{E}}) \coloneqq 0$ と定義する.
		\item $\sigma \in \mathscr{W}$ の表示
		\begin{align}
			\sigma = \sigma_{\alpha_1} \circ \cdots \circ \sigma_{\alpha_{\len(\sigma)}} \WHERE \alpha_{i} \in \Delta
		\end{align}
		のことを $\Delta$ に関する $\sigma$ の\textbf{簡約表示} (reduced presentation) と呼ぶ.
	\end{itemize}
\end{mydef}

\begin{mylem}[label=lem:reduced-basic]{簡約表示の長さ}
	$\Phi$ の\hyperref[def:base-root]{底} $\Delta$ を任意に与える.
	このとき $\forall \sigma \in \mathscr{W}$ に対して
	\begin{align}
		\len (\sigma) = \abs{\bigl\{\, \alpha \in \Phi \bigm| \alpha \succ 0 \AND \sigma(\alpha) \prec 0 \,\bigr\}}
	\end{align}
	が成り立つ.右辺を $\bm{n(\sigma)}$ とおく.
\end{mylem}

\begin{proof}
	$\forall \sigma \in \mathscr{W}$ を1つ固定する.$\len (\sigma)$ に関する数学的帰納法により示す.
	$\len (\sigma)= 0$ のときは $\sigma = \mathrm{id}$ なので良い\footnote{\hyperref[ax:root-system]{\textsf{\textbf{(Root-1)}}}より $0 \notin \Phi$ である.}.

	$\len (\sigma) > \len (\tau)$ を充たす任意の $\tau \in \mathscr{W}$ に対して補題が成り立っているとする.
	$\sigma = \sigma_{\alpha_1} \circ \cdots \circ \sigma_{\alpha_{t}}$ を $\Delta$ に関する $\sigma$ の\hyperref[def:Weylgroup-presentation]{簡約表示}とし,$\alpha \coloneqq \alpha_t \in \Delta$ とおく.
	すると系\ref{col:simpleroot-C}より $\sigma(\alpha) \prec 0$ である.よって補題\ref{lem:simpleroot-B}から $n(\sigma \circ \sigma_\alpha) = n(\sigma) - 1$ となる.一方で $\len(\sigma \circ \sigma_\alpha) = \len(\sigma_{\alpha_1} \circ \cdots \circ \sigma_{\alpha_{t-1}}) = \len(\sigma) - 1 < \len (\sigma)$ であるから,帰納法の仮定より $\len (\sigma \circ \sigma_\alpha) = n(\sigma \circ \sigma_\alpha)$ であり,$n(\sigma) = \len (\sigma)$ が言えた.
\end{proof}

位相空間 $X$ の部分空間 $A \subset X$ の閉包\footnote{$\overline{A} \coloneqq \bigcap_{F:\, \text{closed},\;A \subset F} F$}を $\overline{A}$ と書く.

\begin{mylem}[label=lem:Weylgroup-FD]{Weyl群の作用の基本領域}
	$\Phi$ の\hyperref[def:base-root]{底} $\Delta$ を任意に与え,$\forall \lambda,\, \mu \in \overline{\mathfrak{C}(\Delta)}$ をとる.

	このとき $\exists \sigma \in \mathscr{W},\; \sigma(\lambda) = \mu$ ならば,$\sigma$ は点 $\lambda$ を固定する鏡映の積で書ける.特に $\lambda = \mu$ である.
\end{mylem}

\begin{proof}
	$\len (\sigma)$ に関する数学的帰納法により示す.$\len (\sigma) = 0$ ならば明らか.

	$\len (\sigma) >0$ とする.補題\ref{lem:reduced-basic}より $\sigma$ はある\hyperref[def:base-root]{正ルート}を負ルートに移す.
	% i.e. $\sigma (\Delta) \not\subset \Phi^+$ である.
	よって $\sigma(\alpha) \prec 0$ を充たす $\alpha \in \Delta$ が存在する.このとき $\lambda,\, \mu \in \overline{\mathfrak{C}(\Delta)}$ であることから
	\begin{align}
		0 \ge \rpair[\big]{\mu}{\sigma(\alpha)} = \rpair[\big]{\sigma^{-1}(\mu)}{\alpha} = \rpair{\lambda}{\alpha} \ge 0
	\end{align}
	となって $\rpair{\lambda}{\alpha} = 0$ が分かった.従って $\sigma_\alpha(\lambda) = \lambda,\; \sigma \circ \sigma_\alpha(\lambda) = \mu$ が言える.i.e. $\sigma_\alpha$ は $\lambda$ を固定する鏡映である.
	さらに補題\ref{lem:simpleroot-B}, \ref{lem:reduced-basic}から $\len (\sigma \circ \sigma_\alpha) = \len(\sigma) -1$ であり,帰納法の仮定が使えて $\sigma = (\sigma \circ \sigma_\alpha) \circ \sigma_\alpha$ は $\lambda$ を固定する鏡映の積で書ける.
\end{proof}

\subsection{既約なルート系}

\begin{mydef}[label=def:irr-root]{ルート系の既約性}
	\hyperref[ax:root-system]{ルート系} $(\mathbb{E},\, \Phi)$ が\textbf{既約} (irreducible) であるとは,$\Phi$ が以下の条件を満たすことを言う:
	
	\begin{description}
		\item[\textbf{(Root-irr)}] 
		
		$\Phi$ の部分集合 $\Phi_1,\, \Phi_2 \subset \Phi$ が $\Phi = \Phi_1 \amalg \Phi_2 \AND \rpair{\Phi_1}{\Phi_2} = 0$ を充たす $\IMP \Phi_1 = \emptyset$ または $\Phi_2 = \emptyset$
	\end{description}
	
\end{mydef}

\begin{myprop}[label=prop:irr-root]{ルート系の既約性の特徴付け}
	\hyperref[ax:root-system]{ルート系} $(\mathbb{E},\, \Phi)$ を与え,$\Phi$ の\hyperref[def:base-root]{底} $\Delta$ を1つ固定する.
	
	このとき,以下の2つは同値である:
	\begin{enumerate}
		\item $\Phi$ は\hyperref[def:irr-root]{既約}
		\item $\Delta$ の部分集合 $\Delta_1,\, \Delta_2 \subset \Delta$ が $\Delta = \Delta_1 \amalg \Delta_2 \AND \rpair{\Delta_1}{\Delta_2} = 0$ を充たす $\IMP \Delta_1 = \emptyset \OR \Delta_2 = \emptyset$
	\end{enumerate}
\end{myprop}

\begin{proof}
	\begin{description}
		\item[\textbf{(1) $\bm{\Longleftarrow}$ (2)}] 
		
		対偶を示す.$\Phi$ の空でない2つの部分集合 $\Phi_1,\, \Phi_2 \subset \Phi$ であって $\Phi = \Phi_1 \amalg \Phi_2 \AND \rpair{\Phi_1}{\Phi_2} = 0$ を充たすものが存在するとする.
		このとき $\Delta_i \coloneqq \Delta \cap \Phi_i \; (i = 1,\, 2)$ とおくと $\Delta = \Delta_1 \amalg \Delta_2$ かつ $\rpair{\Delta_1}{\Delta_2} = 0$ が成り立つ.
		あとは $\Delta_1 \neq \emptyset \AND \Delta_2 \neq \emptyset$ を背理法により示す.
		$\Delta_1,\, \Delta_2$ のどちらかが空だと仮定する.議論は全く同様なので $\Delta_2 = \emptyset$ としよう.このとき $\Delta \subset \Phi_1$ なので仮定より $\rpair{\Delta}{\Phi_2} = 0$ だが,\hyperref[def:base-root]{\textsf{\textbf{(B-1)}}}よりこのことは $\rpair{\mathbb{E}}{\Phi_2} = 0$ を意味する.よって $\Phi_2 = \emptyset$ となり $\Phi_2 \neq \emptyset$ に矛盾.

		\item[\textbf{(1) $\bm{\Longrightarrow}$ (2)}] 
		
		$\Phi$ が既約だとする.このとき背理法によって (2) を示す.
		$\Delta$ の空でない2つの部分集合 $\Delta_1,\, \Delta_2 \subset \Delta$ であって $\Delta = \Delta_1 \amalg \Delta_2 \AND \rpair{\Delta_1}{\Delta_2} = 0$ を充たすものが存在するとする.
		このとき定理\ref{thm:Weylgroup-basic}-(3) より,
		\begin{align}
			\Phi_i \coloneqq \bigl\{\, \alpha \in \Phi \bigm| \exists \sigma \in \mathscr{W},\; \sigma(\alpha) \in \Delta_i \,\bigr\} 
		\end{align}
		とおくと $\Phi = \Phi_1 \cup \Phi_2$ となる.
		ここで $\forall \alpha_i \in \Delta_i$ に対して仮定より $\sspair{\alpha_i}{\alpha_j} = 0 \quad (i \neq j)$ であるから
		\begin{align}
			(\sigma_{\alpha_1} \circ \sigma_{\alpha_2} - \sigma_{\alpha_2} \circ \sigma_{\alpha_1})(\beta) 
			= \sspair{\beta}{\alpha_2} \sspair{\alpha_2}{\alpha_1} \alpha_1
			-\sspair{\beta}{\alpha_1} \sspair{\alpha_1}{\alpha_2} \alpha_2 = 0
		\end{align}
		i.e. $\sigma_{\alpha_1} \circ \sigma_{\alpha_2} = \sigma_{\alpha_2} \circ \sigma_{\alpha_1}$ である.さらに $\sigma_{\alpha_i}(\alpha_j) = \alpha_j \quad (i \neq j)$ であるから
		$\Phi_1 \cap \Phi_2 = \emptyset$ であり,$\Phi_i = \Span_{\mathbb{Z}} \Delta_i$ である.よって背理法の仮定から $\rpair{\Phi_1}{\Phi_2} = 0$ となるが,仮定より $\Phi$ は既約なので $\Phi_1 = \emptyset$ または $\Phi_2 = \emptyset$ である.従って $\Delta_i$ のどちらかが空集合ということになって矛盾.
	\end{description}
\end{proof}

\begin{mylem}[label=lem:irr-root-A]{}
	\hyperref[ax:root-system]{ルート系} $(\mathbb{E},\, \Phi)$ を与え,$\Phi$ の\hyperref[def:base-root]{底} $\Delta$ を任意にとる.

	このとき $\Phi$ が\hyperref[def:irr-root]{既約}ならば,\hyperref[def:base-root]{半順序 $\prec$} に関して $\Phi$ は唯一の極大元 $\beta_0 \in \Phi$ を持つ
	\footnote{i.e. $\forall \alpha \in \Phi,\;\alpha \neq \beta_0 \IMP \hight \alpha < \hight \beta_0$.さらに,$\forall \alpha \in \Delta,\; \rpair{\beta_0}{\alpha} \ge 0$}
	.特に $\beta_0 = \sum_{\alpha \in \Delta} k_\alpha \alpha$ と書いたときに $\forall \alpha \in \Delta,\; k_\alpha \; \textcolor{red}{>}\; 0$ である.
\end{mylem}

\begin{proof}
	\hyperref[ax:root-system]{\textsf{\textbf{(Root-1)}}}より $(\Phi,\, \prec)$ は有限な半順序集合だから極大元 $\beta_0 = \sum_{\alpha \in \Delta} k_\alpha \alpha \in \Phi$ を少なくとも1つ持つ.$\beta_0 \prec 0$ だとすると任意の\hyperref[def:base-root]{単純ルート} $\alpha$ に対して $\beta_0 \prec \alpha$ となり $\beta_0$ の極大性に矛盾.よって $\beta_0 \succ 0$ である.
	% この $\beta_0$ が唯一であることを示そう.
	
	まず $\forall \alpha \in \Delta,\; k_\alpha\; \textcolor{red}{>}\; 0$ を示そう.
	\begin{align}
		\Delta_1 &\coloneqq \bigl\{\, \alpha \in \Delta \bigm| k_\alpha > 0 \,\bigr\}, \\
		\Delta_2 &\coloneqq \bigl\{\, \alpha \in \Delta \bigm| k_\alpha = 0 \,\bigr\}
	\end{align}
	とおくと $\Delta = \Delta_1 \amalg \Delta_2$ が成り立つ.
	% 仮定より $\Phi$ は既約なので,命題\ref{prop:irr-root}より $\Delta_1 = \emptyset$ または $\Delta_2 = \emptyset$ である.
	$\Delta_2 = \emptyset$ を背理法により示す.
	$\Delta_2 \neq \emptyset$ を仮定する.
	% このとき補題\ref{lem:base}より $\forall \alpha \in \Delta_2$ に対して $\rpair{\alpha}{\beta_0} = \sum_{\gamma \in \Delta} k_\gamma \rpair{\alpha}{\gamma} \le 0$ が成り立つ\footnote{$\beta_0 \succ 0$ なので $\forall \gamma \in \Delta,\; k_\gamma \ge 0$ である.}.
	$\beta_0 \neq 0$ なので $\Delta_1 \neq \emptyset$ だが,仮定より $\Phi$ が既約なので命題\ref{prop:irr-root}から $\rpair{\Delta_1}{\Delta_2} \neq 0$,i.e. ある $\alpha_1 \in \Delta_1,\, \alpha_2 \in \Delta_2$ が存在して $\rpair{\alpha_1}{\alpha_2} < 0$ を満たす.従って 
	\begin{align}
		\rpair{\alpha_2}{\beta_0} = k_{\alpha_1} \rpair{\alpha_2}{\alpha_1} + \sum_{\gamma \in \Delta_1 \setminus \{\alpha_1\}} k_\gamma \underbrace{\rpair{\alpha_2}{\gamma}}_{\le 0\; (\because\; \text{補題\ref{lem:base}})} < 0
	\end{align}
	ということになる.
	然るにこのとき補題\ref{lem:root-basic}から $\alpha_2 + \beta_0 \in \Phi$ となり $\beta_0$ の極大性に矛盾する.

	次に,$\forall \alpha \in \Delta$ に対して $\rpair{\alpha}{\beta_0} \ge 0$ であることを背理法により示す.実際ある $\alpha \in \Delta$ に対して $\rpair{\alpha}{\beta_0} < 0$ だとすると補題\ref{lem:base}から $\alpha + \beta_0 \in \Phi$ となり $\beta_0$ の極大性に矛盾する.
	次に,ある $\alpha_0 \in \Delta$ が存在して $\rpair{\alpha_0}{\beta_0} > 0$ となることを背理法により示す.実際 $\forall \alpha \in \Delta$ に対して $\rpair{\alpha}{\beta_0} = 0$ だとすると $\beta_0 \in (\Span_{\mathbb{R}} \Delta)^\perp = \mathbb{E}^\perp = \emptyset$ ということになり矛盾する.
	% $\Delta_2 = \emptyset$ が分かったので $\forall \alpha \in \Delta,\, k_\alpha > 0$ が言えた.このとき 
	
	最後に $\beta_0$ の一意性を示す.別の極大元 $\beta = \sum_{\gamma \in \Delta} \beta_\gamma \gamma \in \Phi$ をとる.上の議論は $\beta'$ にも当てはまるので $\forall \gamma \in \Delta,\; \beta_\gamma > 0$ であり,
	\begin{align}
		\rpair{\beta}{\beta_0} = \beta_{\alpha_0} \rpair{\alpha_0}{\beta_0} + \sum_{\gamma \in \Delta \setminus \{\alpha_0\}} \beta_\gamma \rpair{\gamma}{\beta_0} > 0
	\end{align}
	だと分かった.従って補題\ref{lem:root-basic}から $\beta = \beta_0$ または $\beta - \beta_0 \in \Phi$ である.後者だと $\beta,\, \beta_0$ のどちらかの極大性に矛盾するので,背理法から証明が完成した.
\end{proof}


\begin{mylem}[label=lem:irr-root-B]{}
	$\Phi$ が\hyperref[def:irr-root]{既約}ならば,\hyperref[def:Weylgroup]{Weyl群}の表現
	\begin{align}
		\pi \colon \mathscr{W} \lto \LGL(\mathbb{E}),\; \sigma \lmto \bigl( \gamma \mapsto \sigma(\gamma) \bigr) 
	\end{align}
	は既約である.特に,$\forall \alpha \in \Phi$ に対して
	\begin{align}
		\mathbb{E} = \Span_{\mathbb{R}} \bigl(\mathscr{W} \btr \{\alpha\}\bigr)
	\end{align}
	が成り立つ\footnote{$\btr \colon \mathscr{W} \times \mathbb{E} \lto \mathbb{E},\; (\sigma,\, \gamma) \lmto \pi(\sigma)(\gamma)$ と定義した.}.
\end{mylem}

\begin{proof}
	$\mathbb{E}$ の $0$ でない $\mathscr{W}$-不変な部分ベクトル空間 $W \subset \mathbb{E}$ を任意にとる.
	このとき $W$ の\hyperref[def:radical-bilinear]{直交補空間} $W^\perp$ について $\mathbb{E} = W \oplus W^\perp$ が成り立つ.
	$\forall \gamma \in W,\, \forall \delta \in W^\perp,\, \forall \sigma \in \mathscr{W}$ に対して $\rpair[\big]{\gamma}{\sigma (\delta)} =  \rpair[\big]{\sigma^{-1}(\gamma)}{\delta} = 0$ が成り立つので $\sigma(\delta) \in W^\perp$ であり,$W^\perp$ もまた $\mathscr{W}$-不変である.
	
	さて,$W$ は $\mathscr{W}$-不変なので $\forall \alpha \in \Phi$ に対して $\sigma_\alpha (W) = W$ が成り立つ.
	このときもし $\alpha \notin W$ かつ $W \not\subset P_\alpha$ だとすると $\gamma \in W \setminus P_\alpha$ が存在して $\sigma_\alpha(\gamma) = \gamma - \sspair{\gamma}{\alpha}\alpha \in W$ が成り立つが $\sspair{\gamma}{\alpha} \neq 0$ なので $\alpha \in W$ ということになり矛盾.よって $\alpha \in W$ または $W \subset P_\alpha = (\mathbb{K}\alpha)^\perp$ である.
	i.e. $\forall \alpha \in \Phi$ は $W$ か $W^\perp$ のどちらか一方に含まれ,$\Phi = (\Phi \cap W) \amalg (\Phi \cap W^\perp)$ が成り立つ.然るに $\rpair[\big]{\Phi \cap W}{\Phi \cap W^\perp} = 0$ かつ $W$ は $0$ でないので,$\Phi$ の既約性から $\Phi \cap W^\perp = \emptyset$ だと分かる.$\mathbb{E} = \Span_{\mathbb{R}}\Phi$ なので $W^\perp = 0$ が言えた.

	$\Span_{\mathbb{R}} \bigl(\mathscr{W} \btr \{\alpha\}\bigr)$ は $0$ でない $\mathscr{W}$-不変な部分ベクトル空間なので,$\pi$ が既約表現であることから $\mathbb{E} = \Span_{\mathbb{R}} \bigl(\mathscr{W} \btr \{\alpha\}\bigr)$ である.
\end{proof}


\begin{mylem}[label=lem:irr-root-C]{}
	$\Phi$ が\hyperref[def:irr-root]{既約}ならば,ルートの長さがとりうる値は高々2通りである.
	さらに,同じ長さのルートは互いに\hyperref[def:Weylgroup]{Weyl群}の作用により移り合う.
\end{mylem}

\begin{proof}
	$\forall \alpha,\, \beta \in \Phi$ をとる.補題\ref{lem:irr-root-B}より $\mathbb{E} = \Span_{\mathbb{R}} \bigl\{\, \sigma(\alpha) \bigm| \sigma \in \mathscr{W} \,\bigr\}$ なので,ある $\sigma \in \mathscr{W}$ が存在して $\rpair[\big]{\sigma(\alpha)}{\beta} \neq 0$ を充たす
	\footnote{$\forall \sigma \in \mathscr{W}$ に対して $\rpair{\sigma(\alpha)}{\beta} = 0$ だとすると $\beta \in (\Span_{\mathbb{R}} \bigl\{\, \sigma(\alpha) \bigm| \sigma \in \mathscr{W} \,\bigr\})^\perp = 0$ となり矛盾}.$\alpha$ と $\sigma(\alpha)$ を取り替えることで
	$\rpair{\alpha}{\beta} \neq 0$ として良い.
	このとき表\ref{tab:rootsystem}より $\norm*{\beta}^2 / \norm*{\alpha}^2 = 1,\, 2,\, 3,\, 1/2,\, 1/3$ しかあり得ない.

	さて,長さの異なるルート $\alpha,\, \beta,\, \gamma \in \Phi$ が存在したとする.冒頭の議論からこのとき $\rpair{\alpha}{\beta} \neq 0,\; \rpair{\gamma}{\beta} \neq 0$ を仮定してよく,$\norm*{\beta}^2 / \norm*{\alpha}^2,\, \norm*{\gamma}^2 / \norm*{\beta}^2 = 2,\, 3,\, 1/2,\, 1/3$ かつ $\norm*{\beta}^2 / \norm*{\alpha}^2 \neq \norm*{\gamma}^2 / \norm*{\beta}^2$ である.然るにこのときどの組み合わせであっても $\norm*{\gamma}^2 / \norm*{\alpha}^2 = (\norm*{\beta}^2 / \norm*{\alpha}^2) (\norm*{\gamma}^2 / \norm*{\beta}^2) \neq 2,\, 3,\, 1/2,\, 1/3$ となり矛盾.故に前半が示された.

	後半を示す.$\norm*{\alpha}	= \norm*{\beta}$ とする.冒頭の議論から $\rpair{\alpha}{\beta} \neq 0$ を仮定して良い.
	このとき表\ref{tab:rootsystem}から $\sspair{\alpha}{\beta} = \sspair{\beta}{\alpha} = \pm 1$ となる.必要ならば $\beta$ を $-\beta = \sigma_\beta(\beta)$ に置き換えることで $\sspair{\alpha}{\beta} = 1$ にできる.すると
	\begin{align}
		\sigma_\alpha \circ \sigma_\beta \circ \sigma_\alpha (\beta)
		= \sigma_\alpha \circ \sigma_\beta (\beta - \alpha) = \sigma_\alpha(- \beta - \alpha + \beta) = \alpha
	\end{align}
	であることが分かった.
\end{proof}


\begin{mylem}[label=lem:irr-root-D]{}
	$\Phi$ が\hyperref[def:irr-root]{既約}ならば,補題\ref{lem:irr-root-A}で得た極大ルート $\beta_0 \in \Phi$ の長さの値は,あり得る2通りのうち大きい方である.
\end{mylem}

\begin{proof}
	$\Phi$ の\hyperref[def:base-root]{底} $\Delta$ および $\forall \alpha \in \Phi$ をとる.
	$\rpair{\beta_0}{\beta_0} \ge \rpair{\alpha}{\alpha}$ を示せば十分である.

	さて,定理\ref{thm:Weylgroup-basic}-(3) より,$\mathscr{W}$ の作用によって移すことで $\alpha \in \overline{\mathfrak{C}(\Delta)}$ を仮定して良い.補題\ref{lem:irr-root-A}より $\beta_0 -\alpha \succ 0$ なので,$\forall \gamma \in \overline{\mathfrak{C}(\Delta)}$ に対して $\rpair{\gamma}{\beta_0 - \alpha} \ge 0$ である.$\gamma = \beta_0,\, \alpha$ の場合を考えることで $\rpair{\beta_0}{\beta_0} \ge \rpair{\beta_0}{\alpha} \ge \rpair{\alpha}{\alpha}$ が言えた.
\end{proof}

\section{ルート系の分類}

この節では $(\mathbb{E},\, \Phi)$ を任意の\hyperref[def:rank-root]{ランク $l$} の\hyperref[ax:root-system]{ルート系}とし,その\hyperref[def:Weylgroup]{Weyl群}を $\mathscr{W}$ と略記する. 
さらに $\Phi$ の\hyperref[def:base-root]{底} $\Delta$ を1つ固定する.

\subsection{Cartan行列}

$\Delta = \{\alpha_1,\, \dots,\, \alpha_l\}$ と書こう.

\begin{mydef}[label=def:Cartan-matrix]{Cartan行列}
	\hyperref[ax:root-system]{ルート系} $\Phi$ の\textbf{Cartan行列} (Cartan matrix) とは,
	\begin{align}
		\mqty[\sspair{\alpha_1}{\alpha_1} &\cdots &\sspair{\alpha_1}{\alpha_l} \\ \vdots &\ddots &\vdots \\ \sspair{\alpha_l}{\alpha_1} &\cdots & \sspair{\alpha_l}{\alpha_l}] \in \LGL(l,\, \mathbb{Z})
	\end{align}
	のこと\footnote{$\Delta$ は $\mathbb{E}$ の基底なのでCartan行列は $\LGL(l,\, \mathbb{Z})$ の元である.}.$\sspair{\alpha_\mu}{\alpha_\nu} \in \mathbb{Z}$ のことを\textbf{Cartan整数} (Cartan integer) と呼ぶ.
\end{mydef}

定理\ref{thm:Weylgroup-basic}-(2) および補題\ref{lem:root-basic}-(2) より,Cartan行列は\underline{底 $\Delta$ の取り方によらずに定まる}\footnote{\hyperref[def:base-root]{単純ルート}のラベル付けの任意性を除く.}!

\begin{myprop}[label=prop:Cartan-matrix-basic]{Cartan行列によるルート系の特徴付け}
	\begin{itemize}
		\item \hyperref[def:rank-root]{ランク $l$} の\hyperref[ax:root-system]{ルート系} $(\mathbb{E},\, \Phi),\, (\mathbb{E}',\, \Phi')$ 
		\item $\Phi$ の\hyperref[def:base-root]{底} $\Delta = \{\alpha_1,\, \dots,\, \alpha_l\}$
		\item $\Phi'$ の\hyperref[def:base-root]{底} $\Delta' = \{\alpha'_1,\, \dots,\, \alpha'_l\}$
	\end{itemize}
	を与える.
	このとき以下の2つは同値である:
	\begin{enumerate}
		\item $1 \le \forall \mu,\, \nu \le l$ に対して $\sspair{\alpha_\mu}{\alpha_\nu} = \sspair{\alpha'_\mu}{\alpha'_\nu}$
		\item \hyperref[def:isom-root]{ルート系の同型写像}
			\begin{align}
				\phi \colon (\mathbb{E},\, \Phi) \lto (\mathbb{E}',\, \Phi')
			\end{align}
			が存在する.
	\end{enumerate}
\end{myprop}

\begin{proof}
	\begin{description}
		\item[\textbf{(1) $\bm{\Longrightarrow}$ (2)}] 
		
		写像
		\begin{align}
			\phi \colon \mathbb{E} \lto \mathbb{E}',\; \alpha_\mu \lmto \alpha_\mu'
		\end{align}
		が\hyperref[def:isom-root]{ルート系の同型写像}となることを示す.
		まず $\Delta,\, \Delta'$ はそれぞれ $\mathbb{E},\, \mathbb{E}'$ の基底なので,写像 $\phi \colon \mathbb{E} \lto \mathbb{E}'$ はベクトル空間の同型写像である.

		次に,仮定より
		\begin{align}
			\phi \circ \sigma_{\alpha_\mu} \circ \phi^{-1}(\alpha'_\nu)
			&=  \phi \circ \sigma_{\alpha_\mu}(\alpha_\nu) \\
			&= \phi (\alpha_\nu - \sspair{\alpha_\nu}{\alpha_\mu} \alpha_\mu) \\
			&= \alpha'_\nu - \sspair{\alpha'_\nu}{\alpha'_\mu} \alpha'_\mu \\
			&= \sigma_{\alpha'_\mu}(\alpha'_\nu)
		\end{align}
		が成り立つので $\phi \circ \sigma_{\alpha_\mu} \circ \phi^{-1} = \sigma_{\alpha'_\mu}$ が言える.$\Weyl{\mathbb{E}}{\Phi},\, \Weyl{\mathbb{E}'}{\Phi'}$ はそれぞれ $\bigl\{\, \sigma_{\alpha_\mu} \in \LGL(\mathbb{E}) \bigm| \alpha_\mu \in \Delta \,\bigr\},\; \bigl\{\, \sigma_{\alpha'_\mu} \in \LGL(\mathbb{E}) \bigm| \alpha'_\mu \in \Delta' \,\bigr\} $ により生成されるので
		\begin{align}
			\phi \circ \Weyl{\mathbb{E}}{\Phi} \circ \phi^{-1} = \Weyl{\mathbb{E}'}{\Phi'}
		\end{align}
		だと分かった.よって定理\ref{thm:Weylgroup-basic}-(3) より
		\begin{align}
			\phi(\Phi) = \phi \bigl( \Weyl{\mathbb{E}}{\Phi} \btr \Delta  \bigr) = \phi \circ \Weyl{\mathbb{E}}{\Phi} \circ \phi^{-1} \circ \phi(\Delta) = \Weyl{\mathbb{E}'}{\Phi'} \btr \Delta' = \Phi'
		\end{align}
		が言えた.

		\item[\textbf{(1) $\bm{\Longleftarrow}$ (2)}] 
		
		\hyperref[def:isom-root]{ルート系の同型写像}
		\begin{align}
			\phi \colon (\mathbb{E},\, \Phi) \lto (\mathbb{E}',\, \Phi')
		\end{align}
		が存在するとする.このとき $\phi(\Delta)$ もまた $\Phi'$ の底になるので,補題\ref{lem:Weylchamber-Weylgroup}-(1) よりある $\sigma' \in \Weyl{\mathbb{E}'}{\Phi'}$ が存在して $\sigma' \bigl( \phi(\Delta)\bigr) = \Delta'$ を充たす.
		よって $\phi$ を $\sigma' \circ \phi$ に置き換えることで $\phi(\Delta) = \Delta'$ であるとして良い.さらに $\Delta,\, \Delta'$ の添字を付け替えることで $\phi (\alpha_\mu) = \alpha'_\mu$ が成り立つとして良い.
		すると $\phi$ が\hyperref[def:isom-root]{ルート系の同型写像}であることから
		\begin{align}
			\sspair{\alpha'_\mu}{\alpha'_\nu} = \sspair{\phi(\alpha_\mu)}{\phi(\alpha_\nu)} = \sspair{\alpha_\mu}{\alpha_\nu}
		\end{align}
		が言える.
	\end{description}
\end{proof}

つまり,\underline{任意の\hyperref[ax:root-system]{ルート系}はその\hyperref[def:Cartan-matrix]{Cartan行列}が与えれれれば完全に\footnote{\hyperref[def:isom-root]{ルート系の同型}の不定性を除いて}復元できる}!
従って,ルート系を分類するにはCartain行列としてあり得るものを全て列挙できれば十分である.

\subsection{CoxeterグラフとDynkin図形}

まず,\hyperref[def:Cartan-matrix]{Cartan行列} $\bigl[ \sspair{\alpha_\mu}{\alpha_\nu} \bigr]_{1 \le \mu,\, \nu \le l}$ について
\begin{description}
	\item[\textbf{(CM-1)}] \label{Cartan-matrix-classification}
	$1 \le \forall \mu \le l$ に対して
	\begin{align}
		\sspair{\alpha_\mu}{\alpha_\mu} = 2\frac{\rpair{\alpha_\mu}{\alpha_\mu}}{\rpair{\alpha_\mu}{\alpha_\mu}} = 2
	\end{align}
	なので,対角成分は常に $2$ である.
	\item[\textbf{(CM-2)}] 補題\ref{lem:base}より $\forall 1 \le \mu \neq \nu \le l$ について
	\begin{align}
		\sspair{\alpha_\mu}{\alpha_\nu} \le 0
	\end{align}
	なので,非対角成分は常に非正である.
	\item[\textbf{(CM-3)}] 表\ref{tab:rootsystem}より $\forall 1 \le \mu \neq \nu \le l$ について
	\begin{align}
		\sspair{\alpha_\mu}{\alpha_\nu}\sspair{\alpha_\nu}{\alpha_\mu} = 0,\, 1,\, 2,\, 3
	\end{align}
	のいずれかである.とくに $\sspair{\alpha_\mu}{\alpha_\nu} = -1$ であるか $\sspair{\alpha_\nu}{\alpha_\mu} = -1$ であるかのどちらかである.
\end{description}
が即座にわかる.\textsf{\textbf{(CM-1)}}より,Cartan行列を知りたければその非対角成分を完全に決定できれば十分である.

\begin{mydef}[label=def:Coxeter-Dynkin,breakable]{CoxeterグラフとDynkin図形}
	\hyperref[def:rank-root]{ランク $l$} の\hyperref[ax:root-system]{ルート系} $\Phi$ の\hyperref[def:Cartan-matrix]{Cartan行列} $\bigl[\, \sspair{\alpha_\mu}{\alpha_\nu} \,\bigr]_{1 \le \mu,\, \nu \le l}$ を与える.
	$\Phi$ の\textbf{Dynkin図形} (Dynkin diagram) $\Gamma = \bigl( \mathcal{V}(\Gamma),\, \mathcal{E}(\Gamma) \bigr)$ とは,以下のデータからなる:
	\begin{itemize}
		\item 頂点集合 $\mathcal{V}(\Gamma)$ は添字付きの $l$ 個の点 $\bullet_\mu\quad (1 \le \mu \le l)$ からなる.
		\item 辺集合 $\mathcal{E}(\Gamma)$ は以下を充たす:
		\begin{description}
			\item[\textbf{(Dynkin-0')}]  $\{\bullet_\mu,\, \bullet_\mu\} \notin \mathcal{E}(\Gamma)$
			\item[\textbf{(Dynkin-0)}]  $\mu \neq \nu$ に対して,
			\begin{align}
				\sspair{\alpha_\mu}{\alpha_\nu}\sspair{\alpha_\nu}{\alpha_\mu} = 0 \IMP \{\bullet_\mu,\, \bullet_\nu\} \notin \mathcal{E}(\Gamma)
			\end{align}
			\item[\textbf{(Dynkin-1)}]  $\mu \neq \nu$ に対して,
			\begin{align}
				\sspair{\alpha_\mu}{\alpha_\nu}\sspair{\alpha_\nu}{\alpha_\mu} = 1 \IMP 
				\dynkin[%
					labels={\mu,\nu},
					edge length=1.25cm
				]A{2} 
				\in \mathcal{E}(\Gamma)
			\end{align}
			\item[\textbf{(Dynkin-2)}]  $\mu \neq \nu$ に対して,
			\begin{align}
				\sspair{\alpha_\mu}{\alpha_\nu}\sspair{\alpha_\nu}{\alpha_\mu} = 2 \AND \norm*{\alpha_\nu} < \norm*{\alpha_\mu} \IMP 
				\dynkin[%
					labels={\mu,\nu},
					edge length=1.25cm
				]B{2} 
				\in \mathcal{E}(\Gamma)
			\end{align}
			\item[\textbf{(Dynkin-3)}]  $\mu \neq \nu$ に対して,
			\begin{align}
				\sspair{\alpha_\mu}{\alpha_\nu}\sspair{\alpha_\nu}{\alpha_\mu} = 3 \AND \norm*{\alpha_\nu} < \norm*{\alpha_\mu} \IMP 
				\dynkin[%
					labels={\mu,\nu},
					edge length=1.25cm
				]G2 
				\in \mathcal{E}(\Gamma)
			\end{align}
		\end{description}
	\end{itemize}
	\tcblower 
	Dynkin図形から矢印(i.e. \hyperref[lem:irr-root-D]{単純ルートの長さ}に関する情報)を取り除いたものを\textbf{Coxeterグラフ} (Coxeter graph) と呼ぶ.
\end{mydef}

与えられたDynkin図形からCartain行列を一意に復元できる.例えば
\begin{align}
	\dynkin[%
					labels={1,...,4},
					edge length=1.25cm
				]F4
\end{align}
(これは $F_4$ と呼ばれるルート系を表す)に対応するCartan行列は
\begin{align}
	\mqty[ 
		2 & -1 & 0 & 0 \\
		-1 & 2 & -2 & 0 \\
		0 & -1 & 2 & -1 \\
		0 & 0 & -1 & 2
	]
\end{align}
である.

\subsection{ルート系の既約成分}

命題\ref{prop:irr-root}より,$\Phi$ が\hyperref[def:irr-root]{既約}であることと 
$\Delta$ が互いに直交する2つの真部分集合のdisjoint unionに分解しないことは同値である.このことから,
\begin{align}
	\Phi\;\text{が\hyperref[def:irr-root]{既約}} \IFF \textbf{対応する\hyperref[def:Coxeter-Dynkin]{Dynkin図形}が連結}
\end{align}
である.

\begin{myprop}[label=prop:root-irrdecomp]{}
	任意のルート系 $(\mathbb{E},\, \Phi)$ は\hyperref[def:irr-root]{既約なルート系} $\Phi_1,\, \dots,\, \Phi_t$ のdisjoint unionに分解し,
	$\mathbb{E}_i \coloneqq \Span_{\mathbb{R}} \Phi_i$ とおくと直交直和の意味で
	\begin{align}
		\mathbb{E} = \mathbb{E}_1 \oplus \cdots \oplus \mathbb{E}_t
	\end{align}
	が成り立つ.
\end{myprop}

\begin{proof}
	
\end{proof}

従って,\hyperref[def:irr-root]{既約なルート系} $\IFF$ 連結な\hyperref[def:Coxeter-Dynkin]{Dynkin図形}を分類すれば十分である.

\subsection{ルート系の分類定理}

\begin{mytheo}[label=thm:root-classification,breakable]{既約なルート系の分類定理}
	\hyperref[def:rank-root]{ランク $l$} の\hyperref[def:irr-root]{既約}な\hyperref[ax:root-system]{ルート系} $\Phi$ を与える.このとき $\Phi$ の\hyperref[def:Coxeter-Dynkin]{Dynkin図形}は以下のいずれかである:
	\begin{description}
		\item[$\bm{A_l}\quad (l \ge 1)$ \textbf{型:}] 
		\begin{align}
			\dynkin[%
					labels={1,2,l-1,l},
					edge length=1.25cm
				] A{}
		\end{align}
		
		\item[$\bm{B_l}\quad (l \ge 2)$ \textbf{型:}] 
		\begin{align}
			\dynkin[%
					labels={1,2,l-2,l-1,l},
					edge length=1.25cm
				] B{}
		\end{align}

		\item[$\bm{C_l}\quad (l \ge 3)$ \textbf{型:}] 
		\begin{align}
			\dynkin[%
					labels={1,2,l-2,l-1,l},
					edge length=1.25cm
				] C{}
		\end{align}

		\item[$\bm{D_l}\quad (l \ge 4)$ \textbf{型:}] 
		\begin{align}
			\dynkin[%
					labels={1,2,l-3,l-2,l-1,l},
					edge length=1.25cm
				] D{}
		\end{align}

		\item[$\bm{E_6}$ \textbf{型:}] 
		\begin{align}
			\dynkin[%
					labels={1,...,6},
					edge length=1.25cm
				] E6
		\end{align}

		\item[$\bm{E_7}$ \textbf{型:}] 
		\begin{align}
			\dynkin[%
					labels={1,...,7},
					edge length=1.25cm
				] E7
		\end{align}

		\item[$\bm{E_8}$ \textbf{型:}] 
		\begin{align}
			\dynkin[%
					labels={1,...,8},
					edge length=1.25cm
				] E8
		\end{align}
		\item[$\bm{F_4}$ \textbf{型:}] 
		\begin{align}
			\dynkin[%
					labels={1,...,4},
					edge length=1.25cm
				] F4
		\end{align}
		\item[$\bm{G_2}$ \textbf{型:}] 
		\begin{align}
			\dynkin[%
					labels={1,...,2},
					edge length=1.25cm
				] G2
		\end{align}
	\end{description}
\end{mytheo}

\begin{proof}
	
\end{proof}



\end{document}