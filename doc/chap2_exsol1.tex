\documentclass{ltjsarticle}

%%%%%%%%%%packages%%%%%%%%%%
%% colors and links
\usepackage[svgnames]{xcolor}
\usepackage[colorlinks,citecolor=DarkGreen,linkcolor=DeepPink,linktocpage,unicode]{hyperref} 

%% equations
%%%% math
\usepackage{amsmath,amsfonts,amssymb,amsthm}
\usepackage{mathtools}
\usepackage{mathrsfs}
\usepackage{bm}
\usepackage{cancel}
\usepackage{dsfont}
%%%% physics
\usepackage{siunitx}
\usepackage{physics}
%%%% chemistry
\usepackage[version=3]{mhchem}

%% positioning
\usepackage{array}
\usepackage{arydshln}
\usepackage{float}

%% table
\usepackage{booktabs}
\usepackage{multirow}
\usepackage{hhline}
\usepackage{caption}
\captionsetup{format=hang}
\usepackage{subcaption}

%% figure
\usepackage{graphicx}
\usepackage{tikz}
\usetikzlibrary{intersections,calc,patterns,through,positioning,arrows,backgrounds,cd,tqft,braids}
\usepackage{tikz-feynman}
\usepackage{pgfplots}
\usepgfplotslibrary{patchplots}
\pgfplotsset{compat=1.15}

%% decorations
\usepackage{titlesec}
\usepackage{picture}

%% framing
\usepackage{fancybox}
\usepackage{boites}
\usepackage{tcolorbox}
\tcbuselibrary{skins,theorems,breakable}

%% citation
\usepackage{cite}

%% miscellaneous
%%%% comment
\usepackage{comment}
%%%% Japanese
\usepackage{ascmac}
\usepackage{listings} %日本語のコメントアウトをする場合jlisting(もしくはjvlisting)が必要
\lstset{
    basicstyle={\ttfamily},
    identifierstyle={\small},
    commentstyle={\smallitshape},
    keywordstyle={\small\bfseries},
    ndkeywordstyle={\small},
    stringstyle={\small	tfamily},
    frame={tb},
    breaklines=true,
    columns=[l]{fullflexible},
    numbers=left,
    xrightmargin=0pt,
    xleftmargin=27pt,
    numberstyle={\scriptsize},
    stepnumber=1,
    lineskip=-0.5ex
}

%%macros
\usepackage{mylogics}
\usepackage{myalgebra}
\usepackage{mygeometry}
\usepackage{myQM}

%%%%%%%%%%optional settings%%%%%%%%%%

%%%%%図表並列%%%%%
\makeatletter
\newcommand{\figcaption}[1]{\def\@captype{figure}\caption{#1}}
\newcommand{\tblcaption}[1]{\def\@captype{table}\caption{#1}}
\makeatother

%%%%%itemization%%%%%
\renewcommand{\labelenumi}{\theenumi}
\renewcommand{\theenumi}{(\arabic{enumi})} % 箇条書きをローマ数字に

%%%%%%theorem environments%%%%%
\newtheoremstyle{mystyle}%   % スタイル名
    {}%                      % 上部スペース
    {}%                      % 下部スペース
    {\normalfont}%          % 本文フォント
    {}%                      % インデント量
    {\bf}%                  % 見出しフォント
    {.}%                      % 見出し後の句読点
    {\newline}%                     % 見出し後のスペース
    {\underline{\thmname{#1}\thmnumber{#2}\thmnote{(#3)}}}%
    % 見出しの書式 (can be left empty, meaning `normal')
\theoremstyle{mystyle} % スタイルの適用

\newtheorem{theorem}{定理}[section]
\newtheorem{definition}{定義}[section]
\newtheorem{proposition}[definition]{命題}
\newtheorem{corollary}[theorem]{系}
\renewcommand{\proofname}{証明}

%proof
\makeatletter % use at mark
\renewcommand{\qedsymbol}{$\blacksquare$}
\renewenvironment{proof}[1][\proofname]{\par
    \pushQED{\qed}%
    \normalfont \topsep6\p@\@plus6\p@\relax
    \trivlist
    \item[\hskip\labelsep
        \itshape
    \textbf{\underline{#1}}]\ignorespaces
    % {\bf\underline{#1}\@addpunct{.}}]\ignorespaces % ピリオドあり
}{%
    \popQED\endtrivlist\@endpefalse
}
\makeatother % end at mark

%%%%%%mathtools%%%%%
\mathtoolsset{showonlyrefs=true} % 被参照数式のみ数式番号割り振り
\numberwithin{equation}{section}

\makeatletter
\@addtoreset{equation}{section}
\makeatother

%%%%%%framing%%%%%

\newtcolorbox[auto counter,number within=section]{myaxiom}[2][]{%
colback=red!5!white,colframe=red!75!black,fonttitle=\bfseries,
title=公理~\thetcbcounter: #2,#1}

\newtcolorbox[auto counter,number within=section]{mytheo}[2][]{%
colback=blue!5!white,colframe=blue!75!black,fonttitle=\bfseries,
title=定理~\thetcbcounter: #2,#1}

\newtcolorbox[auto counter,number within=section]{myprop}[2][]{%
colback=blue!5!white,colframe=blue!75!black,fonttitle=\bfseries,
title=命題~\thetcbcounter: #2,#1}

\newtcolorbox[auto counter,number within=section]{mylem}[2][]{%
colback=blue!5!white,colframe=blue!75!black,fonttitle=\bfseries,
title=補題~\thetcbcounter: #2,#1}

\newtcolorbox[auto counter,number within=section]{mydef}[2][]{%
colback=green!5!white,colframe=green!75!black,fonttitle=\bfseries,
title=定義~\thetcbcounter: #2,#1}

\newtcolorbox[use counter from=mytheo]{mycol}[2][]{%
fonttitle=\bfseries,enhanced,
attach boxed title to top left=
{xshift=-2mm,yshift=-2mm},
title=系~\thetcbcounter: #2,#1}

\newtcolorbox{marker}[1][breakable]{enhanced,
  	before skip=2mm,after skip=3mm,
  	boxrule=0.4pt,left=5mm,right=2mm,top=1mm,bottom=1mm,
  	colback=yellow!50,
  	colframe=yellow!20!black,
  	sharp corners,rounded corners=southeast,arc is angular,arc=3mm,
  	underlay={%
  	  	\path[fill=tcbcolback!80!black] ([yshift=3mm]interior.south east)--++(-0.4,-0.1)--++(0.1,-0.2);
  	  	\path[draw=tcbcolframe,shorten <=-0.05mm,shorten >=-0.05mm] ([yshift=3mm]interior.south east)--++(-0.4,-0.1)--++(0.1,-0.2);
  	  	\path[fill=yellow!50!black,draw=none] (interior.south west) rectangle node[white]{\Huge\bfseries !} ([xshift=4mm]interior.north west);
  	  	},
  	drop fuzzy shadow,#1}


\newtcolorbox[auto counter, number within=section]{myproblem}[2][]{
    breakable,
    % empty,
    outer arc=0pt,
    arc=0pt,
    attach boxed title to top left={
    %     % empty,
        xshift=-2mm,
        yshift=-2mm
    },
    % coltitle=black,
    % colframe=black,
    colback=white,
    fonttitle=\bfseries,
    title=問題~\thetcbcounter #2,#1,
}
\newcommand{\probref}[1]{\textbf{【問題\ref{#1}】}}


%%%%%title decorations%%%%%

%% section
% \titleformat{\section}[block]
% {}{}{0pt}
% {
%     \colorbox{teal}{\begin{picture}(0,20)\end{picture}}
%     \hspace{0pt}
%     \normalfont \Huge\bfseries \thesection
%     \hspace{0.5em}
% }
% [
% \begin{picture}(100,0)
%     \put(3,30){\color{teal}\line(1,0){400}}
% \end{picture}
% \
% \vspace{-50pt}
% ]

% % subsection
% \titleformat{\subsection}[block]
% {}{}{0pt}
% {
%     \colorbox{blue}{\begin{picture}(0,10)\end{picture}}
%     \hspace{0pt}
%     \normalfont \Large\bfseries \thesubsection
%     \hspace{0.5em}
% }
% [
% \begin{picture}(100,0)
%     \put(3,18){\color{blue}\line(1,0){300}}
% \end{picture}
% \
% \vspace{-30pt}
% ]

% % subsubsection
% \titleformat{\subsubsection}[block]
% {}{}{0pt}
% {
%     \normalfont \large\bfseries \thesubsubsection
%     \hspace{0.5em}
% }
% [
% \begin{picture}(100,0)
%     \put(3,15){\color{black}\line(1,0){200}}
% \end{picture}
% \
% \vspace{-25pt}
% ]
\newcommand{\lto}{\longrightarrow}
\newcommand{\lmto}{\longmapsto}
\usepackage{blkarray}

\begin{document}

\title{Humphreys Chapter II Exercises \\ 解答例(2023/11/13 実施分)}
\author{高間俊至}
\date{\today}
\maketitle

\setcounter{section}{1}

\cite[p.24, Exercise1, 2]{Humphreys1972introduction}
% と\cite[p.14, Exercise 2]{Humphreys1972introduction}
の解答例です.

何の断りもない場合,体 $\mathbb{K}$ は代数閉体でかつ $\character \mathbb{K} = 0$ であるとする.
また,$\mathfrak{g}$ は\underline{零でない}体 $\mathbb{K}$ 上の\underline{有限次元} Lie代数とする.

\begin{myproblem}[label=ex:2-2-1]{p.24 の Exercise 1}
    % \setcounter{\tcbcounter}{7}
    Lie代数 $\mathfrak{g}$ が冪零Lie代数 $\IMP$ $\mathfrak{g}$ のKilling形式が恒等的に $0$
\end{myproblem}

\begin{proof}
    $\mathfrak{g}$ が冪零Lie代数であるとする.このとき第1回資料の命題1.1.4-(5) より,部分Lie代数 $\Im \ad \subset \Lgl (\mathfrak{g})$ の任意の元 $\ad(x) \in \Im \ad$ は冪零である.
    従って $\Im \ad$ に対して Corollary 3.3~\cite[p.13]{Humphreys1972introduction} を使うことができて,$\forall \ad(x) \in \Im \ad$ の表現行列を同時に $\mathfrak{n}(\dim \mathfrak{g},\, \mathbb{K})$ の元(i.e. 対角成分が全て $0$ であるような上三角行列)にするような $\mathfrak{g}$ の基底が存在する.
    この基底の下では,$\forall \ad(x),\, \ad(y) \in \Lgl(\mathfrak{g})$ に対して $\ad(x) \circ \ad(y)$ の表現行列も $\mathfrak{n}(\dim \mathfrak{g},\, \mathbb{K})$ の元になるので $\Tr \bigl( \ad(x) \circ \ad(y) \bigr) = 0$ が言えた.
    % $\forall x,\, y \in \mathfrak{g}$ を固定する.
    % $\mathfrak{g}$ が冪零Lie代数なので,命題1.1.4-(6) より $\ad(x),\, \ad (y),\, \ad(\comm{x}{y})$ は冪零.
    % Jacobi恒等式から
    % \begin{align}
    %     \ad(x) \circ \ad(y) (z) 
    %     &= \comm{x}{\comm{y}{z}} \\
    %     &= \comm{y}{\comm{x}{z}} + \comm{\comm{x}{y}}{z} \\
    %     &= \ad(y) \circ \ad(x)(z) + \ad(\comm{x}{y})(z)
    % \end{align}
    % であり,
    % \begin{align}
    %     (\ad(x) \ad(y))^2 
    %     &= \ad(x)\ad(y)\ad(x)\ad(y) \\
    %     &= \ad(y)\ad(x)\ad(x)\ad(y) + \ad(\comm{x}{y})\ad(x)\ad(y)
    % \end{align}

\end{proof}


\begin{myproblem}[label=ex:2-2-2]{p.24 の Exercise 2}
    % \setcounter{\tcbcounter}{7}
    Lie代数 $\mathfrak{g}$ およびそのKilling形式
    \begin{align}
        \kappa \colon \mathfrak{g} \times \mathfrak{g} \lto \mathbb{K},\; (x,\, y) \lmto \Tr \bigl(\ad (x) \circ \ad (y)\bigr)
    \end{align}
    を与える.
    このとき以下の2つは同値である\footnote{$\bigl\{\, x \in \mathfrak{g} \bigm| \forall y \in \mathfrak{g},\; \kappa (x,\, y) = 0 \,\bigr\} $  を $\kappa$ の\textbf{radical}と呼ぶのだった.}:
    \begin{enumerate}
        \item $\mathfrak{g}$ が可解
        \item \begin{align}
            \comm{\mathfrak{g}}{\mathfrak{g}} \subset \bigl\{\, x \in \mathfrak{g} \bigm| \forall y \in \mathfrak{g},\; \kappa (x,\, y) = 0 \,\bigr\} 
        \end{align}
    \end{enumerate}
\end{myproblem}

\begin{proof}
    \begin{description}
        \item[\textbf{(1) $\bm{\Longrightarrow}$ (2)}] 
        
        $\mathfrak{g}$ が可解であるとする.このとき第1回資料の命題1.1.3-(1) より,Lie代数の準同型 $\ad \colon \mathfrak{g} \lto \Lgl(\mathfrak{g})$ の像 $\Im \ad \subset \Lgl(\mathfrak{g})$ もまた可解な部分Lie代数である.よってLieの定理~\cite[p.16, Corollary A]{Humphreys1972introduction}から,$\forall \ad(x) \in \Im \ad$ の表現行列を同時に $\mathfrak{t}(\dim \mathfrak{g},\, \mathbb{K})$ の元(i.e. 上三角行列)にするような $\mathfrak{g}$ の基底が存在する.
        以下,$\mathfrak{g}$ の基底をこの特別な基底に固定する.
        
         $\comm{x}{y}$ の形をした任意の $\comm{\mathfrak{g}}{\mathfrak{g}}$ の元をとる.
        このとき $\ad(\comm{x}{y}) = \comm{\ad(x)}{\ad(y)} = \ad(x) \circ \ad(y) - \ad(y) \circ \ad(x)$ が成り立つ\footnote{$\ad$ はLie代数の準同型なので.$\Lgl (\mathfrak{g})$ のLieブラケットは交換子だったことを思い出すと良い.}が,今の基底の下では $\ad(x),\, \ad(y)$ の表現行列が上三角行列なので $\ad(\comm{x}{y})$ の表現行列は $\mathfrak{n}(\dim \mathfrak{g},\, \mathbb{K})$ の元(i.e. 対角成分が全て $0$ であるような上三角行列)になる.
        このとき $\forall z \in \mathfrak{g}$ に対して\footnote{$\mathfrak{g}$ の基底の取り方から $\ad(z) \in \Im \ad$ の表現行列もまた上三角行列である.} $\ad(\comm{x}{y}) \circ \ad(z)$ の表現行列もまた $\mathfrak{n}(\dim \mathfrak{g},\, \mathbb{K})$ の元になるので $\kappa(\comm{x}{y},\, z) = \Tr \bigl( \ad(\comm{x}{y}) \circ \ad (z)\bigr) = 0$ が言える.$\comm{\mathfrak{g}}{\mathfrak{g}}$ が $\comm{x}{y}$ の形をした元によって生成されること,および $\kappa$ が双線型写像であることから証明が完了した.
        
        \item[\textbf{(1) $\bm{\Longleftarrow}$ (2)}] 
        
        Corollary~\cite[p.20]{Humphreys1972introduction} そのものである.
    \end{description}
    
\end{proof}


%参考文献のスタイル(style.bstを参照)
\bibliographystyle{myh-physrev}
%参考文献(reference.bibを参照)
\bibliography{repref}

\end{document}