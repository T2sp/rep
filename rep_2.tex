\documentclass[a4paper,12pt]{ltjsarticle}


% 数式
\usepackage{amsthm}
\usepackage{amsmath,amsfonts,amssymb}
\usepackage[bold-style=ISO]{unicode-math}
\usepackage{physics}
\usepackage{tensor}
\usepackage{ascmac}
\usepackage{here}
% 画像
\usepackage{graphicx}
\usepackage{bmpsize}
\graphicspath{{image/}}

% 追加
\usepackage{url}
\usepackage{hyperref}
% 定理
\usepackage{tcolorbox}
\tcbuselibrary{breakable, skins, theorems}

\newtcolorbox{axi}[2][]{colbacktitle=red!75!black, colframe=red!75!black, 
colback=red!5!white, coltitle=white,
title={#2},fonttitle=\bfseries\gtfamily,#1}

\newtcolorbox{defi}[2][]{colbacktitle=green!75!black, colframe=green!75!black, 
colback=green!5!white, coltitle=white,
title={#2},fonttitle=\bfseries\gtfamily,#1}

\newtcolorbox{thm}[2][]{colbacktitle=blue!75!black, colframe=blue!75!black, 
colback=blue!5!white, coltitle=white,
title={#2},fonttitle=\bfseries\gtfamily,#1}

\newtcolorbox{marker}[1][]{enhanced,
  before skip=2mm,after skip=3mm,
  boxrule=0.4pt,left=5mm,right=2mm,top=1mm,bottom=1mm,
  colback=yellow!50,
  colframe=yellow!20!black,
  sharp corners,rounded corners=southeast,arc is angular,arc=3mm,
  underlay={%
    \path[fill=tcbcolback!80!black] ([yshift=3mm]interior.south east)--++(-0.4,-0.1)--++(0.1,-0.2);
    \path[draw=tcbcolframe,shorten <=-0.05mm,shorten >=-0.05mm] ([yshift=3mm]interior.south east)--++(-0.4,-0.1)--++(0.1,-0.2);
    \path[fill=yellow!50!black,draw=none] (interior.south west) rectangle node[white]{\Huge\bfseries !} ([xshift=4mm]interior.north west);
    },
  drop fuzzy shadow,#1}

\renewcommand{\proofname}{\textbf{証明}}

\begin{document}

\title{表現論ゼミ 第2回}
\author{前田 陵汰}
\date{\today}
\maketitle

\part*{半単純Lie代数}

以下、体$𝔽$は代数閉体とし, 標数は$0$とする ($\mathrm{char}𝔽 = 0$). 
また, ベクトル空間は有限次元とする. 

\setcounter{section}{3}
\section{Lieの定理とCartanの判定条件}
\subsection{Lieの定理}

\begin{thm}{定理4.1}
  $L$を$\mathfrak{gl}(V)$の可解な部分代数とする. $V \neq 0$ならば, ある$v \in V$があって, 任意の$L$の元に対して$v$は固有ベクトルとなる. 
\end{thm}

\renewcommand{\labelenumi}{(\roman{enumi})}
\begin{proof}
  $\mathrm{dim}~L$ に関する帰納法を用いる. $\mathrm{dim}~L = 0$の時は自明. $\mathrm{dim}~L > 0$とする. 
  \begin{enumerate}
    \item $L$は可解なので, $[L, L]$は$L$の真部分代数となる. $[L, L]$を含む余次元$1$の部分集合$K$をとれば, $K$はイデアルとなる. 
    \item $\mathrm{dim}~K = \mathrm{dim}~L - 1$なので, 帰納法の仮定から, 次で定義される$0$でない集合$W$が存在する. 
    \begin{equation}
      \qty{w \in V | 𝒙 w = λ(x) w ~ \mathrm{for} ~∀ 𝒙 \in K}
    \end{equation}
    \item $∀ 𝒙 \in L$は$W$を$W$に写すことを示す. すなわち, $ 𝒙 \in L, w \in W ~ \Rightarrow 𝒙w \in W$.
    
    $𝒚 \in K$をとる. 
    \begin{equation}
      𝒚 𝒙 w = 𝒙 𝒚 w - [𝒙, 𝒚]w = λ(𝒚) 𝒙w - λ([𝒙, 𝒚]) w
    \end{equation}
    なので, $λ([𝒙, 𝒚]) = 0$ であれば良い.

    ここで$𝒙$, $w$をひとつ選ぶ. $w, 𝒙x, \ldots, 𝒙^n-1w$が線型独立となる最大の$n$をとり, 
    \begin{equation}
      W_i := 𝔽\mathrm{-span} \qty{w, }
    \end{equation} 

    \item 
  \end{enumerate}
\end{proof}

この定理から, 以下の系が従う. 

\begin{thm}{系4.1A (Lie's Theorem)}
  $L$を$\mathfrak{gl}(V)$の可解な部分代数とする. このとき, $L$は適当な$V$の基底に対して上三角行列となる\footnote{本には "$L$ stabilizes some flag in $V$. " とありました. flag が分からん...}. 
\end{thm}

\begin{thm}{系4.1B}
  $L$が可解であるとき, 以下を満たすイデアルの列が存在する. 
  \begin{equation}
    0 = L_0 \subset L_1 \subset \ldots \subset L_n = L, \quad \mathrm{dim}L_i = i
  \end{equation}
\end{thm}

\begin{thm}{系4.1.C}
  $L$が可解であるとき, $x \in [L, L]$ $\Rightarrow$ $\mathrm{ad_L} x$は冪零. 特に, $[L, L]$は冪零Lie代数となる. 
\end{thm}

\subsection{Jordan-Chevalley 分解}
一般に行列はJordan標準形で表すことができる. これは対角成分と, その上に$1$または$0$が並んだ行列 (これは冪零) への分解と見ることができる. これを一般化しよう. 

\begin{defi}{定義4.2}
  $x \in \mathrm{End}~ V$ が半単純であるとは, $x$の最小多項式(minimal polynomial)が重解を持たないことをいう. 
\end{defi}

\begin{marker}
  上の定義はわかりにくいが, 実は
  \begin{center}
    $x \in \mathrm{End}~ V$ が半単純 $\Leftrightarrow$ $x$ は対角化可能
  \end{center}
  である. 
  
  また, $x$ の固有ベクトルが $V$ の基底をなすことを半単純の定義とする場合もあり\cite{hantanjun}, これも対角化可能であることと同値である. 
\end{marker}

\renewcommand{\labelenumi}{(\alph{enumi})}

\begin{thm}{命題4.2}
  $x \in \mathrm{End}~V$とする. 
  \begin{enumerate}
    \item 以下を満たす$x_s, x_n$がただ一つ存在する. 
    \begin{equation}
      x = x_s + x_n, \qquad x_s~は半単純,~x_n~は冪零.
    \end{equation}
    \item 定数項をもたない一変数多項式$p(T), q(T)$があって, $x_s = p(x),~ x_n = q(x)$. 特に, $x$と交換する$\mathrm{End}~V$の元は$x_s, x_n$ とも交換する. 
    \item $A \subset B \subset V$が部分空間であって, $x$が$B$を$A$に写すならば, $x_s, x_n$もまた, $B$を$A$に写す. 
  \end{enumerate}
\end{thm}

% \begin{proof}
%   $(a_1, \ldots , a_k)$を$x \in \mathrm{End}~V$
% \end{proof}

この分解を Jordan-Chevalley 分解と呼ぶ. 有用性を見るために随伴表現を考える. 

\begin{thm}{補題4.2}
  \begin{center}
    $x$ が半単純 $\Rightarrow$ $\mathrm{ad}~x$も半単純
  \end{center}
\end{thm}
補題3.2では, $x$ が冪零 $\Rightarrow $ $\mathrm{ad}~x$も冪零となることを示した. 

\begin{thm}{補題4.2A}
  $x \in \mathrm{End}~V$ が $x = x_s + x_n$ のように Jordan-Chevalley 分解されているとき, $\mathrm{ad}~x \in \mathrm{End}(\mathrm{End}~V)$の分解は以下で与えられる.  
  \begin{equation}
    \mathrm{ad}~ x = \mathrm{ad}~x_s + \mathrm{ad}~x_n
  \end{equation}
\end{thm}

\begin{thm}{補題4.2B}
  $\mathfrak{U}$を$𝔽$-代数とする. このとき, $\mathrm{Der}~\mathfrak{U}$の任意の元は, $\mathrm{Der}~\mathfrak{U}$内に半単純成分と冪零成分を持つ. 
\end{thm}

\begin{proof}
  $δ \in \mathrm{Der}~\mathfrak{U}$ とし, $δ = σ (半単純) + ν (冪零)$と分解できたとする. $σ \in \mathrm{Der}~\mathfrak{U}$ を示す. 

  $∀a \in 𝔽$ に対し, $\mathfrak{U}_a = \qty{x \in \mathfrak{U} |~ (δ - a \mathbb{1})^k x = 0 ~\mathrm{for}~∃k}$ を定義. このとき以下が成立. 
  \begin{equation}
    \mathfrak{U} = \bigoplus_{a} \mathfrak{U}_a \qquad (a~は~δ~(またはσ)~の固有値)
  \end{equation}
  つまり, 適当な$U_a$の元を集めて基底とできる. 
  
  ここで次の恒等式を用いる. 
  \begin{equation}
    \qty(δ - (a+b) \mathbb{1})^n [x, y] = \sum_{i=0}^{n} \tensor[_n]{C}{_i} \qty[(δ-a)^{n-i} x, (δ-b)^i y] \label{eq42}
  \end{equation}
  十分大きな$n$を持ってくれば右辺は$0$とできるので, $x \in \mathfrak{U}_a,~ y \in \mathfrak{U}_b ~ \Rightarrow~ [x, y] \in \mathfrak{U}_{a+b}$ が得られる. 

  $x \in \mathfrak{U}_a,~ y \in \mathfrak{U}_b$ に対し, 
  \begin{equation}
    \begin{aligned}
      σ ([x, y]) &= (a + b) [x, y] \\
      &= [ax, y] + [x, by] \\
      &= [σx, y] + [x, σy]
    \end{aligned}
  \end{equation}
  したがって$σ$はライプニッツ則を満たすので, $σ \in \mathrm{Der}~\mathfrak{U}$. 
\end{proof}

\medskip
\footnotesize
恒等式\eqref{eq42}の証明は$n$に関する帰納法による. $n = 0$では明らかに成立. $n > 0$とする. 

\begin{equation}
  \begin{aligned}
    (左辺) &= (δ - (a+b)) ~(δ - (a+b))^{n-1} ~[x, y] \\
    &= (δ - (a+b)) ~\sum_{i=0}^{n-1} \tensor[_{n-1}]{C}{_i} \qty[(δ-a)^{n-1-i} x, (δ - b)^{i} y] \\
    &= \sum_{i = 0}^{n-1} \tensor[_{n-1}]{C}{_i} \qty{ \qty[(δ-a)^{n-i} x, (δ - b)^{i} y] + \qty[(δ-a)^{n-1-i} x, (δ - b)^{i+1} y]} \\
    &= \sum_{i = 0}^{n-1} \tensor[_{n-1}]{C}{_i} \qty[(δ-a)^{n-i} x, (δ - b)^{i} y] + \sum_{i = 0}^{n-1} \tensor[_{n-1}]{C}{_{i-1}}\qty[(δ-a)^{n-i} x, (δ - b)^{i} y] \\
    &= \sum_{i = 0}^{n} \tensor[_n]{C}{_i} \qty[(δ-a)^{n-i} x, (δ - b)^{i} y] = ~(右辺)
  \end{aligned}
\end{equation}


\normalsize
\subsection{Cartanの判定条件}
\begin{thm}{補題4.3}
  $A \subset B$を $\mathfrak{gl}(V)$ の部分空間とし, 集合$M$を $M = \qty{x \in \mathfrak{gl}(V)|~[x, B] \subset A}$ とする.
  このとき, $x \in M$ が $∀y \in M$ に対して $\mathrm{Tr}(xy) = 0$ を満たすならば, $x$は冪零である. 
\end{thm}

\begin{proof}
  $x = s + n$と分解し, $V$の基底を$s$が対角行列となるようにとる. $s = \mathrm{diag}~(a_1, \ldots, a_m) = 0$ を示す. 

  $E$を$(a_1, \ldots, a_m)$で張られる$𝔽$の部分空間とする. なお係数体は有理数体$ℚ$とする. $E = 0$を示せばよく, 有限次元なので, $E^* = 0$(双対空間) を示せば良い. \\
  $∀f \in E^* ~(f: E \rightarrow ℚ, 線型写像)$をとる. また, $y \in \mathfrak{gl}~(V)$を, $y = \mathrm{diag}(f(a_1), \ldots, f(a_m))$とする. 
  $e_{ij} \in \mathfrak{gl}(V)$に対し, 
  \begin{align}
    \mathrm{ad}~s ~(e_{ij}) &= (a_i - a_j) e_{ij} \label{ads} \\
    \mathrm{ad}~y ~(e_{ij}) &= (f(a_i) - f(a_j)) e_{ij} \label{ady}
  \end{align}
  であった(4.2節). ここで, 定数項を持たない$𝔽$係数多項式$r(t)$を
  \begin{equation}
    r(a_i - a_j) = f(a_i) - f(a_j) \\
  \end{equation}
  を満たすようにとる\footnote{このような多項式の存在はラグランジュ補間 (Lagrange interpolation) によって保証される. }. 式\eqref{ads}, \eqref{ady}より, 
  \begin{equation}
    \mathrm{ad}~y = r(\mathrm{ad}~s)
  \end{equation}
  である. 命題4.2, 補題4.2Aより, 
  % label変更
  $\mathrm{ad}~s$は$\mathrm{ad}~x$の定数項を持たない多項式で書けるので, $\mathrm{ad}=y$も同様である. $x \in M$より, $\mathrm{ad}~x$は$B$を$A$に写すから, $\mathrm{ad}~y$もまた$B$を$A$に写し, よって$y \in M$. 

  仮定より, $\mathrm{Tr(xy)} = 0 \Rightarrow \sum a_i ~ f(a_i) = 0$. 
  $f$を作用させて, $\sum f(a_i)^2 = 0$. $f(a_i)$は有理数なので, $f(a_i) = 0 ~\mathrm{for}~ ∀a_i$. したがって, $f = 0$となり, 題意は示された. 
\end{proof}


ここで有用な恒等式を述べておく. \\
$x, y, z \in \mathrm{End}~(V)$に対し, 
\begin{equation}
  \mathrm{Tr} ([x, y]~ z) = \mathrm{Tr} (x ~[y, z]) \label{trace}
\end{equation}


\begin{thm}{定理4.3 (Cartan's Criterion)}
  $L$ を $\mathfrak{gl}(V)$の部分代数とする. 任意の $x \in [L, L]$, $y \in L$ に対して $\mathrm{Tr} (xy) = 0$ であるならば, $L$は可解Lie代数である. 
\end{thm}


% label変更
\begin{proof}
  $L$が可解であることを示すためには, $[L, L]$が冪零Lie代数であることを示せば良い (命題4.2Cの逆) \footnote{教科書は "it is obvious" と言ってますが, 全然分かりません. }. 
  上の補題を, $A = [L, L], ~B = L$として適用したいので, \\
  $M = \qty{x \in \mathfrak{gl}~(L)| [x, L] \subset [L, L]}$ とする. 明らかに, $L \subset M$である\footnote{この定理の仮定は, $∀x \in [L, L],~ ∀y \in L~ \mathrm{s.t.} ~ \mathrm{Tr}(xy) = 0$であり, $y \in M$ではないので, このままでは補題を適用できない. }. 
  
  % 

  $∀z \in M,~ p, q \in L ~(よって[p, q] \in [L, L])$ をとる. 恒等式\eqref{trace}より, 
  \begin{equation}
    \mathrm{Tr}~([p, q] z) = \mathrm{Tr}~(p [q, z]) = \mathrm{Tr}~([q, z] p)
  \end{equation}
  $z \in M$より, $[q, z] \in [L, L]$なので, 定理の仮定から, $\mathrm{Tr}~([p, q] z) = 0$. 任意の$[L, L]$の元は$[p, q]$の和の形で書けるから, 
  \begin{equation}
    ∀x \in [L, L],~ ∀z \in M~ \mathrm{s.t.}~ \mathrm{Tr}~(xz) = 0
  \end{equation}
  よって補題より$x$は冪零であり, $\mathrm{ad}~x$ も冪零. Engelの定理から, $[L, L]$は冪零Lie代数となる. 
\end{proof}


\begin{thm}{系4.3}
  $L$ を Lie代数とする. 任意の $x \in [L, L]$, $y \in L$ に対して $\mathrm{Tr} (\mathrm{ad}x ~ \mathrm{ad}y) = 0$ であるならば, $L$は可解Lie代数である.
\end{thm}

\begin{proof}
  定理4.3を$V \rightarrow L, \mathfrak{gl}~(V)の部分代数~L \rightarrow \mathfrak{gl}~(L)の部分代数~\mathrm{ad}~(L)$ として適用する. 定理の仮定を満たすことを確認する. $x \in [L, L]$ より $\mathrm{ad}~x \in [\mathrm{ad}~L, \mathrm{ad}~L]$, $y \in L$ より $\mathrm{ad}~y \in \mathrm{ad}~L$. また系の仮定から $\mathrm{Tr}~(\mathrm{ad}~x ~ \mathrm{ad}~y) = 0$. ゆえに定理を適用できて, $\mathrm{ad}~L$は可解である. 

  $\mathrm{ad} : L \rightarrow \mathfrak{gl}(L)$ に対して $\mathrm{Ker~ad}= Z(L)$, $\mathrm{Im~ad} = \mathrm{ad}~L$なので, 準同型定理より, 
  \begin{equation}
    L / Z(L) \simeq \mathrm{ad}~L
  \end{equation}
  中心化代数$Z(L)$は定義から可解であるので, 命題3.1(b) \footnote{$\mathfrak{g}$のイデアル$\mathfrak{i}$と商代数$\mathfrak{g/i}$がどちらも可解ならば, $g$自身も可解である. }により, $L$は可解である. 
\end{proof}


\section{Killing 形式}

\subsection{半単純性の判定条件}

\begin{defi}{定義5.1A: Killing形式}
  $L$を任意のLie代数とする. $x, y \in L$に対し, Killing形式$κ(x, y)$を次式で定義する. 
  \begin{equation}
    κ(x, y) = \mathrm{Tr}(\mathrm{ad}x ~ \mathrm{ad}y)
  \end{equation}
\end{defi}
$κ$は対称な双線型写像であり, 次式の意味で結合則を満たす. 
\begin{equation}
  κ([x,y] ,~z) = κ(x,~ [y,z]) \label{ketsugou}
\end{equation}

\begin{thm}{補題5.1}
  $I$は$L$のイデアルとする. $κ$が$L$上のKilling形式, $κ_I$が$I$上のKilling形式とすると, $κ_I = κ|_{I \times I}$ 
\end{thm}

\footnotesize
記号の意味がややこしいので補足. 
\begin{table}[H]
  \centering
  \footnotesize
  \begin{tabular}{rp{10cm}}
    $κ_I$ &: $I$上のKilling形式. 引数の$x,y$は初めから$I$の元のみ. $\mathrm{ad}~x$は大きさ$\mathrm{dim}I \times \mathrm{dim}I$の行列. \\
  $κ|_{I \times I}$ &: $L$上のKilling形式で, 定義域を$I \times I$に制限したもの. 引数の$x, y$は制限によって$I$の元のみになる. $\mathrm{ad}~x$は大きさ$\mathrm{dim}L \times \mathrm{dim}L$の行列.
  \end{tabular}
\end{table}

\normalsize
\begin{proof}
  まず, 線形代数における事実を確認しよう. 

  $V$をベクトル空間, $W$をその部分空間とする. 写像$ϕ: V \rightarrow V$の像が$\mathrm{Im}~ϕ \subset W$を満たす時, 
  \begin{equation}
    \mathrm{Tr} ϕ = \mathrm{Tr} (ϕ|_W)
  \end{equation}
  ここで, $ϕ|_W$は定義域を$W$に制限した, $\mathrm{dim}W \times \mathrm{dim}W$の行列である. 

  補題に戻る. $x, y \in I$なので, $\mathrm{Im}~(\mathrm{ad}~x)(\mathrm{ad}~y) \subset I$. よって, 
  \begin{equation}
    \begin{aligned}
      κ|_{I \times I} &= \mathrm{Tr}\qty((\mathrm{ad}~x)(\mathrm{ad}~y)) \\
      &= \mathrm{Tr}\qty((\mathrm{ad}~x)(\mathrm{ad}~y)|_I) \\
      &= \mathrm{Tr}\qty((\mathrm{ad}_I~x)(\mathrm{ad}_I~y)) = κ_I
    \end{aligned}
  \end{equation}
\end{proof}



\begin{defi}{定義5.1B: 非退化}
  $L$上の対称な双線型写像$β(x, y)$のradical $S$が$0$ (零集合) のとき, $β$は非退化であるという. ここで, $S = \qty{x \in L| β(x, y) = 0 ~\mathrm{for}~ ∀y \in L}$\footnote{この$S$は双線型写像に対する根基 (radical). 前回出てきたのはLie代数$\mathfrak{g}$に対する根基 (radical). 名前は同じだが別のもの. } . 
\end{defi}
Killing形式に対する$S$は, $L$のイデアルとなる ($\because$結合則). 

$κ$が非退化であるかは次のようにして判定できる\footnote{教科書に判定法として載っていましたが, なぜこれで判定できるのかが不明です. わかる人助けて. }. \\
$L$の基底$\qty{x_1, \ldots, x_n}$をとり, 行列 $K_{ij} = κ(x_i, x_j)$ を定義する. 
このとき, 
\begin{equation}
  κ が非退化 ~ \Leftrightarrow \mathrm{det} K \neq 0
\end{equation}


\begin{thm}{定理5.1}
  $L$をLie代数とする. このとき
  \begin{equation}
    L が半単純 ~ \Leftrightarrow ~ κ が非退化
  \end{equation}
\end{thm}

\begin{proof}
  \noindent \underline{$(\Rightarrow)$}

  仮定より, $\mathrm{Rad}~L = 0$ である. $S$を$κ$の radical とする. $S$の定義より,
  \begin{equation*}
    \mathrm{Tr}~(\mathrm{ad}~x ~\mathrm{ad~}y) = 0 ~ \mathrm{for}~ ∀x \in S, ∀y \in L ~(特に, y \in [S, S])
  \end{equation*}
  よって補題4.3により, $S$は可解イデアルとなり, $S \subset \mathrm{Rad}~L = 0$. したがって$κ$は非退化である. 

  \noindent \underline{$(\Leftarrow)$}

  仮定より, $S=0$. 
  \begin{equation}
    \mathrm{Rad}~L = 0 ~\Leftrightarrow ~ 0でない可換イデアルは存在しない\footnote{対偶で考えると分かりやすい. 可換イデアルとは, イデアル内の元同士がすべて可換であることを指す. }
  \end{equation}
  なので, 右側の命題を示す. \\
  $I$を$L$の可換イデアルとする. $∀x \in I, y \in L$に対し, 
  \begin{equation}
    \begin{aligned}
      \mathrm{ad}~x ~ \mathrm{ad}~y &: L \rightarrow L \rightarrow I \\
      (\mathrm{ad}~x ~ \mathrm{ad}~y)^2 &: L \rightarrow [I, I] = 0
    \end{aligned}
  \end{equation}
  よって, $\mathrm{ad}~x ~ \mathrm{ad}~y$は冪零であり, $\mathrm{Tr}~(\mathrm{ad}~x ~ \mathrm{ad}~y) = κ(x, y) = 0$. 
  ゆえに, $x \in S ~\Rightarrow ~ I \subset S $となるが, $S = 0$なので$I = 0$であり, 題意は示された. 
\end{proof}


\subsection{単純イデアル}

\begin{thm}{定理5.2}
  $L$を半単純Lie代数とする. このとき, $L$の単純なイデアル$L_1, \ldots , L_t$が存在して, $L$は$L_i$の直和で書ける. すなわち, 
  \begin{equation}
    L = L_1 ⊕ \ldots ⊕ L_t
  \end{equation}
  また, $L$上の単純なイデアルは$L_i$以外に存在しない. 
  さらに, $L_i$上のKilling形式は, $L$上のKilling形式で定義域を$L_i \times L_i$上に制限したものと一致する. 
\end{thm}

\renewcommand{\labelenumi}{(\roman{enumi})}
\begin{proof}
  \noindent (1) まず, 半単純なLie代数がイデアルの直和で書けることを示す. 
  
  $L$を半単純Lie代数とし, $I$を任意のイデアルとし, \\
  $I^{⊥} = \qty{x \in L|~ κ(x, y) = 0 ~\mathrm{for}~∀y \in I}$とおく. $κ$が結合則(式\eqref{ketsugou})を満たすので, $I^{⊥}$はイデアルである\footnote{
  $x \in I^{⊥}, z \in L \Rightarrow [x, z] \in I^{⊥}$を示す. $y \in I$として, 
  \begin{equation}
    κ([x, z]~y) = κ(x~[y, z]) = 0 \quad (∵ [y, z] \in I)
  \end{equation} 
  よって, $[x, z] \in I^{⊥}$. }. 

  $∀x \in [I ∩ I^{⊥}, I ∩ I^{⊥}] \subset I^{⊥}, ∀y \in I ∩ I^{⊥} \subset I$ に対し, 
  \begin{equation}
    \mathrm{Tr}~ (\mathrm{ad}~x \mathrm{ad}~y) = κ (x, y) = 0
  \end{equation}
  よって, Cartanの判定条件 (系4.3) より, $I ∩ I^{⊥}$は可解となるが, $L$が半単純であることから, $I ∩ I^{⊥} = {0}$. 
  これと $\mathrm{dim}I + \mathrm{dim}I^{⊥} = \mathrm{dim}L$\footnote{文献\cite{watanabe}の27ページに証明が載っていましたが, イマイチ不完全な感じがします (行列$𝔹'$がフルランクな理由がわからない). }により, $L = I ⊕ I^{⊥}$ と書ける. 
  
  \medskip
  \noindent (2) 次に, $L$が単純なイデアルの直和で書けることを示す. 

  $L$が非自明なイデアルを持たない場合, すでに$L$は単純である. 

  $L$が非自明なイデアルを持つ場合, 極小イデアルの一つを$L_1$として, $L = L_1 ⊕ L_1^{⊥}$と分解できる. $L$が$0$でない可解イデアルを持たないので, $L_1$もまた$0$
  でない可解イデアルを持たず半単純である. これと極小であることから, $L_1$は単純イデアルとなる. 同様の理由で$L_1^{⊥}$も半単純なので, この操作を繰り返し行うことで, $L$を単純なイデアルの直和で表せる. 

  \medskip
  \noindent (3) 最後に$L_i$以外の単純イデアルが存在しないことを示し, 直和の分解が一意的であることを示す. 

  $I$を$L$の単純イデアルとする. $Z(L) = 0$なので, $[I, L]$は$0$でない$I$のイデアルとなる ($∵ [I, L] = 0 ~\Rightarrow ~ I \subset Z(L)$). さらに$I$が単純であることから, $[I, L] = I$. 一方で, $L$の直和分解より, 
  \begin{equation}
    [I, L] = [I, L_1] ⊕ \ldots ⊕ [I, L_t]
  \end{equation}
  各$[I, L_i]$は$I$のイデアルとなるから, $0$または$I$. よって左辺と比較して, ただ一つの$k$に対して$I = [I, L_k]$で, それ以外の$L_i$に対して$[I, L_i] = 0$. 
  ゆえに$I \subset L_k$となるが, $L_k$が単純なので$I = L_k$. 

  以上より, $L_i$以外の単純イデアルは存在せず, 直和分解は一意的である. 

  \medskip
  \noindent (4) 補題5.1により, $L_i$上のKilling形式は, $L$上のKilling形式で定義域を$L_i \times L_i$に制限したものと一致する. 
\end{proof}

\renewcommand{\labelenumi}{(\alph{enumi})}
\begin{thm}{系5.2}
  $L$を半単純Lie代数とする. 
  \begin{enumerate}
    \item $L = [L, L]$. 
    \item $L$の任意のイデアル, および$L$を定義域とする任意の準同型写像の像も半単純Lie代数となる. 
    \item $L$の任意のイデアルは, 適当な$L$の単純イデアルの直和で書ける. 
  \end{enumerate}
\end{thm}

\begin{proof}
  \noindent (a)
  \begin{equation}
    \begin{aligned}
      [L, L] &= \bigoplus_{i} [L, L_i] \\
      &= \bigoplus_{i} L_i \qquad \qty(∵ [L, L_i] \neq \qty{0}) \\
      &= L
    \end{aligned}
  \end{equation}

  \medskip
  \noindent (b)
  
  $L$のイデアル$\tilde{L}$が半単純でないと仮定して矛盾を導く. $\tilde{L}$が半単純でないので, 定理5.1より, $\tilde{L}$のKilling形式$\tilde{κ}$は退化である. すなわち, 
  \begin{equation}
    ∃a \in \tilde{L} \backslash \qty{0} ~\mathrm{s.t.} \tilde{κ}(a, x) = 0 ~ \mathrm{for} ∀x \in \tilde{L}
  \end{equation}
  補題5.1より, $a$は$κ(a, x) = 0 ~ \mathrm{for}~∀x \in \tilde{L}$を満たすので, $a \in \tilde{L}^{⊥}$\footnote{$\tilde{L}^{⊥} = \qty{x \in L|~ κ(x, y) = 0 ~ \mathrm{for}~ ∀y \in \tilde{L}}$}となるが, $\tilde{L} ∩ \tilde{L}^{⊥} = \qty{0}$なのでこれは矛盾. よって$\tilde{L}$は半単純である. 

  次に, 準同型像も半単純であることを示す\footnote{調べても出てこなくて, 合ってるか微妙です. }. $f: L \rightarrow \mathrm{Im}~f$ (準同型) に対し, 準同型定理より, 
  \begin{equation}
    \mathrm{Im}~f \simeq L / \mathrm{Ker}~f \simeq (\mathrm{Ker}~f)^{⊥}
  \end{equation}
  定理5.2の証明(1)より, $(\mathrm{Ker}~f)^{⊥}$も$L$のイデアルなので, 半単純. よって$\mathrm{Im}~f$も半単純である. 

  \medskip
  \noindent (c)
  
  上記(b)の主張, および直和分解の一意性から従う. 
\end{proof}

\subsection{内部微分}

\begin{thm}{定理5.3}
  $L$が半単純Lie代数ならば, $\mathrm{ad}~L = \mathrm{Der}~L$. すなわち, 任意の微分は内部微分.   
\end{thm}

\footnotesize
一般のLie代数$L$に対して, $\mathrm{ad}~L$は$\mathrm{Der}~L$のイデアルであることを確認しておこう. \\
$δ \in \mathrm{Der}~L, \mathrm{ad}~x \in \mathrm{ad}~L, v \in L$に対し, 
\begin{equation}
  \begin{aligned}
    [δ, \mathrm{ad}~x] (v) &= δ [x, v] - \mathrm{ad}~x (δv) \\
    &= [δx, v] + [x, δv] - [x, δv] \\
    &= \mathrm{ad}~δx (v)
  \end{aligned}
\end{equation}
よって, $[δ, \mathrm{ad}~x] \in \mathrm{ad}~L$. 

\normalsize

\begin{proof}
  $L$が半単純なので, $Z(L) = \qty{0}$. つまり, $\mathrm{Ker~ad} = \qty{0}$なので, $L$と$\mathrm{ad}~L$は同型であり, 特に$\mathrm{ad}~L$は半単純. \\
  % $M = \mathrm{ad}~L, D = \mathrm{Der}~L$とおく. $M$は非退化な$κ_M$をもつ. 
  % また, $[M, D] \subset M$. 
  % 補題5.1より, $κ_M = 
  $M = \mathrm{ad}~L$とおく. $M^{⊥} = \qty{x \in D|~ κ(x,y) = 0 ~\mathrm{for}~∀y \in M}$とすれば, 定理5.2の証明(1)より, $M ∩ M^{⊥}$. 
  よって, $[M, M^{per}] \subset M$かつ$[M, M^{per}] \subset M^{⊥}$より, $[M, M^{per}] = \qty{0}$. 
  
  したがって, $∀δ \in I$に対して, 
  \begin{equation}
    \mathrm{ad}~δx = [δ, \mathrm{ad}~x] = 0 ~\Rightarrow ~ δx = 0 ~\mathrm{for}~∀x \in L
  \end{equation}
  つまり, $δ = 0$であり, $I = \qty{0}$. 
  したがって, $\mathrm{Der}~L = M = \mathrm{ad}~L$. 
\end{proof}

\subsection{抽象 Jordan 分解}
$L$を半単純Lie代数とする. 
補題4.2Bより, $\mathrm{Der}~L$は$\mathrm{Der}~L$内に半単純成分と冪零成分を持つ\footnote{この補題は$L$が半単純でなくても成り立つ}. いま, $\mathrm{Der}~L = \mathrm{ad}~L$であり, $L$と$\mathrm{ad}~L$は一対一対応である. 
よって, $∀x \in L$に対して$∃s, n \in L$ $\mathrm{s.t.}$
\begin{equation}
  \mathrm{ad}~x = \mathrm{ad}~s + \mathrm{ad}~n \qquad (sは半単純, nは冪零)
\end{equation}
よって, $x = s + n$と分解することができ, これを抽象Jordan分解と呼ぶ. $s, n$はそれぞれ半単純成分, 冪零成分と呼ばれる\footnote{線型写像でないLie代数に対して "半単純成分", "冪零成分"を定義できる. }.  

特に$L$が線形Lie代数である場合, $∀x \in L$は$x = x_s + s_n$とJordan-Chevalley分解できる. これが抽象Jordan分解によって得られた$s, n$と一致することは6.4節で確認する. 

% \section{表現の可約性}







\begin{thebibliography}{99}
  \bibitem{wakayama} 田川 裕之, Lie環論入門, 
  \newblock \url{https://web.wakayama-u.ac.jp/~tagawa/lecture/liealgh.pdf}
  \bibitem{hantanjun} 対角化と固有値問題, 
  \newblock \url{https://w.atwiki.jp/nopu/pages/138.html}
  \bibitem{watanabe} 渡邉 究, 数学特別講義X, 代数学特論V (リー代数入門)
  \newblock \url{http://www.rimath.saitama-u.ac.jp/lab.jp/kwatanab/lie-algebra2015.pdf}
\end{thebibliography}


\end{document}